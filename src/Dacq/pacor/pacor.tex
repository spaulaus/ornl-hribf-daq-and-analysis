   29-Dec-17 .... U210  PACOR -  Physics Acq Compiler - Revise ...... PAGE   1
 
 
 
   Sec Page Contents
 
   010   1  Introduction
 
   020   2  Hardware Description & Compiler Directives - Syntax
 
   030   3  Hardware Description & Compiler Directives - Definitions
 
   035   4  Module Type Table (MOTYP)                  - Definitions
 
   040   5  Hardware Description & Compiler Directives - Discussion
 
   050   7  Conditional (programmed) CAMAC Readout     - Syntax
 
   060   7  Conditional (programmed) CAMAC Readout     - Discussion
 
   070   8  Conditional (programmed) CAMAC Readout     - Example
 
   080   9  PACOR Output Tables (for McConnell & Milner only)
 
 
   U210.010  Introduction
 
   User-supplied  front-end  programming  of  the  DECstation/VME-based   data
   acquisition system consists of two principal parts:
 
   (1)....A  hardware  description table in which the locations of all modules
          to be read out are given along with associated device  names  (ADCs,
          TDCs, etc.), parameter-IDs, function codes for module clearing, etc.
 
 
   (2)....A  conditional  readout section in which certain CAMAC sub-addresses
          may be read (or not) depending on pattern word  bits.  Given  module
          types of Fastbus or Ferabus may also be conditionally read.
 
   It  is  intended that the hardware description table should be more or less
   self documenting with the device names serving as the primary label as well
   being   used   in  any  subsequent  conditional  readout  section  to  make
   connections to hardware locations, etc.
 
                               Data Stream Format
 
   At this first stage of development, the data stream format will be what  we
   will  call  L003  -  that is: a sequence of (parameter-ID, data-word) pairs
   terminated by two hex FFFFs.
 
                               How to Get Started
 
   Type: /usr/hhirf/pacc prog    ;To compile prog.pac.
                                 ;Expanded listing & diagnostics
                                 ;are written to prog.lst
 
   Type: /usr/hhirf/pacc prog L  ;To compile prog.pac and
                                 ;download to VME front-end.
    
   29-Dec-17 .... U210  PACOR -  Physics Acq Compiler - Revise ...... PAGE   2
 
   U210.020  Hardware Description & Compiler Directives - Syntax
 
   $ini  cnaf C,N,A,F,DATA      ;Executed at start-up time only    (200 max)
   $dun  cnaf C,N,A,F,DATA      ;Executed at end-of-event or kill  (200 max)
   $run  cnaf C,N,A,F,DATA      ;Executed at TRUN or STARTVME      (200 max)
   ---------------------------------------------------------------------------
   $dla  uncondit  DT           ;Delay-time for  unconditional CAMAC   readout
   $dla  condit    DT           ;Delay-time for    conditional CAMAC   readout
   $dla  camac     DT           ;Delay-time for block-transfer CAMAC   readout
   $dla  fastbus   DT           ;Delay-tine for                Fastbus readout
   $dla  fera      DT           ;Delay-time for                Fera    readout
   $dla  vme       DT           ;Delay-time for                VME     readout
                                ;Valid DT range is 0 to 255 microseconds
   ---------------------------------------------------------------------------
   $lat  c0C  n0N  f0F  a0A     NAME:0N              fc0F  ac0A
   $lat  c0C  n0N  f0F  a0A     NAME:0N     id0N     fc0F  ac0A
   ---------------------------------------------------------------------------
   $cam  c0C  n0N  f0F  a0A-0B  NAME:0N,0I  id0N,0I  fc0F  ac0A
 
   $cam  c0C  n0N       a0A-0B  NAME:0N,0I  id0N,0I  fc0F  ac0A  mt=MOTYP
 
   $fer  c0C  n0N       a0A-0B  NAME:0N,0I  id0N,0I              mt=MOTYP
 
   $fas  c0C  n0N       a0A-0B  NAME:0N,0I  id0N,0I              mt=MOTYP
 
   $xia  c0C  n0N  v0V  g0G
 
   $vme  adc0N          a0A-0B  NAME:0N,0I  id0N,0I              mt=MOTYP
 
   $vme  tdc0N          a0A-0B  NAME:0N,0I  id0N,0I              mt=MOTYP
 
   $vme  sis0N          a0A-0B  NAME:0N,0I  id0N,0I              mt=SIS_3820
 
   $did  FASI,FERI,CAMI         ;IDs for illegal Fastbus, Fera, CAMAC  readout
 
   $cid  idhi idlo              ;ID#s for hi- & lo-order parts of 100HZ clock
   ---------------------------------------------------------------------------
   $rgat GNAM(N) PNAM(I) LO,HI              ;define raw gate GNAM(N)
   ....                                     ;on parameter    PNAM(I)
   ....                                     ;gate limits  =  LO,HI
 
   $cgat CNAM(N) = <N>GNAM(I) OP<N>GNAM(J)..;define constructed gate CNAM(N)
   ....                                     ;from raw gates GNAM(I),GNAM(J)...
   ....                                     ;<N> is optional and implies .not.
   ....                                     ;OP is either .and. or .or.
   ....                                     ;evaluation is left-to-right
   ---------------------------------------------------------------------------
   $kil any   LW(N),MSK         ;Kill event if any latch & mask bits match
   $kil none  LW(N),MSK         ;Kill eveny if no  latch & mask bits match
   $kil true  GNAM(N)           ;Kill event if gate GNAM(N) is true
   $kil false GNAM(N)           ;Kill event if gate GNAM(N) is false
 
   $rif LW(N)  MSK   MOTYP      ;If any latch & mask bits match, read MOTYP
   $rif true GNAM(N) MOTYP      ;If gate GNAM(N) is true,        read MOTYP
 
   $cdn GNAM(N) ICOUNT          ;If gate GNAM(N) true, "countdown" via ICOUNT
   ---------------------------------------------------------------------------
   $pat  PAT =  NAME(1),I1,I2  NAME(2),J1,J2  NAME(3),K1,K2 ....
   ---------------------------------------------------------------------------
   $lat, $cam, $fer, $fas, $xia, $vme denote: Gated-latch, CAMAC, Fera,
                                              Fastbus, XIA & VME  modules
   ---------------------------------------------------------------------------
    
   29-Dec-17 .... U210  PACOR -  Physics Acq Compiler - Revise ...... PAGE   3
 
                   IMPORTANT COMMENTS RELATED TO GATES
 
   (1)....All gates are inclusive.
 
   (2)....All parameters on which gates are set must be CAMAC parameters.
 
   (3)....Gates are only  supported  for  CAMAC  modules  which  allow  randon
          access  to  the  data. No gate support is provided for modules which
          generate only compressed data readout (i.e. sparce data  readout)  -
          the LECROY-2277 TDC is a prime example.
 
   U210.030  Hardware Description & Compiler Directives - Definitions
 
   The  following  list  of  definitions  apply to quantities specified in the
   $lat, $cam, $fas, $fer, $xia, and $vme directives above.
 
   c     ;Labels the crate number (see Note-1)   - denoted by 0C
 
   n     ;Labels the slot  number                - denoted by 0N
 
   a     ;Labels sub-address range for read      - denoted by 0A-0B
 
   f     ;Labels function-code for read          - denoted by 0F
 
   fc    ;Labels function-code for clear         - denoted by 0F
 
   ac    ;Labels sub-address for clear           - denoted by 0A
 
   v     ;Labels the virtual station number      - denoted by 0V - $xia only
 
   g     ;Labels the group number (optional)     - denoted by 0G - $xia only
 
   tdc   ;Labels CAEN TDC 775 number             - denoted by 0N - $vme only
 
   adc   ;Labels CAEN ADC 785, 791 number        - denoted by 0N - $vme only
 
   sis   ;Labels SIS 3820 number                 - denoted by 0N - $vme only
 
   mt    ;Labels Fera or Fastbus module-type     - denoted by MOTYP
         ;(Required for all Fera & Fastbus modules - optional for CAMAC)
 
   NAME  ;User-supplied name for program reference (12 bytes max)
         ;0N denotes first index (corresponds to sub-addr 0A)
         ;0I denotes the   index increment
 
   id    ;Labels parameter-ID and parameter-ID-increment entry
         ;0N denotes parameter-ID corresponding to sub-address 0A
         ;0I denotes parameter-ID increment
   ---------------------------------------------------------------------------
   PAT   ;See discussion of $pat (SECT# 040)below
   ---------------------------------------------------------------------------
   Note-1 CAMAC crate numbers are supported as indicated below:
 
   Crate# 0-7           are supported for CAMAC data readout (via $lat & $cam)
 
   Crate# 0-7 and 10-17 are supported for Fera  data readout (via $fer)
 
   Crate# 0-7 and 10-17 are supported for writing (i.e. via $ini, $dun & $run)
   ---------------------------------------------------------------------------
    
   29-Dec-17 .... U210  PACOR -  Physics Acq Compiler - Revise ...... PAGE   4
 
 
   U210.035  Module Type Table (MOTYP) - Definitions
 
   MOTYP = LRS_4300     for LECROY-4300   (Fera    ADC) - Code-01
   MOTYP = GAN_812F     for GANELEC-812F  (Fera    TDC) - Code-02
   MOTYP = SILENA_4418  for SILENA-4418   (Fera    ADC) - Code-03
 
   MOTYP = BAKLASH      for Clover module (Fera    ADC) - Code-06 ***
   MOTYP = BAKLASH      for Clover module (Fera    ADC) - Code-07 ***
   MOTYP = BAKLASH      for Clover module (Fera    ADC) - Code-08 ***
 
   MOTYP = AD_413       for ORTEC AD_413  (Fera    ADC) - code-09
 
   MOTYP = LRS_1885     for LECROY-1885   (Fastbus ADC) - Code-11
   MOTYP = PHIL_10C6    for PHILLIPS-10C6 (Fastbus TDC) - Code-12
   MOTYP = LRS_1872     for LECROY-1872   (Fastbus TDC) - Code-13
 
   MOTYP = LRS_1875     for LECROY-1875   (Fastbus TDC) - Code-13
   MOTYP = LRS_1881     for LECROY-1881   (Fastbus ADC) - Code-14
   MOTYP = LRS_1877     for LECROY-1877   (Fastbus ADC) - Code-15
 
   MOTYP = PHIL_7164    for PHILLIPS-7164 (CAMAC   ADC) - Code-21
   MOTYP = PHIL_7186    for PHILLIPS-7186 (CAMAC   TDC) - Code-22
   MOTYP = LRS_2277     for LECROY-2277   (CAMAC   TDC) - Code-23
 
   MOTYP = SILENA_4418C for SILENA-4418   (CAMAC   ADC) - Code-24
   MOTYP = LRS_4300C    for LECROY-4300C  (CAMAC   ADC) - Code-26
   MOTYP = LRS_3377C    for LECROY-3377C  (CAMAC   ADC) - Code-27
 
   MOTYP = XIA_TIME     for XIA TIME      (CAMAC   MOD) - Code-28
   MOTYP = AD_413C      for ORTEC AD_413C (CAMAC   ADC) - Code-29
 
   MOTYP = CAEN-775     for CAEN-775 TDC                - Code-41
   MOTYP = CAEN-785     for CAEN-785 ADC                - Code-42
   MOTYP = CAEN-792     for CAEN-792 ADC                - Code-43
 
   MOTYP = SIS-3820     for SIS-3820 scaler             - Code-44
 
   MOTYP = MYRIAD       for Myriad trigger, clock       - Code-45
   ---------------------------------------------------------------------------
 
   *** Data readout by the BAKLASH module types:
 
   Currently there are three module types:  BAKLASH,  BAKLASH2  and  BAKLASH3.
   Zero-suppression  is  done  on  each  module  individually. The data from a
   clover module and corresponding output vs module type is:
 
   Word#  Parameter                 Module types which readout listed param
   -----  -----------------------   ---------------------------------------
       1  VSN (internal use only)
       2  ??? (junk word)
       3  Hit pattern               BAKLASH, BAKLASH2, BAKLASH3
       4  Hires GE                  BAKLASH, BAKLASH2, BAKLASH3
       5  Hires GE                  BAKLASH, BAKLASH2, BAKLASH3
       6  Hires GE                  BAKLASH, BAKLASH2, BAKLASH3
       7  Hires GE                  BAKLASH, BAKLASH2, BAKLASH3
       8  GE time                   BAKLASH, BAKLASH2
       9  GE time                   BAKLASH, BAKLASH2
      10  GE time                   BAKLASH, BAKLASH2
      11  GE time                   BAKLASH, BAKLASH2
      12  side channel              BAKLASH, BAKLASH2
      13  side channel              BAKLASH, BAKLASH2
      14  side channel              BAKLASH, BAKLASH2
    
   29-Dec-17 .... U210  PACOR -  Physics Acq Compiler - Revise ...... PAGE   5
      15  BGO                       BAKLASH
      16  BGO                       BAKLASH
    
   29-Dec-17 .... U210  PACOR -  Physics Acq Compiler - Revise ...... PAGE   6
 
 
   U210.040  Hardware Description & Compiler Directives - Discussion
 
   $ini...defines CAMAC operations which are to be  executed  at  "init  time"
          only  (i.e. when the INITVME command is issued). Only operations for
          which F.GT.7 are allowed.
 
   $run...defines CAMAC operations which are to be executed each time  a  TRUN
          or  STARTVME  command  is executed. Only operations for which F.GT.7
          are allowed.
 
   $dun...defines CAMAC operations which are to be  executed  at  the  end  of
          every  event  (whether  normal or killed). Only operations for which
          F.GT.7 are allowed.
 
   $dla...specifies a time delay (in microseconds) to be imposed prior to  the
          readout  of  the  specified  "class"  of  modules  or  readout  type
          (conditional or unconditional).
 
   $lat...specifies that a gated latch is to be read  and  saved.  That  latch
          may  be  subsequently referenced (in pattern register definitions or
          in conditional CAMAC readout) by  the  associated  NAME.  The  latch
          will be loaded into the data stream only if the id field is entered.
 
 
   $cam...defines  the sub-addresses and associated parameter NAMEs and IDs to
          be read from one CAMAC module. A given $cam can only  reference  one
          module  but  one  module may be referenced by more than one $cam. If
          the fc & ac fields are entered, that module will be cleared  at  the
          end of each event. Otherwise, an external clear is expected.
 
          Any  $cam  entries which are not referenced in the conditional CAMAC
          readout program will be read unconditionally for  every  "un-killed"
          event.
 
   $fer...defines  Ferra  modules,  sub-addresses,  associated parameter NAMEs
          and IDs to be read. The module  type  mt=MOTYP  must  be  given  and
          MOTYP  must  be either LRS_4300 or GAN_812F. All existing parameters
          specified for a given module type  will  be  loaded  into  the  data
          stream  unless  a  conditional  readout  is  specified  via the $rif
          directive defined below.
 
   $fas...defines Fastbus module  sub-addresses,  associated  parameter  NAMEs
          and  IDs  to  be  read.  The  module type mt=MOTYP must be given and
          MOTYP must be either LRS_1885 or PHIL_10C6. All existing  parameters
          specified  for  a  given  module  type  will be loaded into the data
          stream unless a  conditional  readout  is  specified  via  the  $rif
          directive defined below.
 
   $xia...defines  the  crate#, slot#, virtual-station#, and group# (c0C, n0N,
          v0V, g0G) for XIA modules which are to be read. The group number  is
          optional at this time.
 
   $vme...defines  the  adc#  (or  tdc#),  sub-addresses, associated parameter
          NAMEs and IDs to be read. The module type must also be given.
    
   29-Dec-17 .... U210  PACOR -  Physics Acq Compiler - Revise ...... PAGE   7
 
 
 
   U210.040  Hardware Description & Compiler Directives - Discussion (cont)
 
   $rif...defines the conditional readout of a given Fera,  Fastbus  or  CAMAC
          module  type.  Only  one  conditional  entry  is allowed for a given
          module type. It says, if any bits in the specified latch word  LW(N)
          and mask MSK match,  read module type MOTYP.
 
          Note:  LW  must  be  a name defined in a previous $lat specification
          and the index N must be a number (not a symbol) for now at least.
 
   $kil...defines the conditional "killing"  of  events.  The  any-type  says,
          kill  the  event  if  any bits in the specified latch word LW(N) and
          mask MSK match. The none-type says kill the event if no bits match.
 
   $did...specifies default parameter IDs FASI &  FERI  to  be  used  for  any
          illegally  reported  slot#  and  sub-address  from  Fastbus and Fera
          readout, respectively. This is for  diagnostic  purposes  only.  For
          example,  you  could  histogram  parameters FASI & FERI and look for
          non-zero spectra. If you do not enter $did, PAC will assign to  FASI
          and  FERI  the  values  MAX-ID+1  and MAX-ID+2, respectively, where,
          MAX-ID is the maximum parameter-ID specified  anywhere  in  the  PAC
          program.
 
   $cid...specifies  two ID-numbers to be associated with the hi- and lo-order
          parts of VME processor's  internal  100HZ  clock  which  are  to  be
          entered  into the data stream. If the #cid directive is not entered,
          then no clock parameters will be recorded.
 
   $pat...provides the user with a method of defining one or more  (up  to  5)
          indexed  pattern  word  names  (denoted PAT in the definition table)
          each of whose elements refer to one bit of one latch word. The  idea
          is  to  make  it  easier  to  construct loops which involve multiple
          latch words.
 
          In   the   definition   table,  PAT  denotes  a  user-supplied  name
          associated  with a multi-word   bit-pattern  composed  of  bits from
          one  or  more  latch-words  which  have  been  defined  via the $lat
          directive. The example given in  the  table  uses  bits  I1-I2  from
          NAME(1),  J1-J2  from  NAME(2)  ....   In   subsequent   conditional
          programming PAT(I) refers to 1 bit of one latch-word.
 
 
                                    COMMENTS
 
   (1)....At  this time no scalar symbols (representing numbers) are supported
          for any of the $-type directives. Examble: for the  $ini  directive,
          C,N,A,F,DATA  must  all  be  numbers.  DATA must be hex and the DATA
          field must be present for ALL CAMAC function codes whether  used  or
          not.
 
   (2)....Hex  numbers  are  entered with a numeric first-digit and a trailing
          "h" or "H". That is, in the usual way.
 
   (3)....A ! or ; introduces a comment field on a statement line.
 
   (4)....Any   non-printing   characters  (including  TABs)  will  cause  the
          compiler to ABORT!
 
    
   29-Dec-17 .... U210  PACOR -  Physics Acq Compiler - Revise ...... PAGE   8
 
 
 
   U210.050  Conditional (Programmed) CAMAC Readout - Syntax
 
   In the following list:
   LW(I)   denotes Ith latch-wd      defined via $lat      - always indexed
   PAT(J)  denotes Jth pattern-bit   defined via $pat      - always indexed
   ADC(J)  denotes Jth named-hwd dev defined via $cam      - always indexed
   GNAM(J) denotes jth gatename defined via $rgat or $cgat - always indexed
   MSK is a mask - number or scaler symbol
   LAB is a statement label
 
   IFU(PAT(J))GOTO LAB    ;If latch-bit implied by PAT(J) not set,  GOTO LAB
   IFS(PAT(J))GOTO LAB    ;If latch-bit implied by PAT(J)  is set,  GOTO LAB
 
   IFT(GNAM(J))GOTO LAB   ;If raw or calculated gate GNAM(J) true,  GOTO LAB
   IFF(GNAM(J))GOTO LAB   ;If raw or calculated gate GNAM(J) false, GOTO LAB
 
   IFA(LW(I),MSK)GOTO LAB ;If any bits set in MSK are set in LW(I), GOTO LAB
   IFN(LW(I),MSK)GOTO LAB ;If no  bits set in MSK are set in LW(I), GOTO LAB
 
   GOTO LAB               ;Unconditional GOTO LAB
 
   READ ADC(J)            ;Read ADC(J) (C,N,A,F defined in $CAM statement)
 
   CNAF C,N,A,F,DATA      ;Explicit CNAF (DATA field is optional)
 
   LABL  CONTINUE         ;Labeled continue statement
 
 
 
   U210.060  Conditional (Programmed) CAMAC Readout - Discussion
 
 
   (1)....Statement labels  must  start  in  col-1  and  be  no  more  than  4
          characters.
 
   (2)....Statements must start in col-7 or greater (like FORTRAN).
 
   (3)....A ! or ; introduces a comment field on a statement line.
 
   (4)....Hex  numbers  are  entered with a numeric first-digit and a trailing
          "h" or "H". That is, in the usual way.
 
   (5)....Any   non-printing   characters  (including  TABs)  will  cause  the
          compiler to ABORT!
    
   29-Dec-17 .... U210  PACOR -  Physics Acq Compiler - Revise ...... PAGE   9
 
 
 
   U210.070  Conditional (Programmed) CAMAC Readout - Example
 
   The $lat directive specifies devices (normally gated latches) which are  to
   be  read  first  and  saved  under  the  name  specified  for  later use in
   bit-tests or in constructing multi-word bit patterns. Such  data  may  also
   be $kil-tested for event crashing.
 
   If  fc0F & ac0A are included in the $lat & $cam directives, clearing of the
   modules will be automatic. If you  do  not  enter  this,  you  must  supply
   explicit CNAFs or clear by some other method.
 
   The  $lat & $cam directives on the preceding page imply full readout unless
   "conditional" readout is specified by a  subsequent  "conditional  readout"
   program  section. Anything not included in the conditional section is still
   readout. A conditional readout program section might look as follows:
 
   Example--1 ----------------------------------------------------------------
 
         I=0                     ;Init pattern-word bit counter
         J=20                    ;Init ADC, TDC, index value
 
         LOOP 40                 ;Loop over 40 "detectors"
         I=I+1                   ;Increment bit-counter
         J=J+1                   ;Increment ADC, TDC index
         IFU(PAT(I))GOTO LAB1    ;Tst pattern bit - read if set
         READ ADC1(J)            ;Read ADC1
         READ ADC2(J)            ;Read ADC2
         READ TDC1(J)            ;Read TDC1
         READ TDC2(J)            ;Read TDC2
   LAB1  ENDLOOP                 ;End-of-loop
 
   Example-2 -----------------------------------------------------------------
 
   $ini cnaf 0,5,0,9,0H
 
   $dun cnaf 0,5,0,16,222h
 
   $lat c00 n02 f00 a00    GLAT:1              fc09 ac00
 
   $cam c00 n05 f00 a00-07 CADC:1,1  id01,1    fc09 ac00   dt120
 
   $fas c01 n02 a00-95     FADC:1,1  id101,1   mt=lrs_1885
 
   $fas c01 n10 a00-95     FTDC:1,1  id201,1   mt=phil_10c6
 
   $pat PAT = GLAT(1)1,16
 
   $kil none GLAT(1) 0FFFFh
 
         I=0
         LOOP 4
         I=I+1
         IFU(PAT(I)) GOTO L1
         READ CADC(I)
   L1    ENDL
    
   29-Dec-17 .... U210  PACOR -  Physics Acq Compiler - Revise ...... PAGE  10
 
 
   U210.080  PAC Output Tables - POB Structure
 
   ******************  WD#  *****************************************DIRECTORY
 
   IOF  #ENT  DELAY -  1-3  ;IOF locates PAC-SOURCE FILENAME
   IOF  #ENT  DELAY -  4-6  ;IOF locates CAMAC CRATE TABLE
   IOF  #ENT  DELAY -  7-9  ;IOF locates CNAF-INIT LIST
   IOF  #ENT  DELAY - 10-12 ;IOF locates CAMAC   MODULE TABLE
   IOF  #ENT  DELAY - 13-15 ;IOF locates FASTBUS MODULE TABLE
   IOF  #ENT  DELAY - 16-18 ;IOF locates FERA    MODULE TABLE
   IOF  #ENT  DELAY - 19-21 ;IOF locates GATED LATCH    TABLE
   IOF  #ENT  DELAY - 22-24 ;IOF locates GATE PARAMETER READOUT TABLE
   IOF  #ENT  DELAY - 25-27 ;IOF locates RAW         GATE SPEC  TABLE
   IOF  #ENT  DELAY - 28-30 ;IOF locates CALCULATED  GATE SPEC  TABLE
   IOF  #ENT  DELAY - 31-33 ;IOF locates COUNT-DOWN LIST
   IOF  #ENT  DELAY - 34-36 ;IOF locates CONDITIIONAL  KILL TABLE
   IOF  #ENT  DELAY - 37-39 ;IOF locates UNCONDITIONAL READOUT TABLE
   IOF  #ENT  DELAY - 40-42 ;IOF locates CONDITIOINAL  READOUT PROGRAM
   IOF  #ENT  DELAY - 43-45 ;IOF locates CNAF-LIST FOR CONDITIONAL READOUT
   IOF  #ENT  DELAY - 46-48 ;IOF locates ID-LIST   FOR CONDITIONAL READOUT
   IOF  #ENT  DELAY - 49-51 ;IOF locates CAMAC   COND MODULE-TYPE  READOUT
   IOF  #ENT  DELAY - 52-54 ;IOF locates FASTBUS COND MODULE-TYPE  READOUT
   IOF  #ENT  DELAY - 55-57 ;IOF locates FERA    COND MODULE-TYPE  READOUT
   IOF  #ENT  DELAY - 58-60 ;IOF locates CAMAC     ID-TABLE
   IOF  #ENT  DELAY - 61-63 ;IOF locates FASTBUS   ID-TABLE
   IOF  #ENT  DELAY - 64-66 ;IOF locates FERA      ID-TABLE
   IOF  #ENT  DELAY - 67-69 ;IOF locates WINDUP  CNAF LIST
   IOF  #ENT  DELAY - 70-72 ;IOF locates RUN     CNAF LIST
   IOF  #ENT  DELAY - 73-75 ;IOF locates XIA MODULE TABLE
   IOF  #ENT  DELAY - 76-78 ;IOF locates VME COND MODULE-TYPE READOUT
   IOF  #ENT  DELAY - 79-81 ;IOF locates CAEN ADC HARDWARE MAP
   IOF  #ENT  DELAY - 82-84 ;IOF locates CAEN TDC HARDWARE MAP
   IOF  #ENT  DELAY - 85-87 ;IOF locates CAEN QDC HARDWARE MAP
   IOF  #ENT  DELAY - 88-90 ;IOF locates CAEN ADC ID-TABLE
   IOF  #ENT  DELAY - 91-93 ;IOF locates CAEN TDC ID-TABLE
   IOF  #ENT  DELAY - 94-96 ;IOF locates CAEN QDC ID-TABLE
   IOF  #ENT  DELAY - 97-99 ;IOF locates 100HZ CLOCK-ID TABLE
   IOF  #ENT  DELAY -100-102;IOF locates SIS3820 Hardware Map
   IOF  #ENT  DELAY -103-105;IOF locates SIS3820 Table
   IOF  #ENT  DELAY -106-108;IOF locates MYRIAD Hardware Map
   IOF  #ENT  DELAY -109-111;IOF locates MYRIAD Table
    
   29-Dec-17 .... U210  PACOR -  Physics Acq Compiler - Revise ...... PAGE  11
 
   U210.080  PAC Output Tables - POB Structure (continued)
 
   *************************************************PAC FILENAME.......(01-03)
   ASCII FILENAME (80 BYTES)
   *************************************************CRATE-LIST.........(04-06)
   CRATE#
   *************************************************CNAF INIT-LIST.....(07-09)
   CNAF DATA
   *************************************************CAMAC MODULE TABLE (10-12)
   CN0T           (Packed C,N,dummy,MODTYP)
   *************************************************FAST  MODULE TABLE (13-15)
   CN0T           (Packed C,N,dummy,MODTYP)
   *************************************************FERA  MODULE TABLE (16-18)
   CN0T           (Packed C,N,dummy,MODTYP)
   *************************************************GATED-LATCH TABLE..(19-21)
   CNAF
   *************************************************GATE READ TABLE....(22-24)
   C N A F MOTYP
   *************************************************RAW GATE SPEC TABLE(25-27)
   GPI LO HI PATNDX MSK
   *************************************************CAL GATE SPEC TABLE(28-30)
   See separate discription
   *************************************************COUNT-DOWN-TABLE...(31-33)
   PATNDX MSK COUNT
   *************************************************COND KILL TABLE....(34-36)
   PATNDX MSK KILTYP
   *************************************************UN-COND CAMAC READ.(37-39)
   CNAF   ID
   *************************************************COND-CAMAC PROGRAM.(40-42)
   PATNDX MSK T-IDX T-NUM T-NXT F-IDX F-NUM F-NXT
   *************************************************COND-CAMAC CNAFs...(43-45)
   CNAF
   *************************************************COND-CAMAC ID-TABL.(46-48)
   ID
   *************************************************CAMAC   MOTY READ..(49-51)
   PATNDX MSK MOTY
   *************************************************FASTBUS MOTY READ..(52-54)
   PATNDX MSK MOTY
   *************************************************FERA    MOTY READ..(55-57)
   PATNDX MSK MOTY
   *************************************************CAMAC     ID-TABLE (58-60)
   ID (32,32,8) = 8192
   *************************************************FASTBUS   ID-TABLE (61-63)
   ID (256,32)  = 8192
   *************************************************FERA      ID-TABLE (64-66)
   ID (32,576)  = 18432
   *************************************************WINDUP/CLR TABLE...(67-69)
   CNAF DATA
   *************************************************RUN CNAF   TABLE...(70-72)
   CNAF DATA
   *************************************************XIA MODULE TABLE...(73-75)
   CNVG           (Packed C,N,VSN,GR)
   *************************************************CAEN    MOTY READ..(76-78)
   PATNDX MSK MOTY
   *************************************************CAEN ADC HWD MAP...(79-81)
   ADC (24)
   *************************************************CAEN TDC HWD MAP...(82-84)
   TDC (12)
   *************************************************CAEN QDC HWD MAP...(85-87)
   QDC (12)
    
   29-Dec-17 .... U210  PACOR -  Physics Acq Compiler - Revise ...... PAGE  12
 
 
 
   U210.080  PAC Output Tables - POB Structure (continued)
 
 
   *************************************************CAEN ADC ID-TABLE..(88-90)
   ID (34,24) = 816
   *************************************************CAEN TDC ID-TABLE..(91-93)
   ID (34,12) = 408
   *************************************************CAEN QDC ID-TABLE..(94-96)
   ID (34,12) = 408
   *************************************************CLOCK ID-TABLE.....(97-99)
   CLOCID(2)
   *************************************************SIS 3820 HWD MAP...(100-102)
   SIS3820 (192)
   *************************************************SIS 3820 ID-Table.(101-103)
   ID (32,6) = 192
   *************************************************MYRIAD HWD MAP....(107-109)
   MYRIAD (3)
   *************************************************MYRIAD ID-Table...(110-112)
   ID (1,3) = 3
   ===========================================================================
 
 
   PATNDX = Pattern-word index
 
   MSK    = mask - (do IAND(LATCH(PATNDX),MSK) and test for .TRUE. or .FALSE.
 
   T-IDX  = CNAF/ID table indices     if test TRUE
 
   T-NUM  = CNAF/ID #entries to use   if test TRUE
 
   T-NXT  = next COND_PROG index      if test TRUE
 
   F-IDX  = CNAF/ID table indices     if test FALSE
 
   F-NUM  = CNAF/ID #entries to use   if test FALSE
 
   F-NXT  = next COND-PROG index      if test FALSE
 
    
   29-Dec-17 .... U210  PACOR -  Physics Acq Compiler - Revise ...... PAGE  13
 
 
   U210.080  PAC Output Tables - POB Structure (continued)
 
   -------------------------------------------------------------------------
 
   GATE PARM READ TABLE
 
   C  N  A  F  MOTYP
   C  N  A  F  MOTYP
   .
   .
   RAW GATE SPECIFICATION TABLE
 
   GPI LO HI PATNDX MSK  ;GPI    = gate parm index (index in read table)
                         ;LO, HI = gate limits
                         ;PATNDX = pattern word number (index)
                         ;MSK    = defines bit# to set if gate true
   .
   .
   CALCULATED GATE SPECIFICATION TABLE
 
   PATNDX                ;pattern array index (word#)
   MSK                   ;MSK defines bit# to set
   NOTCODE               ; 0  implies normal  1  implies .not.
   IDX                   ;index in raw gate specification table
   OPCODE                ; 0  implies .and.  1  implies .or.
   NOTCODE
   IDX                   ;index in raw gate specification table
   ENDCODE = -1          ; -1  implies end of spec for this bit
   PATNDX
   MSK
   NOTCODE
   IDX                   ;index in raw gate specification table
   ENDCODE = -1
   PATNDX
   MSK
   NOTCODE
   IDX
   OPCODE
   NOTCODE
   IDX
   OPCODE
   NOTCODE
   IDX
   ENDCODE = -1
   .
   .
   COUNT DOWN LIST
 
   PATNDX  MSK  COUNT    ;down-count if bit is set
   .
   .
