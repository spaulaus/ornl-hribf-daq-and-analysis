   22-Sep-17 ....... PACMAN II - ORPHAS Acquisition Manager ......... PAGE   1
 
 
 
   Sec Page               Function
 
   010   1  Introduction
 
   020   1  Getting Started - Things you MUST do
   030   2  Location of Important Files
   040   3  PACMAN Configuration File
   050   6  PACMAN Run-time Help File
   060   7  Running PACMAN
   070   9  Histogramming Tasks
   080   9  Control of VME Front-End Processor
   090  11  FILE Commands
   100  12  Miscellaneous Commands
   110  14  System Status Commands
   120  16  Testing VME/Workstation Connection
   130  18  What to Do if PACMAN Refuses to RUN
   140  19  How to Exit PACMAN
   150  19  Run-time Help Text
   160  23  Built-in command list
   170  23  How to Access On-Line Documentation
 
   PACM.010 Introduction
 
   PACMAN  is  an  executive which controls all processes necessary to run the
   data acquisition system and write event  data  to  a  list-data  file  (LDF
   file).  It  initially loads all processes required and continually monitors
   those processes throughout  data  acquisition.  If  any  essential  process
   terminates,  the  user  is  notified of what corrective action is required.
   The commands to PACMAN fall into 4 categories:
 
   1).. VME Front-end processor control
 
   2).. File control
 
   3).. System status and test
 
   4).. USER commands defined in a configuration file
 
   Throughout this document examples of screen output will be  reproduced.  In
   these examples, user input will be in Bold print.
 
 
   PACM.020 Getting Started - Things YOU need to do.
 
       1).. Setup your VME System .......................................
 
            see Section 030 of VME Front-End System document
            see document file - /usr/hhirf/vme/doc/VMEsys.doc
 
       2).. Configure the VME/Workstation system .........................
 
            see Section 060 of VME Front-End System document
            see document file - /usr/hhirf/vme/doc/VMEsys.doc
 
    
   22-Sep-17 ....... PACMAN II - ORPHAS Acquisition Manager ......... PAGE   2
 
       3).. Prepare your PACOR program ..................................
 
            PACOR user documentation
            see document file - /usr/hhirf/doc/pacor.doc
 
       4).. [optional] Make a PACMAN configuration file .................
 
            see Section 040
 
       5).. [optional] Make a custom run-time help file .................
 
            see Section 050
 
       6).. Start PACMAN and start data acquisition .....................
 
            see Section 060
 
       7).. Start your histogram tasks ...................................
 
            see document file - /usr/hhirf/doc/scanor.doc
 
 
   PACM.030
 
 
   PACM.030 Location of Important Files
 
   You  may  start  PACMAN  in any directory you wish.  However, there are two
   files which PACMAN processes will access  or  create  if  necessary.  These
   files  will  ALWAYS  be  in your home directory (i.e. login directory). The
   files are orphas.vme? and pacman.fig.
 
   orphas.vme? ...........................................
 
   Before starting PACMAN, you MUST set the environment variable  VME  to  the
   VME  processor  you  will  be  using.  For example, if the VME processor is
   vme2, you should
 
          setenv VME vme2
 
   The log file name will be orphas.vme2.
 
   If the log file does not exist in your home directory, PACMAN  will  create
   it.   Otherwise,  PACMAN  appends to the existing file.  Each entry has the
   name of the process which originated  the  message,  a  time  stamp  and  a
   message.   Most  of  the  commands to PACMAN are logged.  The log file is a
   fairly complete record of what has been done.
 
   Remember, the log file is ALWAYS in your home directory.
 
   pacman.fig ............................................
 
   This your custom PACMAN configuration file.  PACMAN  expects  this  file to
   be  in  your  home directory unless you specify a path and file name on the
   command line invoking PACMAN.  If  there  is  no  pacman.fig  in  your home
   directory  and no file is specified on the pacman command line, the default
    
   22-Sep-17 ....... PACMAN II - ORPHAS Acquisition Manager ......... PAGE   3
 
   configuration file is /usr/acq/wks/pacman.fig.
 
 
   PACM.040 PACMAN Configuration File
 
   Several   PACMAN   parameters   may   be  customized  by  making  your  own
   configuration file.   This  section  describes  those  parameters  you  may
   specify.
 
   The  syntax  rules are very simple.  Each statement type is identified by a
   keyword.  First level keywords must be the first word on  a  line  and must
   be  upper  case.  If a statement requires more than one line, there must be
   a left curly bracket on the first line and a right  curly  bracket  at  the
   end of the last line.
 
   The  statement  $USER_CMD  allows  a second level of keywords which must be
   lower case and keywords are separated by semicolons.
 
   $BANNER ...............................................
 
   This is the PACMAN startup banner.  If not  present  in  your configuration
   file,  the  default  is  identical  to  that  shown below in the listing of
   /usr/acq/wks/pacman.fig.
 
   $BUFFER_SIZE ..........................................
 
   This parameter defines the size of event data buffers in bytes. The  buffer
   size should be a power of 2 and in the range of 2048 thru 32768.
 
   If  this parameter is not present in the configuration file, the default is
   32768.
 
   $LOG_INTERVAL .........................................
 
   The   acquisition   system   writes   rate  information  to  the  log  file
   periodically. This parameter is the time interval in seconds.  For example,
 
         $LOG_INTERVAL 600
 
   means to log the data rates every 10 minutes.
 
   If the parameter is not present in your configuration file, the default  is
   300.
 
   $USER_PFTOIPC .........................................
 
   The  acquisition  system process which receives data from the VME system is
   pftoipc  (Packet  Filter  to  Interprocess  Communication).   This  process
   receives  ethernet  packets  from  the VME and builds event data buffers in
   shared memory.  All other processes get event data from shared memory.
 
   The default process  is  /usr/acq/wks/pftoipc.   The  user  may  provide  a
   custom  process to replace pftoipc.  If you have a custom process, you need
   to define it in your configuration file.  The statement
 
        $USER_PFTOIPC  /usr/users/myacq/mypftoipc
    
   22-Sep-17 ....... PACMAN II - ORPHAS Acquisition Manager ......... PAGE   4
 
 
   means that the process /usr/users/myacq/mypftoipc is to be used instead  of
   the default.
 
   $HELP_FILE ............................................
 
   If you have a custom help file, you need to define it in your configuration
   file.  The statement,
 
        $HELP_FILE  /usr/users/mcsq/my.help
 
   means  that  the file /usr/users/mcsq/my.help will be read as the help file
   instead of the default /usr/acq/wks/pacman.fig.
 
   $WINDOW ...............................................
 
   While it is quite  easy  to  create  new  xterm  windows,  there  are  some
   drawbacks.   Usually all windows have the same name and the position of the
   Icon is selected by the XWindow manager.  Sometimes it is convenient  for a
   window  to have a unique name associated with the task usually performed in
   that window.  The $WINDOW statement allows you to specify  a  window  to be
   created when PACMAN starts.  For example, the statement
 
       $WINDOW scan = {/usr/bin/X11/xterm -geometry 80x32-0+10 #-41-38
          -sb -ls -iconic }
 
   creates  a  window named SCAN.  The position of the window and the Icon are
   specified.  Also, the window starts as an Icon. See the manual pages  X and
   xterm for additional information.
 
   Windows  may  not  have the same name as an existing command.  A command by
   the same name as the window will be  added  to  PACMAN.   This  command  is
   available to recreate the window should you accidentally delete it.
 
   $USER_CMD .............................................
 
   Typically  users  have  hardware  specific  codes  to  set  high  voltages,
   discriminator thresholds, etc.   You  may  elect  to  run  these  codes  in
   temporary  windows  from  PACMAN.   The  statement $USER_CMD may be used to
   define the code to be executed and the default  directory  while  the  code
   executes.  For example, the statement
 
       $USER_CMD { dmon = /usr/bin/X11/xterm -geometry 80x50+10+10 -sb -e
          /usr/acq/wks/dmon }
 
   creates  a PACMAN command dmon.  When you type dmon in the PACMAN window, a
   temporary   window   is  created  and  the  code  /usr/acq/wks/dmon  begins
   execution in that window.  When dmon terminates,  the  temporary  window is
   deleted and you return to PACMAN.
 
        NOTE:  PACMAN will not accept additional commands until
               a user defined command terminates.
 
   There  are  two additional command statements valid only within a $USER_CMD
   statement. These are second level commands which begin with a  $  and  MUST
   be  lower  case.  Commands within the $USER_CMD statement MUST be separated
    
   22-Sep-17 ....... PACMAN II - ORPHAS Acquisition Manager ......... PAGE   5
 
   by semicolons.
 
      $stopvme; ........................
 
   Often, hardware setup codes require that acquisition in the  VME  front-end
   system  be  stopped.  This statement directs PACMAN to do a STOPVME command
   prior to running your code.
 
      $cd directory; ...................
 
   This statement specifies the default directory while you code is executing.
 
   The following example shows the usage of these second level commands.  Note
   the semicolons at the  end  of  the  second  and  third  lines.   They  are
   essential!
 
       $USER_CMD { mycode = /usr/bin/X11/xterm -geometry 80x50+10+10 -sb -e
                            ~myname/setup/mycode;
          $stopvme;
          $cd ~myname/setup }
 
   $ALIAS ................................................
 
   If  you  don't  like the command mnemonics we have chosen, you may assign a
   new name to any existing command using the $ALIAS statement.   However, the
   new  name  MUST  be  unique.   Furthermore, the log file entries retain the
   original command name.  For example, the statement
 
         $ALIAS  go  trun
 
   defines go to mean the same thing of the built-in command trun.
 
   The   following   is   a   listing   of  the  default  configuration  file,
   /usr/acq/wks/pacman.fig.
 
     ---------------------------------------------------------------------
 
   $BANNER {
   ****************************************************************************
   *                                ORPHAS                                    *
   *                                                                          *
   *                               pacman II                                  *
   *                      Physics Acquisition Manager                         *
   *                                                                          *
   *     Welcome to DOE land - land of safety and efficiency!                 *
   * New command requirements - courtesy of DOE and the OMB                   *
   * Before starting -                                                        *
   *   setenv EXPT RIBxxx - where xxx is your experiment number               *
   * In pacman:                                                               *
   *   TRUN [bon, boff]  for "beam on" (bon) or "beam off" (boff)             *
   *   TON  [bon, boff]  for "beam on" (bon) or "beam off" (boff)             *
   ****************************************************************************}
   $BUFFER_SIZE 32768
   $LOG_INTERVAL 300
   $HELP_FILE  /usr/acq/wks/pacman.hep
 
    
   22-Sep-17 ....... PACMAN II - ORPHAS Acquisition Manager ......... PAGE   6
 
   #if   tektronix  display
 
   $WINDOW scan = {/usr/bin/X11/xterm -geometry 80x32-0+10 #-52-46
   -sb -ls -iconic }
 
   $WINDOW {
   damm = { /usr/bin/X11/xterm -geometry 80x32+10+400 #-57-24 -sb -ls
   -iconic } }
 
   #endif
 
   #if   alpha  display
 
   $WINDOW scan = {/usr/bin/X11/xterm -geometry 80x32-0+10 #-52-46
   -font 8x13 -sb -ls -iconic }
 
   $WINDOW {
   damm = { /usr/bin/X11/xterm -geometry 80x32+10+432 #-57-24 -sb -ls
   -font 8x13 -iconic } }
 
   #endif
 
   $USER_CMD {emacs = /usr/local/bin/emacs}
 
   $USER_CMD {
   dmon = {
          /usr/bin/X11/xterm
          -geometry
          80x50+10+10
          -sb
          -e
          /usr/acq/wks/dmon };
   }
 
   #if   decstation  or unknown type displays
 
   $WINDOW scan = {/usr/bin/X11/xterm -geometry 80x32+345+10 #+955+722 -sb -ls
   -font 8x13bold -iconic }
 
   $WINDOW {
   damm = { /usr/bin/X11/xterm -geometry 80x32+10+320 #+950+744 -sb -ls
   -font 8x13bold -iconic } }
 
   #endif
 
     ---------------------------------------------------------------------
 
 
   PACM.050 PACMAN Run-time Help File
 
   If you have a custom configuration file,  you  may  wish  to  also  make  a
   custom help file.
 
   A  line  for  which  the first 4 characters are $$$$, is the beginning of a
   Help topic.  All characters on the line following the $$$$ become  an entry
   in  the  help  directory  which  is  displayed  when  you  type help. The 4
    
   22-Sep-17 ....... PACMAN II - ORPHAS Acquisition Manager ......... PAGE   7
 
   characters  immediately  following  the  $$$$  must  be  unique.   These  4
   characters  are  the  help topic.  Help topics may have trailing spaces but
   they may NOT have leading spaces.
 
   Lines following the help topic line are saved  as  is  until  another  help
   topic  line or the end-of-file is encountered.  What you see in the file is
   what you get on the screen.
 
   Section 150 has a complete listing of the default help file.
 
 
   PACM.060 Running PACMAN
 
   Login at the console  of  the  workstation.  You  probably  have  only  one
   window. In that window, type
 
         pacman [file.fig]
 
     NOTICE:  It is important, for reasons I will not attempt to explain, that
          you do this in your login window. So, why don't you just  humor  the
          old fat man and do it my way?
 
   The  file.fig  argument on the command line is optional. If the argument is
   present, it should be path and file name  of  a  configuration  file.  Note
   that  all examples in this document assume use of the default configuration
   file - /usr/acq/wks/pacman.fig.
 
   After about 5 seconds, PACMAN should be started  and  initialized.  At  the
   bottom left of the screen is a long 10 line window. Every message appearing
   in  this  window  is  written  to  the  log file - orphas.vme? in your home
   directory. At the bottom right of the screen there  are  two  icons,  DAMM,
   and  SCAN.  Nothing  is  running  in  either  of  these windows. These were
   created and named for functions  you  are  likely  to  use.  The  size  and
   position  of  these windows may be changed as a you would any other window.
   There are no restrictions on what you may run in these windows.
 
   In your PACMAN window, you will see the startup banner and a  directory  of
   the help available.
 
     ---------------------------------------------------------------------
   Using configuration file: /usr/acq/wks/pacman.fig
   ****************************************************************************
   *                                ORPHAS                                    *
   *                                                                          *
   *                               pacman II                                  *
   *                      Physics Acquisition Manager                         *
   *                                                                          *
   *     Welcome to DOE land - land of safety and efficiency!                 *
   * New command requirements - courtesy of DOE and the OMB                   *
   * Before starting -                                                        *
   *   setenv EXPT RIBxxx - where xxx is your experiment number               *
   * In pacman:                                                               *
   *   TRUN [bon, boff]  for "beam on" (bon) or "beam off" (boff)             *
   *   TON  [bon, boff]  for "beam on" (bon) or "beam off" (boff)             *
   ****************************************************************************
     --------DOE Logging Started---------
    
   22-Sep-17 ....... PACMAN II - ORPHAS Acquisition Manager ......... PAGE   8
 
    Expt: TEST, VME vme3, Beam OFF, Events            0, Records 0
 
   FILEOU->PID = 28428, Old Priority = 0, New Priority = -15
   PID = 28429, Old Priority = 0, New Priority = -10
 
   The latest PACMAN documentation is dated: Tue Sep  7 17:10:04 2004
   If your copy is older, discard it and print a new copy using the
   command -  prtdoc.
 
   Type: H SCAT - For Scaler-to-File related commands
   Type: H VME  - For commands to control and get status of Front-End
   Type: H ACQ  - Commands related to data acquisition
   Type: H FILE - For File control commands
   Type: H DISP - For commands which display Data-records
   Type: H STAT - Commands to display system status and test VME system
   Type: H MISC - Miscellaneous Commands
   Type: H EXIT - How to STOP the acquisition system
   Type: H CMDS - Command list
   Type: H NEW  - How to update Help messages
   pacman:
 
     ---------------------------------------------------------------------
 
   These  help topics contain information about commands that may be useful in
   your work. The help includes most of  the  commands  available.  In  latter
   sections, we will explore the commands in greater detail.
 
   The  remainder  of  this  section  is an example of what is needed to start
   acquisition to disk file after all hardware setup is  complete.  The  steps
   required are:
 
          1).. Load VME acquisition software
 
          2).. Compile and load the proper PAC file into the VME processor
 
          3).. Setup the FILEOU process to record the data stream on disk
 
          4).. Start acquisition
 
     ---------------------------------------------------------------------
 
   pacman: loadacq
   Delete the acquisition tasks in the VME processor
 
   Now load the acquisition tasks.  Any error messages here are BAD news.
 
   Acquisition system now loaded
   pacman: pacor linux l
    #BYTES OBJECT CODE GENERATED =     584
    CONSIDER YOURSELF LOADED WITH    584  BYTES
    NO ERRORS
   pacman: hnum 1
   pacman: FILEOU->htit testing 1,2,3
   pacman: FILEOU->ouf /tera/mcsq/test.ldf
   pacman:  Open LDF file: /tera/mcsq/test.ldf
   FILEOU->trun boff
    
   22-Sep-17 ....... PACMAN II - ORPHAS Acquisition Manager ......... PAGE   9
 
   FILEOU->pacman:  PAC file: /tera/mcsq/Dlinux/Dpac/linux.pac   -       999001
     testing 1,2,3                                               -            1
   pacman:
 
     ---------------------------------------------------------------------
 
 
   PACM.070 Histogramming Tasks
 
   See  the  SCANOR documentation for details on making a histogramming task -
   /usr/hhirf/doc/scanor.doc.
 
 
   PACM.080 Control of VME Front-End Processor
 
   From the PACMAN window you can do everything you need to do  with  the  VME
   front-end  processor.  You  can  boot  the  VME  processor,  load  the  VME
   acquisition code, compile and load your PAC  program,  initialize  the  VME
   acquisition  system,  start  VME  acquisition, stop VME acquisition and get
   get the acquisition status of the VME system.
 
   BOOTVME ...............................................
 
   This command loads the code for a minimal operating  system  into  the  VME
   processor.  After the basic operating system is loaded, the workstation can
   communicate  directly  with  hardware  such  as  FASTBUS  and  CAMAC.  This
   operating  system  must  be  loaded  and functional before using your setup
   codes which directly access CAMAC or FASTBUS.
 
   You may ask when do I need to boot the VME system? The  most  obvious  time
   is  after any time when power has been off in the VME crate. There are also
   various other  hints.  If  your  setup  programs  give  connection  failure
   errors,  it  is  a  good hint that the VME operating system is not working.
   This can be checked with the LT command (see Section 110).
 
   Just remember that after BOOTVME, there are several other things  you  must
   do in preparing to take data.
 
   LOADACQ ...............................................
 
   Once  the basic operating system is loaded(BOOTVME above), you can load the
   acquisition code into the VME processor. This is  the  code  which  decodes
   your  readout  instructions,  responds to Event inputs, reads the specified
   hardware and sends event data to the workstation. This is a  necessary  but
   not sufficient step in starting data acquisition.
 
   LOG your comments......................................
 
   The  log  file  maintains  a  record  of  what  has  been  done during data
   acquisition.  The LOG command provides a way  for  you  to  insert comments
   into  the  log  file.   This line, including the keyword LOG, is written to
   the log file along with the date and time of day.
 
   CD directory ..........................................
 
   Change PACMAN's working  directory  to  the  directory  you  specify.  This
    
   22-Sep-17 ....... PACMAN II - ORPHAS Acquisition Manager ......... PAGE  10
 
   applies  only  to the EDIT and PAC commands below. This is a convenient way
   to avoid typing complete paths for these two commands.
 
   PWD ...................................................
 
   Show PACMAN's current working directory.
 
   DIR [options]..........................................
 
   List the files in the current working directory.  Use this command  just as
   you  would  use ls in a normal window.  In fact, you may type ls instead of
   dir.
 
   EDIT filename .........................................
 
   This command creates a new window on top of your PACMAN window  and  starts
   a  vi  session  on  the  specified file. When you exit your vi session, the
   window is deleted and you return to  PACMAN.  Note  that  PACMAN  will  NOT
   execute  another command until you exit the vi session! Also, when you exit
   the vi session, an entry will be made in the log file show  that  the  EDIT
   command  has  been  used.  NO, it does not indicate if changes were make in
   the PAC file! Sorry about that.
 
   PAC filename [L] ......................................
 
   When satisfied with your PAC program, you need to compile and load the  PAC
   into  the  VME  processor.  This  command  is identical to what you need to
   execute PACOR in  any  other  window.   See  the  PACOR  documentation  for
   details.
 
   If  you  specified  the  L  option  and  PAC  completed without errors, the
   following commands are also executed:
 
          1)... STOPVME - Stops the VME Acquisition
 
          2)... INITVME - Initialize the VME acquisition system
 
   If the INITVME command completes without error, the next TRUN command  will
   cause the PAC file to be written to disk file.
 
   INITVME ...............................................
 
   The  VME  acquisition system must be initialized after a new PAC is loaded.
   Since the PACOR command now does  this,  you  no  longer  need  to  do  the
   INITVME  command.   However,  it  does  no  harm.  Note  that  you  can NOT
   initialize the VME acquisition system if VME acquisition is running.
 
   The workstation which sends the INITVME command is the one  which  receives
   event data and messages from the VME system.
 
   Many  errors  are possible during the initialization of the VME(nonexistent
   CAMAC crates or FASTBUS modules etc. etc.). If there are errors,  you  will
   get  an error message in the PACMAN window. However, additional information
   about the error(s) may be present in the logger window.
 
   STARTVME ..............................................
    
   22-Sep-17 ....... PACMAN II - ORPHAS Acquisition Manager ......... PAGE  11
 
 
   If  the  INITVME  executed  without  error,  you  may  now  start  the  VME
   acquisition  system.  You  must  STARTVME  if  you  want  data  sent to the
   workstation.  The TRUN command also starts VME acquisition.
 
   STOPVME ...............................................
 
   Stop the VME acquisition. Readout by the VME processor stops  and  no  more
   data are sent to the workstation.  TSTOP also stops VME acquisition.
 
   STATVME ...............................................
 
   You  may ask for the state of the VME acquisition system with this command.
   Possible responses are 1) not initialized, 2) running and 3) stopped.  This
   status is reported only in the PACMAN window.
 
   VMEHOST ...............................................
 
   Show  the  name  of the workstation which receives event data and messaages
   form the VME system.  This the  workstation  from  which  the  last INITVME
   command was issued.
 
   PACM.090 FILE Commands
 
   The process which records event data to an LDF file is called FILEOU.  This
   is  stripped down version of LEMOR.  It has a limited command set with some
   syntax differences.  There are many  commands  for  positioning  the output
   file,  examining  input data.  If you are a LEMOR user, the main difference
   is that FILEOU has only one output stream and commands don't  now  have  an
   output  stream specification. If you are not a LEMOR user, you should first
   read the LEMOR documentation(See document  file  /usr/hhirf/doc/lemor.doc).
   In  any  case,  the help topics ACQ, FILE and DISP have a brief description
   of all commands. FILEOU  automatically  writes  an  End-of-File  and  after
   every 10000 records.
 
   There  are a few commands which are significantly different from LEMOR. So,
   lets look at them.
 
   OUF /tera/username/directory/file.ldf .................
 
   You MUST specify the full path name of the list-data disk  file.   You  may
   specify  the  file  in  two  parts using the FDIR command. For example, the
   following sequence is equivalent to the above OUF command:
 
         fdir /tera/username/directory
         ouf file.ldf
 
   Note that the  FDIR  command  without  an  argument  displays  the  current
   directory.
 
   FDIR ..................................................
 
   As  noted  above,  the  output  filename MUST be a FULL PATH name (i.e. the
   name MUST begin with a /).  The command  FDIR  allows  you  to  specify the
   full  path  of  the  directory  in which the output list-data file is to be
   created.  The OUF  filename  is  then  just  a  file  within  the directory
    
   22-Sep-17 ....... PACMAN II - ORPHAS Acquisition Manager ......... PAGE  12
 
   specified in a previous FDIR command.
 
   An FDIR command without an argument displays the current directory.
 
   The  directory  path  as  specified  in a FDIR command, is used for all OUF
   commands until a new  directory  is  specified  or  the  current  directory
   string is erased.  To erase the directory string, do
 
         FDIR xx
 
   since 'xx' is NOT a directory path, the directory string will be erased.
 
   TRUN ..................................................
 
   This  command starts recording event data on list-data disk file. TRUN also
   starts the VME acquisition.
 
   If you have loaded a new PAC, the PAC file is written to the  ourput  file.
   The  header  number  for  PAC  files  begins with 999001 and is incremented
   after the file is written.  The header title is the PAC file  name. Example
   PAC file header:
 
         PAC file: /usr/orph22/users/mcsq/test.pac         999001
 
   You  can  recover  a PAC file from the list-data disk file by searching for
   the header number and then using the STEX command in LEMOR.
 
   TCONT .................................................
 
   This command should be used when you are adding records to the last file.
 
   TSTOP .................................................
 
   In LEMOR, CTRL records.  PACMAN does  not  support  this  use  of  CTRL the
   command tstop any time you would have used CTRL
 
   SCAT ..................................................
 
   The  SCAT  commands control writing scaler data to the list-data disk file.
   For details, see the  document  /usr/hhirf/doc/scat.doc.   For  a  list  of
   commands, type scat help.
 
   AFON ..................................................
 
   Enables  automatic  file  marks on the output file.  A file mark is written
   on the output file after every 10000 records.  This is the DEFAULT.
 
   AFOF ..................................................
 
   Disables automatic file marks on the output file.
 
 
   PACM.100 Miscellaneous Commands
 
   CMD and CMDF file .....................................
 
    
   22-Sep-17 ....... PACMAN II - ORPHAS Acquisition Manager ......... PAGE  13
 
   Normally commands to PACMAN are  input  via  the  keyboard.   CMD  or  CMDF
   direct  PACMAN  to  get  commands from the specified file.  By default, the
   PACMAN appends .cmd to the file argument.  For example, the command
 
        cmd disk
 
   reads commands from the file disk.cmd in the current working directory.  If
   there  is  a  period  in the argument file, PACMAN does not append the .cmd
   and file is taken as the complete file name.  For example, the command
 
        cmd  disk.xxx
 
   executes commands from the file disk.xxx.
 
   DATE and TIME .........................................
 
   Display the date and time.  DATE and TIME are identical.
 
   DIR and LS ............................................
 
   List the files in the current working directory.  Use this command  just as
   you  would  use ls in a normal window.  In fact, you may type ls instead of
   dir.
 
   HUP ...................................................
 
   If your count rate is very low, it may take a long time to get a buffer  of
   event  data  to  be histogrammed.  The HUP command forces any event data in
   the VME processor to be sent to  the  workstation.   The  workstation  then
   pads  the remainder of the current buffer with End-of-Events and passes the
   buffer to all processes getting event data from shared memory.
 
   IPCS [options] ........................................
 
   IPCS displays the status of the interprocess communication facilities.  See
   the manual page ipcs for a description of the options field.
 
   KILL ALL ..............................................
 
   This  is  the  command  to  terminate PACMAN and exit.  See Section 140 for
   details.
 
   LOG ...................................................
 
   The log file  maintains  a  record  of  what  has  been  done  during  data
   acquisition.   The  LOG  command  provides a way for you to insert comments
   into the log file.  This line, including the  keyword  LOG,  is  written to
   the log file along with the date and time of day.
 
   PRTDOC ................................................
 
   Use PRTDOC to print a copy of the PACMAN documentation.
 
   PS [options] ..........................................
 
   PS  displays  the  status  of processes running in the workstation. See the
    
   22-Sep-17 ....... PACMAN II - ORPHAS Acquisition Manager ......... PAGE  14
 
   manual page for ps for a description of the options field.
 
   PWD ...................................................
 
   Show PACMAN's current working directory.
 
   USERCONFIG ............................................
 
   The PACMAN parameters which may  set  from  a  configuration  file  may  be
   displayed using the USERCONFIG command.  An example output follows:
 
     ---------------------------------------------------------------------
 
   pacman: userconfig
   ***** Windows *****
   SCAN  /usr/bin/X11/xterm -n SCAN -geometry 80x32-0+10 #-41-38 -sb -ls -iconi
   DAMM  /usr/bin/X11/xterm -n DAMM -geometry 80x32+10+400 #-45-20 -sb -ls -ico
 
 
   ***** Command Aliases *****
 
   ***** User Commands *****
   Command: dmon
          : /usr/bin/X11/xterm -geometry 80x50+10+10 -sb -e /usr/acq/wks/dmon
 
   Help file: /usr/acq/wks/pacman.hep
   Shared Memory buffer size (bytes): 8192
   Log Interval (seconds): 300
   pacman:
 
     ---------------------------------------------------------------------
 
 
   PACM.110 System Status Commands
 
   There  may  be  rare  occasions when you wonder if everything is working as
   you planned.  Well, here are a  few  sanity  checks.   I  will  show  a few
   commands  and  what  you should expect to see in the PACMAN window. Getting
   the expected response to these commands is a necessary but  not  sufficient
   condition to ensure that everything is working!
 
   LT ....................................................
 
   This  is  a check of the VME system.  The response shown below assumes that
   you have previously executed the BOOTVME and LOADACQ commands.
 
     ---------------------------------------------------------------------
 
   pacman: lt
 
   VME Processor: vme2
    ID          Memory/Task    Size       Address      pc    evt1/evt2  pri/time
     0  T00             Task0    8    7000    9000  FF002CC4   97    0   64    1 W
     1  T01        Lan_Driver   16    9000    D000      ABF8   56   64   70    1 W
     2  T02           Mem_mgr   76    D000   20000      FDA8    0    0   69    1 R
     3  M          Acq_Params  128   20000   40000
    
   22-Sep-17 ....... PACMAN II - ORPHAS Acquisition Manager ......... PAGE  15
 
     4  T03            cnafxx   42   40000   4A800     424B0   -1    1   69    1 W
     5  T04            fastxx   40   4A800   54800     4C9A8   -1    1   63    1 W
     6  T05         data_proc   42   54800   5F000     56C38   72    0   69    1 W
     7  T06            VMEacq 1174   5F000  184800     6873C   -1    1   69    1 W
     8  T07            vmemon   42  184800  18F000    186C02 -128    0   69    1 W
     9  T08             vmexx   40  18F000  199000    19123C   -1    1   69    1 W
   pacman:
 
     ---------------------------------------------------------------------
 
   This is a list of the processes running in the  VME  processor.   The  fact
   that  you  can  get  this  list  is a very good sign that the VME system is
   working.  Note that this list may at times have a different order  and that
   additional  processes  may  be  loaded.   What is shown above is simply the
   minimum required.
 
   If the VME processor has gone  belly  up,  the  response,  after  about  15
   seconds, would be:
 
     ---------------------------------------------------------------------
 
   pacman: lt
   Connection failure, no acknowledgment - /usr/acq/vme/lt
 
     ---------------------------------------------------------------------
 
   So  what  do  I  do  now?  Well, try BOOTVME followed by LOADACQ.  If these
   commands seem to work, try LT again.  If LT now  works,  you  will  need to
   load the PAC and STARTVME to resume data acquisition.
 
   The  next  thing  to try is manually resetting the FORCE CPU-40.  Go to the
   VME crate and push the Reset switch. The Reset switch is the  upper  switch
   on  the module front panel.  Now at the Workstation do BOOTVME, LOADACQ and
   LT again.  If LT works, you still need to  load  the  PAC  and  STARTVME to
   resume data acquisition.
 
   If none of that works, you are in deep dodo.  Quit while you're ahead, call
   for HELP and find something else to amuse yourself with until HELP arrives.
 
   PS ....................................................
 
   This  shows  the  processes  in  the workstation.  The response shown below
   only includes the essential processes.  Your list will most  likely include
   many  others.   Just  check  to  see that the ones listed below are in your
   list.
 
     ---------------------------------------------------------------------
 
   pacman: ps
     PID TTY          TIME CMD
    9457 pts/2    00:00:00 csh
    9533 pts/2    00:00:00 pacman
    9536 pts/2    00:00:00 pftoipc
    9537 pts/2    00:00:00 tape
    9538 pts/2    00:00:00 xterm
    9539 pts/2    00:00:00 femsg
    
   22-Sep-17 ....... PACMAN II - ORPHAS Acquisition Manager ......... PAGE  16
 
    9545 pts/2    00:00:00 xterm
    9546 pts/2    00:00:00 xterm
    9599 pts/2    00:00:00 ps
   pacman:
 
     ---------------------------------------------------------------------
 
   If any of the processes above is not present,  you  have  a  problem.   The
   proper thing to do is KILL ALL in the PACMAN window and run PACMAN again.
 
   STATVME ...............................................
 
   You  may ask for the state of the VME acquisition system with this command.
   Possible responses are 1) not initialized, 2) running and 3) stopped.  This
   status is reported only in the PACMAN window.
 
   VMEHARDWARE ...........................................
 
   VMEHARDWARE  dislays a list of the hardware modules the VME processor found
   when it was booted.  Compare the list of  interface  modules  with  what is
   physically installed in the VME crate. Example output follows:
 
     ---------------------------------------------------------------------
 
   pacman: vmehardware
   VME System Hardware Configuration
 
       VME Processor Logical Name: vme2
   VME Processor Ethernet Address: 00-80-42-00-29-79
    Default Host Ethernet Address: 08-00-2b-24-e0-5b
           Boot Multicast Address: 03-6d-63-73-71-00
   Available Interface Modules are:
 
   KSC 2917 -  CAMAC Interface Module
   LRS 1131 -  FASTBUS Interface Module
   CES 8170 -  FERA Readout Module
   TRIGGER  -  ORNL Trigger Module
   pacman:
 
     ---------------------------------------------------------------------
 
 
   PWD ...................................................
 
   Show  PACMAN's  working  directory.  This command can be used to check what
   directory you are in prior to using an EDIT or PAC command.
 
   DIR ...................................................
 
   List the files in the current working directory.
 
   PACM.120 Testing VME/Workstation Connection
 
   You may load the VME  processor  with  a  code  which  generates  an  event
   stream.   Each  event has 28 parameters.  The IDs are 1 thru 28.  Within an
   event the data for every parameter has the same value.  That data  value is
    
   22-Sep-17 ....... PACMAN II - ORPHAS Acquisition Manager ......... PAGE  17
 
   generated  by  a  random number routine and has a range of 0 thru 1023. The
   command is
 
   TESTVME ...............................................
 
   An example using TESTVME follows:
 
     ---------------------------------------------------------------------
 
   pacman: testvme
   *********************** WARNING ************************
   *  Normal Acquisition Codes in the VME processor are   *
   *  being replaced with test programs.  These test      *
   *  programs generate an event stream for testing the   *
   *  VME/Workstation connection.                          *
   *                                                      *
   * NOTE: When finished testing, your should execute the *
   *       the command - loadacq                          *
   *********************** WARNING ************************
   Delete the acquisition tasks in the VME processor
 
   Now load the VME/Workstation test programs.
 
   VME/Workstation test codes are now loaded.
   pacman: initvme
   pacman: startvme
   FILEOU->rdi
   RECORD, # BYTES READ =       1    8192    STAT =GOOD
 
   FILEOU->dev 1,256
   pacman: EVENT: Start word =    1 Number of parameters =  28
      1  3CD;     2  3CD;     3  3CD;     4  3CD;     5  3CD;     6  3CD;
      7  3CD;     8  3CD;     9  3CD;    10  3CD;    11  3CD;    12  3CD;
     13  3CD;    14  3CD;    15  3CD;    16  3CD;    17  3CD;    18  3CD;
     19  3CD;    20  3CD;    21  3CD;    22  3CD;    23  3CD;    24  3CD;
     25  3CD;    26  3CD;    27  3CD;    28  3CD;
   EVENT: Start word =   59 Number of parameters =  28
      1   29;     2   29;     3   29;     4   29;     5   29;     6   29;
      7   29;     8   29;     9   29;    10   29;    11   29;    12   29;
     13   29;    14   29;    15   29;    16   29;    17   29;    18   29;
     19   29;    20   29;    21   29;    22   29;    23   29;    24   29;
     25   29;    26   29;    27   29;    28   29;
   EVENT: Start word =  117 Number of parameters =  28
      1   BC;     2   BC;     3   BC;     4   BC;     5   BC;     6   BC;
      7   BC;     8   BC;     9   BC;    10   BC;    11   BC;    12   BC;
     13   BC;    14   BC;    15   BC;    16   BC;    17   BC;    18   BC;
     19   BC;    20   BC;    21   BC;    22   BC;    23   BC;    24   BC;
     25   BC;    26   BC;    27   BC;    28   BC;
   EVENT: Start word =  175 Number of parameters =  28
      1  28C;     2  28C;     3  28C;     4  28C;     5  28C;     6  28C;
      7  28C;     8  28C;     9  28C;    10  28C;    11  28C;    12  28C;
     13  28C;    14  28C;    15  28C;    16  28C;    17  28C;    18  28C;
     19  28C;    20  28C;    21  28C;    22  28C;    23  28C;    24  28C;
     25  28C;    26  28C;    27  28C;    28  28C;
   FILEOU->
 
    
   22-Sep-17 ....... PACMAN II - ORPHAS Acquisition Manager ......... PAGE  18
 
     ---------------------------------------------------------------------
 
   PACM.130 What to Do If PACMAN Refuses to RUN
 
   If you attempt to run PACMAN in a second  window  or  some  other  user  is
   running  PACMAN  using the same VME processor, it will refuse your request.
   You will get a list of  the  processes  and  be  informed  that  PACMAN  is
   already running.
 
     ---------------------------------------------------------------------
 
   orph38> pacman
    9429 p1 I      0:26 /usr/acq/wks/pacman
    9432 p1 S <    1:04 /usr/acq/wks/pftoipc -d vme2
    9433 p1 I      0:00 /usr/acq/wks/femsg -d vme2
    9434 p1 I <    0:00 /usr/acq/wks/tape vme2
    9442 p2 S      0:00 /usr/acq/wks/logger vme2
   pacman is already running!!
   orph38>
 
     ---------------------------------------------------------------------
 
   It  is  also  possible  that  all  of  PACMAN's  essential processes do not
   terminate when you exit PACMAN.  In this case, you will get a  list  of one
   or more processes which must be terminated before PACMAN can run again.
 
     NOTE:  The  DAMM  and   SCAN  windows are NOT essential processes. PACMAN
          never terminates these but does recognize their  presence  on  start
          up.
 
   The following is an example:
 
     ---------------------------------------------------------------------
 
   orph38> pacman
    9434 p1 I      0:00 /usr/acq/wks/tape vme2
 
   One or more of pacman's processes are already running
   orph38>
 
     ---------------------------------------------------------------------
 
   In this example, you need to kill process with PID 9434.  The first attempt
   should be:
 
          kill 9434
 
   Now  either  try  to  run  PACMAN again or use the ps command to see if the
   process is gone.  If the process is  still  present,  the  following should
   terminate it:
 
          kill -9 9434
 
 
   PACM.140 How to Exit PACMAN
 
    
   22-Sep-17 ....... PACMAN II - ORPHAS Acquisition Manager ......... PAGE  19
 
   After  several  hectic  days  and sleepless nights,  the experiment is over
   and it is now time to shutdown everything, pack and  go  catch  our  plane.
   How do you stop PACMAN?  The command is
 
   KILL ALL ......(this one MUST be upper case)...........
 
   If  you  were  writing  event  data  to disk, two End-of-Files(EOF) will be
   written on the disk file. Then the disk file is positioned between the  two
   EOFs.  All  of PACMANs resources will be released and all of it's processes
   are terminated. PACMAN will exit and you will get a system prompt.
 
   There are circumstances when PACMAN will prompt you to  use  the  KILL  ALL
   command and then run PACMAN again.  What can I say, it is good advice.
 
 
   PACM.150 Run-time Help Text - /usr/acq/wks/pacman.hep
 
   $$$$SCAT - For Scaler-to-File related commands
   Commands used to control scaler dumps to the data file ------------
   SCAT  filename      ;Initializes SCAT from filename
 
   SCAT  ON            ;Dumps scalers before EOF only
   SCAT  ON  CLR       ;Dumps scalers before EOF & clears post dump
   SCAT  ON  TSEC      ;Dumps scalers every TSEC & before EOF
   SCAT  ON  TSEC  CLR ;Dumps scalers every TSEC & before EOF & clears
   SCAT  OFF           ;Disables scaler dumps
 
   SCAT  NORT          ;Normalize count rates to internal clock
   SCAT  NORS          ;Normalize count rates as specified by snit-file
   SCAT  UNOR          ;Count rates un-normalized
 
   SCAT  STAT          ;Displays/logs setup status
   SCAT  HELP          ;Display on-line help
   SCAT  H             ;Display on-line help
 
   SCAT  LERR ON       ;Turns error display/logging ON
   SCAT  LERR OFF      ;Turns error display/logging OFF
 
   SCAT  ZERR          ;Zeros error counters
   SCAT  ZDUM          ;Zeros scaler dump counter
   SCAT  ZERO          ;Zeros all scalers
   $$$$VME  - For commands to control and get status of Front-End
   COMMANDS TO THE FRONT-END VME PROCESSOR ..............................
   BOOTVME    - Reboot the VME processor
   LOADACQ    - Download code for the data acquisition system to VME processor
   LOADDSSD   - Acquisition for the DSSD detector system
 
   SETUP file - Run MODU_SETUP to initialize CAMAC modules.
 
   INITVME [file]  - Initialize the acquisition software.  MUST be done after PAC.
                     If a file is specified, MODU_SETUP is run prior to
                     initializing the VME.
 
   STARTVME   - Start acquisition in the VME processor
   STOPVME    - Stop acquisition in the VME processor
   STATVME    - Show status of acquisition in VME processor
    
   22-Sep-17 ....... PACMAN II - ORPHAS Acquisition Manager ......... PAGE  20
 
 
   ZEROCLK    - Zero the 100Hz VME clock
 
   VMEHOST    - Show workstation which receives event data
   PACFILE    - SHow name of PAC file
 
   PACOR file l - Compile and load your front-end acquisition code.
   EDIT file    - Edit user front-end acquisition code (vi editor).
 
   CD dir     - Change directory to dir.  Useful if set to directory of your
                xxx.pac files.  If so set, you need not specify the directory
                path for PAC and EDIT commands.
   PWD        - Show current working directory.
   DIR        - List the files in the current working directory.
   $$$$ACQ  - Commands related to data acquisition
   ACQUISITION COMMANDS .................................................
   FDIR      - Display directory for LDF files
   FDIR dir  - Set directory for LDF files
   OUF file  - Write list data to LDF file.
 
   HTIT      - Set title for output header generated by FILEOU
   HNUM      - Set next header number for output header
   HOUT      - Write locally generated header
 
   EOF       - Write EOF on OUTPUT
 
   LOG mess  - Insert comments into log file
 
   TRUN beam - To START writing the Acquisition Data Stream to file
               beam="boff" (beam off) or "bon" (beam on)
   TCONT     - To Continue writing the Acquisition Data Stream to file
   TSTOP     - To STOP writing the Acquisition Data Stream to file
 
   TON beam  - Starts file but does not start VME acquisition
               beam="boff" (beam off) or "bon" (beam on)
 
   HUP       - Histogram update.  Useful for very low rate acquisition.
 
   ZBUF      - Clear all shared memory buffers.
 
   AFON      - Enable automatic file mark after 10000 records (DEFAULT)
   AFOF      - Disable automatic file mark
   $$$$FILE - For File control commands
   FILE CONTROL COMMANDS ................................................
   FDIR      - Display directory for LDF files
   FDIR dir  - Set directory for LDF files
   OUF file  - Write list data to LDF file.
 
   RDI  N    - Read    N records from Data Stream
   RDO  N    - Read    N records from OUTPUT
   FRO  N    - Forward N records on   OUTPUT
   BRO  N    - Backup  N records on   OUTPUT
   FFO  N    - Forward N files   on   OUTPUT
   BFO  N    - Backup  N files   on   OUTPUT
   RWO       - Rewind           OUTPUT
   BTO       - Go to bottom of  OUTPUT (to DBL-EOF, backup 1 EOF)
    
   22-Sep-17 ....... PACMAN II - ORPHAS Acquisition Manager ......... PAGE  21
 
 
   CLO       - Close            OUTPUT
   ULO       - Unload and close OUTPUT
 
   AFON      - Enable automatic file mark after 10000 records (DEFAULT)
   AFOF      - Disable automatic file mark
   $$$$DISP - For commands which display Data-records
   COMMANDS FOR READING & DISPLAY OF DATA-RECORDS .......................
   RDI  N    - Read    N records from Data Stream
   RDO  N    - Read    N records from OUTPUT
 
   PEV IA,IB - Print   16-bit word IA thru IB in EVENT format
   DEV IA,IB - Display 16-bit word IA thru IB in EVENT format
 
   PZ  IA,IB - Print   16-bit word IA thru IB in HEX   format
   DZ  IA,IB - Display 16-bit word IA thru IB in HEX   format
 
   PA  IA,IB - Print   16-bit word IA thru IB in ASCII format
   DA  IA,IB - Display 16-bit word IA thru IB in ASCII format
 
   PI  IA,IB - Print   16-bit word IA thru IB in INT*2 format
   DI  IA,IB - Display 16-bit word IA thru IB in INT*2 format
 
   PIF IA,IB - Print   32-bit word IA thru IB in INT*4 format
   DIF IA,IB - Display 32-bit word IA thru IB in INT*4 format
   $$$$STAT - Commands to display system status and test VME system
   COMMANDS TO SHOW STATUS AND TEST SYSTEM..................................
   USERCONFIG- Display commands and parameters from the configuration
               file.
 
   TESTVME   - Download code to the VME processor which generates a data
               stream.  This is a fixed pattern of 28 parameters/event.
               IDs are 0 thru 27.  Parameter values are all the same but
               the value increments after each event (range = 0 thru 4095).
 
   LT        - List processes in the VME processor.  If this gives no
               error, the VME processor is alive and kicking.
 
   STATVME   - Show status of acquisition in the VME processor.
 
   PS        - Lists your processes in the workstation.
   PWD       - Show current working directory.
   DIR       - List the files in the current working directory.
 
   VMEHARDWARE - Show hardware installed in the VME crate
   $$$$MISC - Miscellaneous Commands
   MISCELLANEOUS COMMANDS ..................................................
   CMD file  - Get PACMAN commands from file.cmd
 
   LOG mess  - Write message to acquisition log file
 
   DATE      - Show date and time of day
 
   TIME      - Same as DATE
 
   IPCS opts - Show status of system IPC resources
    
   22-Sep-17 ....... PACMAN II - ORPHAS Acquisition Manager ......... PAGE  22
 
 
   LS        - Show directory
 
   DIR       - Same as LS
 
   PRTDOC    - Print PACMAN user document
 
   PWD       - Show current working directory
   $$$$EXIT - How to STOP the acquisition system
   HOW TO EXIT PACMAN ......................................................
 
   KILL ALL  - STOP all acquisition.  This terminates all acquisition
               processes in the workstation.  To run acquisition again
               you must run pacman again.  If file output is running,
               two EOFs are written and then the file is positioned
               between the two EOFs.
 
               WARNING:  This command MUST be upper case!
   $$$$CMDS - Command list
 
   VME commands:
       bootvme       cd             edit          initvme
       kt            lt             loadacq       loaddssd
       pacor         pacfile        setup         startvme
       statvme       stopvme        testvme       vi
       vmehost       vmehardware    zeroclk
 
   Help commands:
       911           end            exit          newf
       h             help           quit
 
   Miscellaneous commands:
       cmd           cmdf           date          dir
       hup           ipcs           kill          log
       ls            prtdoc         ps            pwd
       time          userconfig     zbuf
 
   File commands:
       afof          afon          bfo           bro
       bto           clo           da            dev
       di            dif           dz            eof
       fdir          ffo           fro           hnum
       hout          htit          ouf           ulo
       pa            pev           pi            pif
       psav          pz            rdi           rdo
       rwo           scat          tcont         ton
       trun          tstop
   $$$$NEW  - How to update Help messages
   HOW TO UPDATE YOUR HELP MESSAGES.........................................
 
   NEWF - Help file may be revised when acquisition is running.  This
          command will replace the current help with the revised version.
 
 
   PACM.160 Built-in Command List
 
    
   22-Sep-17 ....... PACMAN II - ORPHAS Acquisition Manager ......... PAGE  23
 
   This  section  lists  all  of  PACMAN's  built-in  commands.   User defined
   commands and windows may NOT be identical to any of these commands.
 
   VME commands:
       bootvme       cd             edit          initvme
       kt            lt             loadacq       loaddssd
       pacor         pacfile        setup         startvme
       statvme       stopvme        testvme       vi
       vmehost       vmehardware    zeroclk
 
   Help commands:
       911           end            exit          newf
       h             help           quit
 
   Miscellaneous commands:
       cmd           cmdf           date          dir
       hup           ipcs           kill          log
       ls            prtdoc         ps            pwd
       time          userconfig     zbuf
 
   File commands:
       afof          afon          bfo           bro
       bto           clo           da            dev
       di            dif           dz            eof
       fdir          ffo           fro           hnum
       hout          htit          ouf           ulo
       pa            pev           pi            pif
       psav          pz            rdi           rdo
       rwo           scat          tcont         ton
       trun          tstop
 
 
   PACM.170 How to Access On-Line Documentation
 
   Almost all documents covering data acquisition and analysis  are  available
   on-line.  These are in the following three directories:
 
          1).. /usr/hhirf/doc    -  Data analysis
 
          2).. /usr/hhirf/wks/doc - Data acquisition
 
          3).. /usr/hhirf/vme/doc - VME acquisition system
 
   The  .doc  files  in these directories are document files and are inputs to
   txx text formating code.  There are three output formats available.
 
          1).. Plain ASCII file   - Suitable for viewing with a text editor
 
          2).. LN03 file - May be printed on a DEC LN03 printer or compatible
 
          3).. Postscript Printer output
 
   Plain ASCII files can be viewed with your favorite editor  and  printed  on
   almost any printer.  To get a plain ASCII file for this document, type
 
      /usr/hhirf/txx /usr/hhirf/wks/doc/pacmanII
    
   22-Sep-17 ....... PACMAN II - ORPHAS Acquisition Manager ......... PAGE  24
 
 
   The  output  file is in your current working directory and is pacmanII.tex.
   Many documents are  already  available  in  this  format  in  the  document
   directories.
 
   To produce a file for a LN03 or compatible printer, type
 
      /usr/hhirf/txx /usr/hhirf/wks/doc/pacmanII x
 
   The  output is file pacmanII.txx which must be queued to the printer to get
   a hard copy.  The output file is in your current working directory.
 
   Finally, if you have a Postscript printer, type
 
      /usr/hhirf/dodoc pacmanII
 
   The output is queued to the  default  Postscript  printer.   However,   two
   temporary  files, pacmanII.txx and pacmanII.ps, are created in your current
   working directory.  These files are  deleted  after  the  print  request is
   queued.
