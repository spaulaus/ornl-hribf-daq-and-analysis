   22-Sep-17 .... ACQS  - Setup of workstation and VME for acq ...... PAGE   1
 
 
 
   Sec Page     Contents
 
   000   1  Linux workstation software
   010   1  VMEterm window
   020   1  Preparing your VME processor bootstrap program - CPU-40
   030   2  Edit file /usr/acq/etc/ORPH_nodes.lst - CPU-40
   040   2  Initial setup of the VME processor - CPU-40
   050   2  Loading bootstrap code into the VME processor - CPU-40
   060   5  Initial setup of the VME processor - CPU-60
   070   6  Edit file /usr/acq/etc/ORPH_nodes.lst - CPU-60
   080   6  Loading bootstrap code into the VME processor - CPU-60
 
 
   Setup of the Linux workstation acquisition software is documented at
 
   ACQS.000  Linux workstation software
 
   Setup  of  the  Linux  workstation  acquisition  software  is documented at
   http://www.phy.ornl.gov/local/computer/daq/daqsupport.shtml
 
 
   ACQS.010  VMEterm window
 
   Connect the first serial port on the workstation to the first  serial  port
   on  the Force processor module. The cable connecting the workstation to the
   Force processor must include the RS-232 signals RTS  and  CTS.   The  cable
   must also be a "null modem".
 
   Open a window on the workstation and start /usr/acq/vme/VMEterm.
 
   For  additional  information  on  VMEterm,  see /usr/hhirf/vme/doc/VMEterm.
   Use dodoc to print a copy or viewdoc to view online.
 
 
   ACQS.020  Preparing your VME processor bootstrap program - CPU-40
 
   FORCE Computers should supply with the documentation the hardware  ethernet
   address  assigned  to  your  VME  processor  module.  That hardware address
   should be put in the bootstrap code for the VME processor.  The file  to be
   changed  is  in  the  directory mcsq/Dlinux/Dvme.  The file is vme1_boot.s.
   Edit this file and find the following line:
 
   LAN_ADDR        DC.B    $00,$80,$42,$00,$29,$79         ;Our address
 
   This file was prepared for a VME  processor  having  an  ethernet  hardware
   address of  00-80-42-00-29-79.
 
   Change the address to that assigned to YOUR processor.
 
   Now  you  must  assemble  and  link  the   program.    Use   the   makefile
   vme1_boot.make.
 
      make -f vme1_boot.make
 
    
   22-Sep-17 .... ACQS  - Setup of workstation and VME for acq ...... PAGE   2
 
   If  you  have  a second VME processor, you can use the file vme2_boot.s for
   it's bootstrap.  This file  is  identical  to  vme1_boot.s  except  for the
   ethernet address.
 
 
   ACQS.030  Edit file /usr/acq/etc/ORPH_nodes.lst - CPU-40
 
   This  file  assigns  names to physical ethernet addresses.  It is extremely
   important that names vme1  and  vme2  be  changed  to  reflect  the  actual
   physical  ethernet  addresses  of  YOUR  VME  processors.  Furthermore, the
   ethernet addresses you enter here MUST agree with those  in  the  bootstrap
   files(SEC  020).   All  workstation  software uses this file to translate a
   name to a physical address.  The distribution file is:
 
   #  Host table for Oak Ridge PHysics VME based acquisition system
   #   Format-  hostname: address (hexadecimal)
   #   Text after # is ignored.
   vme1:   eth1 00-80-42-00-29-79
   vme2:   eth1 00-80-42-04-65-26
   vme20:  eth1 00-80-42-0d-03-62       #CPU-60
 
   Assuming you have two VME processor modules, vme1 should  be  assigned  the
   physical  address  of  the  first processor and vme2 should be the physical
   address of the other.
 
   You must set the environment variable VME, to select a specific  processor.
   For example,
 
         setenv VME vme2
 
   would  cause  all codes which communicate with the VME processor to use the
   physical address associated with the name vme2.
 
 
   ACQS.040  Initial setup of the VME processor - CPU-40
 
   Step one is to install the battery.  FORCE  ships  the  module  without the
   battery  installed.   This  is important because the bootstrap code will be
   stored in the battery backed RAM.
 
   Check all jumpers and switches to see that  they  are  installed  in  their
   DEFAULT positions.  See FORCE documentation for the default settings.
 
   Install  the  VME  processor  in slot 1 and the Kinetics CAMAC interface in
   slot 2. Turn on power to the VME crate.   Within  approx.  45  seconds, you
   should get a banner and a prompt in the VMEterm window.
 
   Verify  that  the  VME  processor  works  and  in  particular  that the VME
   interface works.  Using the VMEPROM, access registers in  the  Kinetics VME
   module to verify that the VME processor can access the VMEbus.
 
 
   ACQS.050  Loading bootstrap code into the VME processor - CPU-40
 
   You can load the bootstrap code using VMEterm.
 
    
   22-Sep-17 .... ACQS  - Setup of workstation and VME for acq ...... PAGE   3
 
   ? load vme1_boot.run
   Loading File: vme1_boot.run
   bp $b01,1
 
   ? lo <1 ,,
 
 
 
   ? 0300B547
 
   VMEPROM Error 50   Invalid file name
 
   ? S9030000FC
 
   VMEPROM Error 50   Invalid file name
 
   ?
   ?
 
   The  two  "VMEPROM  Error  50"  messages are normal. Using VMEPROM, you can
   examine memory starting at address FFC10000 to see if  the  bootstrap  code
   has  been  loaded.   If  everything  looks good, start the VME processor at
   address FFC10000.
 
          go $ffc10000
 
   When you start the VME processor, you get a banner in  the  VMEterm  window
   which includes:
 
          VME Processor ready for download
 
   Connect  the  Ethernet  between  the  workstation and the VME processor and
   login at the workstation console.  If  the  VME  processor  is  working, it
   will  be sending an ethernet packet to the workstation every 10 seconds. To
   test this type the following:
 
         /usr/acq/vme/listen eth1
 
   You should get the following(with date, time and  ethernet  source  address
   changes):
 
   daqdev2> /usr/acq/vme/listen eth1
   Device H/W address: 00:02:b3:96:b0:f1
   Interface flags = 0x1043
   Interface eth1 is UP
   Interface MTU = 1500
   *********************  Packet  *************************
   Time - Fri Oct 22 11:50:46 2004
    Packet Length: 64
   Multicast
      Destination: 03-6d-63-73-71-00
           Source: 00-80-42-04-65-26
         Protocol: 4f-51
    Buffer Length: 50
     Ack: 0
   Order: 0
    
   22-Sep-17 .... ACQS  - Setup of workstation and VME for acq ...... PAGE   4
 
     PID: 0
   56 32 2e 30 20 49 50 4c 20 46 6f 72 63 65 20 43    V2.0 IPL Force C
   50 55 2d 34 30 20 3a 20 52 65 61 64 79 20 66 6f    PU-40 : Ready fo
   72 20 64 6f 77 6e 6c 6f 61 64 0d 00                r download..
 
   *********************  Packet  *************************
   Time - Fri Oct 22 11:50:56 2004
    Packet Length: 64
   Multicast
      Destination: 03-6d-63-73-71-00
           Source: 00-80-42-04-65-26
         Protocol: 4f-51
    Buffer Length: 50
     Ack: 0
   Order: 0
     PID: 0
   56 32 2e 30 20 49 50 4c 20 46 6f 72 63 65 20 43    V2.0 IPL Force C
   50 55 2d 34 30 20 3a 20 52 65 61 64 79 20 66 6f    PU-40 : Ready fo
   72 20 64 6f 77 6e 6c 6f 61 64 0d 00                r download..
 
   daqdev2>
 
   There  are  just  a  few more steps to setup the VME processor.  We need to
   load a code /usr/acq/vme/vme_listen and then load and  execute  a  code  to
   make  a  backup  copy  of  our  bootstrap  in the onboard flash EEPROM. The
   following should do all this:
 
       1) Set the enviroment variable VME to the processor you
          are using.  For example, if you are using vme2, type
 
          setenv VME vme2
 
       2) Reset VME processor and type  go $ffc10000
 
       3) At the workstation type:
 
          /usr/acq/vme/srload /usr/acq/vme/vme_listen
 
          In the VMEterm window, you should get a message:
 
          CPU-40 Loaded with VME_LISTEN
 
       4) Reset VME processor and type  go $ffc10000
 
       5) At the workstation type:
 
          /usr/acq/vme/srload /usr/acq/vme/sav_res
 
          In the VMEterm window, you should get a message:
 
          CPU-40 Loaded with SAV_RES
 
       6) Reset VME processor and type  go $80000
 
       7) To copy to flash EEPROM, type:
 
    
   22-Sep-17 .... ACQS  - Setup of workstation and VME for acq ...... PAGE   5
 
          prog $100000 $ffc80000 0
 
 
   This last step takes a few seconds.  When finished you have  a  backup copy
   of the bootstrap.
 
   To restore the SRAM from the flash EEPROM,
 
          go $ffc80000
 
   If  you made it here successfully, things are going to be a lot easier from
   now on.  Happy days are here again!
 
   From now on we will use the ethernet for all  communication  with  the  VME
   processor.   To  do  this, change switch 1 on the VME processor front panel
   from F to B and reset the processor. Now after reset,  the  processor  will
   start at address ffc10000 and can be downloaded via ethernet.
 
 
   ACQS.060  Initial setup of the VME processor - CPU-60
 
   FORCE  ships  the  module  with  battery backup for the SRAM disabled.  You
   MUST enable battery backup of the SRAM because the bootstrap code  will  be
   stored in SRAM.  To enable battery backup set the following switches to ON:
 
            SW5 - 2    ON    Enable battery backup
            SW5 - 3    ON    Enable RTC and SRAM
 
   In  addition, it is recommended that you write proctect the System PROM. To
   do this switch to ON:
 
            SW10 - 3   ON    Write proctect System PROM
 
   All other switches on the CPU-60 module should be set to  OFF.   The  front
   panel switches 1 and 2 should be set to F.
 
   Install  the  VME  processor  in slot 1 and the Kinetics CAMAC interface in
   slot 2. Turn on power to the VME crate.   Within  approx.  45  seconds, you
   should get a banner and a prompt in the VMEterm window.
 
   Verify  that  the  VME  processor  works  and  in  particular  that the VME
   interface works.  Using the VMEPROM, access registers in  the  Kinetics VME
   module to verify that the VME processor can access the VMEbus.
 
   There  is  a single bootstrap program for ALL CPU-60 processors.  FORCE now
   has the Ethernet address stored on-board.  The code is mcsqvme_boot60.s.
 
   However, you still need to determine the ethernet address for  each  CPU-60
   so  that  the  proper entry can be made in the ORPH_nodes.lst file( See SEC
   070).
 
   To get the ethernet address, use the VMEPROM command
 
         info
 
 
    
   22-Sep-17 .... ACQS  - Setup of workstation and VME for acq ...... PAGE   6
 
   ACQS.070  Edit file /usr/acq/etc/ORPH_nodes.lst - CPU-60
 
   This file assigns names to physical ethernet addresses.   It  is  extremely
   important  that names vme1, vme2 and vme20 be changed to reflect the actual
   physical ethernet addresses  of  YOUR  VME  processors.   Furthermore,  the
   ethernet  addresses  you  enter  here MUST agree with those stored on-board
   the VME module (See SEC 130). All workstation software uses  this  file  to
   translate  a  name  to  a physical address.  CPU-60 modules MUST have names
   starting with vme20.  The distribution file is:
 
   #  Host table for Oak Ridge PHysics VME based acquisition system
   #   Format-  hostname: address (hexadecimal)
   #   Text after # is ignored.
   vme1:   eth1 00-80-42-00-29-79
   vme2:   eth1 00-80-42-04-65-26
   vme20:  eth1 00-80-42-0d-03-62       #CPU-60
 
   Assuming you have two VME processor modules, vme1 should  be  assigned  the
   physical  address  of  the  first processor and vme2 should be the physical
   address of the other.
 
   You must set the environment variable VME, to select a specific  processor.
   For example,
 
         setenv VME vme20
 
   would  cause  all codes which communicate with the VME processor to use the
   physical address associated with the name vme20.
 
 
 
   ACQS.080  Loading bootstrap code into the VME processor - CPU-60
 
   You can load the bootstrap code using VMEterm. Type
 
          load vme_boot60.run
 
   The two "VMEPROM Error 50" messages are  normal.  Using  VMEPROM,  you  can
   examine  memory  starting  at address FFC08000 to see if the bootstrap code
   has been loaded.  If everything looks  good,  start  the  VME  processor at
   address FFC08000.
 
          go $ffc08000
 
   When  you  start  the VME processor, you get a banner in the VMEterm window
   which includes:
 
          VME Processor ready for download
 
   Connect the Ethernet between the workstation  and  the  VME  processor  and
   login  at  the  workstation  console.   If the VME processor is working, it
   will be sending an ethernet packet to the workstation every 10 seconds.  To
   test this, type the following:
 
         /usr/acq/vme/listen eth1
 
    
   22-Sep-17 .... ACQS  - Setup of workstation and VME for acq ...... PAGE   7
 
   You  should  get  the following(with date, time and ethernet source address
   changes):
 
   daqdev2> /usr/acq/vme/listen eth1
   Device H/W address: 00:02:b3:96:b0:f1
   Interface flags = 0x1043
   Interface eth1 is UP
   Interface MTU = 1500
   *********************  Packet  *************************
   Time - Fri Oct 22 11:50:46 2004
    Packet Length: 64
   Multicast
      Destination: 03-6d-63-73-71-00
           Source: 00-80-42-0d-03-62
         Protocol: 4f-51
    Buffer Length: 50
     Ack: 0
   Order: 0
     PID: 0
   56 32 2e 30 20 49 50 4c 20 46 6f 72 63 65 20 43    V3.0 IPL Force C
   50 55 2d 34 30 20 3a 20 52 65 61 64 79 20 66 6f    PU-60 : Ready fo
   72 20 64 6f 77 6e 6c 6f 61 64 0d 00                r download..
 
   *********************  Packet  *************************
   Time - Fri Oct 22 11:50:56 2004
    Packet Length: 64
   Multicast
      Destination: 03-6d-63-73-71-00
           Source: 00-80-42-0d-03-62
         Protocol: 4f-51
    Buffer Length: 50
     Ack: 0
   Order: 0
     PID: 0
   56 32 2e 30 20 49 50 4c 20 46 6f 72 63 65 20 43    V3.0 IPL Force C
   50 55 2d 34 30 20 3a 20 52 65 61 64 79 20 66 6f    PU-60 : Ready fo
   72 20 64 6f 77 6e 6c 6f 61 64 0d 00                r download..
 
   daqdev2>
 
   There are just a few more steps to setup the VME  processor.   We  need  to
   load  a  code /usr/acq/vme/vme_listen60 and then load and execute a code to
   make a backup copy of our bootstrap  in  the  on-board  flash  EEPROM.  The
   following should do all this:
 
       1) Set the enviroment variable VME to the processor you
          using.  For example, if you are using vme20, type
 
          setenv VME vme20
 
       2) Reset VME processor and type  go $ffc08000
 
       3) At the workstation type:
 
          /usr/acq/vme/srload /usr/acq/vme/vme_listen60
 
    
   22-Sep-17 .... ACQS  - Setup of workstation and VME for acq ...... PAGE   8
 
          In the VMEterm window you should get a message:
 
          CPU-60 Loaded with VME_LISTEN60
 
       4) Reset VME processor and type  go $ffc08000
 
       5) At the workstation type:
 
          /usr/acq/vme/srload /usr/acq/vme/sav_res60
 
          In the VMEterm window you should get a message:
 
          CPU-60 Loaded with SAV_RES60
 
       6) Reset VME processor and type  go $80000
 
       7) To copy to flash EEPROM, execute the following two commands:
 
          ferase user_flash
 
          fprog user_flash,$100000
 
 
   This  last  step takes a few seconds.  When finished you have a backup copy
   of the bootstrap.
 
   To restore the SRAM from the flash EEPROM,
 
          go $ffc80000
 
   If you made it here successfully, things are going to be a lot easier  from
   now on.  Happy days are here again!
 
   From  now  on  we  will use the ethernet for all communication with the VME
   processor.  To do this, change switch 1 on the  VME  processor  front panel
   from  F  to  7 and reset the processor. Now after reset, the processor will
   start at address ffc08000 and can be downloaded via ethernet.
