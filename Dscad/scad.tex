   10-Feb-06 ...... U230  SCAD - CAMAC Scaler Display Program ....... PAGE   1
 
   Sec Page Contents
 
   010   1  Introduction
 
   020   2  Main Features
 
   030   3  Tabular Display - Command List - General
 
   040   4  Tabular Display - Command list - for Rate Limits
 
   050   4  Tabular Display - Discussion of Audio Alarms - BONGO
 
   060   5  Graphic Display - Introduction
 
   070   6  Graphic Display - Command List
 
   080   6  Graphic Display - Comments
 
   090   7  Preparation of Scalar Initialization (snit) Files
 
   100   8  Example snit file
 
   110   9  Example Display with Display Limits set
 
   120  10  How to Copy CAMAC Scalers to Tape
 
 
   U230.010  Introduction
 
   scad  is New & Improved - it now provides for graphical rate displays (like
   scudd) in addition  to  the  standard  tabular  displays  of  earlier  scad
   versions.  It also provides for printing scaler tables on a printer of your
   choice and simultaneously copying same to a "snap file" (see SEC# 030).
 
   The tabular display is defined by one "scaler initialization"  (snit)  file
   (you  are  prompted  for  this one when scad is first started). This is the
   only opportunity that you have to specify this file.
 
   Graphical rate   displays  are  defined  by  another  snit  file  which  is
   specified by the command:
 
   gnit filename
 
   This  file  can  be the same as the file for tabular displays if the number
   of scalers are no more than 8. This file may be redefined at  any  time  or
   the graphic display eliminated by the command:
 
   clrg
 
                               How to Get Started
 
   Make a snit-file for the tabular display (required)
 
   Make a snit-file for any graphic display (optional) (max# = 8)
 
   Type: setenv VME vmex   ;where vmex denotes the name of the vme
                           ;processor that you are using - vme1, vme2..
 
   Type: /usr/hhirf/scad   ;To start program
                           ;Enter tabular snit file name as directed
                           ;Then enter commands from following lists
 
    
   10-Feb-06 ...... U230  SCAD - CAMAC Scaler Display Program ....... PAGE   2
 
 
   U230.020  Main Features
 
   o......Supports  "live" display of raw and computed (linear combinations of
          hardware readings and/or computed values) scaler information.
 
   o......Supports scaler logging to file scad.log via the command log or  via
          auto-logging (see LSEC command SEC# 30).
 
   o......A  maximum of 240 scalers (raw plus computed) may be displayed and a
          maximum of 500 scalers may be logged.
 
   o......The display interval may be changed at run-time from  1  to  20  sec
          (the default is 5 sec). Display intervals are approximate.
 
   o......Provides  for the setting of rate limits which generates an alerting
          display (as well as an audio alarm - see SEC#  050)  if  limits  are
          violated.
 
   o......Automatically  adjusts display to window size and warns if window is
          too small.
 
   o......Supports Command File processing
 
   o......Provides for graphic  display  of  up  to  8  scalers  via  software
          generated meters (see SEC# 060 to 080).
    
   10-Feb-06 ...... U230  SCAD - CAMAC Scaler Display Program ....... PAGE   3
 
 
   U230.030  Tabular Display - Command List - General
 
   h               Displays on-line help
 
   lon             Turns ON  output to scad.log  (default)
   lof             Turns OFF output to scad.log
 
   zero            Zero all scalers defined in snit-file
 
   sec  nsec       Set display interval to nsec seconds (default = 5 sec)
 
   lsec nsec       Set autolog interval to nsec seconds (default = 0, OFF)
                   (allowed values are 0 (OFF) and 300 to 3600 seconds)
 
   nort            Normalize rates to cts/sec using internal clock (default)
                   (any NOR specification in snit-file resets default to nors)
 
   nors            Normalize rates as specified in snit-file (NOR spec)
 
   tst             Tests display configuration only - uses random data
 
   tsta            Tests ALARM generation (takes 15 seconds) (see SEC# 050)
 
   run             Do repeative read/display operations
 
   Ctrl-C          Interrupt RUN for subsequent commands
 
   log             Record scalers on scad.log
 
   snap            Records scalers on scad-yyyy-mm-dd_hh:mm:ss.snap
                   Example ---------- scad-2005-03-08_10:53:12.snap
                   Also prints file on printer (default printer or that
                   specified by the namp command below)
 
   Ctrl-
                   That is Ctrl-backslash
                   Does nothing in command-line mode
 
   namp prname     Says print snaps on printer prname where
                   prname is like (ps01, cp6000, jet6000, etc)
 
   namp            Says print snaps on the default printer
 
   stat            Displays/logs current status information
 
   end             End program
 
    
   10-Feb-06 ...... U230  SCAD - CAMAC Scaler Display Program ....... PAGE   4
 
 
   U230.040  Tabular Display - Command List - for Rate Limits
 
   rlim NAME LO HI  Sets rate-limits (LO, HI) for NAMEd scaler (max# = 10)
 
   rlim NAME off    Deletes rate-limits for NAMEd scaler
 
   rlim off         Deletes all defined scaler rate-limits
 
   rlim sho         Displays all scaler rate-limits
 
   bpof NUM         NUM = #beeps before beeper is turned OFF  (dflt=100)
                    (allowed values are 0 to 1000)
 
   bpon NUM         NUM = #in-limit displays before beeper ON (dflt=100)
                    (allowed values are 0 to 1000)
 
   hush NUM         Suspend bongo audio alarm for NUM minutes
 
   See  the  SEC#  110  for an example display with 3 rate-limits set. This is
   just a random data test display.
    
   10-Feb-06 ...... U230  SCAD - CAMAC Scaler Display Program ....... PAGE   5
 
 
 
   U230.050  Tabular Display - Discussion of Audio Alarms - BONGO
 
   This is what scad does
 
   o......When scad is started, it opens or creates a  file  scadalarm.dat  in
          the directory in which scad is running.
 
   o......If  one  or  more  limits  are defined and violated, scad writes the
          epoch time (# seconds since Jan 1,1970) into record 1 of that file.
 
   o......This occurs every 5 seconds as long as a limit is being violated.
 
   This is how BONGO works
 
   o......You must open a window on some LINUX CPU console that is  nearby  so
          that you can hear it's speakers.
 
   o......Set the directory to same as that where scad is running and type:
 
   /usr/hhirf/bongo
 
   o......BONGO  is a looping script which runs programs to read scadalarm.dat
          every 5 seconds to see if  the  time  written  there  is  within  10
          seconds of the current epoch time.
 
   o......If it is, then a play process is executed which sounds a loud BONG!
 
   o......This  BONG sound is repeated every 5 seconds until no limit is being
          violated or the associated scad is halted via CTRL-C or  terminated.
          The BONGing can be suspended by interrupting scad and typing:
 
   hush MIN
 
          Where  MIN  is  the  number  of  minutes  to  SHUT UP. This lets you
          continue running scad without having to hear all of that BONGing!
 
                    Well, this is how it is supposed to work!
                                       but
      The last upgrade to Redhat-10 seems to have broken the play function!
         So I guess we'll have to see if it can be fixed by the experts!
    
   10-Feb-06 ...... U230  SCAD - CAMAC Scaler Display Program ....... PAGE   6
 
 
   U230.060  Graphic Display - Introduction
 
   scad also provides for a graphic display of  a  limited  number  of  scaler
   rates. Features are listed below:
 
   o......Up  to  8  scalers  to be graphically displayed may be defined via a
          standard snit file using the gnit command (see SEC# 070).
 
   o......The count rate in  counts/sec  (based  on  the  computer's  internal
          clock) is displayed in the form of software generated "meters".
 
   o......Displays  may  be  either linear or log (all displays will be of the
          same type).
 
   o......For log display, The display range is 1 to  20  with  a  power-of-10
          scale  factor  (shown  in  the  window  banner). The scale factor is
          automatically changed when the count rate falls outside the  current
          range.
 
   o......For  linear display, the display range is 0 to 12 with a power-of-10
          scale factor (shown in the  window  banner).  The  scale  factor  is
          automatically  changed  when the count rate exceeds the current high
          limit or falls below 0.8 times the current scale factor.
 
             The Figure Below Illustrates the Graphic Scaler Display
 
 
 
 
 
 
 
 
         (See ORPHAS Handbook Tab-8 for a picture of an example display)
 
 
 
 
 
 
 
 
      GLIM EVT 40  80         GLIM PMT 0  500          GLIM MCP_PSD 500 2000
 
   Note: Rate limits have been defined by the GLIM command as  indicated.  See
   the next page for command syntax which is (glim name lo hi).
 
   Note:  The inside pointer indicates the power-of-10 scale factor and is the
   same as that given in the banner.
 
   Limits specified via the glim command specify "rate  values"  for  which  a
   "visual alarm" will be displayed.
 
   IF(rate.GT.hi) - Displays flashing yellow disks at upper left & right
   IF(rate.LT.lo) - Displays flashing white  disks at lower left & right
 
   The  number  at  the  bottom of each display gives the rate averaged over a
   greater time than that of the meter (so you can read it). This  fact  leads
   to  some  disagreement  between the two when rates change rapidly which was
   the case here.
    
   10-Feb-06 ...... U230  SCAD - CAMAC Scaler Display Program ....... PAGE   7
 
 
   U230.070  Graphic Display - Command List
 
 
   gnit  filename   Opens & processes standard snit-file
 
   glin             Set graphic display to linear
 
   glog             Set graphic display to log
 
   clrg             Clear graphic display & deletes gnit definition
 
   rav   N          Sets to display an average of N readings
                    allowed value of N is 0 to 100
                    N = 0 or blank sets N = 5 (the default)
 
   dps   N          Specifies display rate to be N/sec
                    allowed value of N is 0 to 20
                    N = 0 or blank sets 10 displays/sec (the default)
 
   glim NAME lo hi  Set limits (lo hi) for NAMEd graphic scaler
 
   glim NAME on     Enable  limits for NAMEd graphic scaler display
   glim NAME off    Disable limits for NAMEd graphic scaler display
   glim NAME null   Deletes limits for NAMEd graphic scaler display
 
   glim on          Enable  limits for ALL   graphic scaler displays
   glim off         Disable limits for ALL   graphic scaler displays
   glim null        Deletes limits for ALL   graphic scaler displays
 
   glim sho         Display all graphic scaler limits
 
   revv             Sets to reverse color mapping on subsequent FIG
 
   fig              Does a FIG for the # of scalers defined (1 to 8)
 
   stat             Displays/logs current status information
 
 
 
   U230.080  Graphic Display - Comments
 
   When a snit-file is processed via the gnit command,  any  computed  scalers
   defined in that file are ignored.
 
   Limits  specified  via  the  glim command specify "rate values" for which a
   "visual alarm" will be displayed.
 
   IF(rate.GT.hi) - Displays flashing yellow disks at upper left & right
 
   IF(rate.LT.lo) - Displays flashing white  disks at lower left & right
 
    
   10-Feb-06 ...... U230  SCAD - CAMAC Scaler Display Program ....... PAGE   8
 
 
   U230.090  Preparation of Initialization (snit) files
 
   Use the editor to prepare a scaler initialization file  (snit  file  -  the
   filename  extension  is  usually  .sca).  The snit file is divided into two
   sections. First the hardware scalers to be read are specified, followed  by
   definitions  of  any computed scalers. The general form of the snit file is
   shown below:
 
   ----------------------------------------------------------------------
   LABEL C N A
     .
   LABEL C N A       ;NOR NORF (normalize all rates to this scaler)
     .
   LABEL C N A ECL   ;NOD
     .
   LABEL C N A ECL
     .
   $END                          (flags the end of scalers to be read)
   LABEL : EXPRESSION ...        (computed scaler - no display)
   &       EXPRESSION-continued  (up to 20 lines per definition)
   LABEL = EXPRESSION            (computed scaler - to be displayed)
                                 (see SEC# 350.170 for EXPRESSION info)
   ----------------------------------------------------------------------
 
   LABEL..denotes a unique label (11 characters max)  which  must  contain  no
          imbedded blanks or the characters + - / *
 
   C......denotes the scaler module crate number.
 
   N......denotes the scaler module slot number.
 
   A......denotes the scaler sub-address (numbers start at 0).
 
   ECL....denotes an ECL-scaler (requires special read/clear functions).
 
   NOD....specifies no LIVE display when using program scad.
 
   NOR....specifies  that  all  rates  (counts  accumulated during the display
          interval) are to be normalized to this scaler.
 
   NORF...specifies a rate normalization factor which is used as follows:
 
          Rate = NORF*Count-Increment/(Normalizing Scaler Increment)
   ----------------------------------------------------------------------
 
   Discussion of Computed Scaler Expressions
 
   All expressions are of the form:  OP LABEL OP LABEL OP LABEL ...
 
   Where, OP denotes an operator (+ - / *) and LABEL  denotes  any  previously
   defined  label.  If  the OP-field is omitted, + is assumed. All expressions
   are evaluated LEFT-TO-RIGHT. Due to  the  limited  expression  syntax,  you
   will  probably  need to define intermediate LABELs which you do not wish to
   display. This is accomplished by using a : instead of an  =  in  the  label
   definition, as illustrated above.
    
   10-Feb-06 ...... U230  SCAD - CAMAC Scaler Display Program ....... PAGE   9
 
 
   U230.100  Example snit file
 
   Time.100   9 17  0 -2 ; NOR 100
   Live.Time  9 17  1
   Integrator 9 17  2
   Live.Int   9 17  3
                                 (A blank line is displayed)
   Ge.Trigs   9 17  4
   See.Trigs  9 17  5
   Acc.Trigs  9 17  6
   #Clean & Dirty -------------- (27 characters of title is displayed)
   Clean01..S3     9  1 3 0 ECL
   Dirty01.BHB     9  2 3 0 ECL
   Clean02..S2     9  1 2 0 ECL
   Dirty02.NHC     9  2 2 0 ECL
   Clean03..S8     9  1 1 0 ECL
   Dirty03.BPE     9  2 1 0 ECL
   Clean04.S10     9  1 0 0 ECL
   Dirty04.NPC     9  2 0 0 ECL
   Clean05..S1     9  1 7 0 ECL
   Dirty05.BHC     9  2 7 0 ECL
   Clean06..S4     9  1 6 0 ECL
   Dirty06.BHD     9  2 6 0 ECL
   Clean07..S7     9  1 5 0 ECL
   Dirty07.BPC     9  2 5 0 ECL
   Clean08..S0     9  1 4 0 ECL
   Dirty08.NPA     9  2 4 0 ECL
   Clean09..S5     9  1 11 0 ECL
   Dirty09.NHD     9  2 11 0 ECL
   Clean10..S6     9  1 10 0 ECL
   Dirty10.BPD     9  2 10 0 ECL
   Clean11..N5     9  1 9 0 ECL
   Dirty11.NHB     9  2 9 0 ECL
   Clean12..N6     9  1 8 0 ECL
   Dirty12.BPB     9  2 8 0 ECL
   Clean13..N8     9  1 15 0 ECL
   Dirty13.NPB     9  2 15 0 ECL
   Clean14..N0     9  1 14 0 ECL
   Dirty14.NPE     9  2 14 0 ECL
   Clean15..N4     9  1 13 0 ECL
   Dirty15.BHA     9  2 13 0 ECL
   Clean16.N10     9  1 12 0 ECL
   Dirty16.BPA     9  2 12 0 ECL
   Clean17..N7     9  1 31 0 ECL
   Dirty17.BPF     9  2 31 0 ECL
   Clean18..N2     9  1 30 0 ECL
   Dirty18.NHA     9  2 30 0 ECL
   Clean19..N3     9  1 29 0 ECL
   Dirty19.BHE     9  2 29 0 ECL
   Clean20..N1     9  1 28 0 ECL
   Dirty20.BHF     9  2 28 0 ECL
   #Computed Scalers ----------- (27 characters of title is displayed)
   $END
   GeTr
   AccTr
   Cln13
    
   10-Feb-06 ...... U230  SCAD - CAMAC Scaler Display Program ....... PAGE  10
 
 
   U230.110  Example Display with Rate Limits Set
 
   The  following  command file was used to set up the rate limits and start a
   test display (just random data).
 
   mxrc 30 128
   rlim Time.100 0 200
   rlim Ge.Trigs 0 10000
   rlim See.Trigs 0 2000
   tst
 
 
   Time.100      10015  100.00    Dirty11.NHB   10064 18.6879
   Live.Time      9256-11511-2    Clean12..N6    9791  110.74
   Integrator     9357 26.4414    Dirty12.BPB    9437-83698-3
   Live.Int       9420-14195-2    Clean13..N8    9721 21.6700
                                  Dirty13.NPB    9556-93241-3
   Ge.Trigs       9645-20676-3    Clean14..N0    9909-14115-3
   See.Trigs      9845 30.6163    Dirty14.NPE    9897-39364-3
   Acc.Trigs      9489-14115-2    Clean15..N4    9965 70.9742
   #Clean & Dirty ------------    Dirty15.BHA    9329-13837-2
   Clean01..S3    9411-15348-2    Clean16.N10    9448 46.1233
   Dirty01.BHB    9636 64.8111    Dirty16.BPA    9457-55666-4
   Clean02..S2    9670  6.3618    Clean17..N7    9395-63419-3
   Dirty02.NHC    9880       0    Dirty17.BPF    9986-19483-3
   Clean03..S8    9655 53.2803    Clean18..N2    9654-29225-3
   Dirty03.BPE    9862-34990-3    Dirty18.NHA   10143-41750-4
   Clean04.S10    9802 88.0716    Clean19..N3   10064  6.5606
   Dirty04.NPC    9553-82306-3    Dirty19.BHE    9255-57853-3
   Clean05..S1    9657-91849-3    Clean20..N1    9446 40.1590
   Dirty05.BHC    9210-51292-3    Dirty20.BHF    9834-46720-3
   Clean06..S4   10127 30.4175    #Computed Scalers ---------
   Dirty06.BHD    9932-14314-3    GeTr
   Clean07..S7    9233-18390-2    AccTr
   Dirty07.BPC    9650 47.1173    Cln13
   Clean08..S0    9409-11948-2
   Dirty08.NPA    9364 13.1213
   Clean09..S5    9239-50696-3
   Dirty09.NHD    9545 33.0020
   Clean10..S6    9281-34791-3
   Dirty10.BPD    9721 20.4771
   Clean11..N5    9809 74.1551
   *******************************
   Time.100    100.0     0 200.0
   Ge.Trigs   -207-1     0 10000 L
   See.Trigs   30.62     0  2000
   -------------------------------
    
   10-Feb-06 ........ U230  SCAT  -  Scaler Dump Procedure .......... PAGE  11
 
 
   U230.120  How to Copy CAMAC Scalers to Tape
 
   CAMAC scalers may be read and written to the data acquisition  output  tape
   during the acquisition process. The following features are supported:
 
   (1)....Scalers  to  be  read  are  specified  in  exactly  the  same way as
          described for program scad (i.e. for real time  display).  See  SEC#
          090 and 100.
 
   (2)....Scaler  records  may  be  written  to tape at timed intervals (10 to
          1000 seconds are allowed), preceeding file-marks or both.
 
   (3)....Scalers may be cleared after each reading if desired.
 
   (4)....The   date  &  time  of  recording,  recording  time  interval,  and
          clear-flag are included in each record.
 
   (5)....Scaler records are 32000 bytes long and  written  in  ASCII.  If  we
          assume an I*4 array SCAL(10,800), then:
 
          (SCAL(I,1),I=1,10) - Contains 'SCALER DUMP'
          (SCAL(I,2),I=1,10) - Contains Date & Time, Time-interval, Clear-flag
          (SCAL(I,N),I=1,10) - Contains Label, Count, Rate
                               FORMAT   3A4,   I12,   1PE12.3, 4X
          Unused portions of the 32000-byte record are blank-filled.
 
   (6)....Program  lemor  supports  the retrival, display and logging of these
          records. While running lemor, Type: H LIST, for relevant commands.
 
   (7)....The maximum number of scalers  supported  at  this  time  (raw  plus
          computed) is 790.
 
   PACMAN Commands
   ---------------------------------------------------------------------------
   SCAT  filename         ;Process scaler specification file (See scad Doc)
 
   SCAT  ON               ;Dump before EOF only
   SCAT  ON  CLR          ;Dump before EOF and (clr post dump)
   SCAT  ON  TSEC         ;Dump every TSEC and before EOF
   SCAT  ON  TSEC  CLR    ;Dump every TSEC and before EOF (clr post dump)
   SCAT  OFF              ;Disable scaler dumps
 
   SCAT  NORT             ;Normalize count rates to internal clock
   SCAT  NORS             ;Normalize count rates as specified by snit-file
   SCAT  UNOR             ;Count rates un-normalized
 
   SCAT  STAT             ;Displays/logs setup status
   SCAT  HELP             ;Displays on-line help
   SCAT  H                ;Displays on-line help
 
   SCAT  LERR ON/OFF      ;Turns error display/logging ON
   SCAT  LERR OFF         ;Turns error display/logging OFF
 
   SCAT  ZERR             ;Zeros read-error  counters
   SCAT  ZDUM             ;Zeros scaler dump counter
   SCAT  ZERO             ;Zero all scalers
 
