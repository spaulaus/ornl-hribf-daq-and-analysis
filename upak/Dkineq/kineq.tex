   02-Feb-02 ....... U610  KINEQ (UNIX & VMS versions) - WTM ........ PAGE   1
 
                     KINEMATICS AND Q-VALUES - PROGRAM KINEQ
 
   KINEQ  calculates   kinematic   quantities   either   relativistically   or
   classically  as  a  function  of either lab energy of the projectile or lab
   angle of the outgoing particle. Q-values only may also be  calculated.  The
   different  modes  of  operation  are selected by means of "commands" (input
   one per line - starting in col-1). Use whatever commands  are  required  to
   "set up" the desired "conditions" and then type "GO" to actually do it.
 
   All commands and element names may be typed in upper or lower case.
 
   ---------------------------------------------------------------------------
   CMD  ASSOCIATED DATA     ;MEANING OR ACTION
 
   CMD  FIL                 ;Open and process commands from FIL.CMD
   CMD  FIL.EXT             ;Open and process commands from FIL.EXT
 
   HELP                     ;Displays this list again
 
   REL                      ;Specifies relativistic kinematics
   CLAS                     ;Specifies classical    kinematics
   Q                        ;Specifies Q-value calculations only
 
   LEV  ELEV                ;ELEV gives level-energy in Mev
 
   FUNA ELAB,AMIN,AMAX,DA   ;Says kinematics vs lab-angle  of outgoing
   FUNA                     ;Switch to "FUNA" & use previous data
 
   FUNE ALAB,EMIN,EMAX,DE   ;Says kinematics vs lab-energy of outgoing
   FUNE                     ;Switch to "FUNE" & use previous data
                            ;ELAB,EMIN,EMAX & DE are in Mev
                            ;ALAB,AMIN,AMAX & DA are in degrees
 
   TARG(PROJ,OUT)RES        ;Specifies reaction of interest
   TARG(PROJ,OUT)           ;Same as above but KINEQ computes RES
 
   STAT                     ;Displays current "conditions"
 
   GO                       ;DO IT for current conditions
                            ;Output to kineq.log on DECstation
                            ;Output to KINEQ.PRT on VAX
 
   GOS                      ;DO IT - output to screen (for REL only)
 
   END                      ;Ends program
   ---------------------------------------------------------------------------
 
   Type   kineq             ;To start with kineq defined in your .login,
                            ;.cshrc or login.com files, otherwise:
 
   Type:  /usr/hhirf/kineq  ;To start execution on a HHIRF DECstation
 
   Type:  /home/upak/kineq  ;To start execution on a SPARCstation
 
   Type: @U1:[MILNER]KINEQ  ;To start execution on the HHIRF VAX
    
   02-Feb-02 ....... U610  KINEQ (UNIX & VMS versions) - WTM ........ PAGE   2
 
 
 
   For Q-values, type:
 
   >Q
 
   >TARG(PROJ,OUTGOING)RESIDUAL  or  TARG(PROJ,OUTGOING)
                                 or  TARG(PROJ)RESIDUAL  if they stick
 
   >GO
 
 
   For kinematic quantities vs angle, type:
 
   >REL  or  CLAS
 
   >LEV  ELEV
 
   >FUNA ELAB,AMIN,AMAX,DA
 
   >TARG(PROJ,OUTGOING)RESIDUAL  or  TARG(PROJ,OUTGOING)
 
   >GO
 
   Where,
   ELEV = The energy (Mev) of excited state (0 if ground state)
   ELAB = Lab energy (Mev) of projectile        - fixed value
   AMIN = Lab angle  (deg) of outgoing particle - minimum value
   AMAX = Lab angle  (deg) of outgoing particle - maximum value
   DELA = Lab angle  (deg) of outgoing particle - step size
 
   For Kinematic Quantities vs Lab-Energy, Type:
 
   >REL  or  CLAS
 
   >LEV  ELEV
 
   >FUNE ALAB,EMIN,EMAX,DE
 
   >TARG(PROJ,OUTGOING)RESIDUAL  or  TARG(PROJ,OUTGOING)
 
   >GO
 
   Where,
   ALAB = Lab angle  (deg) of outgoing particle - fixed value
   EMIN = Lab energy (Mev) of projectile        - minimum value
   EMAX = Lab energy (Mev) of projectile        - maximum value
   DELE = Lab energy (Mev) of projectile        - step size
   ---------------------------------------------------------------------------
 
         * * * SEE NEXT PAGE FOR COMMENTS ON WRITING THE REACTION * * *
 
                    * * * SEE NEXT PAGE FOR AN EXAMPLE * * *
    
   02-Feb-02 ....... U610  KINEQ (UNIX & VMS versions) - WTM ........ PAGE   3
 
                   How to Write the Reaction - Other Comments
 
   The  reaction  is  written in the usual way, however no imbedded blanks are
   allowed and at  least  one  blank  must  follow  the  reaction  field.  For
   example,  12C(16O,4HE)24MG  or  C12(16O,HE4)MG24  are  both acceptable. All
   particles except N (for neutron) must contain the mass number. The  proton,
   deuteron,  and  triton  are denoted by 1H, 2H, and 3H or by H1, H2, and H3.
   The reaction is decoded and "exact"  nuclear  masses  are  found  by  using
   recommended  masses  from  "The  1995 Update to the Atomic Mass Evaluation"
   by G. Audi and A. H. Wapstra, Nuclear Physics, Vol. A595 Vol 4,  p.409-480,
   December  25, 1995 or a set of mass formula coefficients due to G. T. Garvy
   et al, Rev. Mod. Phys., Vol. 41, No. 4, Part II  (1969)  Page  S1.  If  the
   mass formula is involved the Q-value is flaged with ***.
 
   If  only  Q-values  are  to  be  calculated,  there may be up to 7 outgoing
   particles or only a residual particle. Multiple outgoing particles must  be
   written  as  a  string of single particles seperated by commas. For example
   two alphas would be written as  4HE,4HE  not  2HE4.  The  maximum  reaction
   length is 40 characters.
 
   Example:   - Note: it types the ">" - you don't.
 
   >Q                              ;Specify Q-value calculation only
 
   >12C(16O,4HE)24MG               ;Enter reaction of interest
 
   >GO                             ;DO IT
 
   >REL                            ;Specify relativistic calculations
 
   >LEV 0.0                        ;Specify ground state
 
   >FUNA 150 10 90 2               ;Specify calculation vs lab-angle
 
                                   ;(enter ELAB,AMIN,AMAX,DA)
 
   >GO                             ;DO IT (Note: reaction already given)
 
   >FUNE 45 50 150 5               ;Specify calculation vs lab-energy
 
                                   ;(enter ALAB,EMIN,EMAX,DE)
 
   >GO                             ;DO IT
 
   >12C(16O,16O)                   ;Enter a new reaction
 
   >GO                             ;Do relativistic calc vs energy
 
   >FUNA                           ;Specify calculation vs lab-angle
 
                                   ;(using previous specifications)
 
   >GO                             ;DO IT
 
   >END                            ;Ends the program
 
    
   02-Feb-02 ....... U610  KINEQ (UNIX & VMS versions) - WTM ........ PAGE   4
 
 
 
                     Definition of The Calculated Quantities
 
   The  meaning  of most of the calculated quantities should be clear from the
   labels. The relativistic kinematic routine  calculates  both  solutions  in
   those cases where there are two. The two solutions are labeled 1 and 2.
 
   E(MEV)...denotes the lab energy of the projectile.
 
   THLB-O,THCM-O...denote  lab  and  center-  of- mass angles (in degrees) for
          the outgoing particle.
 
   THLB-R,THCM-R...denote lab and center- of- mass  angles  (in  degrees)  for
          the residual nucleus.
 
   E(MEV)-O,E(MEV)-R...denote  lab  energies  of  the  outgoing  and  residual
          particles.
 
   V-R...denotes the velocity (in cm/ns) of the residual particle.
 
   DWO/DWCM,DWR/DWCM...denotes the ratio of solid angle in the lab to that  in
          the center- of- mass system for the outgoing and residual particles.
 
 
   SIGR(MB/STR)CM...denotes  the  Rutherford  cross  section  (millibarns  per
          steradian in the center-of-mass system) - calculated  for  "elastic"
          scattering as a function of angle, only.
 
                                     WARNING
 
   SIGR(MB/STR)CM  is semi-relativistic only and is intended for practical and
   not theoretical (purist) use. If you are fussy about it, you  should  check
   to see how it is computed. Also, see 1987 Handbook, SEC# 520 RUTH.
