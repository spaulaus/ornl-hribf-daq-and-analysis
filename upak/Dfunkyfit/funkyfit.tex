   02-Feb-02 ..... U620  FUNKYFIT (UNIX & VMS versions) - WTM ....... PAGE   1
 
                      LINEAR-LEAST-SQUARES FITTING PROGRAM
 
   SEC PAGE CONTENTS
 
   010   1  INTRODUCTION
 
   020   2  COMMANDS RELATED TO SYMBOLS AND LOOPS
 
   030   3  COMMANDS RELATED TO FIT SETUP AND CONTROL
 
   040   3  EXPANDED DEFINITION OF CERTAIN COMMANDS
 
   050   5  DEFINITION OF SOME OUTPUT QUANTITIES
 
   060  COMMENTS
 
   U620.010  INTRODUCTION
 
   FUNKYFIT  is  a  general purpose linear-least-squares fitting program which
   fits sets of (X,Y) points to a linear combination of  up  to  20  algebraic
   functions. Directives are used to choose one of the following forms for the
   fit.
 
                  J=N
         Y(X) = SUM   B(J)*F(J,X)
                  J=1
 
                  J=N
         Y(X) = SUM   B(J)*F(J,LOG(X))
                  J=1
 
                       J=N
         LOG(Y(X)) = SUM   B(J)*F(J,X)
                       J=1
 
                       J=N
         LOG(Y(X)) = SUM   B(J)*F(J,LOG(X))
                       J=1
 
   Where  N  is  the  no. of terms (functions of X) and F(J,X) denotes the Jth
   function of X. B(J) denotes the Jth coefficient which is determined in  the
   fit.  Experimental  and  calculated  values, coefficients, errors and CHISQ
   values are listed. Although most of the  output  from  FUNKYFIT  should  be
   self-explanatory, some quantities are defined in the last section.
 
   ---------------------------------------------------------------------------
   Type:  funkyfit            ;To start if funkyfit is defined in your
                              ;.login, .cshrc or login.com files, otherwise:
 
   Type:  /usr/hhirf/funkyfit ;To start on HRIBF DECstation or Alpha platforms
   or
   Type:  /home/upak/funkyfit ;To start execution on a SPARCstation
   ---------------------------------------------------------------------------
    
   02-Feb-02 ..... U620  FUNKYFIT (UNIX & VMS versions) - WTM ....... PAGE   2
 
 
 
   U620.020  COMMANDS RELATED TO FILENAME VARIABLES, SYMBOLS & LOOPS
 
   Filname Variables
 
   One  symbol  (integer  variable)  may  be  incorporated   in   a   FILENAME
   specification as the following examples illustrate:
 
   ...........................................................................
   SYM=3
   IN FIL"SYM".DAT   ;Opens FIL3.DAT for input
   ...........................................................................
   I=0
   LOOP 3
   I=I+1
   IN  FIL"I".DAT   ;Opens (in succession)  FIL1.DAT, FIL2.DAT, FIL3.DAT
   .
   .
   ENDLOOP
 
   LOOP execution and symbol definition
 
 
   Commands related to LOOP execution and SYMBOL definition
 
   SYM = EXPRESSION - Define symbol (SYM) up to 100 symbols supported
                    - symbols M, N, O, P, Q, R, S are reserved
                    - expression syntax is same as in CHIL
                    - no imbedded blanks are allowed in expressions
                    - symbols may contain up to 4 characters (5-8 ignored)
 
   DSYM             - Displays list of currently defined sumbols & values
 
   LOOP N           - Starts LOOP (executed N-times) N=SYM or CONST
   CMD  ....        - Nesting supported
   CMD  ....        - # lines between 1st LOOP & matching ENDL = 100
   ENDL             - Defines end-of-loop
                    - KILL (entered before ENDL) kills LOOP
                    - Ctrl/C - aborts loop-in-progress
                    - opening of CMD-file within a LOOP not allowed
    
   02-Feb-02 ..... U620  FUNKYFIT (UNIX & VMS versions) - WTM ....... PAGE   3
 
 
   U620.030  COMMANDS RELATED TO FIT SETUP AND CONTROL
 
   CMD  Data-list   Definition or Action
 
   LINX             Set to use FUNCS of X - F(X)      DFLT
   LOGX             Set to use FUNCS of Log(X) - F(Log(X))
   LINY             Set to fit Y to F(X) or F(LOG(X)) DFLT
   LOGY             Set to fit Log(Y) to F(X) or F(Log(X))
 
   CSUN             Set UNCERT to counting statistics
   ABUN             Says DATA-SET UNCERTS are absolute
   PCUN             Says DATA-SET UNCERTS are in %
   ALUN UVAL        Says set all UNCERTS to UVAL (in %)
   USUN UVAL        Says set unspecified UNCERTS to UVAL(%)
   MULU FAC         Says mult all given UNCERT by FAC
 
   XLIM XMIN,XMAX   Set fit-limits in terms of X-values
   ILIM IMIN,IMAX   Set fit-limits in terms of index-I
 
   NUFU             Says new Function (i.e sets #TERMS=0)
   XPOW E1,E2..     Exponent list for Pwr-series terms
   LPOL J1,J2..     Ordinal-list for Legendre poly terms
 
   TABL XLO,XHI,DX  Specs for computed Y vs X table
   HFMT,DFMT        Table header & data formats (see SEC#040)
 
   *                A * in col-1 of a command line results in the full
                    line being copied to log-file as a comment
 
   DATA             Introduces Y vs X DATA-SET
   X  Y <YERR>      Up to 500 DATA-SET entries
   ENDA             Ends DATA-SET
 
   FIT  <ID>        Fits set# ID (ID omitted says in core)
   FITU <ID>        Same as above but UNWEIGHTED
 
   CMD  filnam      Open and process commands from filnam.cmd
   CMD  filnam.ext  Open and process commands from filnam.ext
   IN   filename    Open new data file (full filename required)
 
   H                Displays directory to on-line help
 
   END              Ends program
 
   U620.040  EXPANDED DEFINITION OF CERTAIN COMMANDS
 
   TABL...XMIN,  XMAX,  DELX  (range  and  step-size  for  table  of values of
          calculated Y vs X). TABL without subsequent list gets rid  of  table
          request.
 
   XLIM...XMIN,  XMAX  (range  of  fit limits on X - default is no limit) XLIM
          without subsequent list gets rid of limit.
 
                            (continued on next page)
    
   02-Feb-02 ..... U620  FUNKYFIT (UNIX & VMS versions) - WTM ....... PAGE   4
 
 
   U620.040  EXPANDED DEFINITION OF CERTAIN COMMANDS (continued)
 
   ILIM...ILO,IHI (range of fit limits on data point ordinals - default is  no
          limit). ILIM without subsequent list gets rid of limit.
 
   ALUN...Value(%) of uncertainty to be assigned to all data points.
 
   USUN...Value(%)  of  uncertainty  to  be  assigned to all data points whose
          uncertainty is not specified in the data set.
 
   MULU...Number by which all given uncertainties are to be  multiplied  prior
          to  computing  weights  (default  is  1.0).  Does  not effect values
          specified by ALUN and USUN.
 
   XPOW...List of powers-of-X to be included as terms in the  "fit  function".
          For  example;  XPOW=0,1,2 says include the terms X**0, X**1 and X**2
          (i.e. 1.0, X and X*X).
 
   LPOL...List of legendre polynomial "ordinals". For example;  LPOL=2,4  says
          include the terms P2(X) and P4(X).
 
   HFMT...Heading  FORMAT for first line of table of calculated values of Y vs
          X. This line contains 10 entries (0.0, DELX, 2*DELX - - -9*DELX).
 
   DFMT...Data FORMAT for subsequent lines of table of calculated values of  Y
          vs  X.  These  lines  contain  11  entries (X-value corresponding to
          first Y-value and 10 Y-values).
 
          Note- HFMT and DFMT are not free-form  entries.  These  labels  must
          start  in column 1 and the actual format must follow immediately. No
          other information can be included on this line. Examples  (also  the
          defaults) are;
 
   HFMT(1H ,10X,10F10.3/)
 
   DFMT(1H ,11F10.3)
 
   DATA...Introduces  a  list  of  (X,Y,U) or (X,Y) entries (i.e. a data set).
          Here U denotes the uncertainty in Y. The first data point is on  the
          line  following the DATA label. Data points are entered one-per-line
          and up to 100 points are allowed. The data set  is  terminated  with
          an  ENDA  directive  which must start in column 1. The ENDA line can
          contain no other information.
 
   FIT....IDN - says retrieve data set number IDN from  the  input  disk  file
          and  carry out a fit as previously specified. If IDN is not entered,
          it is assumed that the data set is already in memory  (either  typed
          in  or  previously  retrieved  from  disk).  Only one data set is in
          memory at any given time.
 
   FITU...has the same meaning as  FIT  except  that  the  fit  is  unweighted
          (actually all points are given equal weight - namely, unity).
    
   02-Feb-02 ..... U620  FUNKYFIT (UNIX & VMS versions) - WTM ....... PAGE   5
 
 
 
   U620.050  DEFINITION OF SOME OUTPUT QUANTITIES
 
   FIT-ERR(%)  gives  the calculated uncertainty in the associated coefficient
          (determined in the fit) based on the quality of fit  (QFN)  and  the
          scatter of the data points about the calculated values.
 
   EST-ERR(%)  gives  the calculated uncertainty in the associated coefficient
          based on the quality of fit and the  scatter  of  experimental  data
          that  is  to  be expected based on the uncertainties assigned to the
          data.
 
   U620.060  COMMENTS
 
   (1)....Unless you have a very  small  number  of  data  points  to  fit,  I
          suggest  that you create a file (fil.dat for example) which contains
          the data sets to be fitted. After the program is started,  open  the
          file with the command - IN fil.dat
 
   (2)....Note  that  the data file does not contain actual ID-number entries.
          The IDN referred to in the FIT command is just  the  ordinal  number
          of  the  data set in the file. This may not be the best way to do it
          but that's the way it is.
 
   (3)....If your fitting procedure will  involve  numerous  commands,  it  is
          probably  best to create a command file (fil.cmd for example) rather
          than entering them at run-time. If you have both  a  fil.dat  and  a
          fil.cmd, then all you have to do is:
 
   Type:  FUNKYFIT
   Type:  IN   fil.dat
   Type:  CMD  fil.cmd
 
   And that's it.
