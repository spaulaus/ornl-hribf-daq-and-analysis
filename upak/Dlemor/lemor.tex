   24-May-05 ....... U310  LEMOR - Unix/Linux Version - WTM ......... PAGE   1
 
 
               LEMOR (List-tape Examine, Modify, Output - Revised)
 
   Sec Page Contents
 
   010   2  Introduction - General
   012   3  Introduction - Definition of L001 List-Data Format
   014   4  Introduction - Definition of L002 List-Data Format
   016   5  Introduction - Definition of L003 List-Data Format
   018   6  Introduction - Definition of List Data Files (LDFs)
 
   020   7  Commands General
   030   7  Command  File Operations
 
   040   7  Commands to  Assign Tapes and Files
   050   8  Commands for Tape & File Control
   060   8  Commands to  Examine a Tape or LDF
 
   070   9  Commands to  Display/Log Special Records from a Tape or LDF
 
   080  10  Commands for Simple  Copy from (Tape of LDF) to (Tape or LDF)
   090  11  Commands for Modify--Copy from (Tape or LDF) to (Tape or LDF)
 
   100  12  Commands to  Copy Simulations from Disk to (Tape or LDF)
   110  13  Commands to  Copy PAC-files from (Tape or LDF) to Disk
   120  13  Commands to  Copy ASCII Files (Tape-to-Disk or Disk-to-Tape)
 
   130  14  Discussion of Certain Commands
   140  15  Discussion of Byte-Swapping
   150  16  Discussion of Tape Conversion
   160  17  Discussion of Prescan (Modify-Copy)
   170  19  Discussion of Banana Gating (Free-Form-Gating)
 
   180  20  How to Create a Customized Prescan Program
 
   190  21  Lemor Routines Which aid in Customization
   200  21  User  Routines Which aid in Customization
 
   210  22  User  Routine -  USERCMP  - User Command Processor
   220  22  User  Routine -  USERMOC  - User Processing of Raw Data
   230  23  User  Routine -  USERMOC  - Example with Banana Gating
 
   240  24  Lemor Routines which Unpack Events
   250  26  Lemor Routines which Output Events
   260  27  Lemor Routine  for Testing  1-D Gates
   270  27  Lemor Routines for Testing  2-D Gates (Bananas)
 
   300  28  Logical Units and COMMON Blocks
 
   310  29  Processing of User Defined Files - UDFs - Linux Version Only!
 
   ===========================================================================
   Type:  lemor              ;To start with /usr/hhirf/ defined in your .login
                             ;Otherwise:
 
   Type:  /usr/hhirf/lemor   ;To start on HHIRF DECstation or Alpha
   ===========================================================================
    
   24-May-05 ....... U310  LEMOR - Unix/Linux Version - WTM ......... PAGE   2
 
 
 
   U310.010  Introduction - General
 
   lemor  is  a  tape  processing  utility  which  can be customized (via user
   supplied routines) to produce a  prescan  program.  The  stock  version  of
   lemor can do a number of useful things: Some of these are:
 
                                                                      Refer to
                                                                      --------
   o....L001, L002 & L003 formatted data tapes may be copied to        SEC#018
        list-data-files on disk (called  LDFs) for subsequent          SEC#080
        processing with scanur histogramming programs.
 
   o....Any  tape  or LDF containing records no longer than 32768      SEC#060
        bytes, may be examined (records read and displayed in various
        formats).
 
   o....Any tape or LDF may be copied to another tape or LDF           SEC#080
 
   o....Input data  records may be byte-swapped for compatibility      SEC#140
        with other platforms.
 
   o....The text records from L002 or L003 formatted tapes or LDFs     SEC#110
        (which normally contain the Data Acquisition PAC program) can
        be restored to disk.
 
   o....In the modify-copy mode, the stock lemor can copy (tapes       SEC#150
        or LDFs) in data formats (L001, L002, L003) to another tape
        or LDF in the same or different format.
 
   o....The  usual  tape  control  functions  (forward  and  backward  SEC#050
        spacing of records and files, rewind etc.) are provided.
 
   o....Simulated list-data files may be copied from disk to tape.     SEC#100
 
   ===========================================================================
   See SEC#220 & 230 for discussion of the user-supplied customizing routine
   USERMOC
   ===========================================================================
    
   24-May-05 ....... U310  LEMOR - Unix/Linux Version - WTM ......... PAGE   3
 
 
   U310.012  Introduction - Definition of L001 List-Data Format
 
   L001 formatted list-data (often generated in external labs) consists of:
 
   o......16-bit data words.
 
   o......Fixed  length  data records which are no shorter than 2048 bytes and
          no longer than 32768 bytes.
 
   o......A fixed number of parameters per event.
 
   o......A fixed (integer) number of events per record  (no  event  splitting
          across record boundries).
 
   o......A  number  of data-record header-words may be specified which are to
          be ignored during processing.
 
   o......Header records of 256 bytes in length and structured as  defined  in
          SEC#014 may be utilized for display, searches, etc.
 
   o......All  other  records  not of the specified data-record length will be
          ignored during processing.
    
   24-May-05 ....... U310  LEMOR - Unix/Linux Version - WTM ......... PAGE   4
 
   U310.014  Introduction - Definition of L002 List-Data Format
 
   The L002 data stream (from the old Concurrent system)  consists  of  Header
   records,  Event  Handler Source Code (EVS) records, Data records and Scaler
   records - as illustrated below:
 
   Header    record   ;256 bytes
   EVS-file  record   ;1600 bytes
    .
   Data      record   ;8192 bytes
   Data      record   ;8192 bytes
    .          .
   Scaler    record   ;6780 bytes
   End-of-File        ;File mark
   Header    record
   Data      record
    .          .
    .          .
   End-of-File        ;Double file-mark
   End-of-File        ;ends all data on tape
 
   L002 events consist of parameter ID flags followed  by  one  or  more  data
   words corresponding to sequential parameter IDs  - as illustrated below:
 
   8000-hex + ID     ;Parameter ID
   Data              ;for Parameter ID
   Data              ;for Parameter ID+1
    .
   8000-hex + JD     ;Parameter JD
   Data              ;for Parameter JD
   Data              ;for Parameter JD+1
    .
   FFFF-hex          ;End-of-event
 
   o......Events may be split across record boundries.
   o......Data words cannot have the hi-order bit set.
   o......All parameter IDs and data words are 16-bit.
 
   ===========================================================================
   Structure of the 256-byte header record
   Header Word Number
      16-BIT   32-BIT   # BYTES  CONTENTS                        TYPE
     01 - 04  01 - 02         8  'HHIRF   '                      ASCII
     05 - 08  03 - 04         8  'L002    '                      ASCII
     09 - 16  05 - 08        16  'LIST DATA       '              ASCII
     17 - 24  09 - 12        16  MO/DA/YR HR:MN                  ASCII
     25 - 64  13 - 32        80  User Title                      ASCII
     65 - 66       33         4  Header Number                   BINARY
     67 - 70  34 - 35         8  Reserved (set to 0)             BINARY
     71 - 72       36         4  # of Secondary Header Records   BINARY
                   37         4  Record Length (bytes)           BINARY
                   38         4  # Blocked Line Image Records    BINARY
                   39         4  Record Length (bytes)           BINARY
                   40         4  Parameters/Event (ref only)     BINARY
                   41         4  Data Record Length (bytes)      BINARY
              42 - 64            Reserved (set to 0)             BINARY
    
   24-May-05 ....... U310  LEMOR - Unix/Linux Version - WTM ......... PAGE   5
 
 
   U310.016  Introduction - Definition of L003 List-Data Format
 
 
   The L003 data stream (currently in use) may consist of:
 
   Header    records - 256   bytes
   PAC-file  records - 1600  bytes
   Scaler    records - 32000 bytes
   Dead-time records - 128   bytes
   Data      records - 2048 to 32768 bytes (but not 32000 bytes)
 
   A typical data stream might look like:
 
   Header    record
   PAC-file  record
   PAC-file  record
   Data      record
   Data      record
    .          .
    .          .
   Scaler    record
   Data      record
    .          .
    .          .
   Scaler    record
   Dead-time record
   End-of-File
   Header    record
   Data      record
    .          .
    .          .
   End-of-File
   End-of-File
 
 
   L003  events consist of pairs of 16-bit numbers, the first of which defines
   the parameter ID and the second the data - as shown below.
 
   8000-hex + ID     ;Parameter ID
   Data              ;Parameter data
   8000-hex + ID
   Data
   8000-hex + ID
   Data
     .
     .
   FFFF-hex          ;End-of-event word-1
   FFFF-hex          ;End-of-event word-2
 
 
   o......Events are not split across record boundries.
 
   o......Unfilled records are padded with FFFF-hex.
    
   24-May-05 ....... U310  LEMOR - Unix/Linux Version - WTM ......... PAGE   6
 
   U310.018  Introduction - Definition of List Data Files (LDFs)
 
   The list data file (LDF) ia a structured disk  file  for  containing  L001,
   L002  or  L003  data.  It is processed as a direct access file whose record
   length is 8194 32-bit words.
 
   The General Record Structure is:
 
   32-bit word  1       TYPE  - Record type (DIR, HEAD, PAC, SCAL, DEAD, DATA)
                2       NFW   - number of 32-bit words of data
                3-8194  DATA  - 8192 full-words (32768 bytes) data plus pads
 
   Directory record structure:
 
   32-bit word  1       'DIR '  - type = Directory
                2       8192    - full-word data size
                3       8194    - full-word blocksize
                4       NREC    - number of records written on file
                5               - un-used
                6       NHED    - number of header records written on file
 
                7       HED-ID  - header ID number
                8       RECN    - record number where header is written
 
                9       HED-ID  - header ID number
               10       RECN    - record number where header is written
                .
 
   Header record structure
 
   32-bit word  1       'HEAD'  - type = header
                2        64     - number of full-words of header
                3-66            - header record - 256 bytes
 
   PAC record structure
 
   32-bit word  1       'PAC '  - type = PAC
                2        NFW    - number of full-words of data
                3-NFW+2         - PAC file records (up to 409 lines)
 
   Scaler record structure
 
   32-bit word  1       'SCAL'  - type = Scaler
                2        NFW    - number of full-words of data
                3-NFW+2         - Data for up to 400 scalers
 
   Deadtime record structure
 
   32-bit word  1       'DEAD'  - type = Deadtime
                2        32     - 128 bytes of deadtime data
                3-34            - Deadtime data
 
   Data record structure
 
   32-bit word  1       'DATA'  - type = Data
                2        8192   - number of full-words of data
                3-8194          - list data plus padding
    
   24-May-05 ....... U310  LEMOR - Unix/Linux Version - WTM ......... PAGE   7
 
 
   U310.020  Commands General
 
   H             ;Displays directory to list of commands
 
   LON           ;Enable  output to log-file lemor.log (default)
   LOF           ;Disable output to log-file lemor.log
 
   STAT          ;Displays/logs open tapes, LDFs, options, etc.
 
   STAT GATE     ;Displays/logs currently defined gates
 
   STX           ;Display/log tape status (MB-used, MB-left, Errors/MB)
                 ;(for all tape units which are open)
 
   UCOM TEXT     ;Send TEXT to USERCMP
 
   END           ;Terminates program
 
 
   U310.030  Command File Operations
 
   CMD  FIL.CMD  ;Assign FIL.CMD as CMD-file (not read yet)
 
   CCMD          ;Continue reading  CMDS from CMD-file
 
   CLCM          ;Continue with last CMD from file (backspaces)
 
   CCON          ;Continue reading  CMDS from VDT (Terminal)
 
   MSG  TEXT     ;Display TEXT (44 bytes) on VDT
 
   Ctrl/C        ;Interrupts command file processing
 
 
   U310.040  Commands to Assign Tapes and Files
 
   IN   rxxx     ;Specifies tape (rmxx) for INPUT
 
   OU   ryyy     ;Specifies tape (rmyy) for OUTPUT
                 ;rxxx, ryyy Denote rmt0:,rmt1:, rst0,  etc
 
   INF  file.ldf ;Opens LDF file.ldf for input
 
   UDF  filename ;Specifies input to be a special User Defined File
                 ;(see SEC# 310) Linux Version Only!
 
   OUF  file.ldf ;Opens/creates LDF file.ldf for output
 
   ELDF          ;Erases output LDF (i.e. deletes & recreates)
 
   GAT  file.gat ;Open and read 1-D gates from file.gat
 
   GATZ          ;Reset all previously defined gates to "impossible"
 
   BAN  file.ban ;Open file.ban for banana testing
 
   BANZ          ;Zero banana archive
    
   24-May-05 ....... U310  LEMOR - Unix/Linux Version - WTM ......... PAGE   8
 
   U310.050  Commands for Tape & File Control
 
   RDI  N        ;Read    N records from INPUT
   RDO  N        ;Read    N records from OUTPUT
 
   FRI  N        ;Forward N records on   INPUT
   FRO  N        ;Forward N records on   OUTPUT
 
   BRI  N        ;Backup  N records on   INPUT
   BRO  N        ;Backup  N records on   OUTPUT
 
   FFI  N        ;Forward N files   on   INPUT
   FFO  N        ;Forward N files   on   OUTPUT
 
   BFI  N        ;Backup  N files   on   INPUT
   BFO  N        ;Backup  N files   on   OUTPUT
 
   RWI           ;Rewind                 INPUT
   RWO           ;Rewind                 OUTPUT
 
   BTI           ;Go to BOTTOM of INPUT  (to DBL EOF, Backup 1 F)
   BTO           ;Go to BOTTOM of OUTPUT (to DBL EOF, Backup 1 F)
 
   CLI           ;Close                  INPUT
   CLO           ;Close                  OUTPUT
 
   ULI           ;Unload and Close       INPUT  tape
   ULO           ;Unload and Close       OUTPUT tape
 
   U310.060  Commands to  Examine a Tape or LDF
 
   Open an input tape (or LDF) or an output tape (or LDF), read  one  or  more
   records  and  display  (or log) data from the last record read using one of
   the following commands:
 
   CLID IDH,IDL  ;Specify VME clock parameter IDs (hi & lo parts respectively)
                 ;1st VME-clock entry, VMET, of each buffer displayed via RDI
                 ;format is full decimal value displayed as xxx,xxx,xxx
                 ;Linux Version Only!
 
   CLID          ;Disables VMET search & display - Linux Version Only!
 
   RDI  N        ;Read    N records from INPUT  (displays REC#, #bytes, VMET)
   RDO  N        ;Read    N records from OUTPUT (displays REC#, #bytes, VMET)
 
   PEV  IA,IB    ;Print 16-bit word IA thru IB in EVENT   Format (integer)
   DEV  IA,IB    ;Disp  16-bit word IA thru IB in EVENT   Format (integer)
 
   PEVZ IA,IB    ;Print 16-bit word IA thru IB in EVENT   Format (hex)
   DEVZ IA,IB    ;Disp  16-bit word IA thru IB in EVENT   Format (hex)
 
   PZ   IA,IB    ;Print 16-bit word IA thru IB in HEX     Format
   DZ   IA,IB    ;Disp  16-bit word IA thru IB in HEX     Format
 
   PA   IA,IB    ;Print 16-bit word IA thru IB in ASCII   Format
   DA   IA,IB    ;Disp  16-bit word IA thru IB in ASCII   Format
 
   PI   IA,IB    ;Print 16-bit word IA thru IB in INTEGER Format
   DI   IA,IB    ;Disp  16-bit word IA thru IB in INTEGER Format
 
   PIF  IA,IB    ;Print 32-bit word IA thru IB in INTEGER Format
   DIF  IA,IB    ;Disp  32-bit word IA thru IB in INTEGER Format
    
   24-May-05 ....... U310  LEMOR - Unix/Linux Version - WTM ......... PAGE   9
 
   U310.070  Commands to  Display/Log Special Records from a Tape or LDF
 
   LISF filename ;Open output file for recording records
 
   LISI          ;Display/log HEAD, DEAD, PAC, & SCAL records from
                 ;INPUT tape or LDF
 
   LISI SCAL     ;Display/log SCAL (scaler) records only
 
   LISI TYPA TYPB .. ;Display/log record types TYPA, TYPB, etc. where,
                     ;legal types are HEAD, DEAD, PAC, SCAL
 
   LISO          ;Same function as LISI but for OUTPUT tape or LDF
   ===========================================================================
   If output file is undefined (via LISF command), the specified records
   will be recorded on lemor.log if LON.
   ===========================================================================
    
   24-May-05 ....... U310  LEMOR - Unix/Linux Version - WTM ......... PAGE  10
 
 
   U310.080  Commands for Simple  Copy from (Tape of LDF) or (LDF or tape)
 
   Tape-to-tape copies are double buffered for speed. Also, the maximum  input
   record  length is auto-detected unless you override the auto-detect feature
   via the RECI command. Output record lengths are the same as the  number  of
   bytes actually read from the input.
 
   Tape-to-disk  copies  are  tape-buffered for speed. Also, the maximum input
   record length is auto-detected unless you override the auto-detect  feature
   via  the  RECI  command.  Output record lengths are fixed at 32776 bytes (8
   bytes of ID information and 32768 bytes of data) to  enable  direct  access
   to the file.
 
   Disk-to-Tape  copies  are un-buffered. Output data record lengths are fixed
   at 32768 bytes.
 
   Disk-to-disk copies are un-buffered.
 
   ===========================================================================
   IN   rxxx     ;Specifies input  tape (rxxx denotes rmt0, rmt1, etc)
 
   INF  file.ldf ;Specifies the input LDF file.ldf
 
   UDF  filename ;Specifies input to be a special User Defined File
                 ;(see SEC# 310) Linux Version Only!
 
   OU   ryyy     ;Specifies output tape (rmyy denotes rmt0, rmt1, etc)
 
   OUF  file.ldf ;Specifies the output LDF file.ldf
 
   STX           ;Display/log tape status (MB-used, MB-left, Errors/MB)
                 ;(for all tape units which are open)
 
   ELDF          ;Erases output LDF (i.e. deletes & recreates)
 
   STAT          ;Displays/logs open tapes, LDFs, options, etc.
 
   RECI          ;Says max input record length be auto-detected (default)
   RECI RECL     ;Sets max input record length to be RECL
 
   SWAB          ;Request    byte-swap of input buffers (SEC#140)
   SWOF          ;Request no byte-swap (default)
   SHON          ;Says byte-swap headers once more than data
   SHOF          ;Says byte-swap headers & data the same way (default)
 
   COPY N        ;Copy N files   from INPUT to OUTPUT
   CREC N        ;Copy N records from INPUT to OUTPUT
   CC            ;Continue COPY - saves file- or record-count
 
   Ctrl/C        ;Interrupts copy process
 
   EOF           ;Write EOF on OUTPUT (not normally needed)
    
   24-May-05 ....... U310  LEMOR - Unix/Linux Version - WTM ......... PAGE  11
 
   U310.090  Commands for Modify--Copy from (Tape or LDF) or (LDF to tape)
 
   Tape-to-tape  modify-copies  are  double  buffered  for  speed.  Also,  the
   maximum  input  record  length  is  auto-detected  unless  you override the
   auto-detect feature via the RECI command. Output record lengths  are  32768
   bytes by default but can be set to smaller values via the RECO command.
 
   IN   rxxx       ;Specifies input  tape (rxxx denotes rmt0, rmt1, etc)
   INF  file.ldf   ;Opens LDF file.ldf for input
 
   OU   ryyy       ;Specifies output tape (rmyy denotes rmt0, rmt1, etc)
   OUF  file.ldf   ;Opens/creates LDF file.ldf for output
 
   STX             ;Display/log tape status (MB-used, MB-left, Errors/MB)
                   ;(for all tape units which are open)
 
   ELDF            ;Erases output LDF (i.e. deletes & recreates)
 
   STAT            ;Displays/logs open tapes, LDFs & record pointers
 
   FMTI L001 NS NP ;Specify non-standard           Input (see SEC#012)
   FMTI L002       ;Specify old     HHIRF standard Input  format
   FMTI L003       ;Specify current HIRBF standard Input  format (default)
 
   FMTO L001 NS NP ;Specify non-standard           Output (see SEC#012)
   FMTO L002       ;Specify old     HHIRF standard Output format
   FMTO L003       ;Specify current HIRBF standard Output format (default)
 
   RECI            ;Says max input record length be auto-detected (default)
   RECI RECL       ;Sets max input record length to be RECL
 
   RECO            ;Sets output data record length to 32768 bytes (default)
   RECO RECL       ;Sets output data record length to RECL  bytes
                   ;(Allowed range is 2048 to 32768 bytes)
 
   UPON NPRAW      ;Turn User-processing ON (see SEC#220)
                   ;NPRAW = Max # of raw parameters (for user only)
   UPON            ;Turn User-processing ON (NPRAW=0) (default)
 
   UPOF            ;Disable User-processing (enable CHIL processing)
   MILF file.mil   ;Read & process CHIL generated mil-file
 
   UCOM TEXT       ;Send TEXT to USERCMP
 
   SWAB            ;Request    byte-swap of input buffers (SEC#140)
   SWOF            ;Request no byte-swap (default)
   SHON            ;Says byte-swap headers once more than data
   SHOF            ;Says byte-swap headers & data the same way (default)
 
   INIT            ;Resets Modify-Copy Input & Output buffers
   ZBUC            ;Zero total Input & Output buffer counters
 
   MOC  N,M        ;Modify-Copy (N-files/M-recs - 1st to occur)
   MOCE N,M        ;Modify-Copy (END on request complete)
 
   Ctrl/C          ;Interrupts Modify-Copy process
    
   24-May-05 ....... U310  LEMOR - Unix/Linux Version - WTM ......... PAGE  12
 
   U310.100  Commands to  Copy Simulations from Disk to (Tape or LDF)
 
   lemor  accomodates  the  examining  and  copying  of  disk-files containing
   event-list data. This feature is intended to  be  an  aid  to  those  doing
   simulations. The following commands are available.
 
   RECI RECL     ;Sets input  data record length to be RECL
   RECI          ;Sets input  data record length to be default (32768)
 
   RECO RECL     ;Sets output data record length to RECL  bytes
   RECO          ;Sets output data record length to be default (32768)
 
   INEV filname  ;Specify input file for exam (RDI, DEV) & copy
 
   OU   ryyy     ;Specifies output tape (rmyy denotes rmt0, rmt1, etc)
   OUF  file.ldf ;Opens/creates LDF file.ldf for output
 
   STAT          ;Displays/logs open tapes, evel-files & record pointers
 
   SWAB          ;Request    byte-swap of input buffers (SEC#140)
   SWOF          ;Request no byte-swap (default)
 
   HTIT  TITLE   ;TITLE contains title for next tape header
   HNUM  HN      ;HN specifies next tape header number to use
   HOUT          ;Outputs tape header and increments HN
 
   COPY  1       ;Copies one file input-to-output
 
   Ctrl/C        ;Interrupts copy process
 
   In  addition,  an example program found in /usr/users/milner/Develx/evelx.f
   contains routines for opening an  event-file  (EVELOPEN)  and  for  writing
   data  to it (EVELOUT). The "simulator" might (or might not) wish to include
   these routines in his simulation source code. Feel free to copy evelx.f  to
   your directory for examination etc. The routines are internally documented.
   The idea is this:
 
   o......You do your simulations and write the generated events to a file.
 
   o......Use lemor to examine this file (read, display, etc)
 
   o......You  may  also  specify  a  tape header number and title, output the
          header and finally copy the entire file to tape or LDF.
 
   An typical file-to-LDF copy session might look as follows:
 
   lemor>inev  eventfile.dat           ;open event-file for input
   lemor>ouf   eventfile.ldf           ;open LDF for output
   lemor>htit  simulation-3 no-gates   ;title for header
   lemor>hnum  3                       ;next header number to use
   lemor>hout                          ;output the header
   lemor>copy  1                       ;copy 1-file input-to-output
 
   NOTE: You can also use the header setup and output feature  (hnum,  htit  &
   hout  commands)  to add headers while copying one tape to another. There is
   no provision, however, to delete or modify existing headers.
    
   24-May-05 ....... U310  LEMOR - Unix/Linux Version - WTM ......... PAGE  13
 
 
 
   U310.110  Commands to  Copy PAC-files from (Tape or LDF) to Disk
 
 
   Commands to Copy a PAC-file From Input to Disk
 
   STEX filname  ;Store text records (PAC source) on filename
 
 
 
   U310.120  Commands to  Copy ASCII Files (Tape-to-Disk or Disk-to-Tape)
 
   FCOP filename ;Copies filename to output-tape (previously opened)
                 ;Variable length records from filename are de-tabbed and
                 ;written as fixed length (80 byte) records on tape.
                 ;(DEC's rules for Fortran tabs are used in de-tabbing)
 
   TCOP filename ;Copies 1 file from input-tape to filename (created)
                 ;Fixed length (80 byte) records from tape are
                 ;written as variable length records on filename.
    
   24-May-05 ....... U310  LEMOR - Unix/Linux Version - WTM ......... PAGE  14
 
 
   U310.130  Discussion of Certain Commands
 
   FIND...N attempts to find HEADER # N by  searching  forward  on  the  INPUT
          tape  (TITLES  and  HEADER  numbers are displayed along the way): If
          found, lemor backs up one record. If   not  found,  lemor  will read
          past the first Double-EOF, and back up one File Mark.
 
   BTO....Advances  OUTPUT  past  the  first  Double-EOF and backs up one File
          Mark (i.e. positions properly  for  appending).  Header  titles  and
          numbers are displayed along the way.
 
   BTI....Does the same thing for the INPUT tape.
 
   DEV....PEV,  DA,  PA,  DZ,  PZ,  DI,  PI,  DIF & PIF are all commands which
          display on the terminal or list on the printer some portion  of  the
          last  record  read  from tape (either the INPUT or OUTPUT). When you
          do a read (RIN or ROU), lemor tells you how  many  bytes  were  read
          but  you  must  specify the portion of the buffer to be displayed in
          half-words (16-bit words).
 
          In EVENT FORMAT (DEV or PEV), lemor looks  for  a  hex  FFFF  before
          starting  to  accumulate the first EVENT to be displayed. Therefore,
          the first EVENT in any record is usually not  displayed  by  DEV  or
          PEV (you can see it via DZ or PZ, however).
 
   COPY...N  says  COPY  N-files from INPUT to OUTPUT. All that is required is
          that records be no longer than 32768 bytes.  File-marks  are  copied
          and  the  OUTPUT tape is always positioned between Double File-Marks
          on  normal  completion  of  a  COPY  request.  The  OUTPUT  tape  is
          positioned  ahead  of  Double File-Marks on abnormal completion (via
          SEND STOP, input  error,  etc)  of  a  COPY  request.  The  COPY  is
          terminated  (normally)  if  a  Double-EOF  or  an  End-Of-Medium  is
          encountered on the INPUT.
 
   CREC...N says copy N-records from INPUT to OUTPUT. The same rules apply  as
          for  COPY  except  that  the  OUTPUT  tape is always left positioned
          ahead of a Double File-Mark.
 
   CC.....Says continue previous COPY or CREC. It remembers how many files  or
          records have already been copied and continues the count.
    
   24-May-05 ....... U310  LEMOR - Unix/Linux Version - WTM ......... PAGE  15
 
 
   U310.140  Discussion of Byte-Swapping
 
   The command SWAB causes the following byte-swapping actions:
 
   o......For  L002  or  L003  input,  swaps  bytes  for  all  input   records
          (appropriately) in both COPY and MOC modes.
 
   o......For  L001  input,  swaps bytes of all input records (assuming 16-bit
          integers) in the COPY mode.
 
   o......For L001 input, swaps bytes of data records only in MOC mode.
 
   Under some circumstances, you may need to byte-swap the data  and  not  the
   headers  or  vice  versa.  In  such  cases,  the  command  SHON turns on an
   additional byte-swap for the header only. SHOF turns  it  off  and  is  the
   default.
    
   24-May-05 ....... U310  LEMOR - Unix/Linux Version - WTM ......... PAGE  16
 
 
   U310.150  Discussion of Tape Conversion
 
   The  stock  lemor may be used to convert certain non-standard tapes to L003
   format under the Modify-Copy (MOC) mode if  the  following  conditions  are
   met:
 
   o......The  length  of  event-by-event data records to be processed must be
          different from all other records which will be encountered.
 
   o......All data records must be written in 16-bit mode.
 
   o......If the high-order bit of a data word in the input stream is  set  it
          will be masked off (lost) in the converted data stream.
 
   o......Each  data  record  must contain a fixed (integer) number of events.
          i.e. events may not be split across record boundries.
 
   o......Each event must contain a fixed number of parameters.
 
   o......Bytes may be swapped if requested.
 
   o......A specified number of data words (record header words  etc)  may  be
          skipped at the beginning of each data record.
 
   Execute the following commands:
 
   lemor             ;Starts program lemor
 
   IN   rmtx         ;Specify input  tape rmtx
 
   ou   rmty         ;Specify output tape rmty  -  or
 
   ouf  file.ldf     ;Specify output LDF (file.ldf)
 
   L001 ns np        ;L001 says that input meets the requirements above
                     ;ns = # of header words to skip before reading events
                     ;np = # of parameters per event
 
   swab              ;Only if bytes are to be swapped!!
 
   reco nbyts        ;Specifies output record-length = nbyts bytes
                     ;Default   output record-length = 32768 bytes
 
   upon              ;Enable "User Processing" (the default)
 
   moc nfiles        ;Says process nfiles files from input
    
   24-May-05 ....... U310  LEMOR - Unix/Linux Version - WTM ......... PAGE  17
 
 
 
   U310.160  Discussion of Prescan (Modify-Copy)
 
   Pre-Scanning  (as  interpreted  here)  involves  the processing of an INPUT
   data stream (from Mag Tape) to produce an OUTPUT data stream  (to  Tape  or
   LDF).  The processing may include selection of certain events, selection of
   certain parameters, modification of input parameters  or  creation  of  new
   parameters or any combination of the preceding.
 
   A  prescan task involves the use of lemor (or a customized version thereof)
   to control the process combined with user-supplied routines  which  aid  in
   the  selection  of  events  and/or  parameters. Modification or creation of
   parameters will normally require  one  or  more  user-supplied  subroutines
   USERSUBS.
 
   In  the  MODIFY-COPY mode, records are read from the input tape, events are
   expanded and passed (one at a time) to the USERSUBS.
 
   Input records which are not of the length specified by the command reci  or
   auto-detected  are  taken  to  be non-data records and are simply copied to
   the output tape or LDF. Thus, Headers  are  copied  automatically.  If  the
   record  length  and/or  the Max # of Parms for the output tape is specified
   differently from the input, the primary Header is modified appropriately.
 
   If an End-of-Medium is encountered on an output tape and the  recording  is
   continued  on  a  new  tape,  only  the primary header (not the text blocks
   containing the PAC program etc) will  be  reproduced  on  the  continuation
   tape.
 
                   End-of-File or End-of-Medium on Input Tape
 
   When  an  EOF  or  EOM  is  encountered on the input tape, lemor writes any
   partially filled output buffer and two File Marks  onto  all  output  tapes
   and  backs up one File Mark. To continue with the next input file, type the
   command MOC. To continue with the next input tape,  type  RWI,  mount  next
   input tape and type MOC.
 
                          End-of-Medium on Output Tape
 
   When  an  EOM is encountered on the output tape, lemor backs up one record,
   writes a File Mark, rewinds the output, instructs you to mount a new  tape,
   and  pauses.  When you type: CONTINUE, lemor writes the primary header (not
   the text blocks) and the last output record (the one it was writing when it
   hit EOM) onto the new tape, and continues processing.
 
   See table on subsequent page for more information on termination of processing.
 
 
                            (continued on next page)
    
   24-May-05 ....... U310  LEMOR - Unix/Linux Version - WTM ......... PAGE  18
 
 
 
   U310.160  Discussion of Prescan (Modify-Copy) (continued)
 
   When running in the modify-copy (MOC) mode, one or more  of  the  following
   operations  may  be  initiated  which can result in the modification of the
   data stream. Operations are listed in the order that they will  be  carried
   out  if  requested.  The column headed CMD gives the run-time lemor command
   which requests the associated operation.
 
   CMD    OPERATION-----------------------------------------------------------
 
   SWAB...Swap bytes in data buffers (also HHIRF headers) (see SEC#140).
 
   L001...Convert data buffers from L001 to L003 format (see SEC#???).
 
   UPON...Process buffers via user-supplied routine USERMOC (see SEC#220).
 
   INIT...Resets both the EVENT- and OUTPUT-buffers. Do this if you have  been
          doing  some  tests but are now ready to prescan for real or any time
          you wish to make a clean start.
 
   MOC....N,M Starts the MODIFY-COPY process, where: N is the number of  files
          to  process  and  M  is  the  number  of  records  to  process.  The
          processing terminates on either N or M - the first to be  satisfied.
          The default values of N and M are 1 and 100,000,000 respectvely.
 
   MOCE...N,M  Differs  from the MOC-command described above only in that upon
          completion of the request, both INPUT and OUTPUT tapes are  unloaded
          and the program is terminated.
 
          The  following  table  summarizes  the  different  ways   in   which
          processing  may  be  terminated  and  the  state of the OUTPUT-tape,
          OUTPUT-buffer and EVENT-buffer for each:
 
          ====================================================================
          TERMINATION BY---   RESULTS IN -------------------------------------
 
          Requested # Files   Flush OUT-BUF, 2-EOF, 1-BKFIL, Scrub-EV
          EOM on INPUT        Flush OUT-BUF, 2-EOF, 1-BKFIL, Scrub-EV
          Requested # Recs    Flush OUT-BUF, 2-EOF, 2-BKFIL,  Save-EV
          Ctrl/C                             2-EOF, 2-BKFIL,  Save-EV
          Input Error         Flush OUT-BUF, 2-EOF, 2-BKFIL, Scrub-EV
          MOCE                Flush OUT-BUF, 2-EOF, Unload Tapes, EXIT
          ====================================================================
 
          Where, Scrub-EV indicates that any partial-event which has not  been
          fully  processed  will  be deleted from the EVENT-buffer and Save-EV
          means that any partial-event will be  retained  for  subsequent  MOC
          requests.
 
          NOTE:  Processing  is  always  terminated  with two File-Marks being
          written on all OUTPUT tapes.
 
          ====================================================================
    
   24-May-05 ....... U310  LEMOR - Unix/Linux Version - WTM ......... PAGE  19
 
   U310.170  Discussion of Banana Gating (Free-Form-Gating)
 
   2-D free-form gate testing  must  be  carried  out  by  user-supplied  code
   executed  by  routine  USERMOC.  There are some intrinsic aids for 2-D gate
   testing and ban-file processing. This is how it goes:
 
   Reading in ban-files-------------------------------------------------------
 
   One or more ban-files are opened and read in via the commands:
 
   ban file1.ban
   ban file2.ban
        .
 
   All entries (bananas) in each file are read in and stored in memory in  the
   order  in which they occur in the file. Subsequent references to individual
   bananas (stored in memory)  may  be  via  the  stacking  ordinal  (sequence
   number  as  read  in) or via the ban-ID from the ban-file. If more than one
   file is read in, one must make sure that all IDs are  unique  or  one  must
   reference  bananas by sequence number rather than ID-number. Non-unique IDs
   will generate a warning at read-in time but it is not a fatal error.  Also,
   the  ID numbers of bananas to be referenced by ID number must be limited to
   the range of 1 to 8000. Any out-of-range IDs will  generate  a  warning  at
   read-in time but, again, the error is not fatal.
 
   Zeroing the in-memory bananas ---------------------------------------------
 
   You  may use the command banz to clear all in-memory bananas. Subsequently,
   a new set may be read in as described above.
 
   Functions which test 2-D Gates --------------------------------------------
 
   After the appropriate ban-files have been read in,  functions  BANTESTI  or
   BANTESTN  may  be  used  to  test  X  and  Y  parameters against individual
   bananas. (see SEC#270).
 
   The following list summarizes the features and limitations of this type  of
   free-form gating support.
 
   o......Multiple  ban-files  may  be  read  in. All bananas in each file are
          stored in memory.
 
   o......A ban-ID directory is built as files are read in. If more  than  one
          banana  has  the  same ID, only the last one read will be entered in
          the directory. A warning will be displayed at read-in time  but  the
          error  is  not  fatal  (you  can always reference via sequence # via
          routine BANTESTN).
 
   o......Up to 3000 bananas may be stored by the standard support routines.
 
   o......Up to 8000 banana IDs can be accomodated by the  standard  routines.
          ID  numbers  must  be  in  he range of 1 through 8000 in order to be
          entered into the ID-directory. If  any  IDs  are  out  of  range,  a
          warning  will  be  given  at read-in time but the error is not fatal
          (you can always reference via sequence# via routine BANTESTN).
 
   o......Up to 1024000 banana channels  (2000  512-channel  bananas)  may  be
          stored in memory.
    
   24-May-05 ....... U310  LEMOR - Unix/Linux Version - WTM ......... PAGE  20
 
 
 
   U310.180  How to Create a Customized Prescan Program
 
 
   For Alphas Workstations
 
   Use  the  make  file /usr/hhirf/lemor.make (listed below) as a template for
   creating your own make file for generating a customized prescan program.
 
   FFLAGS= -fpe2
   #
   OBJS= /usr/hhirf/lemor.o
         /usr/users/milner/Dlemor/dummysubs.o
   #
   LIBS= /usr/hhirf/lemorlib.a
         /usr/hhirf/miliba.a
         /usr/hhirf/orphlib.a
   #
   LDF = -laio -lpthreads
   #
   lemor: $(OBJS) $(LIBS)
   #
           f77 -O4 $(FFLAGS) $(OBJS) $(LIBS) $(LDF) -o lemor
 
 
   Assume that your username is userdoe and your customizing routines  are  in
   your  directory Dlemosubs and there are two files to be included, subsa and
   subsb and you wish to produce a program  mylemo.  Copy  the  template  file
   into  your  directory  and  create  a make file mylemo.make by changing the
   template file as/where indicated by the bold face type. The generic  result
   is shown below.
 
   FFLAGS= -fpe2
   #
   OBJS= /usr/hhirf/lemor.o
         /usr/users/userdoe/Dlemosubs/subsa.o
         /usr/users/userdoe/Dlemosubs/subsb.o
   #
   LIBS= /usr/hhirf/lemorlib.a
         /usr/hhirf/miliba.a
         /usr/hhirf/orphlib.a
   #
   LDF = -laio -lpthreads
   #
   mylemo: $(OBJS) $(LIBS)
 
           f77 -O4 $(FFLAGS) $(OBJS) $(LIBS) $(LDF) -o mylemo
 
   ===========================================================================
   The procedures is the same for DECstations except that the LDF
   entry is not defined.
   ===========================================================================
    
   24-May-05 ....... U310  LEMOR - Unix/Linux Version - WTM ......... PAGE  21
 
 
   U310.190  Lemor Routines Which aid Customization
 
   Routine   Function or use                                               See
 
   UNPACKL   Unpacks L001, L002 or L003 buffers into single events     SEC#240
             Form is ID-list, Data-list and expanded data-array
 
   UNPACKAA  Unpacks L003 buffers into single events                   SEC#240
             Form is ID-list and associated data-list
 
   UNPACKBB  Unpacks L003 buffers into single events                   SEC#240
             Form is ID-list and associated expanded data-array
 
   GATTESTI  Tests X data against a 1-D gate identified by ID-number   SEC#260
 
   BANTESTI  Tests X,Y data against a banana identified by ID-number   SEC#270
 
   BANTESTN  Tests X,Y data against a banana identified by ordinal     SEC#270
 
   EVLISOUT  Sends single events to the output data stream             SEC#250
 
   EVEXPOUT  Sends single events to the output data stream             SEC#250
 
 
 
   U310.200  User-Supplied Routines Which aid Customization
 
   Routine   Function or use                                               See
 
   USERCMP   Receives UCOM commands for the support of other           SEC#210
             user-supplied routines
 
   USERMOC   The user-supplied routine called by lemor when operating  SEC#220
             in the modify-copy mode.                                  SEC#230
 
 
   U310.210  User Routine: USERCMP - Users Command Processor
 
   lemor  supports customization by means of a user supplied command processor
   as follows:
 
   The command:  ucom text
 
   Results in the following actions:
 
   The ucom is removed from the command line and the text field is used as  an
   argument in a call to routine USERCMP as follows:
 
   CALL USERCMP(text)
 
   The  user supplied routine USERCMP may then interpret the 80-byte character
   string text as desired and pass the results to other routines through  user
   defined COMMON or calls to other user supplied routines.
    
   24-May-05 ....... U310  LEMOR - Unix/Linux Version - WTM ......... PAGE  22
 
 
   U310.220  User Routine: USERMOC - User Processing of Raw Data
 
   lemor  provides  users with the means (via a user-supplied routine USERMOC)
   to do the following:
 
   o......Receive raw input data buffers and modify if desired.
 
   o......Call routine UNPACKL (see SEC#240) to return single events.
 
   o......Process (test, modify, etc) events and call an event output  routine
          (EVLISOUT  or  EVEXPOUT)  (see  SEC#250)  to output single events to
          tape or LDF.
 
   At run time, calls to USERMOC are enabled/disabled as follows:
 
   UPOF             ;Disable calling of USERMOC
   UPON             ;Enable  calling of USERMOC (NPRAW=0) (default)
   UPON NPRAW       ;Enable  calling of USERMOC with NPRAW defined
                    ;NPRAW = Max # raw parms (for USERMOC only,  if needed)
 
   The default USERMOC listed below does no event selection,  however  it  may
   be used to convert one of the data formats (L001, L002, L003) to another.
 
   ===========================================================================
   C$PROG USERMOC   - Default USERMOC (can convert data formats)
   C
         SUBROUTINE USERMOC(IBUF,NHW)
         IMPLICIT NONE
   C     ------------------------------------------------------------------
         INTEGER*2    IBUF(*)
         INTEGER*4    IDLST(2000),DALST(2000),EXPEV(2000)
         INTEGER*4    NHW,MXID,NPAR,IERR
         INTEGER*4    NWDS,IX,IY
         CHARACTER*4  DONE
         DATA         MXID/2000/
   C     ------------------------------------------------------------------
   C
     100 CALL UNPACKL(
        &           IBUF,   !I*2 - raw data buffer
        &           NHW,    !I*4 - # of I*2 words in IBUF
        &           MXID,   !I*4 - max-ID (dimension IDLST, DALST, EXPEV)
        &           IDLST,  !I*4 - ID-list   for returned event
        &           DALST,  !I*4 - Data-list for returned event
        &           EXPEV,  !I*4 - expanded-event array
        &           NPAR,   !I*4 - # of parameters in this event
        &           DONE,   !C*4 - YES/NO - requests new IBUF
        &           IERR)   !I*4 - 0 means OK, nonzero means error
   C
   C
         IF(DONE.EQ.'YES ') GO TO 200
         IF(IERR.NE.0)      GO TO 100
         CALL EVLISOUT(IDLST,DALST,NPAR)
         GO TO 100
     200 RETURN
         END
   ===========================================================================
    
   24-May-05 ....... U310  LEMOR - Unix/Linux Version - WTM ......... PAGE  23
 
 
 
   U310.230  User Routine USERMOC - Example with Banana Gating
 
   The following USERMOC:
 
   (1)....Receives data buffers in default (L003) or specified data format.
 
   (2)....Calls UNPACKL to unpack and return single events,
 
   (3)....Calls BANTESTI to test parameters 5 & 7 against banana 1,
 
   (4)....Calls EVLISOUT to output events into a tape or LDF.
 
   ===========================================================================
 
   C$PROG USERMOC   - Example USERMOC which tests one banana-gate
   C
         SUBROUTINE USERMOC(IBUF,NHW)
   C
         IMPLICIT NONE
   C
   C     ---------------------------------------------------------------------
         INTEGER*2    IBUF(*)
         INTEGER*4    IDlST(2000),DALST(2000),EXPEV(2000)
         INTEGER*4    NHW,MXID,NPAR,IERR
         INTEGER*4    NWDS,IG,IX,IY
         CHARACTER*4  DONE
         LOGICAL      BANTESTI
         DATA         MXID/2000/
   C     ---------------------------------------------------------------------
   C
         IG=1               !Banana ID number to test
         IX=5               !X-parameter to test
         IY=7               !Y-parameter to tset
   C
     100 CALL UNPACKL(
        &           IBUF,   !I*2 - raw data buffer
        &           NHW,    !I*4 - # of I*2 words in IBUF
        &           MXID,   !I*4 - max-ID (dimension of IDLST, DALST, EXPEV)
        &           IDLST,  !I*4 - ID-list   for returned event
        &           DALST,  !I*4 - Data-list for returned event
        &           EXPEV,  !I*4 - expanded-event array
        &           NPAR,   !I*4 - # of parameters in this event
        &           DONE,   !C*4 - YES/NO - requests new IBUF
        &           IERR)   !I*4 - 0 means OK, nonzero means error
   C
   C
         IF(DONE.EQ.'YES ')     GO TO 200     !Test for end of buffer
         IF(IERR.NE.0)          GO TO 100     !Test for error
         IF(BANTESTI(IG,                      !Test EXPEV(IX),EXPEV(IY)
        &            EXPEV(IX),               !against banana-IG
        &            EXPEV(IY)) GO TO 110     !Test for BAN-gate satisfied
         GO TO 100                            !If no,  skip it
   C
     110 CALL EVLISOUT(IDLST,DALST,NPAR)      !If yes, send to output
         GO TO 100                            !Go back for next enent
     200 RETURN                               !Return for next buffer
         END
    
   24-May-05 ....... U310  LEMOR - Unix/Linux Version - WTM ......... PAGE  24
 
 
   U310.240  Lemor Routines which Unpack Events
 
 
   UNPACKL - Unpacks L001, L002 & L003 Buffers
 
   The  data  format  is interpreted as (L001, L002 or L003) depending on that
   specified by the user at run time. The default format is  L003  since  that
   is the one we are currently using in data acquisition.
 
     100 CALL UNPACKL(
        &           IBUF,   !I*2 - raw data buffer
        &           NHW,    !I*4 - # of I*2 words in IBUF
        &           MXID,   !I*4 - max-ID (dimension of IDLST, DALST, EXPEV)
        &           IDLST,  !I*4 - ID-list   for returned event
        &           DALST,  !I*4 - Data-list for returned event
        &           EXPEV,  !I*4 - expanded-event array
        &           NPAR,   !I*4 - # of parameters in this event
        &           DONE,   !C*4 - YES/NO - requests new IBUF
        &           IERR)   !I*4 - 0 means OK, nonzero means error
 
   Important  Note:  In  interest  of speed, UNPACKL resets (to -1) only those
   elements of EXPEV which were set by the previous  call  (all  elements  are
   initially  set to -1). If the user sets other elements in this array, it is
   her/his/its responsibility to reset  these  elements  prior  to  subsequent
   calls to UNPACKL.
 
   Important  Note:  In  the  interest  of  speed, UNPACKL uses IDLST to reset
   elements in  EXPEV  which  were  set  in  the  previous  unpack  operation.
   Therefore, the user should NOT modify IDLST.
 
   Important  note:  Routine  UNPACKBB  resets  elements in the expanded array
   EVBUF to 0 (not -1).
    
   24-May-05 ....... U310  LEMOR - Unix/Linux Version - WTM ......... PAGE  25
 
 
   U310.240  Lemor Routines which Unpack Events (continued)
 
 
   UNPACKAA - Unpacks L003 Buffers to Single Events
 
   The L003 format used in the ORPHAS  data  acquisition  system  consists  of
   pairs  of  16-bit words.  The first word is the parameter ID and the second
   word is the parameter data.  The  end-of-event  is  marked  by  a  pair for
   which  both  the  ID and the data are 'FFFF'X.  An event data buffer always
   has an integral number of events.
 
   UNPACKAA unpacks single events from a buffer.  Each call  returns  a single
   event  in  the  EVBUF  array.   When  all  events  in  the buffer have been
   processed, the alternate FORTRAN return is taken.
 
   UNPACKAA returns data in a two dimensional array.  The first element is the
   parameter ID and second is the parameter data.
 
        CALL UNPACKAA(IBUF,NHW,EVBUF,NPARAM,EVSIZE,IERR,IEND)
 
   call:   INTEGER*2  IBUF       - raw event data buffer
           INTEGER*4  NHW        - number of INT*2 words in IBUF
           INTEGER*4  EVSIZE     - Second dimension of array EVDAT
 
   return: INTEGER*4  EVBUF(2,*) - Event data. EVBUF(1,*) is the parameter ID
           INTEGER*4  NPARAM     - Number of parameters in this event.
           INTEGER*4  IERR       - 0 says OK. Nonzero says too many parameters
 
           INTEGER*4  IEND       - 0 says more data - nonzero says end-of-buffer
 
   NOTE: IEND has been implemented because gnu fortran doesn't support the
   alternate RETURN used in previous versions
 
 
 
   UNPACKBB - Unpacks L003 Buffers to Single Events
 
   UNPACKBB performs the same function as UNPACKAA.  The parameter ID  is used
   as  an  index  to  the EVBUF array to get the parameter data.  The EVBUF is
   zero for parameters which were not present in the event.
 
        CALL UNPACKBB(IBUF,NHW,IDBUF,EVBUF,MAXID,NPARAM,IERR,IEND)
 
   call:   INTEGER*2  IBUF      - raw event data buffer
           INTEGER*4  NHW       - number of INT*2 words in IBUF
           INTEGER*4  MAXID     - Max ID. IDBUF and EVBUF must be dimensioned
                                  at least as great as MAXID
 
   return: INTEGER*4  IDBUF()   - List of IDs in this event
           INTEGER*4  EVBUF(id) - Event data for parameter id
           INTEGER*4  NPARAM    - Number of parameters in this event.
           INTEGER*4  IERR      - 0 says OK. Nonzero says too many parameters.
 
           INTEGER*4  IEND       - 0 says more data - nonzero says end-of-buffer
                                 - Also nonzero if L003 format error occurs
 
   NOTE: IEND has been implemented because gnu fortran doesn't support the
   alternate RETURN used in previous versions
    
   24-May-05 ....... U310  LEMOR - Unix/Linux Version - WTM ......... PAGE  26
 
 
   U310.250  Lemor Routines which Output Events
 
   Two event output routines are available in "user" mode, namely; EVLISOUT  &
   EVEXPOUT.  Both  routines can generate output streams in L001, L002 or L003
   format (whichever is specified or defaulted). The calling sequences are:
 
         CALL EVLISOUT(IDLST,  !I*4 array  - ID-list (from UNPACKL)
        &              DALST,  !I*4 array  - data associated with ID-list
        &              NPAR)   !I*4 scaler - number of IDs
 
         CALL EVEXPOUT(EXPEV,  !I*4 array  - expanded event (from UNPACKL)
        &              LO,     !I*4 array  - min parameter# to output
        &              HI)     !I*4 scaler - max parameter# to output
 
 
   The default lemor contains the following USERMOC routine and  uses  routine
   EVLISOUT to output events.
 
   C$PROG USERMOC   - Default user modify-copy routine (USERMOC)
   C
         SUBROUTINE USERMOC(IBUF,NHW)
         IMPLICIT NONE
   C     ------------------------------------------------------------------
         INTEGER*2    IBUF(*)
         INTEGER*4    IDLST(2000),DALST(2000),EXPEV(2000)
         INTEGER*4    NHW,MXID,NPAR,IERR
         INTEGER*4    NWDS,IX,IY
         CHARACTER*4  DONE
         DATA         MXID/2000/
   C     ------------------------------------------------------------------
   C
     100 CALL UNPACKL(
        &           IBUF,   !I*2 - raw data buffer
        &           NHW,    !I*4 - # of I*2 words in IBUF
        &           MXID,   !I*4 - max-ID (dimension IDLST, DALST, EXPEV)
        &           IDLST,  !I*4 - ID-list   for returned event
        &           DALST,  !I*4 - Data-list for returned event
        &           EXPEV,  !I*4 - expanded-event array
        &           NPAR,   !I*4 - # of parameters in this event
        &           DONE,   !C*4 - YES/NO - requests new IBUF
        &           IERR)   !I*4 - 0 means OK, nonzero means error
   C
   C
         IF(DONE.EQ.'YES ') GO TO 200
         IF(IERR.NE.0) GO TO 100
         CALL EVLISOUT(IDLST,DALST,NPAR)
         GO TO 100
     200 RETURN
         END
    
   24-May-05 ....... U310  LEMOR - Unix/Linux Version - WTM ......... PAGE  27
 
   U310.260  Lemor Routine for Testing 1-D Gates
 
   Simple 1-D gating is supported in user-mode as follows:
 
   o......One or more 1-D gates are specified via the syntax:
 
          GATE ID LO HI
 
          Where,  ID  is  the  gate  ID#  and LO & HI are the gate limits. The
          allowed range for ID is 1 to 1000. All quantities are INTEGER*4.
 
   o......LOGICAL FUNCTION GATTESTI(ID,IP) is used to test  the  parameter  IP
          against gate ID. (Note: gates are inclusive)
 
          GATTESTI is .TRUE.  if gate is satisfied
          GATTESTI is .FALSE. otherwise
 
   o......Gates may be redefined - no warning is given.
 
   o......There is no check on gate limits.
 
   o......Gates  must  be  on  the  same  basis (scale) as the parameter being
          tested.
 
   o......Gates may be specified interactively,  or  read  from  a  file  (see
          SEC#020 and SEC#040 for relavant commands).
 
   U310.270  Lemor Routine for Testing 2-D Gated (bananas)
 
   After  the  appropriate  ban-files  have  been  read  in,  logocal function
   BANTESTI may be used by routine USERMOC to test X and Y parameters  against
   individual bananas.
 
         LOGICAL FUNCTION BANTESTI(IG,IX,IY)
   C
         INTEGER*4 IG,IX,IY
 
   BANTESTI  references  bananas  via ID-numbers IG. If the the IX,IY point is
   contained within the  specified  banana,  the  value  of  the  function  is
   .TRUE.,  otherwise  it  .FALSE.  In  NG  or  IG  do  not  exist, .FALSE. is
   returned. There are no error messages.
 
 
   After the  appropriate  ban-files  have  been  read  in,  logical  function
   BANTESTN  may be used by routine USERMOC to test X and Y parameters against
   individual bananas.
 
         LOGICAL FUNCTION BANTESTN(NG,IX,IY)
   C
         INTEGER*4 NG,IX,IY
 
   BANTESTN references bananas via sequence  numbers  NG.  If  the  the  IX,IY
   point  is  contained within the specified banana, the value of the function
   is .TRUE., otherwise it .FALSE. In NG  or  IG  do  not  exist,  .FALSE.  is
   returned. There are no error messages.
    
   24-May-05 ....... U310  LEMOR - Unix/Linux Version - WTM ......... PAGE  28
 
 
 
   U310.300  Logical Units and COMMON Blocks
 
   lemor may use some or all of the following logical unit assignments:
   ===========================================================================
   LU#     USE
 
     5     VDT INPUT
     6     VDT OUTPUT
     7     LIST FILE/DEV
     8     HELP-FILE, MIL-FILE, TEX-FILE, ETC. CLOSED AFTER USE
 
    10     CMD-FILE
    11     INPUT    TAPE/FILE
    12     OUTPUT-1 TAPE/FILE
    13     OUTPUT-2 TAPE
    14     OUTPUT-3 TAPE
    15     INPUT    LDF-FILE
    16     OUTPUT   LDF-FILE
    17     OUTPUT   RECLOG-FILE
   ===========================================================================
 
 
 
 
 
   lemor uses the following labeled COMMON blocks:
   ===========================================================================
   COMMON/FFGA/
   COMMON/FFGB/
   COMMON/FFGC/
   COMMON/FFGD/
   COMMON/FFGE/
   COMMON/LLL/
   COMMON/LM01/ thru COMMON/LM50/
   COMMON/ML01/
   COMMON/ML02/
   ===========================================================================
 
   The  user  should  not  attempt to use any of these logical units or COMMON
   block labels.
    
   24-May-05 ..... U310  Processing UDFs - Linux Version Only! ...... PAGE  29
 
 
   U310.310  Processing of User Defined Files - UDFs - Linux Version Only!
 
   Support is provided for the processing of non-scandard  event  data  files.
   Such  files might be generated via simulations or from some other source. I
   will call such files User Defined Files or just UDFs.
 
   The commands:
 
   UDF filename      ;Opens the UDF for sequential access reading
                     ;and sets flags for UDF processing
 
   UDF filename RECL ;Opens the UDF for direct access reading with
                     ;record length = RECL
                     ;and sets flags for UDF processing
 
   A number of special routines are brought into play. Depending  on  specific
   requirements, one or more of these routines must be customized by the user.
   There are three routines that may need to be customized. These are:
 
   UDFOPEN  which  opens  the  UDF for the type of access required for reading
          (sequential, direct, etc). The default version opens the file  as  a
          formatted (ASCII) file to be read sequentially.
 
   UDFHAN  processes the positioning commands REW, FR, & BR which provides for
          REWIND, BACK-RECORD and FORWARD-RECORD operations.
 
   UDFEVENT reads the UDF and returns one  event  per  call  in  the  form  of
          INTEGER*4  ID  and  DATA arrays. These event arrays are then used to
          build standard L003 buffers. This routine will most  certainly  need
          to be customized, however, it is probably the only one which does.
 
   The archive for default routines are in /usr/hhirf/lemorlib.a
 
   The source  for default routines are in /tera/milner/DDgnu/Dlemorlib/
 
   In  order to use the UDF support, the user must understand the structure of
   the UDF and be able to:
 
   o......Modify UDFOPEN to be compatible with the reading requirements,
 
   o......Modify UDFEVENT so as to read the UDF and return one event per call.
 
 
   The template (default) UDF support routines are internally  documented  and
   listed on subsequent pages.
    
   24-May-05 ..... U310  Routine UDFOPEN - Linux Version Only! ...... PAGE  30
 
 
   C$PROG UDFOPEN   - Opens Input UDF-Files for LEMOR
   C
   C     ******************************************************************
   C     BY W.T. MILNER AT HRIBF - LAST MODIFIED 03/31/2005
   C     ******************************************************************
   C
         SUBROUTINE UDFOPEN
   C
         IMPLICIT NONE
   C
   C     ------------------------------------------------------------------
         COMMON/LLL/ MSSG(28),NAMPROG(2),LOGUT,LOGUP,LISFLG,MSGF
         INTEGER*4   MSSG,NAMPROG,LOGUT,LOGUP
         CHARACTER*4 LISFLG,MSGF
         CHARACTER*112 CMSSG
         EQUIVALENCE (CMSSG,MSSG)
   C     ------------------------------------------------------------------
         COMMON/ML02/ IWDRAW(20)
         INTEGER*4    IWDRAW
   C     ------------------------------------------------------------------
         COMMON/LM20/ LUINF,LUOUF,INFOP,OUFOP
         INTEGER*4    LUINF,LUOUF
         CHARACTER*4              INFOP,OUFOP
   C     ------------------------------------------------------------------
         COMMON/LM23/ INDIR(8192),OUDIR(8192),INTYP,OUTYP,INRECI,OURECI
         INTEGER*4    INDIR,      OUDIR,                  INRECI,OURECI
         CHARACTER*4                          INTYP,OUTYP
   C     ------------------------------------------------------------------
         COMMON/LM33/ UDFNAM(20),UDFRECL,UDFNPAR,UDFRECI
         INTEGER*4    UDFNAM,    UDFRECL,UDFNPAR,UDFRECI
   C     ------------------------------------------------------------------
         COMMON/LM34/ NCEOF,LAUTO
         INTEGER*4    NCEOF
         CHARACTER*4        LAUTO
   C     ------------------------------------------------------------------
         CHARACTER*80 CNAMF
         EQUIVALENCE (CNAMF,UDFNAM)
         INTEGER*4    RECLVALU,IOS,IERR
   C     ------------------------------------------------------------------
         SAVE
   C     ------------------------------------------------------------------
   C     Process: UDF filename <RECL>     ;i.e. get UDFNAM & UDFRECL
   C     ------------------------------------------------------------------
   C
         CALL UDFNAME(IWDRAW,UDFNAM,UDFRECL,IERR)
   C
         IF(IERR.NE.0) GO TO 1000
 
 
 
                            (continued on next page)
    
   24-May-05 ..... U310  Routine UDFOPEN - Linux Version Only! ...... PAGE  31
 
 
   C
   C     ------------------------------------------------------------------
   C     Open User-Defined-File for input
   C     ------------------------------------------------------------------
   C
         CLOSE(UNIT=LUINF)               !Close input file if open
         INTYP='    '                    !Reset input type
         NCEOF=0                         !Reset contiguous EOF counter
   C
         IF(UDFRECL.EQ.0) THEN           !Test for RECL specified
         OPEN(UNIT      = LUINF,         !Otherwise,
        &     FILE      = CNAMF,         !Open UDF for sequential access
        &     STATUS    = 'OLD',
        &     IOSTAT    = IOS)
         GO TO 100                       !Go test status
         ENDIF
   C
         OPEN(UNIT      = LUINF,         !Otherwise,
        &     FILE      = CNAMF,         !Open UDF for DIRECT access
        &     ACCESS    = 'DIRECT',
        &     RECL      = RECLVALU(UDFRECL),!with RECORD LENGTH = UDFRECL
        &     STATUS    = 'OLD',
        &     IOSTAT    = IOS)
   C
     100 IF(IOS.NE.0) THEN               !Test OPEN status for good
         CALL IOFERR(IOS)
         GO TO 1000
         ENDIF
   C
         INTYP='UDF '                    !Set input type to UDF
   C
         UDFRECI=1                       !Next rec# to be read
   C
         UDFNPAR=0
   C
         RETURN
   C
   C     ------------------------------------------------------------------
   C     Send error messages
   C     ------------------------------------------------------------------
   C
    1000 WRITE(CMSSG,1005)
    1005 FORMAT('UDFOPEN ERROR')
         CALL MESSLOG(LOGUT,LOGUP)
         RETURN
         END
    
   24-May-05 ..... U310  Routine UDFHAN - Linux Version Only! ....... PAGE  32
 
 
   C$PROG UDFHAN    - Manages UDF-file position pointers for LEMOR
   C
   C     ******************************************************************
   C     BY W.T. MILNER AT HRIBF - LAST MODIFIED 11/22/2004
   C     ******************************************************************
   C
         SUBROUTINE UDFHAN(IERR)
   C
         IMPLICIT NONE
   C
   C     ------------------------------------------------------------------
         COMMON/LLL/ MSSG(28),NAMPROG(2),LOGUT,LOGUP,LISFLG,MSGF
         INTEGER*4   MSSG,NAMPROG,LOGUT,LOGUP
         CHARACTER*4 LISFLG,MSGF
         CHARACTER*112 CMSSG
         EQUIVALENCE (CMSSG,MSSG)
   C     ------------------------------------------------------------------
         COMMON/ML01/ IWD(20),LWD(2,40),ITYP(40),NF,NTER
         INTEGER*4    IWD,    LWD,      ITYP,    NF,NTER
   C     ------------------------------------------------------------------
         COMMON/LM20/ LUINF,LUOUF,INFOP,OUFOP
         INTEGER*4    LUINF,LUOUF
         CHARACTER*4              INFOP,OUFOP
   C     ------------------------------------------------------------------
         COMMON/LM23/ INDIR(8192),OUDIR(8192),INTYP,OUTYP,INRECI,OURECI
         INTEGER*4    INDIR,      OUDIR,                  INRECI,OURECI
         CHARACTER*4                          INTYP,OUTYP
   C     ------------------------------------------------------------------
         COMMON/LM33/ UDFNAM(20),UDFRECL,UDFNPAR,UDFRECI
         INTEGER*4    UDFNAM,    UDFRECL,UDFNPAR,UDFRECI
   C     ------------------------------------------------------------------
         COMMON/LM34/ NCEOF,LAUTO
         INTEGER*4    NCEOF
         CHARACTER*4        LAUTO
   C     ------------------------------------------------------------------
         INTEGER*4    IERR,KIND,NUM,NV,N
         REAL*4       XV
         CHARACTER*4  KMD
         EQUIVALENCE (KMD,LWD(1,1))
   C     ------------------------------------------------------------------
         SAVE
   C     ------------------------------------------------------------------
   C     LUINF    =  Logical unit# for UDF-file
   C
   C     INTYP    = '    ', TAPE, FILE = input type
   C
   C     UDFRECI  =  Current rec# pointer for input file if DIRECT access
   C                 Otherwise just keeps track of position but not used
   C     ------------------------------------------------------------------
   C
         IERR=0
   C
         CALL MILV(LWD(1,2),NV,XV,KIND,IERR)
         IF(IERR.NE.0) GO TO 1000
         NUM=NV
         IF(NUM.LE.0) NUM=1
   C
         IF(INTYP.NE.'UDF ') GO TO 1000
 
 
                            (continued on next page)
    
   24-May-05 ..... U310  Routine UDFHAN - Linux Version Only! ....... PAGE  33
 
 
   C
         IF(KMD.EQ.'RWI ') GO TO 100
         IF(KMD.EQ.'BRI ') GO TO 150
         IF(KMD.EQ.'FRI ') GO TO 200
   C
         GO TO 1000
   C
   C     ------------------------------------------------------------------
   C     Process REW  -  Rewind input file
   C     ------------------------------------------------------------------
   C
     100 REWIND LUINF               !For SEQUENTIAL access only
         NCEOF=0                    !Reset contiguous EOF counter
         UDFRECI=1                  !Next REC# to be read for DIRECT access
         UDFNPAR=0
         RETURN
   C
   C     ------------------------------------------------------------------
   C     Process BR   -  Backup record/s on input file
   C     ------------------------------------------------------------------
   C
     150 DO 160 N=1,NUM
         BACKSPACE LUINF            !For SEQUENTIAL access only
         UDFRECI=UDFRECI-1
         IF(UDFRECI.LT.1) THEN
         UDFRECI=1
         UDFNPAR=0
         RETURN
         ENDIF
     160 CONTINUE
         RETURN
   C
   C     ------------------------------------------------------------------
   C     Process FR   -  Forward record/s on input file
   C     ------------------------------------------------------------------
   C
     200 DO 210 N=1,NUM
         READ(LUINF,205)KIND         !For SEQUENTIAL access only
     205 FORMAT(A4)
         UDFRECI=UDFRECI+1
         UDFNPAR=0
     210 CONTINUE
         RETURN
   C
   C     ------------------------------------------------------------------
   C     Send error messages
   C     ------------------------------------------------------------------
   C
    1000 IERR=1
         RETURN
         END
    
   24-May-05 .... U310  Routine UDFEVENT - Linux Version Only! ...... PAGE  34
 
 
   C$PROG UDFEVENT  - Reads & returns one event from example UDF-file
   C
   C     ******************************************************************
   C     BY W.T. MILNER AT HRIBF - LAST MODIFIED 11/28/2004
   C     ******************************************************************
   C
         SUBROUTINE UDFEVENT(MXID,PID,DAT,NP,IERR)
   C
         IMPLICIT NONE
   C
   C     ------------------------------------------------------------------
         COMMON/LM23/ INDIR(8192),OUDIR(8192),INTYP,OUTYP,INRECI,OURECI
         INTEGER*4    INDIR,      OUDIR,                  INRECI,OURECI
         CHARACTER*4                          INTYP,OUTYP
   C     ------------------------------------------------------------------
         COMMON/LM20/ LUINF,LUOUF,INFOP,OUFOP
         INTEGER*4    LUINF,LUOUF
         CHARACTER*4              INFOP,OUFOP
   C     ------------------------------------------------------------------
         COMMON/LM33/ UDFNAM(20),UDFRECL,UDFNPAR,UDFRECI
         INTEGER*4    UDFNAM,    UDFRECL,UDFNPAR,UDFRECI
   C     ------------------------------------------------------------------
         INTEGER*4    MXID,PID(*),DAT(*),BUF(2,8),NP,LU,IERR,I
   C
         INTEGER*4    EOFTST,IOS
   C     ------------------------------------------------------------------
         SAVE
   C     ------------------------------------------------------------------
   C
         NP=0
         IERR=0
   C
     100 UDFRECI=UDFRECI+1
   C
         IF(UDFRECL.EQ.0) THEN
         READ(LUINF,110,ERR=200,END=300)BUF      !Read ASCII input line
     110 FORMAT(16I5)
         ENDIF
   C
         IF(UDFRECL.GT.0) THEN
         READ(LUINF,REC=UDFRECI,IOSTAT=IOS)BUF   !Read binary input line
         IF(IOS.NE.0) GO TO 400
         ENDIF
   C
   C
         IF(BUF(1,1).EQ.0.AND.BUF(2,1).EQ.0) RETURN  !If 1st 2 wds 0,
   C                                                 !then its end-event
   C
     140 DO 150 I=1,8                            !Loop on buffer values
         IF(BUF(1,I).EQ.0) GO TO 150             !If parm-ID 0, skip it
         NP=NP+1                                 !Inc cntr
         PID(NP)=BUF(1,I)                        !Save parm-ID
         DAT(NP)=BUF(2,I)                        !Save data
     150 CONTINUE
         GO TO 100                               !Go read next record
 
                            (continued on next page)
    
   24-May-05 .... U310  Routine UDFEVENT - Linux Version Only! ...... PAGE  35
 
 
   C
     200 WRITE(6,205)                            !Error return
     205 FORMAT('Error reading UDF file')
         IERR=5
         RETURN
   C
     300 WRITE(6,305)                            !EOF return
     305 FORMAT('EOF reading UDF file')
         IERR=999
         RETURN
   C
     400 IF(EOFTST(IOS).NE.0) THEN
         WRITE(6,305)
         IERR=999
         RETURN
         ENDIF
   C
         WRITE(6,405)IOS
     405 FORMAT('Error reading UDF, IOS =',I5)
         IERR=5
         RETURN
         END
