   20-Aug-05 ............. U300  DAMM - GENERAL - WTM ............... PAGE   1
 
   Sec Page Contents
 
   010   2  GENERAL      - Introduction and General Features
   020   3  GENERAL      - Getting Started
   030   4  GENERAL      - Assigning Input/Output Files
   040   4  GENERAL      - Loop-Execution & Symbol-Definition
   050   5  GENERAL      - Log File -  damm.log
   060   5  GENERAL      - Comments on Hard Copy
   070   5  GENERAL      - File ID-directories and Count-Sums
   090   6  GENERAL      - Cursor Tracking Problems With Xterminals
   100   6  GENERAL      - Changes in Cursor-Mode Commands
   110   6  GENERAL      - Mouse Button Customizing
   120   7  GENERAL      - Screen Setup and Color Mapping
   130   8  GENERAL      - Display Delays (hangup problems)
   135   8  GENERAL      - Commands Which Suspend or Abort Operations
   140   8  GENERAL      - Zeroing Histograms in Shared Memory
 
   150   9  DISPLAY      - 1-D Display
   155  11  DISPLAY      - 1-D Display (Peak Finding/Logging)
   160  12  DISPLAY      - 2-D Display
   170  14  DISPLAY      - Special XY-Line Display
   175  16  DISPLAY      - Label Generation and Annotation
 
   200  18  MANIPULATION - Syntax Definitions
   210  18  MANIPULATION - Setup
   220  19  MANIPULATION - I/O     1-D Histograms
   230  19  MANIPULATION - Gating  2-D Histograms
   240  19  MANIPULATION - General 2-D Projections
   250  20  MANIPULATION - Operations on Buffers-1 & -2
   260  20  MANIPULATION - Show Data, Count-Sums etc (from Bufs-1 & -2)
   270  20  MANIPULATION - Modify Buffer Contents
   280  21  MANIPULATION - Printer Plots
   290  21  MANIPULATION - Gain-Shifts & Compressions
 
   300  22  MANIPULATION - HIS-file (Create, Expand, Copy, Add, Gain-Shift...)
   310  25  MANIPULATION - HIS-file (Adding & Subtracting Complete HIS-files)
 
   350  26  BANANAS      - Definition, Rules, Construction & Display
   360  27  BANANAS      - Projections
 
   400  28  FITTING      - Introduction & General Features
   410  29  FITTING      - 1-Key (Cursor Mode) Commands
   420  30  FITTING      - Setup Commands
   430  31  FITTING      - Display of Data, Fits & Printer-Plots
   440  32  FITTING      - FIT Execution Commands
   450  32  FITTING      - FIT Parameters - Saving, Setting, Default
   460  33  FITTING      - Commands to Control Relative Peak-intensities
   470  34  FITTING      - FIT Specification Details
   480  35  FITTING      - Peak Shapes
   490  35  FITTING      - Gradient-Search Method (FIT  request)
   500  36  FITTING      - Gaussian Method        (GFIT request)
   510  37  FITTING      - Estimated Uncertainties in Peak Areas
   520  37  FITTING      - Reading Fit Results from damm.log
   530  37  FITTING      - Common Problems
   540  38  FITTING      - Peak Shape vs Asymmetry (Log Plot)
 
   550  39  CUSTOMIZING  - Screen Configuration
   560  41  CUSTOMIZING  - Graphic Screen Color Mapping
   570  42  CUSTOMIZING  - Graphic Screen Black & White Mapping
    
   20-Aug-05 ............. U300  DAMM - GENERAL - WTM ............... PAGE   2
 
   U300.010  Introduction and General Features
 
   DAMM  is a Display, Analysis and Manipulation Module which is configured to
   be used on Alpha workstations running Digital  Unix.  Older  versions  were
   configured  for  DECstations,  SPARCstations and VAXstations. DAMM provides
   all of the features to be found in VAXPAK programs DAM, SAM,  TDX  and  XAM
   as  well  as  a  number  of  additional features. Some general features are
   listed below.
 
   Display features ..........................................................
 
   (1)....Works with Alphas, DECstations, SPARCstations & VAXstations.
 
   (2)....Displays 1-D data from HIS- or SPK-files.
 
   (3)....Displays 2-D data from HIS-files.
 
   (4)....Hardcopy available via screen-copy to networked Laser printers.
 
   (5)....A dialog record may be saved on a Log-file.
 
   (6)....Supports Free-Form (Banana) gate construction.
 
   (7)....Provides for total number of counts within a Banana.
 
   (8)....Supports X- & Y-projections of Bananas (saved on DAMQ8Q.SPK).
 
   (9)....Provides for peak sum, centroid and fwhm.
 
   (10)...Provides for spectrum analysis (fitting - see SEC# U300.400).
 
 
   General features ..........................................................
 
   (1)....Reads 1-D histograms from either HIS-, SPK-files.
 
   (2)....Extracts GATES (on parameters 1 or 2) from 2-D histograms.
 
   (3)....Supports general projections of Bananas on arbitrary axis.
 
   (4)....Forms linear combinations, gain-shifts etc. of 1-D histograms.
 
   (5)....Forms linear combinations, gain-shifts etc. of 2-D histograms.
 
   (6)....Does linear gain and intercept transformations by rebinning.
 
   (7)....Does crunches (sums a specified number of channels together).
 
   (8)....Lists and plots 1-D histograms on the line printer.
 
   (9)....Shows directory (ID'S) contained in HIS-, SPK- & BAN-files.
 
   (10)...Shows count-sums for all ID's in SPK- or HIS-files.
 
 
                            (continued on next page)
    
   20-Aug-05 ............. U300  DAMM - GENERAL - WTM ............... PAGE   3
 
   Program operation .........................................................
 
   The program is controlled by a set of commands (alphabetic directives)  and
   associated  data-lists  (numbers):  I  call  these  command-lists. Input is
   free-form. Command and list-element delimiters are BLANK , ( ) /
 
   U300.020  GETTING STARTED
 
   The steps given below outline how I would do it. Of course, you can  do  it
   any way that you choose or not at all.
 
   (1)....Log onto the workstayion, Xterminal, etc. in the usual manner.
 
   (2)....Open a window and move it to the lower left corner of screen.
 
 
   (3)....Type:  damm                ;to start program on any host
                                     ;where the path to the executable
                                     ;has been defined in your
                                     ;.cshrc, .login or login.com files
                                     ;otherwise,
 
          Type:  /usr/hhirf/damm     ;to start on a HHIRF DECstation
 
          Type   /home/upak/damm     ;to start on a HHIRF SPARCstation
 
   (4)....Type:  H                   ;for HELP directory
   (5)....Type:  H  ITEM             ;for help on directory ITEM
   (6)....Type:  H  FIG              ;for screen configurations
 
   (7)....Try a few FIG commands to get a feel for how they work.
 
   (8)....Note  the  fact  that  the  display  required  for  entering fitting
          information (like peak positions, etc) is via the DS & DSX  commands
          rather than the general display commands D & DX.
 
   (9)....Learn  to  use  the  HELP  facility.  That  will  usually  be   more
          up-to-date than this document.
 
   (10)...Filenames  have been made case-sensitive for the UNIX version. Where
          default extensions apply, upper case is assumed for  VMS  and  lower
          case  is  assumed  for  UNIX.  Acceptable  standard  extensions  now
          include .spk, .SPK, .his, .HIS, .ban, .BAN, .cmd &  .CMD.  Note:  If
          the  his-file  extension  is  lower/upper  case  then  the  drr-file
          extension must be lower/upper case. Also note:
 
   /usr/users/directory/subdirectory/filename   ;is an acceptable form but
 
           ../directory/subdirectory/filename   ;is NOT! (at least for now)
    
   20-Aug-05 ............. U300  DAMM - GENERAL - WTM ............... PAGE   4
 
   U300.030  Commands for Assigning Input/Output Files
 
   IN   FIL.EXT    - Open N-file (EXT = SPK or HIS)
   OU   FIL.EXT    - Open O-file (EXT = SPK or HIS) - OUTPUT for SPK only
   OU   FIL.SPK,NEW- Create and open O-file (SPK-file for output)
   QF   FIL.EXT    - Open Q-file (EXT = SPK or HIS) - for display only
   RF   FIL.EXT    - Open R-file (EXT = SPK or HIS) - for display only
   SF   FIL.EXT    - Open S-file (EXT = SPK or HIS) - for display only
   BAN  FIL        - Open FIL.BAN for store, recall, proj, etc
   BAN  FIL,NEW    - Create & open FIL.BAN for store, recall, etc
                   - (See below for how to specify variable FILENAMES)
   CLO  F          - Closes F-file (where F = N, O, P, Q, R, S or BAN)
   DFIL            - Displays data files currently open
 
   CMD  FIL        - Open and process commands from FIL.CMD
   CMD  FIL.EXT    - Open and process commands from FIL.EXT
 
   LON/LOF         - Turn Log-output (to LU7) ON/OFF (default = OFF)
 
   Explanation of variables in FILENAMES .....................................
 
   One   symbol   (integer   variable)  may  be  incorporated  in  a  FILENAME
   specification as the following examples illustrate:
 
   Example-1 .................................................................
   SYM=3
   OU FIL"SYM".SPK  ;Opens FIL3.SPK
 
   Example-2 .................................................................
   I=0
   LOOP 3
   I=I+1
   IN FIL"I".SPK    ;Opens (in succession) FIL1.SPK, FIL2.SPK, FIL3.SPK
   .
   ENDLOOP
 
   U300.040  Commands Related to Loop-Execution & Symbol-Definition
 
   SYM = EXPRESSION - Define symbol (SYM) up to 100 symbols supported
                    - symbols: UIND CIND ULOC CLOC FIX NONE FITS ALL and
                    - COLR GREY DOTS LIVE BAN  M N O P Q R S X are reserved
                    - expression syntax is same as in CHIL
                    - no imbedded blanks are allowed in expressions
                    - symbols may contain up to 8 characters
 
   DSYM             - Displays currently defined sumbols & values
   LSYM             - Logs currently defined symbols & values on damm.log
   ZSYM             - Resets symbol table (deletes all symbols)
 
   LOOP N           - Starts LOOP (executed N-times) N=SYM or CONST
   CMD  ....        - Nesting supported
   CMD  ....        - # lines between 1st LOOP & matching ENDL = 100
   ENDL             - Defines end-of-loop
                    - KILL (entered before END) kills LOOP
                    - Ctrl/C    - aborts loop-in-progress
                    - opening of CMD-file within a LOOP not allowed
 
                            (continued on next page)
    
   20-Aug-05 ............. U300  DAMM - GENERAL - WTM ............... PAGE   5
 
   LOOP suspension - the WO command ------------------------------------------
 
   A command WO [means the same thing as WOA or WHOA -  i.e  the  opposite  of
   GIDDUP (my preferred spellings)] has been implemented to work within LOOPs.
   Whenever the WO command is encountered, the message:
 
   Type [RETURN] to CONTINUE--->
 
   will  appear  on  the  screen. This gives you an opportunity to look at the
   display etc. before it gets wiped  out.  When  you  are  finished  looking,
   press the [RETURN] key and it will continue.
 
   U300.050  Log File - damm.log
 
   The  VMS version of DAMM always creates a new version of DAMM.LOG while the
   UNIX version creates a new  damm.log  or  appends  to  an  old  version  of
   damm.log  if  it exists. If you enter the command LON, almost all dialog to
   and from the host will be logged, otherwise, only certain "print  commands"
   will  produce output to the log-file or device. You may turn the log ON/OFF
   by entering LON/LOF.
 
   U300.060  Comments on Hard Copy
 
   As I have defined the default the color mapping, it is best to set  up  the
   Workstation to Print Screen in the negative image mode. If you are printing
   on  something  like  a LN03 ScriptPrinter, anything you chose to print will
   be scaled to fit on one page. If you are using an LN03, it may come out  on
   multiple  pages  (and  you  may  miss some) unless you choose a "Portion of
   Screen" that it likes.
 
   U300.070  File ID-directories and Count-Sums
 
   DIR  KF   Displays a list of all ID's in file-KF, where
             KF   left blank says input-file
             KF = N     denotes   input-file
             KF = O     denotes  output-file
             KF = Q     denotes       Q-file
             KF = R     denotes       R-file
             KF = S     denotes       S-file
             KF = BAN   denotes     BAN-file
 
   LDIR KF   Logs a list of all ID's in file-KF on DAMM.LOG
 
   DDIR KF   Displays ID's & # of non-zero channels for SPK-files
   DDIR KF   Displays ID-directory in detail        for HIS-files
             Also logs on DAMM.LOG if LON
 
   DSUM KF   Displays count-sums of all ID's on file-KF
             Also logs on DAMM.LOG if LON
    
   20-Aug-05 ............. U300  DAMM - GENERAL - WTM ............... PAGE   6
 
   U300.090  Cursor Tracking Problems With Xterminals
 
   The software-generated full-window cursor displayed by  DAMM  when  in  the
   "1-key  cursor  mode"  requires  a  lot of real-time response from the host
   computer for live tracking of the mouse. This works fine when the  host  is
   a  local workstation but does not work well for Xterminals hosted by a busy
   VAX. The following commands are intended to alleviate this problem. Type:
 
   CURT LIVE  ;for full-window cursor which tracks mouse "live" (default)
 
   CURT X     ;for new cursor display only for mouse-click or key-press
              ;(works better for Xterminals hosted by busy VAX, etc.)
 
   Execute the desired CURT command and then FIG to make it take effect.
 
   U300.100  Changes in Cursor-Mode Commands
 
   I have eliminated the destinction between  upper  and  lower  case  in  all
   cursor-mode  commands.  The  Shift-  and  Caps-lock keys have no effect. In
   order to do this and retain meaningful command names, it was neccessary  to
   use  two keys for certain commands. These commands (UP, UX, UW, UL, UH, UO,
   and UB) are defined below:
 
   P/UP   Add/Delete peak to Library (pos specified by cursor)
   X/UX   Fix/Free  peak position (for displayed peak nearest to cursor)
   W/UW   Fix/Free  peak width    (for displayed peak nearest to cursor)
   L/UL   Fix/Free  Lo-Side ASYM  (for displayed peak nearest to cursor)
   H/UH   Fix/Free  Hi-Side ASYM  (for displayed peak nearest to cursor)
   O/UO   Turn peak  ON/OFF       (for displayed peak nearest to cursor)
   B/UB   Add/Delete background point at cursor position
 
   Read UP as Unset Peak, for example. As usual, no carriage return  is  used.
   There are also a few other changes in commands. These are listed below:
 
   LF-ARROW    Set expand-region lo-limit
   RT-ARROW    Set expand-region hi-limit
   DN-ARROW    Pan DOWN - move picture so cursor-chan at right-screen
   UP-ARROW    Pan UP   - move picture so cursor-chan at  left-screen
   /           Display XCUR, YCUR, channel# & energy
 
   U300.110  Mouse Button Customizing
 
   The  Mouse  Buttons  can  be used in place of some key-strokes while in the
   1-key cursor-mode. Different button definitions are provided for the  three
   different  types  of  displays  (namely:  the  displays  resulting from the
   commands  D, DD & DS). The following customizing commands are supported:
 
   BUD  L,M,R  ;Defines Left,Middle,Right buttons for cursor in  D-window
   BUDD L,M,R  ;Defines Left,Middle,Right buttons for cursor in DD-window
   BUDS L,M,R  ;defines Left,Middle,Right buttons for cursor in DS-window
   BUD  L,H,S  ;Example (cursor in 1-D)   L-butt sets sum-reg lo-limit,
               ;M-butt sets sum-reg hi-limit,  R-butt requests S-sum
   BUDD A,T,Z  ;Example (cursor in 2-D)   L-butt adds banana points,
               ;M-butt totalizes enclosed counts,  R-butt zots banana
 
   Buttons can't be set to  / or ;  -  set to  ? or :  instead
    
   20-Aug-05 ............. U300  DAMM - GENERAL - WTM ............... PAGE   7
 
   U300.120  Screen Setup and Color Mapping
 
   Screen configuration (placement of graphics  windows  on  the  screen)  and
   color  (or  black  &  white)  mapping  is  discussed in more detail in SEC#
   U300.550, U300.560 and U300.570 (if  you  have  a  B&W  monitor,  you  will
   probably  want to change the color mapping). Here we give the usual list of
   commands and a brief description of each.
 
   COMMANDS RELATED TO SCREEN SETUP & COLOR MAPPING ...........................
 
   FIGI           ;Set  screen configuration library to default
   FIGF FILNAM    ;Read screen configuration library from FILNAM
   FIG  N         ;Set current screen configuration to type-N
 
   WIN  ID        ;Set subsequent displays to be in window-ID (dflt=1)
 
   AXON ID        ;Enable  the drawing of axis for  window-ID (dflt)
   AXOF ID        ;Disable the drawing of axis for  window-ID
 
   CMAP           ;Set color map to default  ("takes" after next FIG)
   CMAP FILNAM    ;Set color map from FILNAM ("takes" after next FIG)
   REVV           ;Reverse all color specs   ("takes" after next FIG)
 
   DLNS N         ;Set # disp-lines = N - for HELP, DDIR & DSUM
 
   CURT LIVE      ;Set full-wind cursor to track mouse LIVE (default)
   CURT X         ;New full-wind cursor generated via mouse-click or key
                  ;The CURT command takes effect only after next FIG
 
   SSI            ;Set screen to initial - erase all graphic windows
 
   Program  damm  has,  by  default,  16  pre-defined  screen  connfigurations
   available.  Each  is  referenced  (via  the FIG command) by an ID-number. A
   list of the ID-numbers along with the  associated  configuration  is  given
   below.  Try a few FIGs and observe the results or if you really want to get
   serious, see SEC# U300.550. In particular, if you have  an  Xterminal  with
   less  than  1024  x  860  pixels,  you  will  probably  need  to modify the
   configuration table as described in SEC# U300.550.
 
    1-[]     2-[][]   3-[][]   4-[][]   5-[][]   6-[][]   7-[][]   8-[][]
                        []       [][]     []       [][]     []       [][]
                                          []       [][]     []       [][]
                                                            []       [][]
 
    9-[][]  10-[][]  11-[--]  12-[--]  13-[--]  14-[--]  15-[][][]
      []       [][]              [--]     [--]     [--]     [][][]
      []       [][]                       [--]     [--]     [][][]
      []       [][]                                [--]     [][][]
      []       [][]                                         [][][]
 
   16-[2D]  17-[2D]  18-[2D]  19-[2D]  20-[][][][]
                                          [][][][]
                                          [][][][]
                                          [][][][]
                                          [][][][]
    
   20-Aug-05 ............. U300  DAMM - DISPLAY - WTM ............... PAGE   8
 
 
   U300.130  Display Delay (hangup problems)
 
   Some display devices (X-terminals for example) may require a delay  between
   successive  displays.  If  your  device  "hangs up" or produces "incomplete
   displays" when executing LOOPS, you might try  increasing  the  appropriate
   delay. A range of 10 to 20000 miliseconds is accepted.
 
   DLAF MS        ;Set FIG-delay     to MS   milliseconds (default=1000)
   DLAF           ;Set FIG-delay     to 1000 milliseconds
   DLAD MS        ;Set Display-delay to MS   milliseconds (default=500)
   DLAD           ;Set Display-delay to 500  milliseconds
 
 
   U300.135  Commands Which Suspend or Abort Operations
 
   WAIT  MS             ;Suspends execution for MS milliseconds
                        ;(60,000 Milliseconds is the maximum allowed)
 
   WO                   ;Suspends execution until you  type [RETURN]
                        ;(it prompts you to: Type [RETURN] to continue
 
   Ctrl/C               ;Interrupts (aborts) a FIT-in-progress
                        ;Interrupts (aborts) a loop-in-progress
                        ;Interrupts (aborts) a command-file being executed
 
   Note:  If  you interrupt a fit-in-progress (being executed within a command
   file), this does not  abort  the  command  file  execution.  An  additional
   Ctrl/C will be required to abort the command file.
 
 
   U300.140  Zeroing Histograms in Shared Memory
 
   The  following  commands  are provided for use when damm is used to monitor
   histogram generation during data acquisition or tape scanning.
 
   Z  ON                ;Enable  Z-command
   Z  OFF               ;Disable Z command (default)
 
   Z  IDA IDB IDC ..    ;Says zero explicit histogram ID-list
   Z  IDA to IDB        ;Says zero IDA thru IDB (in steps of 1)
   Z  IDA to IDB INC    ;Says zero IDA thru IDB (in steps of INC)
 
   For these commands to work, the following conditions are required:
 
   (A)....The his-file must be opened by DAMM via the IN cmd.
 
   (B)....The his-file must be attached to a scan or monitor process
          (i.e. the his-file must be in a shared memory segment)
 
                          The disk-file is NOT zeroed!
    
   20-Aug-05 ............. U300  DAMM - DISPLAY - WTM ............... PAGE   9
 
 
   U300.150  Commands Related to 1-D Display
 
   FIG  NF        ;Choose screen config-NF. See SEC# U300.120 & U300.550
                  ;for screen configuration and color mapping commands
   WIN  NW        ;Set subsequent displays to be in window-NW
 
   LIN/LOG        ;Set display to linear/log  (default is linear)
 
   PLON/PLOF      ;Turn peak logging ON/OFF (dflt OFF) see SEC# U300.155
 
   ST /OV         ;Set to disp mult hist stacked/overlayed (dflt = OV)
 
   CAL  A,B,C     ;Define energy calibration (E=A+B*Chan+C*Chan**2)
 
   COL  I,J,K...  ;Defines color sequence for display
                  ;For I,J.. = 1     2   3     4    5    6       7
                  ;       COL= white,red,green,blue,yellow,magenta,cyan
   GWID WID       ;Define cursor-mode sum-region width (channels)
 
   DNOR LO,HI     ;Normalize displayed data to count-sum of chans LO,HI
 
   DL   LO,HI     ;Set display limits (min,max chan#)
 
   DMM  LO,HI     ;Set display limits (min,max counts)
                  ;LO or HI = X  says use MIN or MAX data value
 
   D    IDLST     ;Display histogram ID's contained in IDLST
 
   DX   IDLST     ;Display IDLST (range defined by expand-region)
 
                  ;-----------------------------------------------------------
                  ;IDLST is of the form:
                  ;KF C ID, C ID.. KF C ID, C ID..
                  ;or
                  ;KF C IDA to IDB,<INC> KF C ID ID ID ... KF C IDA to IDB ...
                  ;-----------------------------------------------------------
                  ;ID, IDA & IDB denote ID-numbers
                  ;INC (optional) denotes an ID-number increment (DFLT=1)
                  ;C is an OPTIONAL floating-point norm-coeff (DFLT=1.0)
                  ;KF = M,N,O,P,Q,R,S (default is N) and denotes:
                  ;MEM-BUF, IN-FIL, OU-FIL, PROJ-FIL, Q-FIL, R-FIL, S-FIL
                  ;If IDLST omitted, uses previously defined IDLST
                  ;-----------------------------------------------------------
 
   SUML LO,HI     ;Define sum region for SUM command below
   SUM  IDLST     ;Sum counts (LO,HI) (IDLST same as D except C illegal)
 
   C              ;Enter cursor-mode
 
   SSI            ;Set screen to initial - erase graphic windows
 
                            (continued on next page)
    
   20-Aug-05 ............. U300  DAMM - DISPLAY - WTM ............... PAGE  10
 
 
   U300.150  Commands Related to 1-D Display (continued)
 
 
   1-KEY CURSOR COMMANDS FOR 1-D DISPLAY .....................................
 
   <-- (LF-ARROW) ;Set expand-region lo-limit
   --> (RT-ARROW) ;Set expand-region hi-limit
 
   E              ;Expand display
 
   UP-ARROW       ;Pan up   - move picture so cursor-chan at  left-screen
   DN-ARROW       ;Pan down - move picture so cursor-chan at right-screen
                  ;(you must be expanded to execute pan)
 
   L              ;Set sum-region lo-limit
   H              ;Set sum-region hi-limit
   G              ;Set sum-region limits (LO=XCUR, HI=XCUR+WID-1)
 
   V              ;Make expand-/sum-region markers visible/invisible (toggles)
   M              ;Turn expand-/sum-region markers ON
   J              ;Junk (delete) all expand-/sum-region markers
 
   S              ;Display sum, centroid, fwhm (2.354*sigma) of sum region
                  ;DATA(LO),DATA(HI) defines BGD for NETS
 
   A              ;Display sum, centroid, fwhm (2.354*sigma) of sum region
                  ;YCUR(LO),YCUR(HI) defines BGD for NETS
 
   C              ;Draw peak-marker and display chan#  at cursor pos
   P              ;Draw peak-marker and display energy at cursor pos
 
   / or ?         ;Display XCUR, YCUR, channel# & energy
   '              ;Same as ? except forces logging (see SEC# U300.155)
 
   Q              ;Quit cursor-mode (return to normal-mode)
 
   See SEC# U300.110 or Type: h mous for use of Mouse Buttons.
    
   20-Aug-05 ............. U300  DAMM - DISPLAY - WTM ............... PAGE  11
 
   U300.155  Peak Finding/Logging
 
   The following commands control peak finding.
 
   FIND BIAS,IFWHM     ;Turn peak-find ON  (see definitions below)
   FIND                ;Turn peak-find ON  (with defaults - see below)
   NOFI                ;Turn peak-find OFF (default is OFF)
 
   BIAS...is the number of standard deviations above background  that  a  peak
          channel  must be in order to be considered as part of a peak. Useful
          values of BIAS are in the range 3 to 10. The default value is 5.0.
 
   IFWHM..is the approximate full-width at half-max (in channels) of peaks  in
          the  region  of interest. This value is not very critical but should
          be within a factor of 2 or so of  the  correct  value.  The  default
          value is 5.
 
   All  peaks  found  within  the display region will be marked & labeled with
   the associated energy-calibration value (default is  the  same  as  channel
   number).  Peak  labels are integers (no decimals - to minimize screen space
   used) so if you want labels to be in units of keV, for  example,  you  must
   enter CAL such that E(keV) is a whole number.
 
   See SEC# U300.430 for how peak-finding is used in fitting operations.
 
   Logging "found" and "marked" peaks on damm.log
 
   Peaks  which  either  found  via  the  FIND command above or marked via the
   1-key command (/ or ? or ;)  may  be  logged  on  damm.log.  The  following
   commands (independent of LON/LOF) turns said logging ON and OFF.
 
   PLON           ;Turns peak logging (to damm.log) ON
   PLOF           ;Turns peak logging (to damm.log) OFF (default)
 
   For found peaks, damm.log may be read (skipping prog, date & time) as:
 
          READ(LU,10)IFLG,ID,CH,HEFT,(FILNAM(I),I=1,16)
       10 (29X,A4,I10,2F10.0,2X,16A4)
 
   Where: IFLG   = 'PEAK' for found peak log entry
          ID     = Spectrum ID number
          CH     = Peak location in channels
          HEFT   = Peak heftiness
          FILNAM = First 64 characters of spk- or his-filename
 
   For marked peaks, damm.log may be read (skipping prog, date & time) as:
 
          READ(LU,10)IFLG,ID,CH,ENER,(FILNAM(I),I=1,16)
       10 (29X,A4,I10,2F10.0,2X,16A4)
 
   Where: IFLG   = 'MARK' for marked peak log entry
          ID     = Spectrum ID number
          CH     = Peak location in channels
          ENER   = Peak "energy" from calibration constants
          FILNAM = First 64 characters of spk- or his-filename
    
   20-Aug-05 ............. U300  DAMM - DISPLAY - WTM ............... PAGE  12
 
   U300.160  Commands Related to 2-D Display
 
   FIG   NF       ;Choose screen config-NF. See SEC# U300.120 & U300.550
                  ;for screen configuration and color mapping commands
 
   WIN   NW       ;Set subsequent displays to be in window-NW
 
   ZLEV  N        ;Set # of color/grey-scale intensity levels to N
   ZLEV           ;Set # of color/grey-scale intensity levels to 10 (dflt)
 
   GRAS  I,J,K..  ;Set grey-scale values (range 0-100) & ZLEV (# entries)
                  ;Must do after first FIG  ("takes" after next FIG)
 
   ZINT  COLR     ;Set 2-D intensity mapping to Color (default)
   ZINT  GREY     ;Set 2-D intensity mapping to Grey-scale
   ZINT  DOTS     ;Set 2-D intensity mapping to Dot-matrix (3x3 or 5x5)
 
   ZSON/ZSOF      ;Z-scale ON/OFF - displays cnts vs colors legend (dflt)
 
   ZLIN/ZLOG      ;Set 2-D display to linear/log (default is log)
 
   XC    LO,HI    ;Set min & max X-channel numbers for display
   YC    LO,HI    ;Set min & max Y-channel numbers for display
 
   ZMM   LO,HI    ;Set min & max counts/channel    for display & count-sum
   ZMM   LO       ;Sets min & searches for max (semi-autoscale)
   ZMM            ;Searches for min & max      (full-autoscale)
 
   DD       ID    ;Display 2-D histogram ID from IN-file
   DD    KF,ID    ;Display 2-D histogram ID from KF-file
 
   DDX      ID    ;Display 2-D histogram ID from IN-file (expand region)
   DDX   KF,ID    ;Display 2-D histogram ID from KF-file (expand region)
                  ;Where KF = N,O,Q,R,S (default is N) and denotes:
                  ;IN-FIL, OU-FIL, Q-FIL, R-FIL, S-FIL
                  ;ID (and KF) omitted says use previously defined spec
 
   DDID           ;Shows ID-number & Filename for current 2-D display
 
   ZBL            ;Zero in-core BAN-library
 
   LBL            ;List in-core BAN-ID numbers
 
   C              ;Enter cursor-mode
 
   SSI            ;Set screen to initial - erase graphic windows
 
                            (continued on next page)
    
   20-Aug-05 ............. U300  DAMM - DISPLAY - WTM ............... PAGE  13
 
 
   U300.160  Commands Related to 2-D Display (continued)
 
 
   1-KEY CURSOR COMMANDS FOR 2-D DISPLAY .....................................
 
   <-- (LF-ARROW) ;Set expand-region lo-left- limit
   --> (RT-ARROW) ;Set expand-region hi-right-limit
 
   V      ;Make expand-region markers visible/invisible (toggles)
   J      ;Junk (delete) expand-region markers
 
   E      ;Expand
 
   1      ;Move display such that cursor is at lo-left
   2      ;Move display such that cursor is at hi-left
   3      ;Move display such that cursor is at hi-right
   4      ;Move display such that cursor is at lo-right
 
   Z      ;Zero  (open) active X,Y-list
   A      ;Add point to active X,Y-list
   D      ;Dele nearest point in active X,Y-list
   M      ;Move nearest point in active X,Y-list to cursor pos
   I      ;Insert a point in active X,Y-list at cursor position
 
   L      ;List active X,Y-list (on VDT)
   B      ;Draw active BAN & BAN's in in-core library
   G      ;Prompt for ID & read into in-core BAN-library
   O      ;Open nearest in-core BAN for modification
   S      ;Prompt for ID & store in in-core library & on disk
   R      ;Store  nearest BAN with original ID
   F      ;Remove nearest BAN from in-core library & erase
   K      ;Delete nearest BAN from in-core library and disk
   T      ;Totalize counts in nearest BAN (active or not)
   P      ;Totalize and & save X- & Y-projections on DAMQ8Q.SPK
 
   / or ? ;Display/(log if PLON)  X,Y-coordinates of cursor
   '      ;Like ? but forces log (see X,Y-log format below)
 
   Q      ;Quit cursor-mode (return to normal-mode)
 
   See SEC# U300.110 or Type: h mous for use of Mouse Buttons.
 
   ---------------------------------------------------------------------------
   X,Y-log format- damm.log may be read (skipping prog, date & time) as:
          READ(LU,10)IFLG,ID,X,Y,(FILNAM(I),I=1,16)
       10 (29X,A4,I10,2F10.0,2X,16A4)
 
   Where: IFLG   = 'MAR2' for 2-D position mark entry
          ID     = Histogram ID number
          X      = X-position in channels
          Y      = Y-position in channels
          FILNAM = First 64 characters of spk- or his-filename
   ---------------------------------------------------------------------------
    
   20-Aug-05 ............. U300  DAMM - DISPLAY - WTM ............... PAGE  14
 
 
 
   U300.170  Special XY-Line Display
 
   DECstation  damm  provides  for  the  display  of  one  or  more  lists  of
   concatenated vectors (lines) onto any of the FIGed  windows.  In  order  to
   use this feature, a ASCII file must be created of the following form:
 
 
   XYDATA  ID  XLEN  YLEN
   X Y
   X Y
   . .
   . .
   XYDATA  ID  XLEN  YLEN
   X Y
   X Y
   . .
   END-OF-FILE     Where:
 
 
   XYDATA  is  the  ASCII  string  "xydata"  in  upper  or lower case (XYDA is
          enough).
 
   ID.....is an integer ID-number by which the  X,Y  data  set  which  follows
          will be referenced.
 
   XLEN...is the X-length basis for the following X-values (optional).
 
   YLEN...is the Y-length basis for the following y-values (optional).
 
   X......is a floating# (used as chan# for 1D & 2D displays).
          (X is decoded via E-format & can contain up to 12 characters)
 
   Y......is a floating# (used as chan# for 2D and count for 1D displays).
          (Y is decoded via E-format & can contain up to 12 characters)
 
 
   If  XLEN  and  YLEN are entered and "histogram lengths" are provided by the
   displayed histograms, the X,Y data will be appropriately  scaled  to  match
   the  histogram  data  contained  in  the  window  in  which  it  is  drawn.
   "Histogram lengths"  are  provided  by  all  his-file  directories  and  by
   spk-file  entries  which  are  copied  from  his-files.  If either of these
   "length entries" are missing, no scaling will be done. For 1D displays,  no
   attempt is made to scale Y.
 
 
 
                            (continued on next page)
    
   20-Aug-05 ............. U300  DAMM - DISPLAY - WTM ............... PAGE  15
 
 
   U300.170  Special XY-Line Display (continued)
 
   List of commands
 
   XYF  filename     ;Opens XY-file and reads in all data
 
   XYI               ;Displays XY IDs which have been read in
 
   XYD  I J K L ...  ;Displays XY data for IDs (I J K L ..) into active
                     ;window (default = window 1 or set by WIN command)
 
   XYP  I J K L ...  ;Same as XYD except that XY-points are shown in
                     ;addition to the connecting vectors
 
                     ;XYD or XYP with no ID-list uses previous ID-list
 
   XYC  KOL          ;Specifies the "color" for subsequent XY displays
                     ;Legal values of KOL are:
 
                     ;WHIT - white      - CMAP entry 34
                     ;RED  - red        - CMAP entry 35
                     ;GREE - green      - CMAP entry 36
                     ;BLUE - blue       - CMAP entry 37
                     ;RG   - red-green  - CMAP entry 38
                     ;RB   - red-blue   - CMAP entry 39
                     ;GB   - green-blue - CMAP entry 40
 
   Note:  Displays  are  done in "complement mode" so that displaying the same
   ID a second time will erase it, displaying a third time will show it, etc.
 
   Current Limits
 
   Maximum number of IDs        in a file = 2048
   Maximum number of XY entries in a file = 256,000
   Maximum number of XY points  in a set  = 500
 
   These limits are rather arbitrary, and can easily be changed.
 
   Comments
 
   (1)....The XY display feature is for display only. The XY data file is  not
          another  data  type  on  a  par  with  HIS-file or SPK-file data. It
          cannot be analysed, summed, or combined with other data, etc.
 
   (2)....If the XY data extend  beyond  the  range  of  the  display,  it  is
          allowed  to  spill  over  into the "scale label" regions. It is easy
          enough to prevent this but I  am  not  sure  if  it  would  be  more
          desirable.
 
   (3)....An  ID label is drawn near to the XY-point which is "nearest" to the
          center of the window. If multiple XY-lines are displayed, there  may
          be  some  confusion  in identification. I will think about trying to
          minimize this.
 
   (4)....All commands and specifications are case insensitive.
    
   20-Aug-05 ............. U300  DAMM - DISPLAY - WTM ............... PAGE  16
 
 
   U300.175  Label Generation and Annotation
 
   damm provides  a  simple  method  for  labeling  and  annotating  graphical
   displays. The following features are provided:
 
   (1)....Normal  text  labels may be generated either interactively or by the
          usual processing of command files.
 
   (2)....The Label Generating Commands (LA7, LA8, LA9)  described  below  are
          used to generate up to three blocks of "label text".
 
   (3)....Each  of the three "label blocks" can contain up to 10 lines of text
          with each line containing up  to  76  characters  (80  -  4  command
          characters).
 
   (4)....Each  damm window supports up to 3 independent, active (relocatable)
          label blocks.
 
   (5)....By "locking" labels already displayed and re-defining label  blocks,
          you can display as many labels as you like.
 
   (6)....Labels  are positioned within graphics windows using the 1-key Label
          Display Commands described below.
 
   (7)....Label Pointer Commands, defined below, may be  used  to  make  close
          associations between labels and specific graphic features.
 
   Note:  LA7,  LA8,  LA9  label  specifiers  are  used  in  order to make the
   association with the label display keys (7,8,9) easier.
 
   Label Generating Commands -------------------------------------------------
 
   LA7  text   ;Adds line of "text" to LA7 label
   LA8  text   ;Adds line of "text" to LA8 label
   LA9  text   ;Adds line of "text" to LA9 label
 
   LAZ7        ;Deletes all lines of   LA7 label
   LAZ8        ;Deletes all lines of   LA8 label
   LAZ9        ;Deleted all lines of   LA9 label
 
   LAL         ;Displays current labels - LA7, LA8, LA9
 
   LAC  KOL    ;Specifies label color for subsequent displays
               ;Legal values of KOL are:
 
               ;WHIT - white      - CMAP entry 34
               ;RED  - red        - CMAP entry 35
               ;GREE - green      - CMAP entry 36
               ;BLUE - blue       - CMAP entry 37
               ;RG   - red-green  - CMAP entry 38
               ;RB   - red-blue   - CMAP entry 39
               ;GB   - green-blue - CMAP entry 40
 
 
                   (see next page for label display commands)
    
   20-Aug-05 ............. U300  DAMM - DISPLAY - WTM ............... PAGE  17
 
 
 
   U300.175  Label Generation and Annotation (continued)
 
   Label Display Commands (1-key cursor-mode) --------------------------------
 
   Key      Action
 
   7        Displays LA7 at cursor location (complement mode)
   8        Displays LA8 at cursor location (complement mode)
   9        Displays LA9 at cursor location (complement mode)
 
   0        Locks previously displayed labels against subsequent change
            (i.e. all labels displayed in a given window will be fixed)
            (enables the drawing of additional labels)
 
   It goes like this:
 
   (1)....The first strike of say  key-7  displays  LA7  with  the  upper-left
          corner of an unclosed box at the cursor location.
 
   (2)....The next strike of key-7 will erase the LA7 label and the box.
 
   (3)....The  next  strike  of  key-7  will  display  LA7  in  any new cursor
          location. Etc, etc. and the same for key-8 and -9.
 
   (4)....Finally, key-0 will close any open boxes and  lock  in  all  current
          labels within a given window. Additional labels are now enabled.
 
   Label Pointer Commands (1-key cursor-mode) --------------------------------
 
   The  following  1-key  commands  provide  for  the  interactive  drawing of
   concatenated straight-line segments. The idea is to  provide  a  method  of
   associating  a  block of text with a very specific region in a display. The
   following 1-key commands are used to draw and modify such lines.
 
   Key      Action
 
   =        Adds a vector point to current list and draws point or line
 
   -        Deletes last vector point in current list and erases line
 
   ;        Locks current vector list against subsequent change and
            enables new list
 
   Comments:
 
   (1)....Each window supports an independent active vector list.
 
   (2)....The active vector-list associated with  any  window  can  contain  a
          maximum of 20 points.
    
   20-Aug-05 ........... U300  DAMM - MANIPULATION - WTM ............ PAGE  18
 
   U300.200  Command Syntax - General Definitions
 
   B1   - Memory Buffer-1
   B2   - Memory Buffer-2
   ID   - The  ID-number of histogram to be read
   NUID - Next ID-number to be assigned to output histogram
   LO   - A first-channel-number (usually of a Gate)
   HI   - A last- channel-number (usually of a Gate)
   FAC  - A multiplication factor
 
   Meaning of the individual command-characters ..............................
 
   I    - Input  or read
   O    - Output or write
   A    - Add or accumulate
   S    - Shift (gain shift)
   GX   - Gate on X-parameter (i.e. parameter # 1)
   GY   - Gate on Y-parameter (i.e. parameter # 2)
   1    - Buffer-1
   2    - Buffer-2
   M    - Multiply
   C    - Crunch
   D    - Divide
 
   U300.210  Commands for Setup (no immediate action)
 
   NUID IV    - Set next ID to be used to IV
 
   IDST N     - Set ID-step to be used in implied I/O loops
                (remains active until changed - default=1)
 
   CRUN IVAL  - Sets standard crunch value to IVAL
 
   GASP XI1,XI2,XF1,XF2,NCF - Standard gain shift specification
 
   SIDA       - Says treat 16-bit HIS-file data as signed
   USDA       - Says treat 16-bit HIS-file data as un-signed (default)
    
   20-Aug-05 ........... U300  DAMM - MANIPULATION - WTM ............ PAGE  19
 
   U300.220  Commands for Input/Output of 1-D Histograms
 
   I    kf ID            Input to B1
   IS   kf ID            Input to B1, gain shift B1
   IA   kf ID,FAC        Input to B1, B2=B2+FAC*B1
   ISA  kf ID,FAC        Input to B1, shift  B1, B2=B2+FAC*B1
   IO   kf ID            Input to B1, output B1
   ISO  kf ID            Input to B1, shift  B1, output B1
   IO   kf IDA,IDB       Input to B1, output B1     (for ID=IDA,IDB)
   ISO  kf IDA,IDB       Input to B1, shift, output (for ID=IDA,IDB)
   O1                    Output B1
   O2                    Output B2
 
   U300.230  Commands for Gating 2-D Histograms
 
   GY   kf ID,LO,HI      Y-gate to B1
   GYS  kf ID,LO,HI      Y-gate to B1, shift  B1
   GYO  kf ID,LO,HI      Y-gate to B1, output B1
   GYO  kf IDA,IDB,LO,HI Y-gate to B1,         output B1 (for ID=IDA,IDB)
   GYSO kf ID,LO,HI      Y-gate to B1, shift & output B1
   GYSO kf IDA,IDB,LO,HI Y-gate to B1, shift & output B1 (for ID=IDA,IDB)
   GYA  kf ID,LO,HI,FAC  Y-gate to B1, B2=B2+FAC*B1
   GYSA kf ID,LO,HI,FAC  Y-gate to B1, shift B1, B2=B2+FAC*B1
 
   GX   kf ID,LO,HI      X-gate to B1
   GXS  kf ID,LO,HI      X-gate to B1, shift  B1
   GXO  kf ID,LO,HI      X-gate to B1, output B1
   GXO  kf IDA,IDB,LO,HI X-gate to B1,         output B1 (for ID=IDA,IDB)
   GXSO kf ID,LO,HI      X-gate to B1, shift & output B1
   GXSO kf IDA,IDB,LO,HI X-gate to B1, shift & output B1 (for ID=IDA,IDB)
   GXA  kf ID,LO,HI,FAC  X-gate to B1, B2=B2+FAC*B1
   GXSA kf ID,LO,HI,FAC  X-gate to B1, shift B1, B2=B2+FAC*B1
   O1                    Output B1
   O2                    Output B2
 
   U300.240  Commands for General 2-D Projections
 
   PJ   kf ID,BID,DEGR               PROJ TO B1
   PJS  kf ID,BID,DEGR               Proj to B1, shift  B1
   PJO  kf ID,BID,DEGR               Proj to B1, output B1
   PJO  kf IDA,IDB,BIDA,BIDB,DEGR -  Proj to B1, output B1
                                     (outer loop on BID, inner loop on ID)
   PJSO kf ID,BID,DEGR               Proj to B1, shift & output B1
   PJSO kf IDA,IDB,BIDA,BIDB,DEGR -  Proj, shift, output
                                     (outer loop on BID, inner loop on ID)
   PJA  kf ID,BID,DEGR,FAC           Proj to B1, B2=B2+FAC*B1
   PJSA kf ID,BID,DEGR,FAC           Proj to B1, shift B1, B2=B2+FAC*B1
 
   PJAL                              Project all bananas in currently open
                                     BAN-file for HIS-files, ID's & DEGR'S
                                     contained therin
   O1                                Output B1
   O2                                Output B2
 
   ID denotes histogram ID, BID denotes Banana ID
   (DEGR = Projection-axis angle in degrees)
   kf is an optional file-code = N,O,P,Q,R,S (default = N)
    
   20-Aug-05 ........... U300  DAMM - MANIPULATION - WTM ............ PAGE  20
 
   U300.250  Commands for Operations on Buffer-1 & Buffer-2
 
   M1  XM             Multiply B1 by XM
   M2  XM             Multiply B2 by XM
   C1  ICRUN          Crunch B1  by ICRUN (standard crunch unchanged)
   C2  ICRUN          Crunch B2  by ICRUN (standard crunch unchanged)
   S1                 Shift  B1  by standard GASP
   S2                 Shift  B2  by standard GASP
   S1  XI1,XI2,XF1,XF2,NCF - Shift B1 as specified (standard GASP unchanged)
   S2  XI1,XI2,XF1,XF2,NCF - Shift B2 as specified (standard GASP unchanged)
   Z1                 Zero B1
   Z2                 Zero B2
   A12  FAC           B2=B2+FAC*B1
   A21  FAC           B1=B1+FAC*B2
   SWAP               Swap B1 & B2
   M2D1 FAC           B2=(FAC*B2)/B1
 
   MX12               B2=MAX(B1,B2) channel-by-channel
   MX21               B1=MAX(B1,B2) channel-by-channel
   MN12               B2=MIN(B1,B2) channel-by-channel
   MN21               B1=MIN(B1,b2) channel-by-channel
 
   O1                 Output B1
   O2                 Output B2
 
   U300.260  Commands which Show Data, Count-Sums etc (from Bufs-1 & -2)
 
   PR1                Print   Buffer-1
   PR2                Print   Buffer-2
 
   D1   LO,HI         Display Buffer-1 (channels LO thru HI)
   D2   LO,HI         Display Buffer-2 (channels LO thru HI)
 
   SUM1 LO,HI         Display sum of counts LO-thru-HI of B1
   SUM2 LO,HI         Display sum of counts LO-thru-HI of B2
 
   COMP NCH           Compare first NCH-channels of B1 & B2
                      (gives # counts and # mis-matches)
   GEN  ID,KO,KX,NCH  Generate test spectrum in B1 (NCH channels)
                      Channel contents = KO+KX*(channel#+1)
 
   U300.270  Commands which Modify Buffer Contents
 
   SET1 ICN,YV        Set channel ICN of B1 to YV
   SET2 ICN,YV        Set channel ICN of B2 to YV
   SET1 LO,HI,YV      Set channels LO-thru-HI of B1 to YV
   SET2 LO,HI,YV      Set channels LO-thru-HI of B2 to YV
   SET1 LO,HI,YA,YB   Set channels LO-thru-HI of B1 to YA-thru-YB
   SET2 LO,HI,YA,YB   Set channels LO-thru-HI of B2 to YA-thru-YB
   ADD1 ICN,YV        Add YV to channel ICN of B1
   ADD2 ICN,YV        Add YV to channel ICN of B2
   ADD1 LO,HI,YV      Add YV to channels LO-thru-HI of B1
   ADD2 LO,HI,YV      Add YV to channels LO-thru-HI of B2
   ADD1 LO,HI,YA,YB   Add YA-thru-YB to channels LO-thru-HI of B1
   ADD2 LO,HI,YA,YB   Add YA-thru-YB to channels LO-thru-HI of B2
                      (i.e. a strait line)
   MSK1 LO,HI,MSK     Mask (with MSK)   channels LO-thru-HI of B1
   MSK2 LO,HI,MSK     Mask (with MSK)   channels LO-thru-HI of B2
                      (MSK is entered in hexadecimal)
    
   20-Aug-05 ........... U300  DAMM - MANIPULATION - WTM ............ PAGE  21
 
   U300.280  Commands Related to Printer Plots
 
   SKRZ               Set to skip repeated-zeros for printer plots
   PLRZ               Set to plot repeated-zeros for printer plots
 
   PLG  ID,LO,HI,NCYC      - Input to B1 & LOG plot
   PLN  ID,LO,HI,NCFS      - Input to B1 & LIN plot
   PLG  IDA,IDB,LO,HI,NCYC - Input to B1 & LOG plot (for ID=IDA,IDB)
   PLN  IDA,IDB,LO,HI,NCFS - Input to B1 & LIN plot (for ID=IDA,IBD)
 
   PLG1 LO,HI,NCYC    Log    Printer-plot of Buffer-1
   PLG2 LO,HI,NCYC    Log    Printer-plot of Buffer-2
   PLN1 LO,HI,NCFS    Linear Printer-plot of Buffer-1
   PLN2 LO,HI,NCYC    Linear Printer-plot of Buffer-2
 
   (NCFS = # of counts full-scale for LIN plots)
   (NCYC = # of cycles            for LOG plots)
 
   U300.290  Discussion of Gain-Shifts and Compressions
 
   Gain  shifts are specified by five parameters - XI1, XI2, XF1, XF2 and NCF.
   XI1 and XI2 represent two locations (in channel-#  units)  in  the  initial
   1-D  histogram  and  XF1  and  XF2 represent corresponding locations in the
   final histogram (i.e. after the transformation). That is:
 
         XF=A+B*XI
   where,
         B=(XF2-XF1)/(XI2-XI1)
   and
         A=XF1-B*XI1
 
   NCF gives the number of channels in the histogram after the transformation.
   If   NCF=0,   the   final   #-of-channels  is  determined  by  the  initial
   #-of-channels NCI and the transformation specified.  If  NCF=-1  the  final
   #-of-channels  is set equal to NCI. Counts are redistributed into the final
   set of channels (bins) by assuming a uniform distribution of counts in  the
   initial  bins. Data shifted below channel-#-0 and above channel-#-NCF-1 are
   lost and gone forever.
 
   Gain-shifts are always "in place"
   CRUN IVAL (i.e. standard crunch) does it at "input time"
   Data is kept internally as floating - is converted to fixed on output
   All output from DAMM is 32 bits/channel
    
   20-Aug-05 ........... U300  DAMM - MANIPULATION - WTM ............ PAGE  22
 
 
   U300.300  HIS-file (Create, Expand, Copy, Add, Gain-Shift, etc)
 
   The commands given below may be used to Create a new HIS-file or to  Expand
   an  existing HIS-file (by specifying additional entries) for support of the
   operations described on the following pages of this section.
 
   HOU  fil.his     - Opens fil.his for histogram Output
                    - If non-existant, Prompts for permission to create it
 
   HDIM N           - HIS dimension (N = 1 or 2) (required)
 
   HWPC N           - half-wds/chan (N = 1 or 2) (required)
 
   HPAR PX PY       - Parameter IDs              (reference only)
 
   LRAW LX LY       - Raw    parm lengths        (reference only)
 
   LSCL LX LY       - Scaled parm lengths        (reference only)
 
   MINC MNX MNY     - Min chan#s for X & Y       (required)
 
   MAXC MXX MXY     - Max chan#s for X & Y       (required)
 
   XLAB labl-x      - X-label (ASCII)            (reference only)
 
   YLAB labl-y      - Y-label (ASCII)            (reference only)
 
   HTIT title       - Title   (ASCII)            (reference only)
 
   HEDD             - Displays current data to be used for next his generation
 
   HGEN HID         - Generates histogram HID using current data
 
 
   When the HGEN command is entered, the data associated  with  HDIM,  MINC  &
   MAXC  will  be  checked  for  legallity and HID, given in the HGEN command,
   will be checked for multiple definition.
 
   Note for Input HIS-file
 
   o......If HIS-file is shorter than implied by DRR-file,  the  open  request
          is rejected.
 
   Notes for Output HIS-file
 
   o......If  HIS  &  DRR files do not exist, you will be asked for permission
          to create them.
 
   o......If DRR-file  exists  but  HIS-file  does  not,  you  are  asked  for
          permission to create the HIS-file and then extend it.
 
   o......If  HIS-file  exists  but  DRR-file  does  not,  the open request is
          rejected.
 
                            (continued on next page)
    
   20-Aug-05 ........... U300  DAMM - MANIPULATION - WTM ............ PAGE  23
 
 
   U300.300  HIS-file (Create, Expand, Copy, Add, Gain-Shift, etc) (continued)
 
   DAMM can copy, add (or subtract) and gain-shift 1-D or 2-D histograms  from
   an input HIS-file to an output HIS-file.
 
   (1)....All  operations  are  from  an  INPUT-file  and INPUT-ID (IDI) to an
          OUTPUT-file and OUTPUT-ID (IDO).
 
   (2)....For HCOP and HADD operations, the output  histogram  must  have  the
          same dimensions and ranges as the input histogram.
 
   (3)....For  SHIF  (gain-shift)  and SHAD (gain-shift & add) operations, the
          dimensions of the output histogram need not match the input.
 
   (4)....The number of bits-per-channel (16 or 32) need not be the  same  for
          input and output.
 
   (5)....Gain-shifts  are  accomplished  by  converting  the data to floating
          point, rebinning (with count fractionation) and  finally  converting
          back to integer.
 
   (6)....Fractional copies and adds are also done in floating point.
 
   (7)....Final  conversion  from  floating  point  to  integer  involves  the
          addition of a  random  number  whose  range  is  0.0  to  1.0.  This
          procedure  results  in  slight  differences  in  the total number of
          counts for the input and output histograms.
 
   ==========================================================================
   The commands given on the preceeding page may be used to create and/or
   expand the output HIS- and DRR-files as needed.
   ==========================================================================
 
   HOU  FIL.HIS               - Opens HIS-file for output
 
   SNEG OFF                   - Turn OFF reset of neg 16bit out data (default)
   SNEG IV                    - Says set negative 16-bit output data to IV
                              - (you MUST use SIDA mode for this to work!!)
 
   GSX  XI1,XI2 XF1,XF2       - Defines   X-gain-shift (described below)
   GSY  YI1,YI2 YF1,YF2       - Defines   Y-gain-shift (described below)
 
   GSXOF                      - Turns X-gain-shift OFF (GSX turns it ON)
   GSYOF                      - Turns Y-gain-shift OFF (GSY turns it ON)
 
   HSTA                       - Shows files open & gain-shift data
 
   HCOP kf IDI,IDO <,F>       - Copies F*IDI (input) to IDO (output)
                                (If F is not entered, F=1)
 
   Where kf is an optional file-code field = N,O,P,Q,R,S (default = N)
 
   ===========================================================================
   For HCOP: If IDO does not exist, it will be created with the same
   attributes (same header data) as IDI.
   ===========================================================================
 
                            (continued on next page)
    
   20-Aug-05 ........... U300  DAMM - MANIPULATION - WTM ............ PAGE  24
 
 
 
   U300.300  HIS-file (Create, Expand, Copy, Add, Gain-Shift, etc) (continued)
 
   HADD kf IDI,IDO <,FI><,FO> - Adds FI*IDI to FO*IDO
                                (If FI is not entered, FI=1)
                                (If FO is not entered, FO=1)
                                (If FO is entered, FI must be entered)
 
   HDIV kf IDI,IDO <,FI>      - Divides FI*IDI by IDO & saves in IDO
 
   SHIF kf IDI,IDO <,FI>      - Gain-shifts IDI & stores in IDO
   SHAD kf IDI,IDO <,FI><,FO> - Gain-shifts IDI & adds   to IDO
 
   Where kf is an optional file-code field = N,O,P,Q,R,S (default = N)
 
   HSET IDO,IV                - Sets IDO on output to IV
   HRAN IDO,IV                - Sets IDO on output to IV (counting statistics)
   HZOT IDO                   - Sets IDO on output to 0
 
                              X- and Y-Gain-shifts
 
   X-gain-shifts are specified by the parameters - XI1, XI2, XF1 & XF2.
   Y-gain-shifts are specified by the parameters - YI1, YI2, YF1 & YF2.
 
   For an X-gain-shift, XI1 and XI2 represent  two  locations  (in  channel  #
   units)  in  the  initial  spectrum  and XF1 and XF2 represent corresponding
   locations in the final spectrum (i.e. after transformation). that is:
 
            XF=A+B*XI
   where:   B=(XF2-XF1)/(XI2-XI1)
   and      A=XF1-B*XI1
 
   The "final" # of channels is determined by the "initial" # of channels  and
   the  transformation  specified. Counts are redistributed into the final set
   of channels (bins) by assuming a uniform  distribution  of  counts  in  the
   initial  bins.  Data  shifted  out  of the range of the final histogram are
   lost and gone forever!
 
   The rules and procedures are identical for Y-gain-shifts.
 
                                    COMMENTS
 
   (1)....If   gain-shift  specifications  are  not  given  (or  turned  off),
          bin-widths will be the same for output and input.
 
   (2)....Any data which does  not  fall  within  the  ranges  of  the  output
          histogram will be lost (without comment).
 
   (3)....Data  will  be  properly  positioned in the output histogram even if
          the ranges of the input and output  are  different.  That  is,  data
          will  appear  in  that region of the output histogram which overlaps
          the gain-shifted input histogram.
    
   20-Aug-05 ........... U300  DAMM - MANIPULATION - WTM ............ PAGE  25
 
 
   U300.310  HIS-file (Adding Subtracting Complete HIS-files)
 
   Commands which add & subtract identically structured  his-files .............
 
   The following commands support the addition  and  subtraction  of  complete
   his-files.  The  associated  drr-files  must have identical structures. The
   files may  contain  any  combination  of  1D,  2D,  INTEGER*2  &  INTEGER*4
   histograms
 
   -----------------------------------------------------------------------------
   FADI   inpfile.his  ;Opens input  his-file - .his & .drr must already exist
                       ;cannot be same as output
 
   FADO   outfile.his  ;Opens output his-file - .his & .drr must already exist
                       ;cannot be same as input but must have same structure
 
   FADN   outfile.his  ;Creates & initializes   outfile.his & outfile.drr
                       ;with same attributes as inpfile.his & inpfile.drr
                       ;outfiles .his & .drr must NOT already exist
 
   FADZ                ;Zeros outfile.his
 
   FADF                ;Displays FAD files which are open
 
   FADD                ;Adds      inpfile.his to   outfile.his
 
   FSUB                ;Subtracts inpfile.his from outfile.his
 
   CLIF                ;Closes input  FAD-file if open
 
   CLOF                ;Closes output FAD-file if open
 
   Ctrl/C              ;Interrupts (aborts) an ADD-in-progress
                       ;Interrupts (aborts) a ZERO-in progress
   -----------------------------------------------------------------------------
   NOTE: The input & output file assignments here are NOT the same as those
         assigned by the damm commands IN, OU & HOU. However, you may open these
         files for display by commands such as IN, QF, etc.
         For example, you may do the following:
 
   fadi infile.his  !Open infile.his for add/subtract input
   fado oufile.his  !Open oufile.his for add/subtract output
   in   infile.his  !Open infile.his for normal display etc.
   qf   oufile.his  !Open oufile.his for normal display etc.
   -----------------------------------------------------------------------------
    
   20-Aug-05 ............. U300  DAMM - BANANAS - WTM ............... PAGE  26
 
   U300.350  Bananas - Definition, Rules, Construction & Display
 
   Free-form-gates  (or  Banana-gates  - Bananas for short) are 2-D regions of
   arbitrary shape which are specified by  a  list  of  X,Y-points  (channel-#
   coordinates).  Each  Banana  on  a given BAN-file is stored and recalled by
   means of an identification number (ID #). Attempts  to  store  two  Bananas
   with the same ID will be rejected. The rules for Bananas are listed below:
 
   (1) Banana coordinates nust be given in CLOCKWISE order.
   (2) The Banana is formed by connecting X,Y-points with strait lines.
   (3) The last point is connected to the first by the program.
   (4) No line segment of the Banana should intersect another.
   (5) A maximum of 63 points may be specified for any one Banana.
   (6) A maximim of 880 Bananas may be stored on a given BAN-file.
 
   Bananas may be displayed in two different forms (OPEN and CLOSED).
 
   A  CLOSED  Banana  is  one which has just been recalled from or stored on a
   BAN-file (i.e. there is an exact image on disk). There  may  be  up  to  20
   CLOSED  Bananas  displayed  at once. You can do the following things with a
   CLOSED Banana:
 
   GET      -  recall from disk (prompted for ID)     by typing  G
   OPEN     -  for modification (change to OPEN)      by typing  O
   FORGET   -  delete from display                    by typing  F
   KILL     -  delete from display and BAN-file       by typing  K
   TOTALIZE -  counts contained within Banana         by typing  T
   PROJECT  -  (X & Y) and save on DAMQ8Q.SPK         by typing  P
 
   An OPEN Banana is one which is open for creation or  modification.  If  the
   Banana  is  being  newly  created  there  will be no corresponding image or
   partial image on a BAN-file. Only one such Banana  can  exist  at  a  given
   time. You can do the following things with a OPEN Banana:
 
   ADD      -  X,Y-point at cursor position           by typing  A
   INSERT   -  X,Y-point at cursor position           by typing  I
   MOVE     -  nearest X,Y-point to cursor position   by typing  M
   SAVE     -  on BAN-file (prompted for ID)          by typing  S
   REPLACE  -  on BAN-file (with old ID)              by typing  R
   ZERO     -  all X,Y-points                         by typing  Z
   TOTALIZE -  counts contained within Banana         by typing  T
   PROJECT  -  (X & Y) and save on DAMQ8Q.SPK         by typing  P
 
   All  Banana  references  are  made in cursor mode. ADD, INSERT, MOVE, SAVE,
   REPLACE and ZERO refer only to the OPEN Banana.  Other  references  (except
   for  GET)  are  made  by  moving  the cursor such that it is closer to some
   point on the Banana of interest than it  is  to  any  point  on  any  other
   Banana.
 
               ALL BANANAS MUST BE CONSTRUCTED IN CLOCKWISE ORDER
    
   20-Aug-05 ............. U300  DAMM - BANANAS - WTM ............... PAGE  27
 
   U300.360  Bananas - Projections
 
   Projections via the PJ-command
 
   Data  which fall within and on the boundries of a Banana are projected onto
   the X-axis of a coordinate system which is rotated through  an  angle  DEGR
   with  respect  to  the  systen  in  which data channel-# (0,0) falls at the
   origin and the first and second indices of the histogram array  define  the
   X-  and  Y-axis,  respectively.  Channel-# XP in the projected histogram is
   calculated from channel-# X,Y in the 2-D histogram by an expression of  the
   following form:
 
         XP=A+COS(DEGR)*X+SIN(DEGR)*Y
 
         where,
 
         A=0.0                               For DEGR =   0 -  90
 
         A=-COS(DEGR)*XMAX                   For DEGR =  90 - 180
 
         A=-COS(DEGR)*XMAX-SIN(DEGR)*YMAX    For DEGR = 180 - 270
 
         A=-SIN(DEGR)*YMAX                   For DEGR = 270 - 360
 
 
   XMAX  and  YMAX  are  the  "dimensions" of the 2-D histogram. The effect of
   this transformation is  to  make  all  channel  numbers  in  the  projected
   histogram positive.
 
   NOTE:  The  "length"  of  the  projected  histogram  may  be  as  large  as
          SQRT(XMAX**2+YMAX**2).
 
   Projections via the P-command
 
   Each time DAMM is executed it  will  create  a  new  version  of  the  file
   DAMQ8Q.SPK  for  the  storage  of  projections. The file is only created if
   projrctions are actually made.
 
   When you project  a  Banana,  both  X-  and  Y-projections  are  stored  on
   DAMQ8Q.SPK  under  the  ID-numbers  displayed.  These 1-D histograms may be
   displayed (or otherwise used) in the normal manner for a SPK-file. Use  the
   P-qualifier  to  display  spectra  from  DAMQ8Q.SPK  without   explicitally
   opening it. For example, to display ID numbers 1,3,5 from DAMQ8Q.SPK, type:
 
   D P 1,3,5
    
   20-Aug-05 ............. U300  DAMM - FITTING - WTM ............... PAGE  28
 
   U300.400  FITTING - Introduction & General Features
 
   You  specify  how fitting is to be carried out by supplying a number of Fit
   Specification Data Sets which may be given in  any  order.  Many  of  these
   have  default  values  (see  SEC#  U300.450).  After the fitting process is
   specified, one or more Fit Requests are entered. Subsequently, some or  all
   of  the  Fit  Specifications  may  be changed and more Fit Requests entered
   etc. etc.
 
                                GENERAL FEATURES
 
   (1)....Fit specifications may be entered interactively or read from a  file
          or a combination of the two methods may be used.
 
   (2)....Peak  and background intensities are determined in a weighted linear
          least-squares  fit  while  peak  positions,  widths,  and  asymmetry
          parameters  are  determined  by  a  non-linear  least-squares search
          (either Gradient search or Gauss method -  See  SEC#s  U300.490  and
          U300.500).
 
   (3)....Peak  positions  may  be typed in or selected interactively or found
          automatically.
 
   (4)....Spectra are fitted one section at a time and can  be  no  more  than
          2048 channels in length.
 
   (5)....In  the gradient search mode (FIT command), each section may contain
          a total of 30 peaks and background terms. That  is,  the  number  of
          linear  coefficients  to  be  determined in the linear least squares
          fit (# of peaks plus # of background terms) may not  exceed  30.  In
          the  Gauss  mode,  only  5  peaks  are  allowed and asymmetry is not
          supported.
 
   (6)....Initial values of peak positions, widths  and  asymmetry  parameters
          must  be  specified  by  the  user.  Different  values  of width and
          asymmetry may be assigned to each peak or all peaks may be  assigned
          the same values.
 
   (7)....The  FWHM  for  peaks  in  a   section   may   vary   independently,
          conditionally,  or  be  held  fixed.  All  peaks in a section may be
          forced to have the same width or fixed relative widths.
 
   (8)....Peak positions may be adjusted or held fixed.
 
   (9)....Peaks may be gaussian or asymmetric (see SEC# U300.480 & U300.540
 
   (10)...The background may be specified (by  up  to  50  X,Y-points)  or  be
          determined  in  the  fit.  If  determined in the fit, the background
          takes the form, Y = A + B*X + C*X*X + D*X*X*X + . .  .  .  with  the
          number of terms in the power series specified by the user.
 
   (11)...The  output  includes  the  Fit  Specification Data, peak positions,
          widths, energies, areas and uncertainties (in percent) as well as  a
          printer plot of the fit on a 0.5 to 5 cycle plot.
    
   20-Aug-05 ............. U300  DAMM - FITTING - WTM ............... PAGE  29
 
   U300.410  Commands for 1-key (cursor mode)
 
   One-Key cursor commands (valid following a DS or DSX command)
 
   Type: C  - To enter cursor-mode
 
   P/UP   Add/Delete peak to Library (pos specified by cursor)
 
   M/M    Move nearest displayed peak to cusor pos (FW, ASYM unchanged)
 
   X/UX   Fix/Free  peak position (for displayed peak nearest to cursor)
 
   W/UW   Fix/Free  peak width    (for displayed peak nearest to cursor)
 
   L/UL   Fix/Free  Lo-Side ASYM  (for displayed peak nearest to cursor)
 
   H/UH   Fix/Free  Hi-Side ASYM  (for displayed peak nearest to cursor)
 
   O/UO   Turn peak  ON/OFF       (for displayed peak nearest to cursor)
 
   B/UB   Add/Delete background point at cursor position
 
   <--    Set Expand Region   Lo-Limit
 
   -->    Set Expand Region   Hi-Limit
 
   [      Set Fit Region      Lo-Limit
 
   ]      Set Fit Region      Hi-Limit
 
   V      Make Expand- & Fit-region markers visible/invisible (toggles)
 
   J      Junk (delete) all expand- & fit-region markers
 
   / or ? Display chan#, cursor Y-value, chan contents
 
   S      Disp sum, cent & fwhm of Fit-Reg - DAT([),DAT(]) defines BGD
 
   A      Disp sum, cent & fwhm of Fit-Reg - CUR([),CUR(]) defines BGD
 
   Q      Return from cursor control routine
 
   E      Expand display (region defined by <--  -->)
 
 
 
   See SEC# U300.110 or Type: h mous for use of Mouse Buttons.
 
    
   20-Aug-05 ............. U300  DAMM - FITTING - WTM ............... PAGE  30
 
   U300.420  Setup Commands
 
   Commands for entry of peak, background & skip-regions ---------------------
 
   PZOT                   - Zero the Peak Library
   PK     X,W,ASLO,ASHI   - List of complete peak specifications
 
   BZOT                   - Delete Fixed Background array
   BACK   X1,Y1 X2,Y2 ..  - X,Y-points for fixed background
 
   SKIP                   - Without List turns SKIP OFF
   SKIP   I1,I2 J1,J2 ..  - Up to 4 regions to omit from Fit
 
   Commands for defining FWHM, ASYM, WLIM, ALIM, NBC, WOOD, ECAL -------------
 
   FW     FWA,FWB,FWC     - Coefficients for standard width function
   WLIM   FWLO,FWHI       - Variation limit factors for peak widths
 
   ASYM   ASLO,ASHI       - Standard Lo-Side and Hi-Side asymmetries
   ALIM   FALO,FAHI       - Variation limit factors for peak asymmetries
 
   NBC    NBC             - Number of power series terms in variable BGD
   WOOD   ON/OFF          - Turn Woods-Saxon BGD term ON/OFF (default OFF)
                          - ON creates an additional background component
                          - with a Woods-Saxon "jog" under each peak which
                          - is porportional to the peak intensity.
 
   ECAL   ECO,ECA,ECB     - Coefficients for standard energy calibration
 
   Commands for control of non-linear parameter variation --------------------
 
   DPX    XSTEP,DXMAX     - Initial step size and limit for peak pos
   DEL    DEL,DELFAC,NDEL - Initial step size, step size multiplier and
                          - number of DEL-values to use
 
   VB                     - Use Variable Background (the default)
   FB                     - Use Fixed Background if available
 
   VX     KVAR            - Kind of variation for peak positions
   VW     KVAR            - Kind of variation for peak widths
   VALO   KVAR            - Kind of variation for Lo-Side asymmetries
   VAHI   KVAR            - Kind of variation for Hi-Side asymmetries
 
          KVAR = UIND     - says vary Unconditionally, Independently
               = CIND     - says vary   Conditionally, Independently
               = ULOC     - says vary Unconditionally, Locked
               = CLOC     - says vary   Conditionally, Locked
               = FIX      - says keep Fixed - this the default assignmemt
 
   Conditional says hold Fixed if peak so specified.
 
   Unconditional says vary regardless of peak specifications.
 
   Independent says given parameter-types are varied independently.
 
   Locked  says  given parameter-types (width for example) are varied together
          (multiplied by the same factor) in the non-linear search.
    
   20-Aug-05 ............. U300  DAMM - FITTING - WTM ............... PAGE  31
 
   U300.430  Display of Data, Fits and Printer-plots
 
   Commands for general display control --------------------------------------
 
   FIG    N           - Select screen configuration number-N
   WIN    N           - Select window-N for subsequent displays
   LIN/LOG            - Set graphic display to LIN (default) or LOG
   DMM    YMIN,YMAX   - Set display-range (YMIN & YMAX)
   DL     ILO,IHI     - Set display-range (channel# limits)
   DS     ID          - Display spectrum# ID (range defined by DL)
   DS     ID,ILO,IHI  - Display spectrum# ID (DL values replaced)
                        (MAX value of IHI-ILO = 4095)
   DSX    ID          - Display Data defined by Expand Region
   C                  - Enters 1-key cursor-mode
 
   Commands related to display of FITS ---------------------------------------
 
   MON/MOF            - Peak Markers ON/OFF for DF (default = ON)
   DFI                - Set to display (DF ) DATA,FIT,BGD (default)
   DPK                - Set to display (DF ) DATA,FIT,PEAKS,BGD
   DPPB               - Set to display (DF ) DATA,FIT,(PEAKS+BGD),BGD
   DF                 - Display Fit (channel-limits given by Fit-range)
   DF     ILO,IHI     - Display Fit (channel-limits given by ILO,IHI)
   DC     NPK         - Display Calculated peak #  NPK+ RESIDUAL
 
   PRP    XLO,XHI     - Display peaks from Library in range XLO thru XHI
   PRP                - Display all peaks from Library
   PRB                - Display all fixed Bgd-points
   FSTAT              - Display current fit-parameters
   DR                 - List results of last Fit on VDT (terminal)
 
   Commands related to printer-plots of results ------------------------------
 
   KPPL   NONE        - Says do no printer plots
   KPPL   FITS        - Says plot FITS only (the default)
   KPPL   ALL         - Says plot FITS, COMPONENTS and RESIDUALS
 
   PR                 - Print and Plot results of last Fit on printer
 
   Commands related to peak-finding ------------------------------------------
 
   FIND  BIAS,FWHM    - Enables  peak-finding (see SEC# U300.155)
   FIND               - Enables  peak-finding with (BIAS=5, FWHM=5)
   NOFI               - Disables peak-finding
 
   If FIND is enabled (see SEC# U300.155 for general details), DAMM will do  a
   peak  find  within  the  display  region each time a DS (or DSX) command is
   given. An attempt will then be made to add the newly  found  peaks  to  the
   internal  peak  library.  If  a  newly  found  peak is closer than 0.5*FWHM
   channels to an existing library peak, it will not be  added.  Finally,  all
   library  peaks  will  be  marked  on  the  display  in the usual manner. No
   distinction is made between "found peaks" and "manually entered peaks".
    
   20-Aug-05 ............. U300  DAMM - FITTING - WTM ............... PAGE  32
 
   U300.440  FIT Execution Commands
 
   Commands for FIT execution ------------------------------------------------
 
   FIT    ID,ILO,IHI  - Fit Request - (non-linear gradient search)
   GFIT   ID,ILO,IHI  - Fit request - (gaussian method)
   RFIT   ID,ILO,IHI  - Resume FIT/GFIT start with Parms from last Fit
   LFIT   ID,ILO,IHI  - Linear Fit - no non-linear search
                      - (Fit-range specified by ILO,IHI)
   FIT    ID X        - Fit Range specified by cursors (Fit Region)
   GFIT   ID X        - Fit Range specified by cursors (Fit Region)
   RFIT   ID X        - Fit Range specified by cursors (Fit Region)
   LFIT   ID X        - Fit Range specified by cursors (Fit Region)
 
   Ctrl/C             - Terminates Fit-in-prograss
 
   U300.450  FIT Parameters - Saving, Setting, Default
 
   Commands which save FIT parameters in memory library ----------------------
 
   SAV    I,J  - Save all Parms from peaks I thru J of last Fit in PK-LIB
   SAX    I,J  - Save X-Parms   for  peaks I thru J of last Fit
   SAW    I,J  - Save W-Parms   for  peaks I thru J of last Fit
   SAL    I,J  - Save ASL-Parms for  peaks I thru J of last Fit
   SAH    I,J  - Save ASH-Parms for  peaks I thru J of last Fit
   (If I,J ommitted, indicated Parms from ALL peaks are saved)
 
   Commands which set FIT parameters -----------------------------------------
 
   SET- X1,X2          - Set STD WIDTH and ASYM for peaks in range X1-X2
                         values (defined by FWA, FWB, FWC, ASLO, ASHI)
 
   SETW X1,X2          - Set WIDTH for peaks in range X1-X2 to STD value
   SETW X1,X2,WA,WB,WC - Set WIDTH for peaks in range X1-X2 to value
                         defined by WA,WB,WC (FWA,FWB,FWC unchanged)
 
   SETL X1,X2          - Set ASLO  for peaks in range X1-X2 to STD value
   SETL X1,X2,ASLOT    - Set ASLO=ASLOT for peaks in range X1-X2
 
   SETH X1,X2          - Set ASHI  for peaks in range X1-X2 to STD value
   SETH X1,X2,ASHIT    - Set ASHI=ASHIT for peaks in range X1-X2
 
   (If X1,X2,.. omitted, indicated Parms for ALL peaks are set)
 
   List of default FIT parameters --------------------------------------------
 
   DEL   = 0.05    FWLO = 0.5      VX   = CIND    NBC  = 2
   DELFAC=0.25     FWHI = 2.0      VW   = CLOC    WOOD = OFF
   NDEL  = 3       FALO = 0.5      VALO = FIX     KPPL = FITS
   XSTEP = 0.2     FAHI = 2.0      VAHI = FIX
   DXMAX = 5.0     ASLO = 0.0
                   ASHI = 0.0
    
   20-Aug-05 ............. U300  DAMM - FITTING - WTM ............... PAGE  33
 
   U300.460  Commands to Control Relative Peak-intensities
 
   The following commands may be used to fix the  intensity  of  two  or  more
   peaks relative to each other within a section being fitted:
 
   RELI  X,R   Sets relative intensity of library-peak nearest chan-X to be R
   RELI  ZOT   Deletes all relative intensity entries
   RELI  OFF   Disables relative intensity control but saves previous entries
   RELI  ON    Enables  relative intensity control (default)
 
   NOTE:  RELI  specifications  must  be  entered  AFTER  the  associated peak
   library entries are completed. RELI has no effect unless there two or  more
   peaks  with specified relative intensities within the region being fitting.
   R may be in any units however there is an 8 character limit on  the  number
   entered!  Peak-areas will be in the same ratios as the relative intensities
   specified ONLY if width & asymmetry parameters are the same for each peak.
 
   U300.470  FIT Specification Details
 
                   PK  Data Set - Complete Peak Specifications
 
   The PK Data Set accomodates a full specification of the characteristics  of
   each  individual  peak.  Up  to 100 peaks may be included in the list. Each
   peak is specified by the following parameters.
 
   X......Gives the initial peak position in channels.
 
   W......Specifies the initial peak FWHM in channels. If  not  entered,  FWHM
          is set to the value specified by FWA, FWB & FWC.
 
   ASLO...Specifies the Lo-Side asymmetry parameter.
 
   ASHI...Specifies the Hi-Side asymmetry parameter.
 
                              Other Specifications
 
   ECO,ECA,ECB...Defines  the  spectrum  energy  calibration (not required for
          fitting) through the relation;
 
          E = ECO + ECA*(CHAN #) + ECB*(CHAN #)**2
 
   FWA,FWB,FWC...Defines the peak  WIDTH  as  a  function  of  channel  number
          through the relation;
 
          FWHM(CHANNELS)=FWA+FWB*SQRT(CHAN #)+FWC*(CHAN #)
 
   ASLO,ASHI...Are  the  initial  values  of the Lo-Side and Hi-Side asymmetry
          parameters. If this specification is used, the initial  values  will
          be the same for all peaks.
 
   FWLO...Is  the  minimum  fraction of the initially specified value by which
          any peak width may be reduced.
 
   FWHI...Is the maximum fraction of the initially specified  value  by  which
          any peak width may be increased.
 
                            (continued on next page)
    
   20-Aug-05 ............. U300  DAMM - FITTING - WTM ............... PAGE  34
 
   U300.470  FIT Specification Details (continued)
 
   FALO...Is  the  minimum  fraction of the initially specified value by which
          any peak asymmetry parameter may be reduced.
 
   FAHI...Is the maximum fraction of the initially specified  value  by  which
          any peak asymmetry parameter may be increased.
 
   DEL....Specifies  the  fraction  by  which  the  peak  width  and  the peak
          asymmetry parameters are to be changed in  each  step  of  the  non-
          linear search. For example,
 
                      (NEW WIDTH) = (OLD WIDTH)*(1.0+-DEL)
 
   DELFAC-Is  a  factor  by  which  the  current value of DEL is multiplied in
          order to obtain a new (smaller) value. Typically one starts  with  a
          fairly  large value of DEL (say 0.02 to 0.05) and subsequently makes
          one or more reductions in order to  achieve  a  greater  convergence
          speed.
 
   NDEL...Is the number of DEL-values to be used
 
   XSTEP..Is  the maximum amount (in channels) that a peak may be moved in any
          one step in the  non-linear  search  for  the  best  fit.  XSTEP  is
          reduced  at  the  same time and by the same factor (DELFAC) that DEL
          is reduced.
 
   DXMAX..Is the maximum number of channels (either + or -) that any  peak  is
          allowed to be moved from its original position.
 
   SKIP...Defines  up  to  four  regions  within the Fit Range which are to be
          ignored in doing the fit.
 
   KPPL=..NONE says do no printer plots.
 
   KPPL=..FITS says plot the FIT (experimental and calculated spectrum on  the
          same graph).
 
   KPPL=..ALL  says  plot  the  FIT (as in KPPL...=FITS) and in addition, plot
          each component (calculated peak)  together  with  the  corresponding
          "residual  component".  What  do you mean by residual component, you
          ask. When plotting the  Ith  peak  we  calculate  the  Ith  residual
          component  by  subtracting  any background (specified or calculated)
          as well as  all  calculated  peaks  other  than  the  Ith  from  the
          experimental spectrum.
 
   NBC....Denotes  the  number  of background components to be included in the
          power series. NBC=2 Says use the form Y=A+B*X  and  NBC=4  says  use
          Y=A  +  B*X + C*X*X + D*X*X*X. The number of peaks in a section plus
          NBC must not exceed 30.
 
                            (continued on next page)
    
   20-Aug-05 ............. U300  DAMM - FITTING - WTM ............... PAGE  35
 
   U300.470  FIT Specification Details (continued)
 
   WOOD...ON/OFF says turn Woods-Saxon background term ON/OFF. The default  is
          OFF.  If  WOOD is ON, an additional background component is included
          which has  a  Woods-Saxon  type  "jog"  under  each  peak  which  is
          porportional to the peak intensity. The jog form is given by:
 
          Y    =  1.0/(1.0+EXP(ARG))    ;where
          ARG  =  4.714*(X0-X)/FWHM     ;and
          X0=peak-position, X=channel-of-interest, and FWHM=peak-FWHM.
 
          The  use  of such a background form could be helpful in the analysis
          of weak peaks which are located on the  low-energy  side  of  strong
          peaks. You will have to be the judge.
 
   U300.480  Discussion of Peak Shapes
 
   The most general peak shape allowed is given by
 
                    YL=EXP(-(X-XO)**2/(A**2*(1+ASLO*(XO-X)/A)
                    YH=EXP(-(X-XO)**2/(A**2*(1+ASHI*(X-XO)/A)
 
   Where  A  is  the gaussian Half-Width at 1/e max and YL and YH describe the
   curve on the Lo- and Hi-Sides, respectively. If  all  asymmetry  parameters
   are  held  to zero, the shape is gaussian. The ASLO/ASHI parameters broaden
   the Lo/Hi sides of the peak and result in  an  expodential  fall-off  (like
   EXP(-(XO-X)/(A*ASLO))  for  example)  As  you  move  far away from the peak
   maximum (i.e. channel  XO).  To  get  some  idea  of  what  size  asymmetry
   parameters to use see Fig 1.
 
   U300.490  Gradient-search Method (FIT request)
 
   Each  time  the  program encounters a Fit Request, it searches the complete
   Library and includes in the Fit all peaks which are ON and whose  positions
   lie within the Range of Fit (i.e. between ILO and IHI).
 
                   GENERAL PROCEDURE FOR THE NON-LINEAR SEARCH
 
   (1)....The  initial  values  of  all  parameters  which  are  to  vary in a
          non-linear way are set to the initial values specified by the user.
 
   (2)....Each individual parameter is changed (both increased and  decreased)
          by  an  amount  determined  by  DEL or XSTEP in order to establish a
          "direction" (increase or decrease) for each parameter.
 
   (3)....All parameters are changed in the direction determined in  step  (2)
          and  in  step sizes determined by DEL and XSTEP until the Quality of
          Fit is no longer improved.
 
   (4)....Steps (2) and (3) are repeated until no improvment in  the  Fit  can
          be made
 
   (5)....DEL  and  XSTEP  are  multiplied by DELFAC and steps (2) and (3) are
          repeated until no improvment in the Fit can be made.
 
   (6)....Step (5) is repeated (NDEL-1) times.
    
   20-Aug-05 ............. U300  DAMM - FITTING - WTM ............... PAGE  36
 
   U300.500  Gaussian Method (GFIT request)
 
   The GFIT (Gauss-method)  fit  request  initiates  an  alternate  non-linear
   procedure. Commands are:
 
   GFIT ID,ILO,IHI
   or
   GFIT ID X
 
   This  command  initiates  a nonlinear least-squares search by Gauss' method
   as modified by Marquardt.  (See, for example, P.R. Bevington's  book, "Data
   Reduction  and  Error  Analysis for the Physical Sciences", p. 235 ff.  The
   routines used in GFIT are not Bevington's, but are those of M.J.  Saltmarsh
   from the SEL 840-A program PKFT.)
 
   The  search  continues  until  chi-squared  per degree of freedom (QFN) has
   changed by less than 0.0001 or until  25  iterations  have  occurred.   The
   search  may  be  resumed by the RFIT command.  The iteration number and QFN
   appear on the right side of the screen.
 
   The printer output from GFIT includes  absolute  error  estimates  for  the
   peak  positions,  widths,  and areas which are derived from the correlation
   matrix of the fit.  The percentage error in  the  area  is  printed  in the
   column labeled PCE.
 
   The  DAMM  commands  VW ULOC or VW CLOC (the default option) ae interpreted
   to mean that the widths of the peaks are to be the same and vary  together.
   Widths  may  be  varied  independently  by  VW UIND or VW CIND.  Individual
   widths or positions may be  frozen  or  released  by  the  standard  cursor
   commands  of DAMM.  If one or more widths are to be kept fixed while others
   are varied, the command VW CIND or VW CLOC should be given; if UIND or ULOC
   is given, the instruction to fix is ignored.
 
   If GFIT is chosen, the program  attempts  to  estimate  the  width  of  the
   tallest  peak  for  its initial guess of width.  If unsuccessful it reverts
   to the standard DAMM procedure of using whatever was stored from  the  last
   previous FW, SETW commands (or the default option, which is FW = 5).
 
   At  present  The  GFIT  request  is  limited  to  fitting  a sum of up to 5
   Gaussian peaks with a linear background.  The parameters ASLO and  ASHI for
   asymmetric peaks are ignored.
    
   20-Aug-05 ............. U300  DAMM - FITTING - WTM ............... PAGE  37
 
   U300.510  Estimated Uncertainties in Peak Areas
 
   Estimated  uncertainties  should  always  be   viewed   with   considerable
   skepticism,  especially  when  non-linerar  as  well  as   linear   fitting
   processes  are involved, as it is here. The uncertainties in the peak areas
   estimated by both the FIT and GFIT procedures  are  rather  "standard"  and
   involve  CHISQ  (of  the  overall  fit) as well the diagonal element of the
   inverse matrix corresponding  to  the  peak  intensity  in  question.  This
   inverse  matrix  is  found  in  the  standard  linear least-squares fitting
   process. See a book like "Bevington" or Cziffra  et.  al.  UCRL-8523,  1958
   for a real discussion of this subject. I have used:
 
   D(J) = SQRT(AINV(J,J)*QFN)     where;
 
   D(J)      = the estimated uncertainty in the J-th fit parameter B(J)
               (there is a B(J) for each peak-area(J))
 
   QFN       = CHISQ/(#data-points  -  #adjustable-parameters)
 
   AINV(J,J) = the J-th diagonal element of the inverse matrix found in the
               linear least-squares fitting process.
 
   PCE(J)    = 100*D(J)/B(J)  =  percent uncertainty in J-th peak-area.
 
 
   U300.520  Reading Fit Results from damm.log
 
   The  table  of  fit-results  recorded  on  damm.log,  as  a  result of a PR
   command, includes flags of the form LAB$ to facillitate  the  location  and
   decoding  of  relevant  data by other programs. Formats associated with the
   different line-labels are listed below.
 
   Label  Format
   TIT$   (1X,15X,2I6,6X,20A4)
   DEL$   (1X,2F10.0,I10,6F10.0)
   SKP$   (1X,8I10)
   CAL$   (1X,8F10.0,I10)
   VAR$   (1X,6(6X,A4),I10)
   FIT$   (1X,5F10.0,3F8.0,4F7.0,I5,I7)
   GFI$   (1X,5F10.0,3F8.0,4F7.0,I5,I7)
   BGD$   (1X,2F10.2)
   QFN    (1X,6X,F10.0,11X,F10.0)
 
   Alternatively,   one  may  make  use  of  the  routines  contained  in  the
   internally documented demonstration program  samred.  The  source  of  this
   program is in /usr/users/milner/Ddamm/samred.f.
 
   U300.530  Common Problems
 
   (1)....If  you  define  the  standard FWHM (via command: FW FWA,FWB,FWC) or
          standard asymmetry parameters (via command:  ASYM  ASLO,ASHI),  this
          does  not  re-define  such  parameters for previously defined peaks.
          You must use SETW to do this (see SEC# U300.450).
    
   20-Aug-05 ............. U300  DAMM - FITTING - WTM ............... PAGE  38
 
   U300.540  Peak Shape vs Asymmetry (Log plot)
 
            YL=EXP(-(X-XO)**2/(A**2*(1+ASLO*(XO-X)/A)
            YH=EXP(-(X-XO)**2/(A**2))
   Y = 100  ------------------------------------------------------------------
            SYMBOL  ASLO                            0
                 0   0.0                          11 00
                 2   0.2                         1     0
                 4   0.4                        1
                 6   0.6                       10       0
                 8   0.8                      14
                 1   1.0                     140         0
                                            162
              FWHM = 12                    1640           0
                                          18 2
                                         1864
                                            20             0
                                        1864
                                       186
                                      1   420               0
                                     1 86
                                      8  4
                                    1  6  20                 0
                                   1 8  4
                                  1 8 6  2
 
                                 1 8 6 4  0                   0
                                1
                                  8 6 4 2
   Y = 10  ------------------- 1 8 -------------------------------------------
                              1    6     0                     0
                             1  8    4 2
                                  6
                            1  8    4
                           1  8  6    2
                                        0                       0
                          1  8  6  4
                         1
                        1   8  6     2
                           8      4
                       1               0                         0
                      1   8   6  4
                                    2
                     1   8   6
                    1           4
                   1    8   6
                       8           2  0                           0
                  1        6   4
                 1    8
                          6
                1    8        4   2
               1         6
                    8                0                             0
              1    8         4
   Y = 1   -------------------------------------------------------------------
 
                                    Figure 1
    
   20-Aug-05 ........... U300  DAMM - CUSTOMIZING - WTM ............. PAGE  39
 
   U300.550  Screen Configurations
 
   The sizes and locations of display windows are controlled by  one  or  more
   of the commands - FIG, FIGF or FIGI.
 
   FIG ID - Sets the number and size of display windows to that specified
          by configuration number ID. A given configuration ID number
          may specify up to 20 display windows. The default configuration
          library contains ID-numbers 1 thru 16. Type: H FIG for a
          display indicating the default sceen configurations or just
          Type: FIG 1, FIG 2 ... FIG 16 and see what you get.
 
   FIGF FILENAME - Requests that a new configuration library be read from
          a file named FILENAME. The file /usr/hhirf/fig.dat (or the file
          /home/upak/fig.dat) which contains the default configuration is
          listed on the next next two pages. The **** in col-1 of the table
          denote comment lines and are ignored in processing.
 
   ****DEFAULT SCREEN-CONFIGURATION TABLE
   ****--------------------------------------------------
   ****FIG-ID   X0(PIX)   Y0(PIX)    W(PIX)    H(PIX)
            1        30        84       540       480
   ****
            2         0        39       480       480
            2       495        39       480       480
   ****
            3        30       369       540       300
            3        30        39       540       300
            3       585        39       390       390
   ****
            4         0       369       480       300
            4         0        39       480       300
            4       495       369       480       300
            4       495        39       480       300
   ****
            5        30       489       480       195
            5        30       264       480       195
            5        30        39       480       195
            5       525        39       435       435
   ****
            6         0       489       480       195
            6         0       264       480       195
            6         0        39       480       195
            6       495       489       480       195
            6       495       264       480       195
            6       495        39       480       195
   ****
            7        30       534       480       135
            7        30       369       480       135
            7        30       204       480       135
            7        30        39       480       135
            7       525        39       435       435
 
                    (continued on next page)
    
   20-Aug-05 ........... U300  DAMM - CUSTOMIZING - WTM ............. PAGE  40
 
   ****DEFAULT SCREEN-CONFIGURATION TABLE (continued)
   ****--------------------------------------------------
   ****FIG-ID   X0(PIX)   Y0(PIX)    W(PIX)    H(PIX)
            8         0       534       480       135
            8         0       369       480       135
            8         0       204       480       135
            8         0        39       480       135
            8       495       534       480       135
            8       495       369       480       135
            8       495       204       480       135
            8       495        39       480       135
   ****
            9        30       579       480       105
            9        30       444       480       105
            9        30       309       480       105
            9        30       174       480       105
            9        30        39       480       105
            9       525        39       435       435
   ****
           10         0       579       480       105
           10         0       444       480       105
           10         0       309       480       105
           10         0       174       480       105
           10         0        39       480       105
           10       495       579       480       105
           10       495       444       480       105
           10       495       309       480       105
           10       495       174       480       105
           10       495        39       480       105
   ****
           11         0        39       960       525
   ****
           12         0       354       960       270
           12         0        54       960       270
   ****
           13         0       489       960       195
           13         0       264       960       195
           13         0        39       960       195
   ****
           14         0       549       960       135
           14         0       384       960       135
           14         0       219       960       135
           14         0        54       960       135
   ****
           15       240        39       750       750
   ****
           16         0        39       645       645
   ---------------------------------------------------------------------------
   FIG-ID  = Configuration ID number to be associated with window.
   X0(PIX) = X-coordinate of upper left corner of window in pixels.
   Y0(PIX) = Y-coordinate of upper left corner of window in pixels.
   W(PIX)  = Width  of window in pixels.
   H(PIX)  = Height of window in pixels.
   ---------------------------------------------------------------------------
 
   FIGI   Restores the configuration library to the default state.
    
   20-Aug-05 ........... U300  DAMM - CUSTOMIZING - WTM ............. PAGE  41
 
   U300.560  Graphic Screen Color Mapping
 
   The  colors (or grey scale) is controlled by means of a color map which has
   40 entries that  specify  (red,  green,  blue)  intensities  in  the  range
   0-65535.  The  default  table  contained  in  /usr/hhirf/cmap.dat   or   in
   /home/upak/cmap.dat is listed below:
 
       RED  GREEN   BLUE   ;ENTRY# - NORMAL USE ------------------------------
 
         0      0      0  ;01 - BLACK
     65535  65535  65535  ;02 - 1-D DISPLAY - COL(1) - FIT DATA
     65535      0      0  ;03 - 1-D DISPLAY - COL(2)
         0  65535      0  ;04 - 1-D DISPLAY - COL(3) - FIT CALC
         0      0  65535  ;05 - 1-D DISPLAY - COL(4)
     65535  65535      0  ;06 - 1-D DISPLAY - COL(5) - FIT BACK
     65535      0  65535  ;07 - 1-D DISPLAY - COL(6)
         0  65535  65535  ;08 - 1-D DISPLAY - COL(7)
     65535  65535      0  ;09 - 1-D DISPLAY - COL(8)
     65535  65535      0  ;10 - NOT USED FOR NOW
         0      0  32767  ;11 - 2-D COLOR DISPLAY
         0      0  65535  ;12 - 2-D COLOR DISPLAY
     32767      0  32767  ;13 - 2-D COLOR DISPLAY
     65535      0  65535  ;14 - 2-D COLOR DISPLAY
     32767      0      0  ;15 - 2-D COLOR DISPLAY
     65535      0      0  ;16 - 2-D COLOR DISPLAY
     32767  32767      0  ;17 - 2-D COLOR DISPLAY
     65535  65535      0  ;18 - 2-D COLOR DISPLAY
     32767  32767  32757  ;19 - 2-D COLOR DISPLAY
     65535  65535  65535  ;20 - 2-D COLOR DISPLAY
     15000  15000  15000  ;21 - 2-D GREY-SCALE DISPLAY
     20600  20600  20600  ;22 - 2-D GREY-SCALE DISPLAY
     26200  26200  26200  ;23 - 2-D GREY-SCALE DISPLAY
     31800  31800  31800  ;24 - 2-D GREY-SCALE DISPLAY
     37400  37400  37400  ;25 - 2-D GREY-SCALE DISPLAY
     43000  43000  43000  ;26 - 2-D GREY-SCALE DISPLAY
     48600  48600  48600  ;27 - 2-D GREY-SCALE DISPLAY
     54200  54200  54200  ;28 - 2-D GREY-SCALE DISPLAY
     59800  59800  59800  ;29 - 2-D GREY-SCALE DISPLAY
     65535  65535  65535  ;30 - 2-D GREY-SCALE DISPLAY
     32767  32767  32767  ;31 - NOT USED FOR NOW
     32767  65535  65535  ;32 - NOT USED FOR NOW
     65535      0      0  ;33 - NOT USED FOR NOW
     65535  65535  65535  ;34 - GCOR(1)
     65535      0      0  ;35 - GCOR(2), SAM PK MARK,  FIT VAR MARK
         0  65535      0  ;36 - GCOR(3), 1-D PK LAB, 2-D BAN & EX-MARK
         0      0  65535  ;37 - GCOR(4), 1-D REG MARK, FIT VAR MARK
     65535  65535      0  ;38 - GCOR(5), CURSOR
     65535      0  65535  ;39 - GCOR(6)
         0  65535  65535  ;40 - GCOR(7)
 
   Different color mapping is accomplished by the CMAP command as shown below:
 
   CMAP FILENAME  ;Processes a file FILENAME of the structure shown above
                  ;and maps as  specified  therein.  The new mapping only
                  ;takes place subsequent to the next FIG command.
    
   20-Aug-05 ........... U300  DAMM - CUSTOMIZING - WTM ............. PAGE  42
 
   U300.570  Graphic Screen Black & White Mapping
 
   When  using  black  &  white  monitors,  the table /usr/hhirf/bmap.dat, (or
   /home/upak/bmap.dat) listed below, may be more useful.  If  you  are  using
   the  REVV  mode,  then  table /usr/hhirf/bmapr.dat or /home/upak/bmapr.dat,
   not listed here, should be used as a template. You will  probably  need  to
   make  other  adjustments  in  order  to  achieve semi-satisfactory results.
   Note: that table entries are labeled with their uses.
 
       RED  GREEN   BLUE   ;ENTRY# - NORMAL USE ------------------------------
 
         0      0      0  ;01 - BLACK
     65535  65535  65535  ;02 - 1-D DISPLAY - COL(1) - FIT DATA
     65535  65535  65535  ;03 - 1-D DISPLAY - COL(2)
     65535  65535  65535  ;04 - 1-D DISPLAY - COL(3) - FIT CALC
     65535  65535  65535  ;05 - 1-D DISPLAY - COL(4)
     65535  65535  65535  ;06 - 1-D DISPLAY - COL(5) - FIT BACK
     65535  65535  65535  ;07 - 1-D DISPLAY - COL(6)
     65535  65535  65535  ;08 - 1-D DISPLAY - COL(7)
     65535  65535  65535  ;09 - 1-D DISPLAY - COL(8)
     65535  65535  65535  ;10 - NOT USED FOR NOW
         0      0  32767  ;11 - 2-D COLOR DISPLAY
         0      0  65535  ;12 - 2-D COLOR DISPLAY
     32767      0  32767  ;13 - 2-D COLOR DISPLAY
     65535      0  65535  ;14 - 2-D COLOR DISPLAY
     32767      0      0  ;15 - 2-D COLOR DISPLAY
     65535      0      0  ;16 - 2-D COLOR DISPLAY
     32767  32767      0  ;17 - 2-D COLOR DISPLAY
     65535  65535      0  ;18 - 2-D COLOR DISPLAY
     32767  32767  32757  ;19 - 2-D COLOR DISPLAY
     65535  65535  65535  ;20 - 2-D COLOR DISPLAY
     15000  15000  15000  ;21 - 2-D GREY-SCALE DISPLAY
     20600  20600  20600  ;22 - 2-D GREY-SCALE DISPLAY
     26200  26200  26200  ;23 - 2-D GREY-SCALE DISPLAY
     31800  31800  31800  ;24 - 2-D GREY-SCALE DISPLAY
     37400  37400  37400  ;25 - 2-D GREY-SCALE DISPLAY
     43000  43000  43000  ;26 - 2-D GREY-SCALE DISPLAY
     48600  48600  48600  ;27 - 2-D GREY-SCALE DISPLAY
     54200  54200  54200  ;28 - 2-D GREY-SCALE DISPLAY
     59800  59800  59800  ;29 - 2-D GREY-SCALE DISPLAY
     65535  65535  65535  ;30 - 2-D GREY-SCALE DISPLAY
     32767  32767  32767  ;31 - NOT USED FOR NOW
     32767  65535  65535  ;32 - NOT USED FOR NOW
     65535      0      0  ;33 - NOT USED FOR NOW
     65535  65535  65535  ;34 - GCOR(1)
     45000  45000  45000  ;35 - GCOR(2), FIT PK MARK,  FIT VAR MARK
     65535  65535  65535  ;36 - GCOR(3), 1-D PK LAB, 2-D BAN & EX-MARK
     65535  65535  65535  ;37 - GCOR(4), 1-D REG MARK, FIT VAR MARK
     65535  65535  65535  ;38 - GCOR(5), CURSOR
     65535  65535  65535  ;39 - GCOR(6)
     65535  65535  65535  ;40 - GCOR(7)
