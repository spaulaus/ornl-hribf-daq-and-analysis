   11-Sep-04 ..... U100  README - Directory to HRIBF Software ....... PAGE   1
 
 
   Sec Page Contents
 
   010   1  Introduction
 
   020   2  Documentation List
 
   030   6  Data Acquisition - Introduction
 
   040   7  Data Acquisition - Block Diagram of System
 
   050   8  Data Acquisition - List of Software Modules
 
   060   9  Data Acquisition - List of Things to be Done
 
   070  10  Data Acquisition - Brief Description of System Operation
 
   080  12  Data Processing, Analysis and Display
 
   090  13  HRIBF List Data Tape Structure
 
   100  16  HRIBF List Data File Structure
 
   110  17  HRIBF Histogram Tape and File Structures
 
   120  19  UPAK - Software Xport
 
   130  20  UPAK - Comments
 
 
   U100.010  Introduction
 
   The  HRIBF  facility  provides  a number of programs, software packages and
   libraries which are intended to enable the  user  to  efficiently  acquire,
   process  and  analyze  experimental  data.  These  software  tools are also
   available for export. Documentation is avaliable in hardcopy form. See:
 
   ORPHAS MANUAL   for data acquisition related software
   ORPH UBOOK 2000 for data processing and general purpose software
   ORPH XBOOK 2003 for the experts, the brave or the foolish
 
   Most documents are also available online. You may print or  view  on-screen
   the latest version of the document of your choice by using dodoc & viewdoc.
   If you wish to use viewdoc, then your PATH must include /usr/local/bin.
 
   dodoc...Prints  entire  documents  on  the printer of your choice. To get a
          list of the available documents, just type dodoc.  Some  examples of
          using dodoc & pdoc to print the document readme follow:
 
          dodoc readme         ;Prints one-sided on default Postscript printer
          dodoc readme ps04    ;Prints one-sided on printer ps04
          dodoc readme double  ;Prints two-sided on  ps01 in Rm T-210
 
          pdoc  readme         ;Prints one-sided on cp6000
          pdoc  readme double  ;Prints two-sided on cp6000
 
   viewdoc...Displays  the  document on-screen. To get a list of the available
          documents, just type viewdoc.  With viewdoc you can select  pages to
          print  or  print the entire document.  For example, to view the this
          document readme, type
 
          viewdoc readme
    
   11-Sep-04 ..... U100  README - Directory to HRIBF Software ....... PAGE   2
 
 
   U100.020  Documentation List
 
 
   Data Acquisition
 
 
   The following are available in the ORPHAS Handbook & via dodoc & viewdoc
 
 
   VMEsys-----VME front-end acquisition system - comprehensive
 
   hhirf_adc--HHIRF ADC interface - comprehensive description
 
   modu_setup-CAMAC module setup code - for many different module types
 
   vmexxlib---Describes routines for read/write/setup of CAEN ADCs & TDCs
 
   acqlib-----Describes data acquisition routine library for workstations
 
   pacman-----Data acquisition manager - description, operation, etc
 
   pacor------Physics Acquisition Compiler - Hi-level language for data acq
 
   scad-------CAMAC scaler display (ASCII and soft-meter display)
 
   scop-------CAMAC scaler display (for accelerator operator)
 
   scat-------CAMAC scalers (how to copy to tape/file during data acq)
 
   ddman------Digital Display Manager for beam profile displays
 
    
   11-Sep-04 ..... U100  README - Directory to HRIBF Software ....... PAGE   3
 
 
   U100.020  Documentation List (continued)
 
 
   Data processing, Display and Analysis
 
 
   The following are available in the ORPH UBOOK 2000 & via dodoc & viewdoc
 
 
   scanor-----Provides for user customized histogramming of list-data
 
   lemor------Program to examine, copy, modify-copy list-data (tape or file)
 
   damm-------Display, Analysis and Manipulation Module for histograms
 
   chil-------Comprehensive Histogramming Language for tape/file scanning
 
   tapes------Describes structure of HRIBF list-data tapes & files
 
   swapo------Byte swaps HIS-, SPK- & LDF-files (in place)
 
   asap-------Automatic Spectrum Analysis program
 
   banco------Copies Ban-files to/from edit compatible form
 
   bando------Provides for editing & modifying Ban-files
 
   dvd--------Creation and use of DVDs & CDs in data archiving & processing
 
 
 
   Control Systems
 
 
   The following are available via dodoc & viewdoc
 
 
   dcon-------DRS control system - user interface
 
   stepit-----Stepping motor control for DRS target ladders
 
   rcon-------RMS control system - user interface
 
   hvln-------HV-control & LN-filling-control user interface for RMS
 
 
 
   Libraries
 
 
   The following are available via dodoc & viewdoc
 
 
   lnflib-----Library of routines supporting LN-filling & HV-control on RMS
 
   orphlib----General routine library for HRIBF user support
 
   acqlib-----Describes data acquisition routine library for workstations
 
   vmexxlib---Describes routines for read/write/setup of CAEN ADCs & TDCs
 
    
   11-Sep-04 ..... U100  README - Directory to HRIBF Software ....... PAGE   4
 
 
   U100.020  Documentation List (continued)
 
 
   Miscellaneous Utilities
 
 
   The following are available in the ORPH UBOOK 2000 & via dodoc & viewdoc
 
 
   baco-------Converts Concurrent "backup tapes" to "stu format"
 
   charge-----Predicts equilibrium charge state distributions (gas & solid)
 
   exab-------Exabyte exerciser
 
   fex--------File examine program
 
   fitu-------X,Y fitting via Spline, Linear & non-Linear LSQ (with display)
 
   funkyfit---X,Y fitting via linear-least-squares method
 
   formats----List data structure of HRIBF tapes and files (LDFs)
 
   kineq------Kinematics & Q-values (relativistic & non-relavistic)
 
   mep--------Describes energy calibration of HRIBF Tandem by time-of-flight
 
   nuci-------Nuclear information program
 
   stope------Stopping power calculations with    graphical display
 
   stopi------Stopping power calculations with no graphical display
 
   stopx------Stopping power and range calculations
 
   stu--------Simple Transfer Utility for xfer of files between systems
 
   txx--------Text formatting program
 
   txxps------Converts output of program txx to postscript
 
   upak-------Describes HRIBF UPAK, How to get it, tips on using it
 
   upak.ftp---Describes available versions of UPAK and how to retrieve
 
    
   11-Sep-04 ..... U100  README - Directory to HRIBF Software ....... PAGE   5
 
 
   U100.020  Documentation List (continued)
 
   For the Experts only
 
 
   The following are available in the ORPH XBOOK 2003
 
 
   acqsetup---Setup of workstation and VME for data acquisition
 
   alphahard--Alpha Power Supply CPU Board
 
   multivme---Describes setup & use of multiple VME crates for data acq
 
   trigger----Describes VME event trigger module used at HRIBF
 
   lnf--------LN2 filling system - workstation setup
 
   lnflib-----LN2 filling system - routine library
 
   lnfutil----LN2 filling system - asorted utilities
 
   rmslib-----RMS control system - routine library
 
   morerms----RMS control system - routine library (more)
 
   moredrs----DRS conrtol system - routine library
 
   rmsigc-----RMS/DRS ion-gauge readout & control routine library
 
   varian-----Varian - interactive readout & control
 
   txxps------TXX to postscript converter
 
   poblis-----Displays/logs tables generated by program pacor
 
    
   11-Sep-04 ........ U100  README - HRIBF Data Acquisition ......... PAGE   6
 
 
   U100.030  Data Acquisition - Introduction
 
 
   This document is intended to introduce the novice user to the  architecture
 
   of  the  HRIBF data acquisition system and to serve as a directory to other
 
   documents which will provide the details needed for a successful  operation
 
   of the system.
 
 
   This  document  is  not  to  be  viewed  as  a cookbook for preparation and
 
   operation of the HRIBF data acquisition system. But it is hoped  that  this
 
   document  will provide the new user with a better "feel" for how the system
 
   operates and make the system seem less mysterious and arbitrary.
 
    
   11-Sep-04 ........ U100  README - HRIBF Data Acquisition ......... PAGE   7
 
 
 
   U100.040  Data Acquisition - Block Diagram of System
 
 
 
   The HRIBF Data Acquisition System consists of three major components:
 
   (1)-A microprocessor controlled VME system with interfaces to  CAMAC,  Fera
       and Fastbus data acquisition hardware.
 
   (2)-A  Linux workstation connected to the VME system via a private ethernet
       and connected to other facilities via a 100 megabit/sec LAN.
 
   (3)-A large RAID file server connected to the  Linux  workstation  via  the
       100 megabit/sec LAN.
 
 
 
                VME                         LINUX              RAID
                SYSTEM                      WORKSTATION        FILE SERVER
               -----------------           ---------------    ---------------
               |Microprocessor |           |             |    |             |
               |               |           | modu_setup**|    | General     |
     CAMAC     |Interface      |           |             |    | File        |
   <---------->|Modules        |           | pacor**     |    | Storage     |
               |               |           |             |    |             |
     FERA      |Acquisition    |           | pacman      |    | Source code |
   <---------->|Readout,       |           |             |    |             |
               |Control &      |           | logger      |    | Executables |
     FASTBUS   |Communication  |  PRIVATE  |             |    |             |
   <---------->|Software       |<--------->| pftoipc     |    | LDFs        |
               |               |  ETHERNET |             |    |             |
     OTHER     |               |           | tape        |    | Histograms  |
   <---------->|               |           |             |    |             |
               |               |           | scanor      |    |             |
               |               |           |             |    |             |
     SYS BUSY  |               |           | damm        |    |             |
   <-----------|               |           |             |    |             |
               |               |           | scad        |    |             |
     EVENT     |               |           |             |    |             |
   ----------->|Trigger Module |           | scop        |    |             |
     TRIGGER   |               |           |             |    |             |
               |               |           |             |    |             |
               -----------------           ---------------    ---------------
                                                 |                   |
                                                 |                   |
                                                 |  100 megabit LAN  |
                                      <-----------<----------------->-------->
 
 
   ** Used in initialization phase only
    
   11-Sep-04 ........ U100  README - HRIBF Data Acquisition ......... PAGE   8
 
 
   U100.050  Data Acquisition - List of Software Modules
 
   ---------------------------------------------------------------------------
   VME            Directly controls all front-end operations
   Microprocessor Reads/writes CAMAC, Fera, Fastbus & other modules
   Code           Detects event trigger
                  Executes data readout as specified by pac program
                  Forms data packets & sends to host
                  Manages all communication with host workstation
   ---------------------------------------------------------------------------
   modu_setup     Provides for the setup of certain CAMAC modules
                  which require initialization.
                  See ORPHAS Tab-9 or Type: viewdoc modu_setup
   ---------------------------------------------------------------------------
   pacor          Compiles readout specification program (pac)
                  Displays diagnostic information
                  Loads specification table into VME microprocessor
                  See ORPHAS Tab-2 or Type: viewdoc pacor
   ---------------------------------------------------------------------------
   pacman         Spawns other processes including logger & tape
                  Provides user control of the data acqisition operation
                  All control commands are entered into the pacman window
                  Creates iconized labeled windows for damm & scanor for your
                  convenience but you don't have to use them
                  See ORPHAS Tab-4 or Type: viewdoc pacmanII
   ---------------------------------------------------------------------------
   logger         Displays/logs internally generated messages
                  This is an "output process" only
   ---------------------------------------------------------------------------
   pftoipc        Runs in the background - no window
                  Receives real time data buffers from VME system
   ---------------------------------------------------------------------------
   tape           Runs in the background - no window
                  Writes list data to List Data Files (LDFs)
                  It did write tape (hence the name) - but no more!
   ---------------------------------------------------------------------------
   scanor         Receives real time data buffers. User customized to:
                  Generate 1-D & 2-D histograms in shared memory & on disk
                  See ORPH UBOOK 2000 Tab-4 or Type: viewdoc scanor
   ---------------------------------------------------------------------------
   damm           General display, manipulation & fitting program
                  Accesses histograms in shared memory from scanor
                  Accesses histograms on disk from anywhere
                  See ORPH UBOOK 2000 Tab-2 or Type: viewdoc damm
   ---------------------------------------------------------------------------
   scad           Generates numeric display of up to 240 CAMAC scalers and
                  Generates graphic display of up to 8 CAMAC scalers
                  See ORPHAS Tab-8 or Type: viewdoc scad
   ---------------------------------------------------------------------------
   scop           Generates graphic display of up to 8 CAMAC scalers and
                  sends rate-data to EPICS (the accelerator control system)
                  See ORPHAS Tab-8 or Type: viewdoc scop
   ===========================================================================
   If you do not have /usr/hhirf/ defined in your path, then you must type:
 
   /usr/hhirf/viewdoc instead of just viewdoc
   ===========================================================================
    
   11-Sep-04 ........ U100  README - HRIBF Data Acquisition ......... PAGE   9
 
 
 
   U100.060  Data Acquisition - List of Things to be Done
 
 
   o......Review the general setup requirements
          (see ORPHAS Tab-5 or type: viewdoc VMEsys)
 
 
   o......Review any special module setup requirements
          (see ORPHAS Tab-9 or type: viewdoc modu_setup)
 
   o......Review the general features of the data acwuisition manager pacman
          (see ORPHAS tab-4 or type: viewdoc pacmanII)
 
 
   o......Write a data acquisitation specification program (pac program)
          (see ORPHAS Tab-2 or type: viewdoc pacor)
 
 
   o......Produce a customized histogramming program (Customized scanor)
          (see UBOOK 2000 Tab-4 or type: viewdoc scanor)
 
 
   o......Create an initialization file for numeric display of CAMAC scalers
          (see ORPHAS Tab-8 or type: viewdoc scad)
 
 
   o......Create an initialization file for graphic display of CAMAC scalers
          (see ORPHAS Tab-8 or type: viewdoc scad & viewdoc scop)
 
 
   o......Review the features of general purpose display program damm
          (see UBOOK 2000 Tab-2 or type: viewdoc damm)
 
 
   ***************************************************************************
   *                              IMPORTANT!!                                *
   *                           BEFORE YOU START!!                            *
   *                       YOU SHOULD DO THE FOLLOWING                       *
   ***************************************************************************
 
   Add the following lines to your .login file:
 
   alias pacman /usr/acq/wks/pacman
 
   set ignoreeof
 
   In  your login window (from which you will run pacman), enter the following
   command:
 
   setenv VME vmex  ;where x=1,2,3... (i.e. define the vme system that you
                    ;will be using) to be vme1, vme2, vme3, ....
 
   ***************************************************************************
   See pacmanII, Sec-30 for other suggested changes to .login, .cshrc & .twmrc
   ***************************************************************************
    
   11-Sep-04 ........ U100  README - HRIBF Data Acquisition ......... PAGE  10
 
 
   U100.070  Data Acquisition - Brief Description of System Operation
 
   The following list of operations are intended to give the novice a  concept
   of  how  the  system  works  and  is  not  intended  to  present a detailed
   description   of  "everything  you  have  to  do".  All  of  the  following
   operations are to be carried  out  on  the  data  acquisition  workstation.
   First, the user creates (writes or modifies):
 
   A customized histogramming program based on scanor - call it myscanor
 
   A data acquisition specification program  - call it mydacq.pac
 
   A CAMAC scaler display specification file - call it myscals.sca
 
   Compile and test mydacq.pac by typing:=====================================
 
        pacor mydacq
 
   Compile and load mydacq into the VME microprocessor by typing:=============
 
       setenv VME vmex    ;Specify the VME system (vme1, vme2, vme3...)
 
       pacor mydacq L
 
   Start pacman (acquisition control program) by typing:======================
 
       setenv EXPT RIBxxx ;Identify your experiment (for DOE & OMB)
 
       setenv VME vmex    ;Specify the VME system (vme1, vme2, vme3...)
 
       pacman             ;This program manages the data acquisition process
                          ;from the workstation side. It also creates windows
                          ;starts 3 other processes logger, pftoipc & tape
 
       startvme           ;Starts data acquisition (enables histogramming)
       stopvme            ;Stops  data acquisition
 
       trun bon           ;Starts recording to LDF - beam ON
       trun boff          ;Starts recording to LDF - beam OFF
 
       tstop              ;Stops  recording
 
   Start myscanor (customized histogramming program) by typing:===============
 
       myscanor myhis     ;Histograms will be stored on myhis.his
       acq VMEx           ;Assign input to be from VMEx (x=1,2,3..)
       go                 ;Start processing input list data
 
   Start damm (display & analysis program) by typing:=========================
 
       damm               ;Starts program damm
       fig n              ;Define windows for display (n = 1 to 20)
       in myhis.his       ;Opens histogram file for input - Actually uses the
                          ;shared memory segment generated by myscanor
       d id               ;Displays histogram number id
    
   11-Sep-04 ........ U100  README - HRIBF Data Acquisition ......... PAGE  11
 
 
   U100.070  A Brief Description of System Operation (continued)
 
   Start scad (CAMAC scaler display program) by typing:=======================
 
        setenv VME vmex              ;Specify reading from vme1, vme2, vme3...
 
        scad                         ;It prompts you for a file name
        Enter filename->myscals.sca  ;Enter filename
        run                          ;Start it going
 
   ===========================================================================
 
 
   o......The  VME  system  responds  to  event triggers inputted to a special
          Trigger Module and subsequently reads out data as prescribed by  the
          pre-loaded acquisition code (pac).
 
   o......While  the readout is proceeding a "VME system busy" is asserted for
          external use.
 
   o......The data read is formed into ethernet packets and sent to  the  host
          workstation.
 
   o......A  process, pftoipc  running on the workstation collects packets and
          generates standard data buffers which are made  available  to  other
          processes  (such  as scanor & tape) running on the "data acquisition
          platform".
 
   o......Process tape writes data buffers to disk files (LDFs).
 
   o......The   user   customized  histogramming  process  (based  on  scanor)
          generates a number of 1-D and 2-D histograms  as  specified  by  the
          customizing routines.
 
   o......Histograms  are  generated  in a shared menory segment identified by
          the association with a histogram file (.his file) on disk.
 
   o......Program damm is used to access this memory segment and  display  the
          histograms being generated by the (scanor based) process.
 
   o......Program  scad  is  used  for  a  tabular display of up to 240  CAMAC
          scalers at a display interval of 1 second or greater (default  is  5
          sec).  scad  may  also  be  used  to display a list of up to 8 CAMAC
          scalers graphically at a rate of up to 20 displays/sec. Each  scaler
          rate  is  displayed as a software generated meter with either linear
          or log scale.
 
   o......Program scop is used to display up to 8 scalers graphically (as  for
          scad  above)  and  to  send  the  rate data to the EPICS accelerator
          control system for operator monitoring.
    
   11-Sep-04 ..... U100  README - Data Processing and Analysis ...... PAGE  12
 
 
   U100.080  Data Processing, Analysis and Display
 
 
   For information about:
 
   List-data-Copying   -   see   lemor   (ORPH  Ubook  2000,  Sec#  U320)  for
          tape-to-tape, tape-to-file, file-to-tape copy procedures.
 
   List-Data-Preprocessing - see  lemor (ORPH Ubook  2000,  Sec#  U310)  for a
          description of available preprocessing facilities.
 
   Histogramming  -  see  scanor  (ORPH  Ubook  2000, Sec# U320) for list-data
          histogramming from tapes and files.
 
   Histogram-Display   -   see  damm  (ORPH  Ubook  2000,  Sec#  U300)  for  a
          description of on-line (and off-line) 1-D & 2-D histogram display.
 
   Histogram-Analysis   -  see  damm  (ORPH  Ubook  2000,  Sec#  U300)  for  a
          description of histogram analysis (peak fitting,  summing,  gateing,
          etc).
 
   Byte-Swapping  -  see  swapo  (ORPH  Ubook, Sec# U540) for how to byte-swap
          list-data files and histogram files.
 
   (documentation is also available via dodoc and viewdoc)
    
   11-Sep-04 .... U100  README - HRIBF Tape and File Structure ...... PAGE  13
 
 
   U100.090  HHIRF List Data Tape Structure
 
   There are currently three list data structures supported  by  the  standard
   HRIBF  processing  software  (i.e.  SCANOR,  LEMOR  and customized versions
   thereof). These are:
 
   L001 - Old externally   generated data
 
   L002 - Old Perkin/Elmer generated data
 
   L003 - Current HRIBF    generated data
 
   These structures may be processed from either  8mm  tape  or  special  list
   data  files  (LDFs).  See LEMOR SEC#310 for how to copy tapes to LDFs, LDFs
   to tape, etc.
 
   HHIRF List Data - L001 Tape Structure
 
   L001 formatted list-data (often generated in external labs) consists of:
 
   o......16-bit data words.
 
   o......Fixed length data records which are no shorter than 2048  bytes  and
          no longer than 32768 bytes.
 
   o......A fixed number of parameters per event.
 
   o......A  fixed  (integer)  number of events per record (no event splitting
          across record boundries).
 
   o......A number of data-record header-words may be specified which  are  to
          be ignored during processing.
 
   o......Header  records  of 256 bytes in length and structured as defined in
          SEC#014 may be utilized for display, searches, etc.
 
   o......All other records not of the specified data-record  length  will  be
          ignored during processing.
    
   11-Sep-04 .... U100  README - HRIBF Tape and File Structure ...... PAGE  14
 
 
   U100.090  HHIRF List Data Tape Structure (continued)
 
   HHIRF List Data - L002 Tape Structure
 
   The  L002  data  stream (from the old Concurrent system) consists of Header
   records, Event Handler Source Code (EVS) records, Data records  and  Scaler
   records - as illustrated below:
 
   Header    record   ;256 bytes
   EVS-file  record   ;1600 bytes
    .
   Data      record   ;8192 bytes
   Data      record   ;8192 bytes
    .          .
   Scaler    record   ;6780 bytes
   End-of-File        ;File mark
   Header    record
   Data      record
    .          .
    .          .
   End-of-File        ;Double file-mark
   End-of-File        ;ends all data on tape
 
   L002  events  consist  of  parameter  ID flags followed by one or more data
   words corresponding to sequential parameter IDs  - as illustrated below:
 
   8000-hex + ID     ;Parameter ID
   Data              ;for Parameter ID
   Data              ;for Parameter ID+1
    .
   8000-hex + JD     ;Parameter JD
   Data              ;for Parameter JD
   Data              ;for Parameter JD+1
    .
   FFFF-hex          ;End-of-event
 
   o......Events may be split across record boundries.
   o......Data words cannot have the hi-order bit set.
   o......All parameter IDs and data words are 16-bit.
 
   ===========================================================================
   Structure of the 256-byte header record
   Header Word Number
      16-BIT   32-BIT   # BYTES  CONTENTS                        TYPE
     01 - 04  01 - 02         8  'HHIRF   '                      ASCII
     05 - 08  03 - 04         8  'L002    '                      ASCII
     09 - 16  05 - 08        16  'LIST DATA       '              ASCII
     17 - 24  09 - 12        16  MO/DA/YR HR:MN                  ASCII
     25 - 64  13 - 32        80  User Title                      ASCII
     65 - 66       33         4  Header Number                   BINARY
     67 - 70  34 - 35         8  Reserved (set to 0)             BINARY
     71 - 72       36         4  # of Secondary Header Records   BINARY
                   37         4  Record Length (bytes)           BINARY
                   38         4  # Blocked Line Image Records    BINARY
                   39         4  Record Length (bytes)           BINARY
                   40         4  Parameters/Event (ref only)     BINARY
                   41         4  Data Record Length (bytes)      BINARY
              42 - 64            Reserved (set to 0)             BINARY
    
   11-Sep-04 .... U100  README - HRIBF Tape and File Structure ...... PAGE  15
 
 
   U100.090  HRIBF List Data Tape Structure (continued)
 
   HHIRF List Data - L003 Tape Structure
 
   The L003 data stream (currently in use) may consist of:
 
   Header    records - 256   bytes
   PAC-file  records - 1600  bytes
   Scaler    records - 32000 bytes
   Dead-time records - 128   bytes
   Data      records - 2048 to 32768 bytes (but not 32000 bytes)
 
   A typical data stream might look like:
 
   Header    record
   PAC-file  record
   PAC-file  record
   Data      record
   Data      record
    .          .
    .          .
   Scaler    record
   Data      record
    .          .
    .          .
   Scaler    record
   Dead-time record
   End-of-File
   Header    record
   Data      record
    .          .
    .          .
   End-of-File
   End-of-File
 
 
   L003 events consist of pairs of 16-bit numbers, the first of which  defines
   the parameter ID and the second the data - as shown below.
 
   8000-hex + ID     ;Parameter ID
   Data              ;Parameter data
   8000-hex + ID
   Data
   8000-hex + ID
   Data
     .
     .
   FFFF-hex          ;End-of-event word-1
   FFFF-hex          ;End-of-event word-2
 
 
   o......Events are not split across record boundries.
 
   o......Unfilled records are padded with FFFF-hex.
    
   11-Sep-04 .... U100  README - HRIBF Tape and File Structure ...... PAGE  16
 
 
   U100.100  HHIRF List Data File Structure
 
   The  list  data  file  (LDF) ia a structured disk file for containing L001,
   L002 or L003 data. It is processed as a direct  access  file  whose  record
   length is 8194 32-bit words.
 
   The General Record Structure is:
   32-bit word  1       TYPE  - Record type (DIR,HEAD,PAC,SCAL,DEAD,DATA,EOF)
                2       NFW   - number of 32-bit words of data
                3-8194  DATA  - 8192 full-words (32768 bytes) data plus pads
 
   Directory record structure
   32-bit word  1       'DIR '  - type = Directory
                2       8192    - full-word data size
                3       8194    - full-word blocksize
                4       NREC    - number of records written on file
                5               - un-used
                6       NHED    - number of header records written on file
 
                7       HED-ID  - header ID number
                8       RECN    - record number where header is written
 
                9       HED-ID  - header ID number
               10       RECN    - record number where header is written
                .
 
   Header record structure
   32-bit word  1       'HEAD'  - type = header
                2        64     - number of full-words of header
                3-66            - header record - 256 bytes
 
   PAC record structure
   32-bit word  1       'PAC '  - type = PAC
                2        NFW    - number of full-words of data
                3-NFW+2         - PAC file records (up to 409 lines)
 
   Scaler record structure
   32-bit word  1       'SCAL'  - type = Scaler
                2        NFW    - number of full-words of data
                3-NFW+2         - Data for up to 400 scalers
 
   Deadtime record structure
   32-bit word  1       'DEAD'  - type = Deadtime
                2        32     - 128 bytes of deadtime data
                3-34            - Deadtime data
 
   Data record structure
   32-bit word  1       'DATA'  - type = Data
                2        8192   - number of full-words of data
                3-8194          - list data plus padding
 
   EOF record structure
   32-bit word  1       'EOF '  - type = EOF (End-of-File)
                2        8192   - number of full-words of data
                3-8194  'EOF '  - padded with EOFs
    
   11-Sep-04 .... U100  README - HRIBF Tape and File Structure ...... PAGE  17
 
 
   U100.110  HRIBF Histogram Tape and File Structures
 
   Program  STU  produces  histogram  tapes  for  transport  to other computer
   systems. See SEC# U410 for STU operating instructions. If you  have  a  VAX
   and  HHIRF  VAXPAK, STU is probably what you should use. If you have a Unix
   or Linux system, then transfer should be via tar or ftp.
 
   If STU is requested to output a HIS-file to tape,  it  writes  an  80  byte
   ASCII  Title record, the associated DIR-file (or DRR-file) and the HIS-file
   into one TAPE file. The Title record consists of the input filename  (words
   1-19)  and  a byte-order flag BOF (word 20). BOF = DEC says tape byte order
   is DEC compatible, BOF =  IBM  says  byte  order  is  IBM  compatible.  The
   structure of such a tape file is shown below:
 
   REC#  LENGTH(BYTES)  CONTENTS-----------------------------------------
 
      1          80     FILNAM (76 bytes) & Byte-order-flag (4 bytes)
      2         128     1st  DIR-file record
      3         128     2nd  DIR-file record
      -          -              -
      -          -              -
      M         128     M-1  DIR-file record
      -          -              -
      N         128     Last DIR-file record (N = NHIS+2 +(NHIS+31)/32)
    N+1       16384     1ST  16KB of HIS-file
    N+2       16384     2nd  16kb of HIS-file
      -          -              -
    N+I       16384     Ith  16KB of HIS-file
      -          -              -
                        End-of-File .....................................
      1          80     FILNAM (76 bytes) & Byte-order-flag (4 bytes)
      2         128     1st  DIR-file record
      3         128     2nd  DIR-file record
      -          -              -
      -          -              -
      M         128     M-1  DIR-file record
      -          -              -
      N         128     Last DIR-file record (N = NHIS+2 +(NHIS+31)/32)
    N+1       16384     1ST  16KB of HIS-file
    N+2       16384     2nd  16kb of HIS-file
      -          -              -
    N+I       16384     Ith  16KB of HIS-file
      -          -              -
                        End-of-File .....................................
                        End-of-File .....................................
 
                    (see next page for DIR-file definitions)
    
   11-Sep-04 .... U100  README - HRIBF Tape and File Structure ...... PAGE  18
 
 
   U100.110  HHIRF Histogram Tape and File Structures (continued)
 
   STRUCTURE OF DIR-FILE - FIRST RECORD (128 BYTES) ---------------------
 
   JDIRF(1-3)   - 'HHIRFDIR0001'
   JDIRF(4)     - # of histograms on HIS-file
   JDIRF(5)     - # of half-words on HIS-file
   JDIRF(7-12)  - YR,MO,DA,HR,MN,SC (date, time of CHIL run)
   JDIRF(13-32) - Text (entered in CHIL via $TEX command)
 
   STRUCTURE OF DIR-FILE - DIRECTORY ENTRY (128 BYTES) ------------------
 
   IDIRH(1)     - Histogram dimensionality (max = 4)
   IDIRH(2)     - Number of half-words per channel (1 or 2)
   IDIRH(3-6)   - Histogram parm#'s (up to 4 parameters)
   IDIRH(7-10)  - Length of raw    parameters (pwr of 2)
   IDIRH(11-14) - Length of scaled parameters (pwr of 2)
   IDIRH(15-18) - Min channel# list
   IDIRH(19-22) - Max channel# list
   IDIRF(12)    - Disk offset in half-words (1st word# minus 1)
   IDIRF(13-15) - X-parm label
   IDIRF(16-18) - Y-parm label
   XDIRF(19-22) - Calibration constants (up to 4 floating numbers)
   IDIRF(23-32) - Sub-title (40 bytes) (entered via CHIL $TIT cmd)
 
   STRUCTURE OF DIR-FILE - ID-LIST (32 ID'S/RECORD) ---------------------
 
   IDLST(1)     - ID number of 1st histogram defined
   IDLST(2)     - ID number of 2nd histogram defined
                -
 
   Note:  JDIRF, IDIRF & IDLST are 32-bit integers: IDIRH is a 16-bit integer:
   XDIRF is REAL*4. There is one directory entry for  each  histogram  on  the
   file.
 
   HHIRF Histogram Tapes - SPK-file Structure
 
   STU  tape  SPK-files  consist  of an 80 byte Title record (same as that for
   HIS-files, above) followed by as many 2048 byte binary records as  required
   to  contain  the  complete  SPK-file  (i.e.  an  image of the disk file). A
   file-mark separates successive tape files. The last tape file  is  followed
   by a double file-mark.
 
   If  you  don't have a VAX (and HHIRF VAXPAK), this record structure will be
   a bitch to read - so I am not going to try to tell you how to do it.
    
   11-Sep-04 .............. U100  README - HRIBF UPAK ............... PAGE  19
 
 
   U100.120  UPAK - Software Xport
 
   Most of the general purpose  data  acquisition,  processing,  analysis  and
   support software is available from the anonymous ftp site given below:
 
   ftp://ftp.phy.ornl.gov/pub/upak
 
   Click  on  README  and  you  will  see  a  list  of the currently available
   packages. The list at the time of this writing is shown below.
 
   mcsq-acq.tar.Z  :data acquisition libs & source, DEC/Compaq fortran, C
   upak-osf.tar.Z  :Obsosete upak for DEC OSF workstations, DEC fortran
   upakor-osf.tar.gz :Version for Compak TRU64, Compaq fortran
   upak-sun.tar.gz  :Obsolete version of upak for SunOS 4.1.4, Sun f77
   upak-ultrix-mips.tar.Z  :Version of upak for DECstations running Ultrix
   upakor-linux-alpha.tar.gz :Linux running on Alpha CPU's, Compaq fortran
   upak-macosx.tar.gz  :MacOSX, using Fink g77 and XFree86
 
   In the Linux subdirectory (Linux/):
 
   README-upak.txt: Information on installing the UPAK data analysis software
   README-hribf-daq.txt: Information on installing the HRIBF acquisition
   upakor-linux-pgf77-intel.tar.gz  :Linux running on Intel Pentium, pgf77
   DvDController.tgz: A graphical interface for archiving data to DVD
   hribf-daq-linux-src.tgz: Source code for the data acquisition system
   hribf-daq.linux.tgz: Linux 9.0 runtime of the acquisition software
   upak-linux-g77-src.tgz: Source of the UPAK software
   upak-linux-g77-intel.tgz: Linux 9.0, Intel Pentium, g77 compiler
 
   This software is distributed to our nuclear physics colleagues, whereis, asis.
 
   *This is the latest release, G77 will likely become our standard release.
 
   Questions?
   Email varner@phy.ornl.gov
   ---------------------------------------------------------------------
 
   To obtain the files from the anonymous ftp site, use  the  ftp  command  on
   your local system,
 
   csh> ftp ftp.phy.ornl.gov
   Username: anonymous
   Password: (Enter your Email address)
 
   ftp> cd pub/upak
   ftp> bin
   ftp> get upak-osf.tar.Z
      (messages)
   ftp> quit
   csh> uncompress upak-osf.tar
 
 
   The  tar  file  contains a directory, milner/, which is the source code for
   all the upak software, plus a  bit.   The  directory  hhirf/  contains  the
   binary  code  for  the  programs,  as  well  as  the  documentation  in the
   hhirf/doc directory.
    
   11-Sep-04 .............. U100  README - HRIBF UPAK ............... PAGE  20
 
 
   U100.130  UPAK - Comments
 
   To:       UPAK users
   From:     W. T. Milner
   Subject:  Some Things About UPAK (Alpha version - revised)
 
   Program scanor (which replaces scan and scanu)  and  program  lemor  (which
   replaces  lemo)  have  a number of new features including the processing of
   list-data-files. See the documentation on scanor and lemor for details.
 
   UPAK contains two main directories /usr/users/milner/  &  /usr/hhirf/.  The
   /hhirf/  directory  contains standard executables, template make-files, and
   some run-time help-files plus a few other things.  The  /milner/  directory
   contains  a  few  example  spk- and his-files, a .login file, a few example
   data files, this document, and not much else.
 
   Each program or routine library is contained in separate  sub-directory  of
   the  /milner/ directory. All sub-directory names consist of an upper case D
   followed   by   the   program   name   (/milner/Ddamm/    contains    damm,
   /milner/Dscanor/  contains  scanor,  etc).   There   are   no   sub-   sub-
   directories.  Typically,  such  a program sub-directory (/milner/Dprog/ for
   example) will contain some or all of the following:
 
   (1)....Source files specific to prog only.
 
   (2)....prog.make - a make file (shows all required files and libraries).
 
   (3)....prog.doc - a source document file  with  formatting  directives.  To
          learn how to process doc-files, Read /usr/hhirf/doc/txx.tex.
 
   (4)....prog.tex  -  a  formatted document file (ready for screen viewing or
          printing on  any  device).  The  tex-files  may  only  be  found  in
          directory /usr/hhirf/doc/.
 
   (5)....prog.txx  -  a  formatted  document  file  (with escape sequences to
          enable bold and undrelined printout on an LN03 compatible printer.
 
   (6)....prog.hep - a run-time  interactive  help  file  (not  used  for  all
          programs).  These  files  are  also  contained  in  the  /usr/hhirf/
          directory.
 
   (7)....Possibly some undocumented test programs, data files, etc.
 
   Libraries sources for (lemorlib, scanorlib,  miliba,  milibb,  and  orphlib
   are all contained in separate directories, Dlemorlib, .... Dorphlib.
 
                                General Comments
 
   A  simple  way  to  make  /UPAK/  executables  available  is to include the
   following statement in your .cshrc or .login files.
 
   setenv PATH "${PATH}:/usr/hhirf"
 
   The UPAK tape is in TAR format.
 
   See next page for a discussion of how to implement run-time help.
    
   11-Sep-04 .............. U100  README - HRIBF UPAK ............... PAGE  21
 
 
   U100.070  UPAK - Comments (continued)
 
   A number of programs utilize files  to  provide  for  run-time  help.  Some
   examples are shown below:
 
   Program       Associated Help File
   damm          damm.hep
   scanor        scanor.hep
   lemor         lemor.hep
 
   In  the initialization phase of all programs which use run-time help files,
   an attempt is first made to open prog.hep where, prog denotes  the  program
   name.   If   that   attempt   fails,   an   attempt   is   made   to   open
   /usr/hhirf/prog.hep.  The  path part, /usr/hhirf/, is "built in" to routine
   helpopen in library orphlib.a and the prog.hep  part  is  supplied  by  the
   program  itself.  If  this attempt also fails, an attempt is made to open a
   file named upak_help_path in the users  main  (login)  directory.  If  this
   file  exists,  it  should  contain  (one line) the path to all files of the
   type prog.hep (including beginning and ending  slashes)  as  shown  in  the
   default example below.
 
   /usr/hhirf/        ;Contents of the default upak_help_path
                      ;This path name can contain 60 characters maximum
 
   If  the  attempt to open upak_help_path fails or if the attempt to open the
   specified path and file fails, then run-time help will not be available.
 
                  Making Run-time Help Available on Your System
 
   There are several ways to  do  this  and  the  one  that  you  choose  will
   probably depend upon the resources available to you.
 
   (1)....If  you  create  a directory /usr/hhirf/ and place all files of type
          prog.hep   in  this  directory,  then  things  should  work  without
          modification.
 
   (2)....If  you  place  all  files  of  type  prog.hep  in  some  directory,
          /home/myname/UPAK/  for  example, and create in your login directory
          the file  named,  upak_help_path,  which  contains  the  path  name,
          /home/myname/UPAK/, then run-time help will be available.
 
   (3)....If  you  change  the  default  path  name  in  routine  helpopen  to
          correspond to the actual path to files  of  the  type  prog.hep  and
          remake  orphlib.a and then remake asap, damm, lemor, etc., then each
          user will not  be  required  th  have  the  file  upak_help_path  in
          his/her/it's login directory.
 
   Method-3  is  quite  a  bit  more  trouble  since you will probably need to
   change all the make-files, etc. and would have to repeat the  process  each
   time  you receive a new version of UPAK. I would suggest either methods (1)
   or (2).
    
   11-Sep-04 .............. U100  README - HRIBF UPAK ............... PAGE  22
 
 
   U100.070  UPAK - Comments (continued)
 
 
 
                                 IMPORTANT NOTE
 
   In order for scanor and lemor to work on  the  Alpha,  you  must  have  the
   "Real  Time  Kernal"  installed  - this is not automatic but is an optional
   feature of the installation. This is used by the asychronous I/O, etc.
 
                                     COMMENT
 
   You can make the "input focus" and "window stacking"  work  better  (in  my
   opinion) by adding the following to your .Xdefaults file:
 
   Mwm*keyboardFocusPolicy:  pointer
   Mwm*focusAutoRaise: False
 
   This  causes  the  input  focus  to  follow  the  pointer  (doesn't require
   clicking) and windows are not restacked unless you click on the banner.
