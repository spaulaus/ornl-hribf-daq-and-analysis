   20-Jul-04 ......... U950  Creating and Using DVDs & CDs .......... PAGE   1
 
 
   Sec Page Contents
 
   010   1  Creation & Use of DVDs & CDs for Data Processing - Outline
 
   020   2  Creation       of DVDs & CDs for Data Processing - Details
 
   030   4  Use            of DVDs & CDs for Data Processing - Details
 
 
   U950.010  Creation & Use of DVDs & CDs for Data Processing - Outline
 
   To Create a DVD or CD
 
   Insert a DVD or CD into the DVD drive (bottom - labeled)
 
   Follow steps 1 through 5 of the next section
 
 
   To use the DVD or CD in program scanor
 
   Insert the DVD or CD in the DVD drive (bottom - labeled)
   or
   Insert the CD in the CD drive         (top  - unlabeled)
   and Type:
 
   mount /dev/cdrom0              ;If using CD  drive (unlabeled - top)
 
   mount /dev/cdrom1              ;If using DVD drive (labeled - bottom)
 
 
   To open a file in  program scanor, Type:
 
   file /mnt/cdrom0/filename.ldf  ;if using CD  drive (top)
 
   file /mnt/cdrom1/filename.ldf  ;if using DVD drive (bottom)
 
   ===========================================================================
   Note: You cannot "punch out" a disk without dismounting it.
   Note: A mounted DVD or CD is readable by all.
   ===========================================================================
       WARNING - FOUR WAYS YOU CAN SCREW THINGS OP WITHOUT REALLY TRYING!!
 
   (1)  Execute  a  creation  process  using  the Gnome window manager without
   disabling the magicdev tool. It will hang the system at the end of  writing
   the DVD and you will need to get a root person to fix it.
 
   (2)  Kill  the  creation  process before it is finished and you will need a
   root person to un-hang the DVD writing system.
 
   (3) Log off without dismounting the DVD or CD and nobody else but you or  a
   root person can dismount it.
 
   (4)  Tell the system that you are going to create a DVD but insert CD media
   into the drive. It may run out of space and hang the system.
 
   ===========================================================================
 
       See also:   http://www.phy.ornl.gov/local/computer   and it's links
 
                                It has PICTURES!
    
   20-Jul-04 ......... U950  Creating and Using DVDs & CDs .......... PAGE   2
 
 
   U950.020  Creation of DVDs & CDs for Data Processing - Details
 
   We presume that you have written your data to an LDF format event  file  on
   disk,  and  now you need to preserve it in a permanent form. We use DVD for
   long-term storage of event data.
 
   A recording process consists of the following steps in order:
 
   (1) Start the application
 
   The application is started by  calling:  /usr/local/bin/DvdIoHandler  which
   brings up the graphical user interface to the DVD recording software.
 
   (2) Create a container file system to temporally hold files to be stored
 
   A  screen  will  present  itself.   You can select to create a CD or a DVD,
   either  will  be  written  with  the  UDF  file  format.  The  choice  here
   determines  the size of the container file system created and thus how much
   data you can write to the CD or DVD. Once you  made  you  selection,  press
   the  Create button to create the container file system. The programm cannot
   be terminated during the time the container file system is  created,  hence
   the Exit button will become unavailable.
 
   While  the container file system is created, you can monitor the process in
   the progress bar in the bottom of  the  window.  Once  the  container  file
   system  is  created,  a  new  window  will appear that allows you to select
   files.
 
   (3) Add files - create a list of files to be copied to container file
 
   You may select files that you want to record on DVD or  CD.  In  this  step
   the  files  are  not  actually copied, they are just collected. You may add
   files or directories using the Add button in  the  screen  presented.   The
   files  and directories will be displayed in tree form.  The available space
   will be listed in the progress bar at the bottom of  the  screen.  You  may
   also delete previously selected files.
 
   In  order  to add new files or directories press the Add button, which will
   present you with a screen to  select.   Select  the  radio  button  labeled
   Directory  if  you  want to add an entire directory. Press OK when you have
   made your selection from this window.
 
   (4) Copy files to the container file
 
   Once you selected all the files you want to record on the DVD  or  the  CD,
   you  may press either the Copy, Copy&Record or the Copy&Record&Exit button.
   This will start the copying  process.   Depending  on  where  your  orignal
   files  reside and how much data you are copying, this may take a while. The
   progress bar at the bottom of the screen let's you know how many data  have
   been copied.
 
   (5) Record files to DVD or CD - copy from container file to DVD or CD
 
   Once  all  files  are  copied,  the  recording  process  can  be started by
   pressing the Record button. However, after the Record  button  is  pressed,
   the  Exit  button  will become unavailable since terminating the program at
   the recording stage will make the recorded DVD useless. Please to not  kill
   the program during the recording stage!
    
   20-Jul-04 ......... U950  Creating and Using DVDs & CDs .......... PAGE   3
 
 
   U950.020  Creation of DVDs & CDs for Data Processing - Details (continued)
 
   If  a  recording  error  occurs,  you are informed by a pop-up window.  The
   error may be due to missing recording media or other problems with the  DVD
   Writer  itself.  Please  make  sure that the DVD Writer contains a writable
   DVD or CD disk and try to record again.
 
   If you selected the  Copy&Record  or  the  Copy&Record&Exit  button  and  a
   record  error occurs, you will be given the opportunity to press the Record
   button after the DVD Writer problem was  resolved.  The  program  will  NOT
   exit.
 
   Recording will take about 30min.
 
   It  is easiest to use the GUI /usr/local/bin/DvdIoHandler to write the DVD.
   Several steps may be combined, once the  container  file  system  has  been
   created  and  the  desired  data  has  been selected. After these tasks are
   completed you have to option to copy and record  in  one  step.  These  two
   tasks  are  the most time intensive, since it may take to actually copy the
   data from their  current  location  into  the  container  file  system.  In
   addition,  recording  a  DVD  or  CD  will take about 30min. If you plan to
   leave, select the  Copy&Record&Exit  button,  which  will  copy  the  data,
   record them and exit the program.
 
   Please  make sure to exit the program after you finish recording the DVD or
   CD. This will clean up resources you use while copying and recording.
 
   Once you started the DVD recording process, you are the  only  person  that
   can  record  data  to  the  DVD  Writer. You may have up to three processes
   simulataneously, however, only one process can record data. If you use  the
   Copy&Record&Exit  or  the  Copy&Record you will be automatically limited to
   one concurrent process.
 
   Limitations and problems we have noted
 
   (1)..The software we are using to make the UDF file system seems  to  crash
        the  system under Linux kernels greater than 2.4.20-18.9.  The problem
        occurs while copying the data to the container filesystem.
 
   (2)..We have observed that if you use the Gnome  window  manager,  it  will
        hang the system at the end of writing the DVD.  The solutions are:
 
        (a) Switch to KDE or a lightweight window manager (e.g. TWM)
 
        (b) Disable the magicdev tool which runs under Gnome.
    
   20-Jul-04 ............ U950  Using the DVDs and CDs .............. PAGE   4
 
 
   U950.030  Use of DVDs & CDs for Data Processing - Details
 
   The  files  stored on the disk have the same format and encodings as on the
   original disk.  Please note that the data are BINARY,  and  that  the Intel
   machines  on  which  we take data are little-endian.  If you read the disks
   on a big-endian machine, it may be necessary to swap the bytes,  especially
   in  histogram files.  Event files  will be swapped automatically by SCANOR,
   at some loss of efficiency.  If you   need  to  byte-swap  use  the program
   /usr/hhirf/swapo.
 
   U950.032  Macintosh and Windows computers
 
   Using  the  DVD  is as easy as inserting it into your DVD drive and waiting
   for the system to mount it. You can then copy the files about as you wish.
 
   U950.034  Linux computers
 
   On Linux, ordinary users  logged in at the console can arrange to  mount CD
   and  DVD  devices.  Some  window  managers, e.g. Gnome and KDE, under Linux
   will try to mount the  DVD for  you,  then  popup  a  file  manager window.
   Then you can manipulate the files as you wish.
 
   Failing  that,  your system should be setup to permit you to mount the DVD.
   If your /etc/fstab file has a line like:
 
   /dev/cdrom1     /mnt/cdrom1      udf,iso9660 noauto,owner,kudzu,r o 0 0
 
   Then you can mount the DVD by using a command like:
 
   mount /dev/cdrom1
 
   It seems that the default will be to mount the DVD  as   /mnt/cdrom1  which
   can  be treated like an ordinary disk. To dismount the disk, use one of the
   following commands:
 
   umount /dev/cdrom1, or
 
   eject /dev/cdrom1
 
 
   U950.036  Other UNIX computers
 
   Our experience with other UNIX systems,  especially  older  ones,  is  that
   only  the  root  user  can mount DVD's. Thus you would have to arrange with
   the sys admin to get root access, or copy all your data at  once  to  local
   storage. The commands for mounting a CD or DVD vary substantially from UNIX
   version to UNIX version, so please consult a local expert.
 
   ==========================================================================
   The website for this documentation will be found at:
 
   http://www.phy.ornl.gov/local/computer
 
   and the links to be found there.
 
   The website has PICTURES as well as text!
   ===========================================================================
