   02-Feb-02 ........... U620  FITU (UNIX version) - WTM ............ PAGE   1
 
                 A LINEAR and NON-LEAST-SQUARES FITTING PROGRAM
 
 
   ---------------------------------------------------------------------------
   Define:     Linear Fitting  =      Linear-Least-Squares Fitting
 
   Define: Non-Linear Fitting  =  Non-Linear-Least-Squares Fitting
   ---------------------------------------------------------------------------
 
 
   Sec Page Contents
 
   010   2  Introduction - for     Linear Fitting
   020   2  Introduction - for Non-Linear Fitting
   030   3  Introduction - Input Data Format
 
   040   4  Commands which Apply to Both Linear & Non-linear Procedures
 
   060   6  Commands for Setup/Control of Linear Fitting
   070   7  Expanded Definitions of some  Linear Fitting Commands
   080   7  Definition of Some Output for Linear Fitting
 
   090   8  Commands for Setup/Control of Non-Linear Fitting
   100   8  A Brief Description of the    Non-Linear Fitting Procedure
 
   120   9  Template make-file (and Comments) for Customized Processes
   130  10  Example (default) Non-linear Support Routine - USERCMP
   140  14  Example (default) Non-linear Support Routine - USERFOX
 
   150  15  Commands Related to Filename Variables, Symbols and Loops
   160  16  Comments
 
 
 
                                    FOREWORD
 
   Program  FITU  is  an  enhanced  version  of program FUNKYFIT that provides
   graphical displays of fit results as well as support  for  user  customized
   non-linear  fitting. There are some changes in the format of the output but
   the input data format is unchanged.
 
 
 
                               HOW TO GET STARTED
 
   ---------------------------------------------------------------------------
   Type:  fitu               ;To start if /usr/hhirf/ is defined in your
                             ;.login, .cshrc or login.com files, otherwise:
 
   Type:  /usr/hhirf/fitu    ;To start on HRIBF Alpha platforms
   ---------------------------------------------------------------------------
    
   02-Feb-02 ........... U620  FITU (UNIX version) - WTM ............ PAGE   2
 
   U620.010  Introduction for Linear Fitting
 
   FITU is a general purpose linear-least-squares fitting program  which  fits
   sets  of  (X,Y)  points  to  a  linear  combination  of  up to 20 algebraic
   functions. Directives are used to choose one of the following forms for the
   fit.
 
                  J=N
         Y(X) = SUM   B(J)*F(J,X)
                  J=1
 
                  J=N
         Y(X) = SUM   B(J)*F(J,LOG(X))
                  J=1
 
                       J=N
         LOG(Y(X)) = SUM   B(J)*F(J,X)
                       J=1
 
                       J=N
         LOG(Y(X)) = SUM   B(J)*F(J,LOG(X))
                       J=1
 
   Where N is the no. of terms (functions of X) and  F(J,X)  denotes  the  Jth
   function  of X. B(J) denotes the Jth coefficient which is determined in the
   fit. Experimental and calculated values,  coefficients,  errors  and  CHISQ
   values  are  listed.  Although  most  of  the  output  from  FITU should be
   self-explanatory, some quantities are defined in SEC#080.
 
 
 
   U620.020  Introduction for Non-Linear Fitting
 
   Non-Linear fitting of data to a  function  Y(X)  is  carried  out  via  the
   gradient  search  method. The user must create a customized version of FITU
   by linking a user supplied  REAL*8  function  USERFOX  which  computes  the
   function value and a user command processor USERCMP which supports it.
 
   o......Create  a  routine  USERFOX to compute the value of the function (or
          functions) using the value of X and the coefficient array  A  passed
          in the argument list.
 
   o......Create a user command processor USERCMP to support USERFOX.
 
   o......Note:  that USERFOX can be a multi-function routine. In such a case,
          USERCMP can be used to select which function is actually used for  a
          given fit request.
 
   o......Use  the  template  make-file  given  in  SEC#120  to  create   your
          customized version of FITU.
 
   See SEC#120 for a listing of the template make-file
 
   See SEC#130 for an example USERCMP
 
   See SEC#140 for an example USERFOX
 
   Note:  that  most  of the functions in the example USERFOX are not actually
   non-linear. I used these to check that  linear  and  non-linear  procedures
   gave nearly the same results.
    
   02-Feb-02 ........... U620  FITU (UNIX version) - WTM ............ PAGE   3
 
 
 
   U620.030  Introduction - Input Data�Format
 
 
   Data  to  be  fitted  can  be read from a file or entered interactively. In
   either case the format  is  the  same.  The  structure  is  illustrated  as
   follows:
 
   ---------------------------------------------------------------------------
   DATA             ;Introduces Y vs X DATA-SET
   X  Y <YERR>      ;Up to 500 DATA-SET entries
   X  Y <YERR>
   .
   .
   ENDA             ;Ends DATA-SET
 
   DATA             ;Introduces Y vs X DATA-SET
   X  Y <YERR>      ;Up to 500 entries (X,Y values)
   X  Y <YERR>      ;YERR is optional
   .
   .
   ENDA             ;Ends this   DATA-SET
   ---------------------------------------------------------------------------
 
   The  ACCII  directive  DATA  Introduces  a list of (X,Y,U) or (X,Y) entries
   (i.e. a data set). Here U denotes the uncertainty  in  Y.  The  first  data
   point  is  on  the  line  following the DATA label. Data points are entered
   one-per-line and up to 500 points are allowed. The data set  is  terminated
   with  an  ENDA  directive  which  must start in column 1. The ENDA line can
   contain no other information.
 
   The file can contain any number of data sets. The  ordinal  number  of  the
   data set is used to identify which data set is to be fitted.
    
   02-Feb-02 ........... U620  FITU (UNIX version) - WTM ............ PAGE   4
 
 
   U620.040  Commands which Apply to Both Linear & Non-linear Procedures
 
   CMD  Data-list   Definition or Action
   ---- ---------   ----------------------------------------------------
 
   H                Displays directory to on-line help
 
   LON              Says record all commands on log-file - fitu.log
   LOF              Says only record output  on log-file
 
   END              Ends program
 
   CMD  filnam      Open and process commands from filnam.cmd
   CMD  filnam.ext  Open and process commands from filnam.ext
 
   IN   filename    Open new data file (full filename required)
 
   CSUN             Says set uncertainties to be counting statistics
   ABUN             Says data-set uncertainties are in absolute units
   PCUN             Says data-set uncertainties are in %
   ALUN UVAL        Says set all uncertainties to UVAL(%)
   USUN UVAL        Says set unspecified uncertainties to UVAL(%)
   MULU FAC         Says multply all given uncertainties by FAC
 
   XMM  XLO,XHI     Sets display X-range (min & max)
   XMM              Sets display X-range defined by data
 
   YMM  YLO,YHI     Sets display Y-range (min & max)
   YMM              Sets display Y-range defined by data
 
   LIN              Set display to be linear
   LOG              Set display to be log (in the Y-direction)
 
   STAT             Displays/logs some setup parameters
 
   UCOM STAT        Displays/logs status provided by USERCMP
 
   WIN  ID          Specify window ID for next display of fit results
 
   REVV             Reverses Video (black-to-white, white-to-black)
                    You must do a FIG, then the REVV, and another FIG
                    for it to take effect.
 
   FIG  N           Execute a standard "FIG" - i.e. like damm
 
   CLR              Clears all FIGed windows
    
   02-Feb-02 ........... U620  FITU (UNIX version) - WTM ............ PAGE   5
    
   02-Feb-02 ........... U620  FITU (UNIX version) - WTM ............ PAGE   6
 
 
   U620.060  Commands for Setup/Control of Linear Fitting
 
   CMD  Data-list   Definition or Action
   ---- ---------   ----------------------------------------------------
 
   LINX             Set to use FUNCS of X - F(X)      DFLT
 
   LOGX             Set to use FUNCS of Log(X) - F(Log(X))
 
   LINY             Set to fit Y to F(X) or F(LOG(X)) DFLT
 
   LOGY             Set to fit Log(Y) to F(X) or F(Log(X))
 
   XLIM XMIN,XMAX   Set fit-limits in terms of X-values
 
   ILIM IMIN,IMAX   Set fit-limits in terms of index-I
 
   NUFU             Says new Function (i.e sets #TERMS=0)
 
   XPOW E1,E2..     Exponent list for power-series terms
 
   LPOL J1,J2..     Ordinal-list for Legendre poly terms
 
   TABL XLO,XHI,DX  Specs for computed Y vs X table
 
   HFMT,DFMT        Table header & data formats (see SEC#070)
 
   *                A * in col-1 of a command line results in the full
                    line being copied to log-file as a comment
 
   FIT  <ID>        Fits set# ID (ID omitted says in core)
 
   FITU <ID>        Same as above but UNWEIGHTED
 
    
   02-Feb-02 ........... U620  FITU (UNIX version) - WTM ............ PAGE   7
 
 
   U620.070  Expanded Definitions of Some Linear Fitting Commands
 
   TABL...XMIN,XMAX,DELX  (range & step-size for table of values of calculated
          Y vs X). TABL without associated list deletes table request.
 
   XLIM...XMIN, XMAX (range of fit limits on X - default  is  no  limit)  XLIM
          without subsequent list gets rid of limit.
 
   ILIM...ILO,IHI  (range of fit limits on data point ordinals - default is no
          limit). ILIM without subsequent list gets rid of limit.
 
   XPOW...List of powers-of-X to be included as terms in the  "fit  function".
          For  example;  XPOW=0,1,2 says include the terms X**0, X**1 and X**2
          (i.e. 1.0, X and X*X).
 
   LPOL...List of legendre polynomial "ordinals". For example;  LPOL=2,4  says
          include the terms P2(X) and P4(X).
 
   HFMT...Heading  FORMAT for first line of table of calculated values of Y vs
          X. This line contains 10 entries (0.0, DELX, 2*DELX - - -9*DELX).
 
   DFMT...Data FORMAT for subsequent lines of table of calculated values of  Y
          vs  X.  These  lines  contain  11  entries (X-value corresponding to
          first Y-value and 10 Y-values).
 
          Note- HFMT and DFMT are not free-form  entries.  These  labels  must
          start  in column 1 and the actual format must follow immediately. No
          other information can be included on this line. Examples  (also  the
          defaults) are;
 
          HFMT(1H ,10X,10F10.3/)  and  DFMT(1H ,11F10.3)
 
   FIT....ID  -  says retrieve data set number ID from the input disk file and
          carry out a fit as previously specified. If ID is  not  entered,  it
          is  assumed  that the data set is already in memory (either typed in
          or previously retrieved from disk). Only one data set is  in  memory
          at any given time.
 
   FITU...has  the  same  meaning  as  FIT  except  that the fit is unweighted
          (actually all points are given equal weight - namely, unity).
 
 
   U620.080  Definition of Some Output for Linear Fitting
 
   FIT-ERR(%) gives the calculated uncertainty in the  associated  coefficient
          (determined  in  the  fit) based on the quality of fit (QFN) and the
          scatter of the data points about the calculated values.
 
   EST-ERR(%) gives the calculated uncertainty in the  associated  coefficient
          based  on  the  quality  of fit and the scatter of experimental data
          that is to be expected based on the uncertainties  assigned  to  the
          data.
    
   02-Feb-02 ........... U620  FITU (UNIX version) - WTM ............ PAGE   8
 
 
 
   U620.090  Commands for Setup/Control of Non-Linear Fitting
 
 
   CMD  Data-list   Definition or Action
   --------------------------------------------------------------------------
 
   UCOM STAT        Displays/logs status provided by USERCMP
 
   TOF  VN          Says turn fit-variable VN OFF
   TON  VN          Says turn fit-variable VN ON
 
   SKPI I J K ...   Says skip entry ordinals I J K in computing sum RESQ
   SKPI             Says remove all skips
 
   UCOM FUNK FID    Says select function# FID (in USERFOX) for fitting
 
   UCOM CPOS CID    Allows adjustable coefficient-CID to be positive only
 
   UCOM CPOM CID    Allows adjustable coefficient-CID to be pos or neg
                    (overrides default set in USERCMP until next UCOM FUNK
 
   NITF NIT         Sets # of iterations for coefficients for best fit
                    to NIT (default = 20, allowed values are 1 to 100)
 
   NITE NIT         Sets # of iterations for coefficient uncertainty estimate
                    to NIT (default = 10, allowed values are 1 to 100)
 
   GO   ID          Do a     non-linear fit on data set# ID
   GOMO ID          Continue non-linear fit of data set# ID
 
 
   U620.100  A Brief Discussion of the Non-Linear Fitting Procedure
 
                  Procedure for Determining Coefficient Values
                  ============================================
 
   1......A  gradient  search  for  the  coefficients  giving  the  best   fit
          (smallest CHISQ) is carried out.
 
   2......The  starting  coefficient values are changed randomly from the best
          fit values by multiplying each by (1.5-RAN(ISEED)).
 
   3......Steps 1 and 2 are repeated NITF times and  coefficients  leading  to
          the smallest CHISQ are recorded.
 
 
               Procedure for Estimating Coefficient Uncertainties
               ==================================================
 
   1......The  input  data  is  randomly  modified  to  simulate what might be
          expected  (from  the  assigned  data  point  uncertainties)  if  the
          experiment were repeated.
 
   2......A search for the best-fit coefficients is carried out.
 
   3......Steps  1  &  2  are  repeated  NITE  times   and   the   coefficient
          uncertainties  are  estimated  from  the  spread  in  the   best-fit
          coefficients computed in steps 1 & 2.
    
   02-Feb-02 ........... U620  FITU (UNIX version) - WTM ............ PAGE   9
 
 
 
   U620.120  Template Make-file (and Comments) for Customized Processes
 
   Template make-file found at /usr/hhirf/fitu.make
   ===========================================================================
   FFLAGS= -fpe2
   #
   OBJS= /usr/hhirf/fitu.o
         /axp/milner/Dfitu/usercmp.o
         /axp/milner/Dfitu/userfox.o
   #
   LIBS= /axp/milner/Dfitulib/fitulib.a
         /axp/milner/Ddammlib/dammlib.a
         /axp/milner/Dxglib/xglib.a
         /axp/milner/Ddammlib/dammlib.a
         /axp/milner/Dxglib/xglib.a
         /usr/hhirf/milibb.a
         /usr/hhirf/orphlib.a
   #
   fitu: $(OBJS) $(LIBS)
   #
           f77  $(FFLAGS) $(OBJS) $(LIBS) -o fitu -lX11
 
 
                             Comments on Customizing
 
   o......You  can  modify  USERCMP  &  USERFOX  to  create  your  own  set of
          non-linear functions. You just need to make sure that your  versions
          of  USERCMP & USERFOX are compatible with each other and the rest of
          the package.
 
   o......Don't change COMMON/FTUSER/ since it is used by  other  routines  in
          the package.
 
   o......Your  version  of  USERCMP can communicate other information to your
          version of USERFOX via other  user  defined  common  but  don't  use
          common  labels  ML..., XL..., FT..., MAINV or LLL since these are or
          may be defined by other routines in the package.
 
   o......Note that the banner labels  BANLAB  defined  in  USERCMP  are  just
          labels  and have no effect on the calculations but they should be be
          consistant with the associated functions in USERFOX.
    
   02-Feb-02 ........... U620  FITU (UNIX version) - WTM ............ PAGE  10
 
 
   U620.130  Example Non-linear Support Routine - USERCMP
 
   Example user command processor for non-linear fitting
   ===========================================================================
 
         SUBROUTINE USERCMP(IWD)
   C
         IMPLICIT NONE
   C
   C     ------------------------------------------------------------------
         COMMON/LLL/  MSSG(28),NAMPROG(2),LOGUT,LOGUP,LISFLG,MSGF
         INTEGER*4    MSSG,NAMPROG,LOGUT,LOGUP,LISFLG,MSGF
         CHARACTER*112 CMSSG
         EQUIVALENCE (CMSSG,MSSG)
   C     ------------------------------------------------------------------
         COMMON/FTUSER/ NONEG(64),NV,IFUNK,UTIT
         INTEGER*4      NONEG,    NV,IFUNK
         CHARACTER*40                      UTIT
   C
         DATA NONEG/64*0/     !Default - allow +/- for all coefficients
         DATA NV   /2/        !Default - 2 adjustable coefficients
         DATA IFUNK/2/        !Default - function ID=2
         DATA UTIT /'A+B*X'/  !Default - Banner label
   C     ------------------------------------------------------------------
         CHARACTER*40   BANLAB(50)
   C
         DATA BANLAB(1)/'A*X'/
         DATA BANLAB(2)/'A+B*X'/
         DATA BANLAB(3)/'A+B*X+C*X*X'/
         DATA BANLAB(4)/'A*DSQRT(X)'/
         DATA BANLAB(5)/'A+B*DSQRT(X)'/
         DATA BANLAB(6)/'A+B*DSQRT(X)+C*X'/
         DATA BANLAB(7)/'DSQRT(A+B*X)+C*X'/
         DATA BANLAB(8)/'A+B*X+C*DEXP(D*X)'/
         DATA BANLAB(9)/'A*(B/(C-B))*(EXP(-B*X)-EXP(-C*X))'/
   C     ------------------------------------------------------------------
         INTEGER*4      IWD(20),LWD(2,40),ITYP(40),NF,NTER
   C
         REAL*4         XV
   C
         INTEGER*4      IV,KIND,IERR,KMD,I
   C
         CHARACTER*8    FLAG
   C
         CHARACTER*18   POMFLAG
   C
         EQUIVALENCE    (KMD,LWD(1,1))
   C     ------------------------------------------------------------------
   C
         CALL GREAD(IWD,LWD,ITYP,NF,1,80,NTER)
   C
         IF(NTER.NE.0) GO TO 1000
   C
         IF(KMD.EQ.'FUNK') GO TO 5
         IF(KMD.EQ.'CPOS') GO TO 400
         IF(KMD.EQ.'CPON') GO TO 400
         IF(KMD.EQ.'CPOM') GO TO 400
         IF(KMD.EQ.'STAT') GO TO 500
   C
         GO TO 1010
   C
    
   02-Feb-02 ........... U620  FITU (UNIX version) - WTM ............ PAGE  11
 
 
   U620.130  Example Non-linear Support Routine - USERCMP (continued)
 
   C
       5 CALL MILV(LWD(1,2),IV,XV,KIND,IERR)     !Get function-ID
   C
         IF(IERR.NE.0) GO TO 1000                !Tst for error
   C
         IF(IV.LT.1)   GO TO 1020                !Tst for in range
         IF(IV.GT.9)   GO TO 1020                !tst for in range
   C
   C     ------------------------------------------------------------------
   C     Set up the specified user function
   C     ------------------------------------------------------------------
   C
         GO TO (10,20,30,40,50,60,70,80,90) IV   !Go set it up
   C
   C
      10 UTIT=BANLAB(1)                          !Display banner title
         NONEG(1)=0                              !Allow negative coeff-1
         IFUNK=1                                 !Function ID
         NV=1                                    !No. of coefficients
         RETURN
   C
      20 UTIT=BANLAB(2)                          !Display banner title
         NONEG(1)=0                              !Allow negative coeff-1
         NONEG(2)=0                              !Allow negative coeff-2
         IFUNK=2                                 !Function ID
         NV=2                                    !No. of coefficients
         RETURN
   C
      30 UTIT=BANLAB(3)                          !Display banner title
         NONEG(1)=0                              !Allow negative coeff-1
         NONEG(2)=0                              !Allow negative coeff-2
         NONEG(3)=0                              !Allow negative coeff-3
         IFUNK=3                                 !Function ID
         NV=3                                    !No. of coefficients
         RETURN
   C
      40 UTIT=BANLAB(4)                          !Display banner title
         NONEG(1)=0                              !Allow negative coeff-1
         IFUNK=4                                 !Function ID
         NV=1                                    !No. of coefficients
         RETURN
   C
      50 UTIT=BANLAB(5)                          !Display banner title
         NONEG(1)=0                              !Allow negative coeff-1
         NONEG(2)=0                              !Allow negative coeff-2
         IFUNK=5                                 !Function ID
         NV=2                                    !No. of coefficients
         RETURN
   C
      60 UTIT=BANLAB(6)                          !Display banner title
         NONEG(1)=0                              !Allow negative coeff-1
         NONEG(2)=0                              !Allow negative coeff-2
         NONEG(3)=0                              !Allow negative coeff-3
         IFUNK=6                                 !Function ID
         NV=3                                    !No. of coefficients
         RETURN
    
   02-Feb-02 ........... U620  FITU (UNIX version) - WTM ............ PAGE  12
 
 
   U620.130  Example Non-linear Support Routine - USERCMP (continued)
 
   C
      70 UTIT=BANLAB(7)                          !Display banner title
         NONEG(1)=1                              !Disallow neg   coeff-1
         NONEG(2)=1                              !Disallow neg   coeff-2
         NONEG(3)=0                              !Allow negative coeff-3
         IFUNK=7                                 !Function ID
         NV=3                                    !No. of coefficiemts
         RETURN
   C
      80 UTIT=BANLAB(8)                          !Display banner title
         NONEG(1)=0                              !Allow negative coeff-1
         NONEG(2)=0                              !Allow negative coeff-2
         NONEG(3)=0                              !Allow negative coeff-3
         NONEG(4)=1                              !Disallow neg   coeff-4
         IFUNK=8                                 !Function ID
         NV=4                                    !No. of coefficients
         RETURN
   C
      90 UTIT=BANLAB(9)                          !Display banner title
         NONEG(1)=0                              !Allow negative coeff-1
         NONEG(2)=0                              !Allow negative coeff-2
         NONEG(3)=0                              !Allow negative coeff-3
         IFUNK=9                                 !Function ID
         NV=3                                    !No. of coefficients
         RETURN
   C
   C     ------------------------------------------------------------------
   C     Display User Status
   C     ------------------------------------------------------------------
   C
     400 CALL MILV(LWD(1,2),IV,XV,KIND,IERR)
   C
         IF(IERR.NE.0) GO TO 1000
   C
         IF(IV.LT.1)   GO TO 1030
         IF(IV.GT.NV)  GO TO 1030
   C
         IF(KMD.EQ.'CPOS') NONEG(IV)=1
         IF(KMD.EQ.'CPON') NONEG(IV)=0
         IF(KMD.EQ.'CPOM') NONEG(IV)=0
   C
         RETURN
   C
    
   02-Feb-02 ........... U620  FITU (UNIX version) - WTM ............ PAGE  13
 
 
   U620.130  Example Non-linear Support Routine - USERCMP (continued)
 
   C
   C     ------------------------------------------------------------------
   C     Display User Status
   C     ------------------------------------------------------------------
   C
     500 WRITE(CMSSG,505)
     505 FORMAT('FUNK#  Function                                 Status')
         CALL MESSLOG(LOGUT,LOGUP)
         WRITE(CMSSG,510)
     510 FORMAT('--------------------------------------------------------')
         CALL MESSLOG(LOGUT,LOGUP)
         DO 530 I=1,9
         FLAG=' '
         IF(I.EQ.IFUNK) FLAG='SELECTED'
         WRITE(CMSSG,520)I,BANLAB(I),FLAG
     520 FORMAT(I5,'  ',A,' ',A)
         CALL MESSLOG(LOGUT,LOGUP)
         CALL MESSLOG(LOGUT,LOGUP)
     530 CONTINUE
   C
         WRITE(CMSSG,510)
         CALL MESSLOG(LOGUT,LOGUP)
         DO 550 I=1,NV
         POMFLAG='Pos or neg allowed'
         IF(NONEG(I).EQ.1) THEN
         POMFLAG='Pos only   allowed'
         ENDIF
         WRITE(CMSSG,535)POMFLAG,I
     535 FORMAT(A,' for coefficient# ',I2)
         CALL MESSLOG(LOGUT,LOGUP)
     550 CONTINUE
         RETURN
   C
   C     ------------------------------------------------------------------
   C     Send error messages
   C     ------------------------------------------------------------------
   C
    1000 WRITE(CMSSG,1005)
    1005 FORMAT('Syntax error or illegal value - command ignored')
         GO TO 2000
   C
    1010 WRITE(CMSSG,1015)
    1015 FORMAT('User command not recognized - command ignored')
         GO TO 2000
   C
    1020 WRITE(CMSSG,1025)
    1025 FORMAT('User function ID out of range - command ignored')
         GO TO 2000
   C
    1030 WRITE(CMSSG,1035)
    1035 FORMAT('User coefficient ID out of range - command ignored')
         GO TO 2000
   C
    2000 CALL MESSLOG(LOGUT,LOGUP)
         RETURN
         END
    
   02-Feb-02 ........... U620  FITU (UNIX version) - WTM ............ PAGE  14
 
 
   U620.140  Example Non-linear Support Routine - USERFOX
 
   Example user-supplied function for non-linear fitting
   ===========================================================================
 
         REAL*8 FUNCTION USERFOX(A,X)
   C
         IMPLICIT NONE
   C
   C     ------------------------------------------------------------------
         COMMON/FTUSER/ NONEG(64),NV,IFUNK,UTIT
         INTEGER*4      NONEG,    NV,IFUNK
         CHARACTER*40                      UTIT
   C     ------------------------------------------------------------------
         REAL*8         A(*),X,DENO,DABS
   C     ------------------------------------------------------------------
   C
         GO TO (10,20,30,40,50,60,70,80,90) IFUNK
   C
      10 USERFOX=A(1)*X
         RETURN
   C
      20 USERFOX=A(1) + A(2)*X
         RETURN
   C
      30 USERFOX=A(1) + A(2)*X + A(3)*X*X
         RETURN
   C
      40 USERFOX=A(1)*DSQRT(X)
         RETURN
   C
      50 USERFOX=A(1) + A(2)*DSQRT(X)
         RETURN
   C
      60 USERFOX=A(1) + A(2)*DSQRT(X) + A(3)*X
         RETURN
   C
      70 USERFOX=DSQRT(A(1) + A(2)*X) + A(3)*X
         RETURN
   C
      80 USERFOX=A(1) + A(2)*X + A(3)*DEXP(A(4)*X)
         RETURN
   C
      90 DENO=A(3)-A(2)
         IF(DABS(DENO).LT.1.0E-9) DENO=-1.0E-9
         USERFOX=A(1)*(A(2)/DENO)*(DEXP(-A(2)*X)-DEXP(-A(3)*X))
   C
         END
    
   02-Feb-02 ........... U620  FITU (UNIX version) - WTM ............ PAGE  15
 
 
 
   U620.150  Commands Related to Filename Variables, Symbols & Loops
 
   Filname Variables
 
   One   symbol   (integer   variable)  may  be  incorporated  in  a  FILENAME
   specification as the following examples illustrate:
 
   SYM=3
   IN FIL"SYM".DAT   ;Opens FIL3.DAT for input
 
   I=0
   LOOP 3
   I=I+1
   IN  FIL"I".DAT   ;Opens (in succession)  FIL1.DAT, FIL2.DAT, FIL3.DAT
   .
   .
   ENDLOOP
 
   LOOP execution and symbol definition
 
 
   Commands related to LOOP execution and SYMBOL definition
 
   SYM = EXPRESSION   Define symbol (SYM) up to 100 symbols supported
                      symbols M, N, O, P, Q, R, S are reserved
                      expression syntax is same as in CHIL
                      no imbedded blanks are allowed in expressions
                      symbols may contain up to 4 characters (5-8 ignored)
 
   DSYM               Displays list of currently defined sumbols & values
 
   LOOP N             Starts LOOP (executed N-times) N=SYM or CONST
   CMD  ....          Nesting supported
   CMD  ....          # lines between 1st LOOP & matching ENDL = 100
   ENDL               Defines end-of-loop
                      KILL (entered before ENDL) kills LOOP
                      Ctrl/C - aborts loop-in-progress
                      opening of CMD-file within a LOOP not allowed
    
   02-Feb-02 ........... U620  FITU (UNIX version) - WTM ............ PAGE  16
 
 
 
   U620.160  Comments
 
   (1)....Unless you have a very  small  number  of  data  points  to  fit,  I
          suggest  that you create a file (fil.dat for example) which contains
          the data sets to be fitted. After the program is started,  open  the
          file with the command - IN fil.dat
 
   (2)....Note  that  the data file does not contain actual ID-number entries.
          The ID referred to in the FIT (or GO) command is  just  the  ordinal
          number  of the data set in the file. This may not be the best way to
          do it but that's the way it is.
 
   (3)....If your fitting procedure will  involve  numerous  commands,  it  is
          probably  best to create a command file (fil.cmd for example) rather
          than entering them at run-time. If you have both  a  fil.dat  and  a
          fil.cmd, then all you have to do is:
 
   Type:  FITU
   Type:  IN   fil.dat
   Type:  CMD  fil.cmd
 
   And that's it.
