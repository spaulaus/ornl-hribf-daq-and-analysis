   02-Feb-02 ........... U510  ASAP (UNIX version) - WTM ............ PAGE   1
 
 
 
 
                       Automatic Spectrum Analysis Program
 
 
   SEC PAGE CONTENTS
 
   010   1  HOW TO GET STARTED
 
   020   2  GENERAL FEATURES
 
   030   2  PROGRAM OPERATION
 
   040   2  LOOP EXECUTION AND SYMBOL DEFINITION
 
   045   3  SYMBOLS IN FILENAMES
 
   050   3  FIT SPECIFICATIONS - COMMAND LIST
 
   060   4  INPUT QUANTITIES - DEFINITIONS
 
   070   4  BACKGROUND COMPUTATION
 
   080   5  PEAK FINDING
 
   090   5  SUM AREA (SAREA) CALCULATION
 
   100   5  CENTROID CALCULATION
 
   110   5  GAUSSIAN AREA (GAREA) CALCULATION
 
   120   6  NON-LINEAR SEARCH FOR PEAK POSITIONS AND WIDTHS
 
   130   6  LISTING OF RESULTS - DEFINITIONS & FORMATS
 
   140   7  COMMENTS
 
 
   U510.010  HOW TO GET STARTED
 
 
   Type:  asap               ;if asap is defined in your .login or .cshrc
                             ;files, otherwise:
 
   Type:  /usr/hhirf/asap    ;For HHIRF DECstation and Alpha users
   or
   Type:  /home/upak/asap    ;For SPARCstation users
 
    
   02-Feb-02 ........... U510  ASAP (UNIX version) - WTM ............ PAGE   2
 
 
   U510.020  GENERAL FEATURES
 
   (1)....ASAP  is  intended  to  be  used for the QUICK, EASY and maybe DIRTY
          analysis of high-resolution gamma-ray spectra.
 
   (2)....Processes data from his- or spk-files.
 
   (3)....Finds peaks, estimates the background, and calculates AREAS  by  two
          methods (a SUMMING method and GAUSSIAN fitting)
 
   (4)....Produces  a  table  of  PEAK  locations,  ENERGIES, AREAS, estimated
          ERRORS, and PEAK-WIDTHS.
 
   (5)....Spectra are analyzed in "sections" chosen by the program.
 
   (6)....Sections may be up to 256 channels in length and contain  up  to  16
          peaks.
 
   (7)....Up to 250 peaks may be processed in one FIT request.
 
   (8)....Spectra with up to 16384 channels may be processed.
 
   U510.030  PROGRAM OPERATION
 
   The  user  specifies  how  the  fitting is to be carried out by supplying a
   number of FIT specifications which are entered free-form  (i.e.  a  command
   NAME  followed  by  zero  or  more  data  items  -  numbers or alphanumeric
   specifiers).  Command  and  data  item  delimiters  are  space  and  comma.
   Commands  may be given in any order. Most of data items have DEFAULT values
   (see SEC# U510.050). SEC#  U510.050  gives  a  complete  list  of  the  FIT
   specifications  available,  some  of  which are described in more detail in
   SEC# U510.060.
 
   After the fitting conditions are specified, one or more  FIT  requests  are
   entered.  Subsequently,  some  or  all  of  the  fitting  conditions may be
   changed and more FIT requests entered etc., etc.
 
   U510.040  LOOP EXECUTION AND SYMBOL DEFINITION
 
   Commands related to LOOP execution and SYMBOL definition.
 
   SYM = EXPRESSION - Define symbol (SYM) up to 100 symbols supported
                    - symbols M, N, O, P, Q, R, S are reserved
                    - expression syntax is same as in CHIL
                    - no imbedded blanks are allowed in expressions
                    - symbols may contain up to 4 characters (5-8 ignored)
 
   DSYM             - Displays list of currently defined sumbols & values
 
   LOOP N           - Starts LOOP (executed N-times) N=SYM or CONST
   CMD  ....        - Nesting supported
   CMD  ....        - # lines between 1st LOOP & matching ENDL = 100
   ENDL             - Defines end-of-loop
                    - KILL (entered before ENDL) kills LOOP
                    - Ctrl/C - aborts loop-in-progress
                    - opening of CMD-file within a LOOP not allowed
    
   02-Feb-02 ........... U510  ASAP (UNIX version) - WTM ............ PAGE   3
 
 
 
   U510.045  SYMBOLS IN FILENAMES
 
   One   symbol   (integer   variable)   may  be  incorporated  in  a  filname
   specification as the following examples illustrate:
 
   Example-1
 
   syn=3
   OU fil"sym".spk,new  ;Creates and opens fil3.spk
 
   Example-2
 
   I=0
   LOOP 3
   I=I+1
   IN fil"I".spk        ;Opens (in succession)  fil1.spk, fil2.spk, fil3.spk
   .
   .
   ENDLOOP
 
   U510.050  FIT SPECIFICATIONS - COMMAND LIST
 
   A list (with definitions) of all ASAP commands is given below.  Note:  that
   commands may be entered in either upper or lower case.
 
   CMD   DATA-------Definition (general commands) ------------
 
   IN    fil.ext    Open data file (ext = .spk or .his)
                    (on VAX, his-file names must include Disk name)
   CMD   fil        Open and process commands from fil.cmd
   CMD   fil.ext    Open and process commands from fil.ext
   TIT   Title      Enter a Title for log-file
   END              Ends program
 
   CMD   DATA-------Definition (associated with fitting) -----
 
   KWD   KWD        See text (usually KWD=2*FWHM)     (DFLT KWD=8)
   WCAL  FWA,FWB    Peak-width=FWA+FWB*SQRT(chan#)    (DFLT FWA,FWB=4,0)
   BIAS  BIAS       Peak-detect threshold             (DFLT BIAS=3.0)
   DEL   DX,WDNC    Step size for pos & width         (DFLT 0.0,0.05)
   KFIT  KIND       Kind of FIT (GAUS OR SUMA)        (DFLT KIND=GAUS)
   KBAK  KIND       Kind of BGD (FULL,FAST,ZERO)      (DFLT KIND=FULL)
   BSTD  BSTD       See text (STD DEV for BGD)        (DFLT BSTD=2.0)
   MPIS  N          Max # Peaks/section (MAX=16)      (DFLT MPIS=8)
   VW    KIND       Width variation (NONE,LOCK,FREE)  (DFLT KIND=LOCK)
   FIT   ID,LO,HI   FIT spect# ID   (chans LO thru HI)
   FIT   ID         FIT spect# ID   (using previous    LO,HI specs)
   FIT              FIT             (using previous ID,LO,HI specs)
 
   Ctrl/C           Interrupts FIT in progress
    
   02-Feb-02 ........... U510  ASAP (UNIX version) - WTM ............ PAGE   4
 
 
 
   U510.060  INPUT QUANTITIES - DEFINITIONS
 
   KWD....KWD  is  a  basic  width  such  that  2*KWD  will include all of the
          stronger peaks (see the following sections for  other  ways  KWD  is
          used)
 
   WCAL...FWA,FWB:    ASAP     computes     FWHM     from     the     relation
          FWHM=FWA+FWB*SQRT(CHAN#).  IF  FWA  =  FWB = 0.0, ASAP estimates FWA
          and FWB by fitting  the  widths  calculated  from  SAREAS  and  PEAK
          HEIGHTS.  If  ASAP computes FWA and FWB and if KWD was initially set
          to 0, ASAP sets KWD = 2*FWHM at mid-range of the FIT request.
 
   BIAS...BIAS: BIAS is  a  PEAK  HEIGHT  selector  (the  number  of  standard
          deviations  above   BACKGROUND that a count must be to be considered
          part of a PEAK) Depending on the statistics, BIAS  may  need  to  be
          set to values ranging from 2.5 to 10.0
 
   DEL....DX,WDNC:  DX  is  the fraction of a channel that PEAKS will be moved
          at each step in the non-linear-search for best  PEAK  position  (DX=
          0.0  says do no non-linear search on PEAK position). A maximum of 10
          attempts is made to find the best position. WDNC is  the  step  size
          (fraction  of  the  initial  value  of  FWHM) used in the non-linear
          search for best peak width/s. Adjustment of FWHM is limited  to  0.3
          to  3.0  times  the  starting value. If WDNC = 0.0  FWHM will not be
          changed.
 
   Peak positioning is not usually required - start with DX = 0.0
 
   ECAL...A,B: A is the spectrum intercept and  B  is  the  spectrum  GAIN  in
          Kev/channel  (or  whatever).  (A   and   B   are   only   used   for
          identification purposes).
 
   BSTD...A  BACKGROUND  computation parameter which effects SAREA only and is
          probably best left alone.
 
   U510.070  BACKGROUND COMPUTATION (FOR SAREA ONLY)
 
   The BACKGROUND is calculated by least squares fitting  the  spectrum  in  a
   number  of  overlapping  sections.  10*KWD  or  256  (whichever is smaller)
   channels are considered in each FIT. Each section is  fitted  three  times-
   the  first  time  to the form BGD(I)=A+B*X(I) and the last two times to the
   form BGD(I)=A+B*X(I)+C*X(I)*X(I). after each fit any channel  containing  a
   count    Y(I)    is    thrown    out     of     the     next     fit     if
   (Y(I)-BGD(I)).GT.BSTD*SQRTF(BGD(I)).  (see  the  list  of definitions) This
   essentially throws away the PEAKS (at least that is what it is supposed  to
   do).  Only  the center 1/2 of each section is retained except for the first
   and last sections. I.E. the first 1/2 of the second  section  overlaps  the
   last  1/2  of  the first section etc. You should probably use BSTD=2.0 (the
   DEFAULT value).
    
   02-Feb-02 ........... U510  ASAP (UNIX version) - WTM ............ PAGE   5
 
 
   U510.080  PEAK FINDING
 
   Peak finding is accomplished using a routine due  to  J.  D.  Larson.  This
   routine  (used  by RIP and STP) computes a "spectrum derivative", looks for
   inflection points, locates the largest count  near  this  inflection  point
   and  tests  against  the  peak  detect  threshold  specified  (BIAS in this
   program). The differentation process uses  KWD/2  as  an  estimate  of  the
   basic  peak  FWHM and is sensitive (but not extremely so) to this value. It
   does what it does and does it rather well, I think.
 
   U510.090  SUM AREA (SAREA) CALCULATION
 
   The SUM AREAS (SAREA'S) are calculated by summing Y(I)-BGD(I)  from  I1  to
   I2  where  I1=(IP-KWD+1)  and I2=(IP+KWD). (see the list of definitions) IP
   is the rough position of the peak. I.E. Y(IP) is the  maximum  peak  count.
   If  peaks  are  closer  together than 2*KWD, the difference is split to get
   the summing range. If IDIFLO is the separation  between  IP  and  the  next
   peak  below  it  and  IDIFHI is the separation between IP and the next peak
   above, the sum is done from I1 to I2 where,
 
   I1=(IP-MINF(KWD,IDIFLO/2)+1)
 
   and
 
   I2=(IP+MINF(KWD,IDIFHI/2)).
 
   Where MINF(A,B) means the smaller of A and B. The quoted ERROR is given by,
 
   ERR=100*SQRTF(SUM(Y(I))+SUM(BGD(I)))/AREA
 
   U510.100  CENTROID CALCULATION
 
   The CENTROID or best peak position may be  gotten  by  summing  as  follows
   from I1 to I2.
 
   CENTROID=SUM(I*(Y(I)-BGD(I))**2)/SUM((Y(I)-BGD(I))**2   I=I1,I2
 
   Notice  that  this  is not really the CENTROID, since the square of the net
   count is used rather than the net count, but this method seems  to  give  a
   better peak position under adverse conditions.
 
   U510.110  GAUSSIAN AREA (GAREA) CALCULATION
 
   Peaks  which  are  well  separated  are  fitted by the linear least squares
   method to a single gaussian plus a linear background over the range  J1  to
   J2  where,  J1=IP-2*KWD  and J2=IP+2*KWD. If peaks are closer together than
   2*KWD the range is extended to include up to 16 peaks, however the  DEFAULT
   maximum  number  is  8. If peaks are still in the way, the range is reduced
   somewhat to exclude them. The errors in the peak areas  are  calculated  in
   the  conventional  way  for  this  type  of  fitting but is too involved to
   discuss here.
    
   02-Feb-02 ........... U510  ASAP (UNIX version) - WTM ............ PAGE   6
 
 
 
   U510.120  NON-LINEAR SEARCH FOR PEAK POSITIONS AND WIDTHS
 
   If DX is entered (greater than 0), an attempt is  made  to  find  the  best
   positions  for  all  peaks  in  the section being fitted by positioning the
   peaks one at a time, starting with the largest  and  working  down  to  the
   smallest,  in  steps of DX (see the list of definitions) channels per step.
   If DX=0.0, no positioning will be done. If no best position  is  found  the
   attempts  will  be terminated after 10 trys. After positioning is completed
   an attempt may be made to find the best peak widths.  Peak  widths  may  be
   held  fixed  (VW=NONE), varied together in a given FIT section (VW=LOCK) or
   allowed to vary freely (VW=FREE). Adjustment of peak widths is  limited  to
   0.3  to  3.0  times  initial  values. If WDNC = 0.0 no width search will be
   done.
 
   U510.130  LISTING OF RESULTS - DEFINITIONS & FORMATS
 
   Most of the output is self explanatory, with a few exceptions:
 
   (1)....XG of the table denotes the best position found in  the  positioning
          operation.
 
   (2)....AFWHM denotes the adjusted FWHM.
 
   (3)....The  output  col  labeled  100(S-G)..../G  compares  GAUSSIAN  AREAS
          (GAREA) and SUM AREAS (SAREA) and flags (with ***) if the  agreement
          is worse than the estimated ERRORS predict.
 
   (4)....If  you  don't  use GAUSSIAN analysis the GAREA col will contain the
          approximate PEAK HEIGHT. (this is just an accident).
 
   In order to facilitate the reading of asap.log by a another  program,  each
   type  of  output  data  line  is specifically flagged for identification. A
   listing of output data, associated flags,  and  record  formats  are  given
   below.
 
   Definition of output quantities flagged with FIL$
 
   MO/DA/YR HR:MN:SC - Month, day, year, hour, minute, second of run
 
   filename          - Name of file from whence spectrum was read
 
   Format is:  (1H ,4X,2(I2,1H/),I2,2X,2(I2,1H:)I2,4X,20A4,9X,'FIL$')
               If line is read on FORMAT(30A4), FIL$ flag is in word-30
 
   Definition of output quantities flagged with TIT$
 
   Title       Entered by user at ASAP run time
 
   Format is:  (1H ,4X,19A4,35X,'TIT$')
               If line is read on FORMAT(30A4), TIT$ flag is in word-30
    
   02-Feb-02 ........... U510  ASAP (UNIX version) - WTM ............ PAGE   7
 
 
   U510.130  LISTING OF RESULTS (continued)
 
   Definition of output quantities flagged with DAT$
 
   ID          Spetrum or histogram ID number
   ILO         First channel # of peak search range (LO from FIT cmd)
   IHI         Last  channel # of peak search range (HI from FIT cmd)
   KWD         A "width parameter" (see HANDBOOK)
   BIAS        Peak-detect threshold (in standard deviation units)
   KEV/CH      Energy calibration (the B from ECAL command)
   E0          Energy calibration (the A from ECAL command)
   FW          Initial width parameter (the A from WCAL command)
   FWB         Initial width parameter (the B from WCAL command)
   WDNC        Peak width    step size (in fractional units)
   DX          Peak position step size (in channel   units)
   VW          Width variation flag (NONE, LOCK, FREE)
   BSTD        Background control parameter for Sum-Area (see HANDBOOK)
   Format is:  (1H ,4I6,7F8.4,4X,A4,F8.4,I8,11X,'DAT$')
               If line is read on FORMAT(30A4), DAT$ flag is in word-30
 
   Definition of output quantities flagged with SAP$
 
   SAREA       Sum-area (see SEC# U510.090 for computation details)
   ERR         Estimated error (via counting statistics with BGD subtraction)
   GAREA       Gaussian-area from fitting process
   ERR         Estimated error from quality-of-fit, etc.
   AFWHM       Adjusted full-width at half-maximum from non-linear search
   XG          Gaussian peak position from non-linear search (found in fit)
   CENTROID    Peak centroid (see SEC# U510.100)
   EGAM        Peak energy from CENTROID, E0 and KEV/CH
   100(S-G)/S  100*(SAREA-GAREA)/SAREA
   **          Says SAREA & GAREA differ by more than "uncertainties predict"
   ID          Spectrum or histogram ID
   Format is:  (1H ,9F10,1X,A2,3X,I8,11X,'SAP$')
               If line is read on FORMAT(30A4), SAP$ flag is in word-30
 
   U510.140  COMMENTS
 
   (1)....I  don't  know  how  well this program will work for your data. Only
          you can determine that. It's not going to  work  very  well  if  the
          whole region of interest is just one mass of overlapping peaks.
 
   (2)....Performance  will depend on the fitting conditions that you specify,
          of course. The DEFAULTS  are  set  to  "reasonable"  values  for  an
          "average"  Ge(Li)  spectrum - whatever that is. If it is possible to
          supply good width coefficients (FWA & FWB) and keep  the  widths  of
          all  peaks in a section locked together (VW LOCK), results should be
          more satisfactory.
 
   (3)....In most cases it is best not to set the maximum number of peaks  per
          section  to  too  large a value. It makes the program slow and gives
          poor results for the smaller peaks in the  section.  You  will  just
          have to find out what works (or doesn't work) for your data.
