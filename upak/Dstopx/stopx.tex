   02-Feb-02 ...... U630  STOPX (UNIX & VMS versions) - AWES ........ PAGE   1
 
                      STOPPING POWER - ENERGY LOSS - RANGE
 
   STOPX  calculates stopping powers, energy loss in finite samples and ranges
   for ions in elements, compounds and mixtures. The formulas  and  parameters
   of  J. F. Ziegler (given in The Stopping and Ranges of Ions in Matter, Vols
   3 & 5, Pergamon Press, 1980.) are used to calculate stopping  powers  which
   are  integrated  for  energy-loss  and  range  calculations.  STOPX  is   a
   "prompting"  program  which requires little or no additional documentation.
   A list of the STOPX commands are given below. All input may be  entered  in
   either upper or lower case.
 
   ---------------------------------------------------------------------------
   CMD   LIST         MEANING
 
   PROJ  P1,P2,P3,... Define stoppee. PI = element or Z.
 
   EA    E1,E2,E3,... Define energy/nucleon of stoppee.
                      (increasing order unless E3=INCR.)
 
   DEDX               Calc. stopping power Mev/(mg/cm**2)
 
   RNGE               Calc. range mg/cm**2
 
   ELOS               Calc. E-loss in a series of absorbers
 
   LP                 Assign output to stopx.log on DECstation
                      Assign output to STOPX.PRT on VAX
 
   CON                Assign output to console (default)
 
   END                Exit (ends program)
 
   FILE               Assign output file for energy losses.
 
   ABSB               Define absorbers (prompted for each absorber)
   ---------------------------------------------------------------------------
   Legal absorber declarations have the form:
   ELEMENT MIXTURE  ["GAS" PRES] THICKNESS
   For example:
 
   .1*(C 1H4)+.9*40AR GAS 500. -30.  - Gas mix of 10% Methane & 90% Argon
                                     - Pressure = 500 torr
                                     - Thickness = 30cm
 
   28SI  3.5                         - Silicon target
                                     - Thickness = 3.5mg/cm**2
 
   Note: A "negative" thickness is taken to be in units of cm.
         A "positive" thichness is taken to be in units on mg/cm**2.
   ---------------------------------------------------------------------------
 
   Type:  stopx             ;To start with stopx defined in your .login,
                            ;.cshrc, or login.com files, otherwise,
   Type:  /usr/hhirf/stopx  ;To start execution on a HHIRF DECstation  or
   Type:  /home/upak/stopx  ;To start execution on a SPARCstation      or
   Type: @U1:[MILNER]STOPX  ;To start execution on the HHIRF VAX
   Type:  help              ;and then try to figure out what to do.
 
                  (See next page for comments by W. T. Milner)
    
   02-Feb-02 ...... U630  STOPX (UNIX & VMS versions) - AWES ........ PAGE   2
 
 
 
                            Comments by W. T. Milner
 
   On  a number of occasions I have been asked questions concerning the syntax
 
   of solid absorber definitions - namely, whether the element mix  should  be
 
   entered  as  mg/cm**2  or  as  elemental concentrations. Relative elemental
 
   concentrations are the appropriate values to enter. Also, the program  does
 
   not  distinguish  between  compounds  and  mixtures.  Note:  that for solid
 
   Silicon Dioxide, all entries in the list shown below give the same results:
 
 
   1.0*(28si 16o2)                1.0 - thickness is 1.0 mg/cm**2
 
   1.0*(28si 16o 16o)             1.0
 
   1.0*28si + 1.0*16o2            1.0
 
   1.0*28si + 1.0*16o + 1.0*16o   1.0
 
   100*(28si 16o2)                1.0
 
   100*28si + 100*16o2            1.0
 
   100*28si + 100*16o + 100*16o   1.0
 
 
   It is probably best to always give solid thickness  in  terms  of  mg/cm**2
 
   since  the program doesn't "know" the density of all elements and certainly
 
   not that of compounds and mixtures.
 
