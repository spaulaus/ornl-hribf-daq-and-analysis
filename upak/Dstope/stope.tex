   02-Feb-02 ..... U640  STOPE (Stopping Power Program) - WTM ....... PAGE   1
 
 
                 A Stopping Power Program With Graphical Display
 
   Sec Page Contents
 
   010   2  Introduction
 
   020   3  List of Commands
 
   030   4  Definition of Output Quantities
 
   040   5  References
 
 
 
                                    Foreword
 
   Program  stope  estimates  the  stopping  power  for  any ion in any target
   element by the three different methods given below.
 
   o......Using the formulas and coefficients given by J.F. Ziegler in Ref 1.
 
   o......Using the method and parameters of Northcliffe  and  Schilling  from
          Ref 2.
 
   o......Using  D.  Ward's  effective  charge  (Ref  3)  and Ziegler's proton
          stopping power (Ref 1).
 
   o......Shell correction  factors  for  Northcliffe's  values  are  computed
          using the Ward's tables and formulas given in Ref 3 and 4.
 
   o......Computation  results  are  presented  in  both tabular and graphical
          form.
 
 
                               How to Get Started
 
   ---------------------------------------------------------------------------
   Type:  stope              ;To start if /usr/hhirf/ is defined in your
                             ;.login, .cshrc or login.com files, otherwise:
 
   Type:  /usr/hhirf/stope   ;To start on HRIBF Alpha platforms
   ---------------------------------------------------------------------------
    
   02-Feb-02 ..... U640  STOPE (Stopping Power Program) - WTM ....... PAGE   2
 
   U640.010  Introduction
 
   Program stope calculates tables of stopping power vs  ion  energy  for  any
   ion  in any elemental target (ZT = 1-92). It's main feature is ease of use.
   For stopping powers, energy loss and ranges in compounds and mixtures,  see
   program stopx by T. C. Awes - SEC# U630.
 
                A Brief Description of Ziegler's Method Used Here
 
   Program  stope uses the formulas and coefficients given by J. F. Ziegler in
   - Stopping and Ranges of Ions in Matter, Vols 3 & 5, Pergamon Press,  1980.
   However,  Ziegler  uses  somewhat  more elaborate procedures in calculating
   his final "curves" than are used here.  The  nuclear  stopping  power  (col
   labeled - SNUC) is calculated from Formula 16 given in Vol 5.
 
   For Protons ===============================================================
 
   EPN=EKEV/AP                ;Energy per nucleon
 
   SE=PSTOP(ZT,EPN)           ;Electronic stopping power
                              ;(in units of eV/(10**15 atoms/cmsq))
 
   For He or Li ==============================================================
 
   EPN=EKEV/AP                ;Energy per nucleon
 
   EFCR=ZEFCR(AP,ZP,ZT,EKEV)  ;Fractional effective charge ratio to that for
                              ;protons at same velocity in target element
 
   ZEF=EFCR*ZP                ;Fractional effective charge of ion
 
   SE=ZEF*ZEF*PSTOP(ZT,EPN)   ;Electronic stopping power of ion in target
                              ;(in units of eV/(10**15 atoms/cmsq))
 
   For Heavy Ions ============================================================
 
   EPN=EKEV/AP                ;Energy per nucleon
 
   EFCR=HICRAT(AP,ZP,ZT,EKEV) ;Fractional effective charge ratio to that for
                              ;protons at same velocity in target element
 
   ZEF=EFCR*ZP                ;Fractional effective charge of ion
 
   SE=ZEF*ZEF*PSTOP(ZT,EPN)   ;Electronic stopping power of ion in target
                              ;(in units of eV/(10**15 atoms/cmsq))
 
   Where:
 
   PSTOP..is  a  routine  which calculates proton stopping power in any target
          element (ZT = 1-92) at energy per nucleon = EPN. It uses a table  of
          coefficients (12 per target element) and formulas from Ref 1.
 
   ZEFCR..is  a  special routine for He & Li which uses formulas from Ref 1 to
          compute the fractional effective charge ratio of  the  ion  to  that
          for protons at the same velocity in the same target element.
 
 
   HICRAT.is  a  routine  for for heavy ions which uses formulas from Ref 1 to
          compute the fractional effective charge ratio of  the  ion  to  that
          for protons at the same velocity in the same target element.
    
   02-Feb-02 ..... U640  STOPE (Stopping Power Program) - WTM ....... PAGE   3
 
 
   U640.020  List of Commands
 
   Command        ;Meaning
   -------        ;---------------------------------------------
   h              ;Displays this list again
 
   soli           ;Specifies SOLID  target - default
   gas            ;Specifies GAS    target
 
   doc            ;Displays/logs definitions & references
 
   lon            ;Enables  output to stope.log- dflt
   lof            ;Disables output to stope.log
 
   lmon           ;Turns Lo-E modification ON - dflt
   lmof           ;Turns Lo-E modification OFF
 
   cmd  filename  ;Processes commands from filename
 
   proj ion       ;ion = name like he4, ar40, u238, etc
 
   targ ele       ;ele = name like 58Ni or H2o, CH2, Mylar
 
   eion ea eb ec  ;Ion energy (MeV) ea to eb in steps of ec
                  ;(if ea,eb,ec omitted - does preset list)
 
   dedx           ;Generates DE/DX table using proj, targ, eion
 
   end            ;Ends program
 
    
   02-Feb-02 ..... U640  STOPE (Stopping Power Program) - WTM ....... PAGE   4
 
 
   U640.030  Definition of Output Quantities
 
   All stopping powers are in units of MeV*CMSQ/MG
 
   NUCL   = Nuclear stopping power - from Ziegler - Ref 1
 
   WARZIG = S(ele) from Ward's effective charge & Ziegler's proton electronic
            stopping power Ref 1,3
 
   ZIEGL  = S(ele) from Ziegler - Ref 1
 
   NORTH  = S(ele) from Northcliff & Schilling corrected for target mass
            Ref 2
 
   CNORTH = S(ele) shell-corrected NORTH - see SHELF - Ref 2,4
 
   SHELF  = Shell correction factor = S(ele,He4)WARD/S(ele,He4)NORTH
            Ref 2,4
 
   WARHE4 = S(ele,He4) from Ward's table for MeV/Amu listed in Ref 4
 
   Entries flagged by * computed from linear velocity dependance in
   range of 0.0 to 0.10 MeV/Amu
 
 
   T-WARZIG = WARZIG+NUCL
   T-ZIEGL  = ZIEGL +NUCL
   T-CNORTH = CNORTH+NUCL
 
   Display-solid-line   = ZIEGL  (electronic)
   Display-open-circle  = WARZIG (electronic)
   Display-close-circle = NORTH  (electronic)
 
 
   U640.040  References
 
   1)  J. F. Ziegler, The Stopping and Ranges of Ions in Matter, Vols 3 & 5,
       Pergamon Press, 1980.
 
   2)  L. C. Northcliff & R. F. Schilling, Nuc Data Tables, Vol 7, 1970.
       (Range .0125 to 12.0 MeV/Amu
 
   3)  D. Ward, et al, Stopping Powers for Heavy Ions, AECL-5313,
       Chalk River, 1976.
 
   4)  D. Ward, et al, Compilation of Realistic Stopping Powers for 4He Ions
       in Selected Materials, AECL-4914, Chalk River, 1975
