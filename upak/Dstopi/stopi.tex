   02-Feb-02 ...... U640  STOPI (UNIX & VMS versions) -  WTM ........ PAGE   1
 
 
                    STOPPING POWER OF ANY ION IN ANY ELEMENT
 
   STOPI  calculates tables of stopping power vs ion energy for any ion in any
   naturally occurring elemental target. It's main feature  is  ease  of  use.
   For  stopping powers, energy loss and ranges in compounds and mixtures, see
   program "STOPX" by T. C. Awes - SEC# U630.
 
   STOPI uses the formulas and coefficients  given  by  J.  F.  Ziegler  in  -
   Stopping  and  Ranges  of Ions in Matter, Vols 3 & 5, Pergamon Press, 1980.
   However, Ziegler uses somewhat more  elaborate  procedures  in  calculating
   his final "curves" than are used here.
 
                         BRIEF DESCRIPTION OF THE METHOD
 
   The  nuclear stopping power (col labeled - SNUC) is calculated from Formula
   16 given in Vol 5.
 
   The electronic stopping power is  calculated  by  two  methods.  The  first
   method (col labeled - SE(EFF-Q)) is calculated from a formula of the form:
 
   S(HI)=S(P)*(ZEFHI/ZEFP)**2
 
   Where  S(HI)  denotes  the  stopping power of the ion of interest, S(P) the
   stopping power of protons in the same medium at the same ion  velocity  and
   (ZEFHI/ZEFP)  the  ratio  of  the fractional effective charge of the ion to
   that of a proton at the same velocity. The stopping  power  of  protons  is
   defined  in  terms  of  12  coefficients for each of the 92 target elements
   (given in Table 2 of Vol 3). The associated formulas are given in Table  1,
   Page  16,  Vol  3. The fractional effective charge ratio is calculated from
   Formula 13, Vol 5.
 
   Above an ion energy of 1.5 Mev/AMU, the electronic stopping power  is  also
   calculated  using  the  so  called High-Energy formula given in Formula 18,
   Vol 5. Results are listed in the col labeled SE(HE-FORM).
 
   Well, it turns out that the effective charge formulas seem to  work  pretty
   well  over  the  range  0.2 to 100.0 Mev/AMU. Above about 0.5 Mev/AMU, this
   formula reproduces Ziegler's result about as  well  as  you  can  read  his
   curves.  There is some deviation below 0.5 Mev/AMU which may be as large as
   10 to 15% at 0.2 Mev/AMU although the  agreement  is  usually  better  than
   that.  Below 0.2 Mev/AMU - your guess is as good as mine. Results below 0.5
   Mev/AMU are flagged with ????  Just  to  remind  you  that  agreement  with
   Zieglers curves is not perfect.
 
   ---------------------------------------------------------------------------
   Type:  stopi              ;To start if stopi is defined in your
                             ;.login, .cshrc or login.com files, otherwise:
 
   Type:  /usr/hhirf/stopi   ;To start execution on a HHIRF DECstation
   or
   Type   /home/upak/stopi   ;To start execution on a SPARCstation
   or
   Type: @U1:[MILNER]STOPI   ;To start execution on the HHIRF VAX
   ---------------------------------------------------------------------------
 
                          (continued on the next page)
    
   02-Feb-02 ...... U640  STOPI (UNIX & VMS versions) -  WTM ........ PAGE   2
 
 
 
   List of STOPI Commands
 
   ---------------------------------------------------------------------------
 
   COMMAND                - MEANING
 
   HELP                   - Gets this list again
 
   CON                    - Directs output to terminal (default)
 
                            (here you get an abreviated output)
 
   PR                     - Directs output to stopi.log on DECstation
 
                          - Directs output to STOPI.PRT on VAX
 
   PROJ TARG              - Computes S(E) for preset list of energies
 
   PROJ TARG EMIN,EMAX,DE - Does it for specified energy list
 
                          - all energies are in Mev
 
 
   EXAMPLE:
 
 
   32S 120SN 10,200,5     - Says 32S IN 120Sn - E=10,200,5 Mev
 
   ---------------------------------------------------------------------------
 
