   02-Feb-02 ........... U350  CHIL (UNIX version) - WTM ............ PAGE   1
 
 
                   Comprehensive HIstogramming Language - CHIL
 
 
   Sec Page Contents
 
 
 
   010   2  How to Compile A CHIL Program
 
   020   3  Introduction
 
   030   4  CHIL Syntax - List
 
   040   7  CHIL Syntax - General Rules
 
   050   8  Parameter Lengths
 
   060   9  Parameter Names
 
   070  10  Gates - Simple    (Lo-limit, Hi-limit Pairs)
 
   080  11  Gates - Simple    (Mapped)
 
   090  12  Gates - Free-form (Bananas)
 
   100  13  IF-statements and Computed GO TOs
 
   110  14  Bit-tests
 
   120  15  Histogram - Bits/channes
 
   130  15  Histogram - ID Numbers
 
   140  15  Histogram - Titles
 
   150  16  Histogram - Statements
 
   160  18  Loops
 
   170  19  Symbol & Expressions
 
   180  20  Pre-scanning - Considerations
 
   190  21  User-Supplied Subroutines
 
   200  23  How to Create Customized CHIL-based Processes
 
   210  24  Directory File - Structure
 
   220  25  Examples
 
   230  30  Comments and warnings
 
    
   02-Feb-02 ........... U350  CHIL (UNIX version) - WTM ............ PAGE   2
 
 
 
   U350.010  HOW TO COMPILE A CHIL PROGRAM
 
   If  chil  is defined in your .login or .cshrc files, you may compile a CHIL
 
   program by typing:
 
 
 
 
   chil cfile              ;For compilation with listing on cfile.prt
 
 
 
   chil cfile OP,OP,..     ;For compilation with options
 
 
 
 
   Where, cfile denotes the filename  prefix  of  a  source  file  whose  name
 
   extension must be .chl (you don't type the .chl part) and:
 
 
   OP = NOT  Says no his-table listing
 
   OP = NOS  Says no source    listing
 
   OP = MIL  Says mil-code listing (for developers only)
 
 
   Two  other  files,  cfil.drr  &  cfil.mil,  are  created or overwritten, at
 
   chil-time. If cfil.his does not exist, you will be prompted  at  "scan  run
 
   time" for permission to create it.
 
 
   ---------------------------------------------------------------------------
 
 
   Note: that all filename extensions (.chl, .drr, .mil, .his) are lower case.
 
    
   02-Feb-02 ........... U350  CHIL (UNIX version) - WTM ............ PAGE   3
 
 
 
   U350.020  INTRODUCTION
 
   CHIL  is a histogramming language used for writing data processing programs
 
   for both on-line (for data acquisition) and off-line  (for  tape  scanning)
 
   analysis. Some features of CHIL-based processing programs are listed below.
 
 
   (1)  Both simple and free-form gating accomodated.
 
   (2)  Input of simple gate lists from files (created graphically).
 
   (3)  IF-statements   of the form - IF (CONDITION) DEST-LABEL
 
   (4)  Computed-GOTO's of the form - IF (COND-SET)  L1,L2,L3,L4..
 
   (5)  Bit testing and branching.
 
   (6)  Loops (no nesting).
 
   (7)  Histograms of dimensionality as large as 4 may be specified.
 
   (8)  Histogram dimensions not restricted to a power of 2.
 
   (9)  Any mixture of legal histogram dimensions may be specified.
 
   (10) Histograms may be any combination of 16- or 32-bits/channel.
 
   (11) Use of Parameter-names as well as Parameter-numbers in CHIL programming.
 
   (13) Multiple USERSUBS (up to 3) called from CHIL at will.
 
   (14) Up to 3 output data-streams from LEMO-based prescan programs.
 
   (15) Specification of TITLES for individual histograms supported.
 
   (16) Specification of ID's   for individual histograms supported.
 
   ---------------------------------------------------------------------------
 
 
   For this version:
 
   Maximum number of parameters supported                 = 2000
 
   Maximum histogram size (if scan-time memory available) = 512 megachannels
 
   Maximum number of bananas (from one file only)         = 880
 
   Maximum banana space (in 256-channel banana units)     = 864
 
    
   02-Feb-02 ........... U350  CHIL (UNIX version) - WTM ............ PAGE   4
 
 
   U350.030  CHIL SYNTAX - DEFINITIONS AND ASSIGNMENTS
 
   $LSTL = LENG                      ;Specify Tape Record Length (bytes)
   $NPR = NPAR                       ;NPAR =Max# Parms (RAW+GENERATED)
 
   $LPR IPA TO IPB,IS  = LENG        ;Specify Parm Lengths (loop)
   $LPR IPA TO IPB     = LENG        ;Specify Parm Lengths (loop)
   $LPR IP1,IP2,IP3,.. = LENG        ;Specify Parm Lengths (list)
   ---------------------------------------------------------------------------
   $BAN (LENG) ID1,ID2,ID3 ......... ;Specify BAN-ID's to get from file
   $BAN (LENG) IDA,IDB,IDC ......... ;Determines stacking order
   $BAN (LENG) ..................... ;LENG must match that from file
 
   $BAF filename.ban                 ;Specify ban-file to process
                                     ;Must follow $BAN specification
   ---------------------------------------------------------------------------
   $GAT (ISN,NG) (ISN,NG) (ISN,NG)   ;Specify (Set#, # of gates)
   $GAT (ISN,NG) (ISN,NG) (ISN,NG)   ;   ''     ''      ''       ''
   $GAT ...........................  ;As many lines as required
 
   $MAPF (LENG) ISA,ISB,ISC .......  ;Map Sets ISA,ISB,ISC .........
                                     ;with length = LENG
                                     ;must follow $GAT-definition
 
   $GAF filename.gaf                 ;Specify Gate-file to process
                                     ;Must follow $MAPF spec if any
   ---------------------------------------------------------------------------
   $MAPL  LENG,ISN (LO,HI) (LO,HI).. ;Map specific Gate-List with
   &(LO,HI) (LO,HI) ................ ;SET# = ISN (must be unique)
   & ............................... ;(any # of continuation lines)
 
   $GLST  LENG,ISN (LO,HI) (LO,HI).. ;Specific unmapped Gate-List with
   &(LO,HI) (LO,HI) ................ ;SET# = ISN (must be unique)
   & ............................... ;(any # of continuation lines)
   ---------------------------------------------------------------------------
   $DIP SYMA(IV),SYMB(JV)...         ;Define Parameter names & dimensions
 
   $ASS SYM(ISA TO ISB,INC)=JLO,JNC  ;Assign values to Parameter names
   $ASS SYM(ISA TO ISB)    =JLO,JNC  ;Assign values to Parameter names
   $ASS SYM(IS1,IS2...)    =JLO      ;Assign value  to Parameter name
   ---------------------------------------------------------------------------
   $DIM SYMA(IV),SYMB(JV)...         ;Define name & dimension of symbols
 
   $DAT SYM(ISA TO ISB)    =V1,V2... ;Assign values to symbol names
   $DAT SYM(IS)            =V1       ;Assign value  to symbol name
   ---------------------------------------------------------------------------
   XX=EXPRESSION                     ;Define or re-define scaler XX
   SYMB(I)=EXPRESSION                ;Set SYMB(I) equal to EXPRESSION
   ---------------------------------------------------------------------------
   $HWD ICOD,ICOD,ICOD .....         ;Insert Half-WD code into MIL
   $FWD ICOD,ICOD,ICOD .....         ;Insert Full-WD code into MIL
   $H16                              ;Specify 16-bits/channel
   $H32                              ;Specify 32-bits/channel
   $HID NUID                         ;Specify next H-ID to use
   $TEX TEXT                         ;Up to 76 bytes of text
   $TIT TITLE                        ;Specify title (40 bytes max)
    
   02-Feb-02 ........... U350  CHIL (UNIX version) - WTM ............ PAGE   5
 
 
 
   U350.030  CHIL SYNTAX - ASP, IF & CALL STATEMENTS
 
   ASP(IP,IV)                        ;Assign to Parm-IP the value IV
 
   IFU(COND)LABEL                    ;IF COND  unsatisfied, GO TO LABEL
   IFS(COND)LABEL                    ;IF COND    satisfied, GO TO LABEL
                                     ;otherwise, drop thru (IFU & IFS)
 
   IFC(CONDITION-SET)   L1,L2,L3.... ;Defines a "Computed GO TO"
   IFC(GS(P,IS,NA,NB))  L1,L2,L3.... ;IF COND(1) satisfied, GO TO L1
   IFC(B(PX,PY,IDA,IDB))L1,L2,L3.... ;IF COND(2) satisfied, GO TO L2
                                     ;IF COND(J) satisfied, GO TO LJ
                                     ;otherwise, drop through
                                     ;Limited use inside loop: ref Note-1
 
   IFX(PARM)LABEL                    ;IF PARM       exists, GO TO LABEL
   IFN(PARM)LABEL                    ;IF PARM    non-exist, GO TO LABEL
                                     ;otherwise, drop thru (IFX & IFN)
 
   IFP(PARM)L1,L2,L3,,LJ,..........  ;IF PARM = 0         , GO TO L1
                                     ;IF PARM = 1         , GO TO L2
                                     ;IF PARM = J-1       , GO TO LJ
                                     ;otherwise, drop through
                                     ;Cannot be used inside loops!
 
   BTAB(PARM,MASK)LALL,LSOME,LNONE   ;GO TO LALL,LSOME,LNONE IF
                                     ;       ALL ,SOME, NONE bits match
                                     ;Goes to LNONE if Parm non-exist
                                     ;no drop thru for BTAB
 
   CALL USERSUB1                     ;CALL USERSUB1
   CALL USERSUB2                     ;CALL USERSUB2
   CALL USERSUB3                     ;CALL USERSUB3
 
   CALL REPACK1   PA,PB              ;Send Parms PA-PB to Out-Stream-1
   CALL REPACK2   PA,PB              ;Send Parms PA-PB to Out-Stream-2
   CALL REPACK3   PA,PB              ;Send Parms PA-PB to Out-Stream-3
 
 
   Important Comment on Banana-Gate Specifications - B(PX,PY IDA,IDB)
 
   At a given X-coordinate - Banana IDA+1 must lie above Banana IDA
 
                           - Banana IDA+2 must lie above Banana IDA+1
 
                           - Banana IDA+N must lie above Banana IDA+N-1
 
 
   Note-1:  IFC- and IFP-statements may be used inside loops ONLY on the
   condition that all associated branches are to labels outside the loop.
    
   02-Feb-02 ........... U350  CHIL (UNIX version) - WTM ............ PAGE   6
 
 
 
   U350.030  CHIL SYNTAX - HISTOGRAM STATEMENTS
 
   THE HISTOGRAM SPECIFICATION STATEMENT TAKES THE GENERAL FORM:
 
   H(I,J..) L(LI,LJ..) "CONDITIONS"
 
   or
 
   H(I,J..) L(LI,LJ..) R(LOI,HII LOJ,HIJ..) "CONDITIONS"
 
   Where:
 
   H(I,J..)                 ;Says histogram Parms I,J..
 
   OH(I,J.)                 ;Says histogram Parms I,J..
                            ;in previously defined "space"
 
   L(LI,LJ,LK..)            ;Defines histogram "Lengths" (Pwr of 2)
                            ;Length of Parm-I = LI
                            ;Length of Parm-J = LJ
                            ;Length of Parm-K = LK
 
   R(LOI,HII LOJ,HIJ ..)    ;Says select Hist-Parm Range (after
                            ;shifting) for
                            ;Parm-I to be LOI thru HII
                            ;Parm-J to be LOJ thru HIJ
   ---------------------------------------------------------------------------
   G(P L,H L,H ..)          ;Explicit Gate-List (P# LO,HI LO,HI ...)
                            ;(not mapped)
 
   GS(P,IS,NA,NB)           ;Gate-List specified by Set# (IS), Gate#'s
                            ;(NA,NB) from $GAT, $GLST- or $MAPL-spec.
                            ;Mapped if included in $MAPF-spec or given
                            ;in $MAPL-spec and if (NA,NB) specifies
                            ;full Gate-Set: Otherwise, not mapped.
 
   B(PX,PY,IDA,IDB)         ;Ban-List from ban-file (X-Parm, Y-Parm,
                            ;list of adjacent Bananas)
 
 
   Important Comment on Banana-Gate Specifications - B(PX,PY IDA,IDB)
 
   At a given X-coordinate - Banana IDA+1 must lie above Banana IDA
 
                           - Banana IDA+2 must lie above Banana IDA+1
 
                           - Banana IDA+N must lie above Banana IDA+N-1
    
   02-Feb-02 ........... U350  CHIL (UNIX version) - WTM ............ PAGE   7
 
 
 
   U350.040  CHIL SYNTAX - GENERAL RULES
 
                       RULES WHICH APPLY TO ALL STATEMENTS
 
   (1)....All 80 cols of a line may be used.
 
   (2)....A blank line shows up as a blank line on the listing.
 
   (3)....All left and right parenthesis must match.
 
   (4)....A semicolon ";" is used to introduce a "comment field".
 
                 RULES FOR THE $-DIRECTIVE AND EQUATE STATEMENTS
 
   (1)....All $-directives ($NPR, $DIM, $ASS, etc) must start in col-1.
 
   (2)....No continuation of $-directives are allowed except for the
          $MAPL- and $GLST-statements.
 
   (3)....Equate-directives may start anywhere but shall not be labeled.
 
                         RULES FOR ALL OTHER STATEMENTS
 
   (1)....Statement labels must be contained within cols 1 thru 5 and be
          no more than 4 characters in length.
 
   (2)....Statement labels may be composed of any combination of numeric
          and alphabetic characters.
 
   (3)....All statements must start in col-7 or beyond.
 
   (4)....Only the H-statement may be "continued" - any char in col-6.
          the max number of continuation lines = 19.
   ---------------------------------------------------------------------------
   SYNTAX IS RIGID - FOLLOW DOCUMENTATION & EXAMPLES CAREFULLY
   ---------------------------------------------------------------------------
 
                                    COMMENTS
 
   Meaning of the Assign directive - $ASS SYM(I TO J)=JLO,JNC
 
   This  means  assign  symbols  SYM(I),  SYM(I+1)  ... SYM(J) the values JLO,
   JLO+JNC, JLO+2*JNC ....
 
   Difference between $DIP and $DIM
 
   Values assigned to symbols dimensioned via the $DIP specification  must  be
   unique  and in the range 1 through NPAR (the max # Parms given by $NPR). In
   addition, only those parameters defined by the $DIP specification  will  be
   represented  by  their  "names" on the histogram summary. Values may not be
   assigned to $DIP-symbols via an "Equate". None  of  these  rules  apply  to
   $DIM-symbols.
    
   02-Feb-02 ........... U350  CHIL (UNIX version) - WTM ............ PAGE   8
 
 
 
   U350.050  PARAMETER LENGTHS
 
   The  "Parameter  Length" defines the maximum value (always a power of 2) to
   be associated with a given parameter. Parameter Length specifications  must
   follow the $NPR-statement. The following specification forms are supported:
 
   $LPR IPA TO IPB,IS  = LENG    ;Sets length of parameters (IPA to IPB
                                 ;in steps of IS) to be LENG.
 
   $LPR IPA TO IPB     = LENG    ;Sets length of parameters (IPA to IPB
                                 ;in steps of 1) to be LENG.
 
   $LPR IP1,IP2,IP3,.. = LENG    ;Sets length of parameters (IP1,IP2,
                                 ;IP3,...) to be LENG.
 
   Lengths  may  be  re-specified  as many times as convenient but all lengths
   must be given at least once before another statement type  is  encountered.
   The following example illustrates the procedure:
 
   $NPR = 240
 
   $LPR   1 TO 240    = 2048
 
   $LPR 220 TO 222    = 1024
 
   $LPR 229 TO 239,2  = 8192
 
   $LPR 219           = 64
    
   02-Feb-02 ........... U350  CHIL (UNIX version) - WTM ............ PAGE   9
 
 
 
   U350.060  PARAMETER NAMES
 
   Names  (beginning  with  a  letter and up to 8 characters in length) may be
   associated with some or all parameter numbers and can be used to  reference
   any parameter, so named, in the CHIL program. Two steps are required.
 
   (1)....First,  the  name  must  appear  in  a   $DIP-statement   (dimension
          parameter statement) even if it's dimension is to be only unity.
 
   (2)....Values  must  then  be  assigned  to  the  names  by  means  of  the
          $ASS-statement (assign statement).
 
   The $DIP-statement takes the form:
 
   $DIP NAME1(IV),NAME2(JV),..     ;Where, NAME1, NAME2, .. denotes the
                                   ;parameter names being defined and
                                   ;IV, JV, .. denotes the associated
                                   ;dimension.
 
   The $ASS-statement takes the form:
 
   $ASS NAME1(IA TO IB,IS)=JLO,JNC ;Where, NAME1 denotes the name to
                                   ;which values are being assigned and
                                   ;(IA TO IB,IS) defines a loop on the
                                   ;name index.
 
                                   ;JLO is the first value to assign.
                                   ;JNC is the value-increment to be
                                   ;added for successive assignments.
                                   ;In this case we have:
 
                                   ;NAME1(IA)      = JLO
                                   ;NAME1(IA+IS)   = JLO+JNC
                                   ;NAME1(IA+2*IS) = JLO+2*JNC
 
   $ASS NAME2(IA TO IB)  =JLO,JNC  ;Assigns values with index increment
                                   ;equal to unity.
 
   $ASS NAME2(IC)        =JLO      ;Assigns the value JLO to NAME2(IC)
 
   THE FOLLOWING EXAMPLE ILLUSTRATES:
 
   $DIP NAI(72),TAC(72),GELI(6)    ;Define parameter names & dimensions
   $DIP TOTK(1)
 
   $ASS NAI(1 TO 72) = 1,3         ;Assign values 1,4,7,10,13 .........
   $ASS TAC(1 TO 72) = 2,3         ;Assign values 2,5,8,11,14 .........
   $ASS TOTK(1)      = 219         ;Assign the value 219 to TOTK(1)
    
   02-Feb-02 ........... U350  CHIL (UNIX version) - WTM ............ PAGE  10
 
 
   U350.070  GATES - SIMPLE (LO-LIMIT, HI-LIMIT PAIRS)
 
   Simple gates (i.e.  Lo-Limit,  Hi-Limit  pairs  to  be  imposed  on  single
   parameters) can be entered in three different ways.
 
   METHOD - 1 ----------------------------------------------------------------
 
   A  single gate of the form: G(PARM LO,HI) may be defined in an IF-statement
   (IFS or IFU) or an H-statement.  The  H-statement  also  allows  the  form:
   G(PARM  LO,HI  LO,HI  LO,HI ..) - i.e multiple gates on a single parameter.
   Here, PARM denotes the test-parameter number or name and LO,HI denotes  the
   pair  of  limits  to be imposed. Gates entered in this way must be given in
   terms of the raw test-parameter length.
 
   METHOD - 2 ----------------------------------------------------------------
 
   Multiple gates may also be entered by means of  the  $GLST-statement  which
   takes the form:
 
   $GLST LENG,ISN LO,HI LO,HI LO,HI .....
 
   Where,  LENG  represents the Parameter-Length on which the limits are based
   and ISN  the  Set-Number  by  which  this  particular  Gate-Set  is  to  be
   referenced.  In this case, limits do not have to be entered on the basis of
   the raw test-parameter length.
 
   METHOD - 3 ----------------------------------------------------------------
 
   Gate-Sets may also be read from a disk file  generated  by  program  GATLIN
   which  allows one to define gates interactively. Input of gates from such a
   file requires two types of statements:
 
   $GAT (ISN,NG) (ISN,NG) ....
   and
   $GAF filename.gaf           ;The .gaf-extension is required
 
   The $GAT-statement specifies which Gate-Set ID's (denoted by  ISN)  are  to
   be  read  from  the file and the maximum number of gates (denoted by NG) to
   be read from each Gate-Set. Note: that the file may actually  contain  more
   or  less gates per Set than is specified. If it contains less, the "excess"
   gates are set to "impossible". If it  contains  more,  only  the  first  NG
   gates  are  read  in.  The  length  basis on which the gates are defined is
   provided automatically by GATLIN. The $GAF-statement specifies the name  of
   the file to be processed
 
                               IMPORTANT COMMENTS
 
   Gates entered via METHOD-2 or -3 may only be used in H-statements  and
   Computed  GOTO's (IFC-statements) and are referenced by Set# (ISN) and
   a Gate# range (NA thu NB) as follows.
 
   GS(PARM,ISN NA,NB)
 
   Gates must be entered in increasing order and cannot overlap.
    
   02-Feb-02 ........... U350  CHIL (UNIX version) - WTM ............ PAGE  11
 
 
   U350.080  GATES - SIMPLE (MAPPED)
 
   If  a large number (5 or more) gates are to be imposed on a given parameter
   (in an H-statement), processing  will  proceed  faster  if  the  gates  are
   "mapped". This may be accomplished in two ways:
 
   METHOD - 1 ----------------------------------------------------------------
 
   One  may  specify the gates to be mapped by means of the $MAPL-statement as
   follows:
 
   $MAPL LENG,ISN  LO,HI LO,HI LO,HI .........................................
   & ............ as many continuation lines as required .....................
 
   Where, LENG represents the Parameter-Length on which the limits  are  based
   and  ISN  the  Set-Number  by  which  this  particular  Gate-Set  is  to be
   referenced. In this case, limits do not have to be entered on the basis  of
   the raw Test-Parameter length.
 
   METHOD - 2 ----------------------------------------------------------------
 
   Gate-Sets  may  also  be  read from a disk file generated by program GATLIN
   which allows one to define gates interactively. Mapping of gates from  such
   a file requires three types of statements given in the order shown below:
 
   $GAT  (ISN,NG) (ISN,NG) ........
   $MAPF LENG ISA,ISB,ISC  ........
   $GAF  filename.gaf               ;The .gaf-extension is required
 
   The  $GAT-statement  specifies  which Gate-Set ID's (denoted by ISN) are to
   be read from the file and the maximum number of gates (denoted  by  NG)  to
   be  read  from each Gate-Set. Note: that the file may actually contain more
   or less gates per set than is specified. If it contains less, the  "excess"
   gates  are  set  to  "impossible".  If  it contains more, only the first NG
   gates are read in.
 
   The $MAPF-statement specifies which of these Gate-Sets  are  to  be  mapped
   (ISA,  ISB  ..  etc) and gives the length basis LENG on which mapping is to
   be done. Note: This can be different from the basis  on  which  gates  were
   entered.
 
   The $GAF-statement specifies the name of the file to be processed
 
                                IMPORTANT COMMENT
 
   Gates  entered  as  described  above  may only be used in H-statements  and
   Computed  GOTO's (IFC-statements) and are referenced by  Set#  (ISN)  and a
   Gate# range (NA thu NB) as follows.
 
   GS(PARM,ISN NA,NB)
 
   For  gate  mapping  to  be  used,  the  full  Gate-Set  specified  by   the
   $GAT-statement or by the $MAPL-statement must be requested by (NA,NB).
    
   02-Feb-02 ........... U350  CHIL (UNIX version) - WTM ............ PAGE  12
 
 
 
   U350.090  GATES - FREE-FORM (BANANAS)
 
   Free-Form-Gates  must always be created "outside" of the CHIL program using
   the   interactive  display  program  RIP  and  stored  on  a  special  file
   (ban-file). Incorporation of  such  Free-Form-Gates  into  a  CHIL  program
   requires one or more statements of the following form:
 
   $BAN  LENG  IDA,IDB,IDC ....
 
   The  $BAN-statement specifies which Banana ID-numbers (IDA, IDB ...) are to
   be retrieved from the ban-file and gives the X-length basis LENG  on  which
   they  were  constructed  or  will be constructed. At this time space is set
   aside to accommodate Free-Form-Gates which are to be processed.
 
   The ban-file to be processed may be specified in the CHIL program  or  this
   specification  may  be  deferred  until run-time. Specification in the CHIL
   program takes the following form:
 
   $BAF filename.ban        ;The .ban-extension is required
 
   If the $BAF-statement is not included in the CHIL program or if it  desired
   to  process  a  new or modified ban-file at run-time, the following command
   may be issued to the monitor task or scan program.
 
   NUBAN  filename.ban
 
   At such time, all Free-Form-Gates are reset to "impossible" and  the  given
   file is processed to provide a new set.
 
                                    COMMENTS
 
   (1)....The  length given in the $BAN-statement must match that contained in
          the ban-file.
 
   (2)....Any ID-numbers specified in the $BAN-statement but not found in  the
          ban-file  will  be  left  set  to  "impossible"  and  a  "not found"
          diagnostic message will be issued when such file is processed.
 
   (3)....If no $BAF-statement is entered at CHIL-time,  no  comment  will  be
          made - you will find out sooner or later, I guess.
    
   02-Feb-02 ........... U350  CHIL (UNIX version) - WTM ............ PAGE  13
 
 
   U350.100  IF-STATEMENTS AND COMPUTED GOTO'S
 
   CHIL supports IF-statements of the following forms: -----------------------
 
   IFS(CON)LABEL           ;IF condition CON   satisfied, GO TO LABEL
   IFU(CON)LABEL           ;IF condition CON unsatisfied, GO TO LABEL
   IFX(PAR)LABEL           ;IF parameter PAR      exists, GO TO LABEL
   IFN(PAR)LABEL           ;IF parameter PAR  non-exists, GO TO LABEL
 
   Here,  CON  denotes a single Simple-Gate or a single Free-Form-Gate and PAR
   denotes a parameter number or name. Note examples below:
 
   IFS(G(1,50,100))100     ;IF gate on Parm-1 (50 thru 100) is
                           ;satisfied, GO TO statement LABEL 100.
 
   IFU(G(NAI(7),50,100))20 ;IF gate on Parm-NAI(7) (50 thru 100)
                           ;is not satisfied, GO TO 20.
 
   IFU(B(E(1),DE(1) 1))ZIP ;IF Free-Form-Gate test on X-Parm-E(1) and
                           ;Y-Parm-DE(1), using Banana-ID# 1, is not
                           ;satisfied, GO TO the statement labeled, ZIP.
 
   IFN(23) NEXT            ;IF Parm-23 does not exist, GO TO NEXT.
 
   CHIL supports computed GOTO's of the following forms: ---------------------
 
   IFC(COND-SET)L1,L2,L3 ..;COND-SET denotes a Condition-Set
                           ;If 1st member is satisfied, GO TO L1
                           ;If 2nd member is satisfied, GO TO L2
                           ;... etc ..., otherwise, drop through.
 
   IFP(PARM)    L1,L2,L3 ..;PARM denotes a parameter number or name
                           ;IF PARM = 0,   GO TO L1
                           ;IF PARM = 1,   GO TO L2
                           ;IF PARM = J-1, GO TO LJ
                           ;... etc ..., otherwise, drop through.
 
   The   "Condition   Set"  refered  to  here,  denotes  an  adjacent  set  of
   Free-Form-Gates (ordered such that at a given X-coordinate,  Banana-N  lies
   below  Banana-(N+1) which are all of the same "length" and whose ID-numbers
   are consecutive or a list of Simple Gates which have been specified by  the
   $GLST-statement  or  defined  by the $GAT-statement and read in from a disk
   file by the $GAF-statement or the NUGAT command. A couple of  examples  are
   given below:
 
   IFC(GS(ETOT(1),ISN 1,3)) 10,20,30  ;Test-Parm=ETOT(1), Gate-Set#=ISN
                                      ;gates 1 thru 3 are to be tested.
 
   IFC(B(10,11 5,8)) 100,200,300,400  ;X-Parm=10, Y-Parm=11, Banana-ID's
                                      ;5 thru 8 are to be tested.
 
   Note: # of gates or Banana-ID's implied must match number of labels!
    
   02-Feb-02 ........... U350  CHIL (UNIX version) - WTM ............ PAGE  14
 
 
 
   U350.110  BIT-TESTS
 
   CHIL  supports  a  bit-test and three-way branch statement of the following
   form:
 
   BTAB(PARM,MASK)LALL,LSOME,LNONE
 
   where,
 
   PARM denotes the parameter number or name to be  tested  and  MASK  defines
   the  bits  of  interest.  MASK  can  be,  most  conveniently,  specified as
   hexadecimal number whose first character must  be  a  decimal  integer  and
   whose last character must be an "H".
 
   The statement means:
 
   GOTO...the  statement labeled LALL, if all of the bits which are set in the
          MASK are also set in the Test-Parameter.
 
   GOTO...the statement labeled LSOME, if some  (but  not  all)  of  the  bits
          which are set in the MASK are also set in the Test-Parameter.
 
   GOTO...the  statement  labeled  LNONE, if none of the bits which are set in
          the MASK are set in the Test-Parameter.
 
   GOTO...the statement labeled LNONE if the  Test-Parameter  does  not  exist
          (i.e. has the hexidecimal value FFFF).
 
   There is no "drop through" condition for this test.
    
   02-Feb-02 ........... U350  CHIL (UNIX version) - WTM ............ PAGE  15
 
 
 
   U350.120  HISTOGRAM - BITS/CHANNEL
 
   The  number of bits per channel used for histogram storage may be set to be
   either 16 (half word) or 32 (full word). This is accomplished in  the  CHIL
   program by the following two direstives:
 
   $H16   ;Sets 16-bits/channel - Max count = 65,535 ------ the default
 
   $H32   ;Sets 32-bits/channel - Max count = 4,294,967,303
 
   There  is no restriction on how many such directives may be entered but the
   user should be aware of the fact  that  each  such  entry  results  in  the
   histogram  space  required being rounded up to the next integer multiple of
   32768 16-bit channels (i.e. next multiple of 64 kb). I will say it again.
 
       Each $H16 or $H32 entry rounds up memory required to a multiple of
                                  64 kilobytes
 
   U350.130  HISTOGRAM - ID NUMBERS
 
   If   you  do  nothing,  histogram  ID-numbers  will  start  at  1  and  run
   consecutively, however, the beginning of  the  ID-number  sequence  may  be
   redefined by the following statement:
 
   $HID  ID
 
   Where,  ID  (a  decimal integer of up to 8 digits) denotes the ID-number to
   be assigned to the next histogram processed. Histogram ID-sequences may  be
   defined  any  number  of  times in the CHIL program so long as one sequence
   does not overlap another.
 
   U350.140  HISTOGRAM - TITLES
 
   A title of up to 40 characters may be associated  with  each  histogram  by
   using the following statement.
 
   $TIT  TITLE
 
   Where  TITLE  denotes the 40 ASCII characters mentioned above. Once a title
   is defined, it becomes associated (stored in the directory  entry)  of  all
   subsequent histograms until re-defined.
 
   Program  HEDDO  may  be  used to display and list directory entries as well
   modify (or add) titles contained therein.
    
   02-Feb-02 ........... U350  CHIL (UNIX version) - WTM ............ PAGE  16
 
 
 
   U350.150  HISTOGRAM - STATEMENTS
 
   The Histogram statement (H-statement) specifies which parameters are to  be
   histogrammed,  the  length  of each histogram parameter (implies the number
   of bits to shift the associated raw parameter), optionally - the  range  to
   be  selected  after  shifting and any conditions (Simple Gates or Free-Form
   Gates) to be imposed. The H-statement must start in  col-7  or  beyond  and
   has the following general form:
 
 
    H(I,J..) L(LI,LJ..) R(LOI,HII LOJ,HIJ..)  "CONDITIONS"
                        !                  !  !          !
                        !---- optional ----!  ! optional !
   or
 
   OH(I,J..) L(LI,LJ..) R(LOI,HII LOJ,HIJ..)  "CONDITIONS"
                        !                  !  !          !
                        !---- optional ----!  ! optional !
   Where:
 
   H(I,J..)                 ;Means histogram parameters I,J .. etc
 
   OH(I,J.)                 ;Means histogram parameters I,J .. etc
                            ;in the same "space" defined by the most
                            ;recent H-statement.
 
   L(LI,LJ,LK..)            ;Defines histogram parameter "lengths".
                            ;always a power-of-2
                            ;Implies # of bits to shift raw parameter.
                            ;Specifies histogram dimensions unless the
                            ;range is specified explicitly by the
                            ;R-specification below.
                            ;Length of histogram-parameter-I = LI
                            ;Length of histogram-parameter-J = LJ
                            ;Length of histogram-parameter-K = LK
 
   R(LOI,HII LOJ,HIJ ..)    ;Means select histogram-parameter Range
                            ;(after shifting - i.e. on the basis of the
                            ;length defined above) for:
                            ;Parameter-I  to be LOI thru HII
                            ;Parameter-J  to be LOJ thru HIJ
 
   I  will  try  to  say  it in a different way - just to fill out the page if
   nothing else. The L-specification L(LI,LJ ..) tells CHIL how many  bits  to
   shift  the  raw  parameters  prior  to  histogramming.  The R-specification
   R(LOI,HII LOJ,HIJ ..), which is optional, can be used to select  a  limited
   portion  of  the histogram dimensions given by the L-specification. See the
   last section for some examples.
 
                 SEE NEXT PAGE FOR A DISCUSSION OF "CONDITIONS"
    
   02-Feb-02 ........... U350  CHIL (UNIX version) - WTM ............ PAGE  17
 
 
 
   U350.150  HISTOGRAM - STATEMENTS (CONTINUED)
 
   Histogram "CONDITIONS" consist of one or more  Condition  Lists  (OR-lists)
   of the following types:
 
   G(P LO,HI LO,HI ..)      ;Explicit list of non-overlapping simple
                            ;gates
                            ;P - denotes test parameter number or name
                            ;LO,HI - denotes gate-limits
                            ;not mapped
 
   GS(P,IS,NA,NB)           ;Gate-List specified by Set# (IS), Gate#'s
                            ;(NA,NB) from $GAT-, $GLST-, or
                            ;$MAPL-statement.
                            ;mapped if included in a $MAPF-statement or
                            ;given in a $MAPL-statement and if (NA,NB)
                            ;specifies the full Gate-Set:
                            ;Otherwise, not mapped.
 
   B(PX,PY IDA,IDB)         ;Free-Form Gate-List from ban-file
                            ;PX = X-test-parameter number or name
                            ;PY = Y-test-parameter number or name
                            ;IDA,IDB - gives first and last ID-numbers
                            ;of a list of "adjacent" Free-Form-Gates.
                            ;That is, at a given X-coordinate, BAN-IDA
                            ;must lie below BAN-(IDA+1).
                            ;All members of list must be of equal
                            ;length and have consecutive ID-numbers.
 
                 WHAT IS THE IMPLICATION OF MULTIPLE GATE-LISTS?
 
   Definition:  OR-list = Simple Gate-List or Free-Form Gate-List.
 
   You  will  get  a  count  "somewhere"  if  and  only if some member of each
   OR-list is satisfied. If you have N OR-lists in the H-statement  and  NG(I)
   represents  the  number  of  gates  in  the  Ith  list,  then the number of
   histograms (NH) actually implied by the H-statement is given by:
 
                       NH=NG(1)*NG(2)*------NG(N-1)*NG(N)
    
   02-Feb-02 ........... U350  CHIL (UNIX version) - WTM ............ PAGE  18
 
 
 
   U350.160  LOOPS
 
   Loops are implemented so as to "look like FORTRAN". However, one must  keep
   in  mind  that  "loops" are actually expanded by the CHIL complier and that
   all "loop parameters" are constants. See SEC# U350.170 for more  discussion
   of symbols and expressions.
 
   Two examples which make use of loops are given below.
 
   EXAMPLE-1: ----------------------------------------------------------------
 
   Produce  72 1-D histograms of parameters 1,4,7...214 with a gate (500,1000)
   set on associated parameters 2,5,8...215.
 
         DO 100 I=1,214,3
         J=I+1
         H(I) L(512) G(J,500,1000)
     100 CONTINUE
 
   EXAMPLE-2: ----------------------------------------------------------------
 
   Produce 72 1-D histograms of parameters 1,4,7...214  with  the  requirement
   that a parameter named ETOT (parameter # 250) be in the range 1000 to 2000.
   Define parameter names and associated values as follows:
 
   $DIP NAI(72),ETOT(1)                 ;Define parameter names
   $ASS NAI(1 TO 72)=1,3                ;Assign values to NAI
   $ASS ETOT(1)=250                     ;Assign value  to ETOT
 
         IFU(G(ETOT(1),1000,2000)) 200  ;Skip loop if gate un-satisfied
 
         DO 100 I=1,72                  ;Loop on 72 NAI detectors
         H(NAI(I)) L(256)               ;Histogram NAI(I)
     100 CONTINUE
 
     200 CONTINUE
    
   02-Feb-02 ........... U350  CHIL (UNIX version) - WTM ............ PAGE  19
 
 
 
   U350.170  SYMBOLS & EXPRESSIONS
 
   Symbols  defined  by  the  "Equate  Directive"  (SYM=EXPRESSION) are called
   compile-time   variables.   Such   variables   may  be  re-defined  without
   restriction. These are not run-time variables!  -  THERE  ARE  NO  RUN-TIME
   VARIABLES  IN  CHIL!!  Compile-time  variables  must be defined in terms of
   numbers and/or previously defined compile-time variables. At a given  place
   in  a  CHIL program, the value associated with a given symbol is always the
   same - NO MATTER HOW YOU GET THERE!!
 
   The CHIL compiler supports simple expressions which are evaluated  left  to
   right.  REPEAT!! EVALUATED! LEFT! TO! RIGHT! Let V represent a single value
   (number or previously defined symbol). Let S represent an algebraic sum  of
   V's. Expressions of the following type are legal:
 
   A=V
   A=S
   A=S*V+S                Means: A=(S)*V+S
   A=S/V+S                Means: A=(S)/V+S
   A=S/V*V+S              Means: A=((S)/V)*V+S
   A=MOD(S,S)             No additional terms allowed
                          (same argument definition as in FORTRAN)
   A=S+[S,S,..]           Where [ ] encloses a list of bit-numbers
                          (Lo-order bit-number = 1)
 
   NUMERICAL EXAMPLES:
 
   ASSIGNMENT             RESULT
 
   A=10                   A=10
   B=A+4                  B=14
   C=A+B-9                C=15
   D=100/B                D=7
   E=A+B+C/6              E=6
   F=A+B-C*7              F=63
   G=A+B-C/6*A            G=10
   H=A+B-C/6*A+C+10       H=35
   I=MOD(A+B,7)           I=3
   J=MOD(A+B,E-1)         J=4
   K=[A+6,3]              K=8004 (HEX)
   L=[A+6,3]+40H          L=8044 (HEX)
   M=[2,1]                M=3
   N=[A+6,M]              N=8004 (HEX)
 
      * * * PARENTHESES ARE NOT ALLOWED EXCEPT IN THE "MOD STATEMENT" * * *
    
   02-Feb-02 ........... U350  CHIL (UNIX version) - WTM ............ PAGE  20
 
 
 
   U350.180  PRE-SCANNING - CONSIDERATIONS
 
   Pre-scanning  (as  interpreted  here)  involves  the processing of an input
   data stream (usually from Mag  Tape)  to  produce  an  output  data  stream
   (usually  to  Mag  Tape). The "processing" may include selection of certain
   events, selection of certain parameters, modification of  input  parameters
   or creation of new parameters or any combination of the preceding.
 
   A  CHIL-based  prescan  task  involves  the  use  of  LEMO (or a customized
   version thereof) to control the process combined with a CHIL program  which
   aides  in  the  selection  of  events  and/or  parameters.  Modification or
   creation of parameters will normally  require  one  or  more  user-supplied
   subroutines USERSUBS.
 
   The  CHIL-based prescan program may include any legal CHIL statement except
   for the H-statement (i.e. concurrent histogramming is not  supported).  The
   CALL  REPACK1 (not legal in histogramming programs) is used to initiate the
   "saving of an event" into the output  data  stream.  The  following  simple
   example illustrates:
 
   ---------------------------------------------------------------------------
 
   Prescan  program  which  selects  only those events for which parameters 18
   and 16 satisfy Free-Form-Gate number-1  from  file  DEC2.ban.  If  gate  is
   satisfied, all parameters of event are saved.
 
   $LSTL        = 8192        ;Specify tape record length in bytes
 
   $NPR 18                    ;Specify number of parameters
 
   $LPR 1 TO  2 = 64          ;Specify length of parameters 1 & 2
   $LPR 3 TO 18 = 2048        ;Specify length of parameters 3 to 18
 
   $BAN (256) 1               ;Request Free-Form-Gate from ban-file
                              ;(X-length = 256, ID-number = 1)
 
   $BAF DCE2.ban              ;Give name of ban-file
 
         IFU(B(18,16 1)) 100  ;Test X,Y-parameters (18,16) against gate
                              ;Skip it if gate not satisfied
 
         CALL REPACK1 1,18    ;Otherwise, save event in output stream
                              ; 1 is lowest  parameter-# to save
                              ;18 is highest parameter-# to save
 
     100 CONTINUE
   ---------------------------------------------------------------------------
    
   02-Feb-02 ........... U350  CHIL (UNIX version) - WTM ............ PAGE  21
 
 
 
   U350.190  USER-SUPPLIED SUBROUTINES
 
   The  user  is able (by means of from 1 to 3 user-supplied subroutines named
   USERSUB1, USERSUB2, USERSUB3) to intercept and  modify  the  Event-by-Event
   data  stream that is being processed by a CHIL-based tape scan, prescan, or
   on-line monitor task. (i.e. standard programs  designed  to  process  HHIRF
   format  L002).  The  interception  occurrs  after  the  event  is unpacked,
   always, and when the CALL USERSUB1, etc is executed in  the  CHIL  program.
   Thus,  new  or  modified  parameters  may  be  ,subsequently,  tested   and
   histogrammed  in  the  same  way  as  any  others. The number of parameters
   specified   by   the  $NPR-statement  must  be  increased  to  include  any
   additional parameters which are generated.
 
   Stock histogramming and prescan tasks include a dummy  USERSUB:  Customized
   packages  are  created by linking a new task in which this dummy routine is
   replaced by the user's routines. A customized task usually includes a  user
   command  processor  USERCMP and possibly other support routines in addition
   to the  USERSUBS.  A  skeleton  USERSUB  which  just  sets  parameter-50  =
   parameter-1 is shown below:
 
         SUBROUTINE USERSUB1(IBUF)
         INTEGER*2 IBUF(512)
         IBUF(50)=IBUF(1)
         RETURN
         END
 
   Your  routine  will  receive  the  "Event"  in  expanded form. That is, all
   parameters will be in their "proper place" and any  parameters  which  were
   not present in the raw event will be set to X'FFFF'.
 
   Suppose  that  the  maximum # of parameters in the original event is 20 and
   USERSUB1 is to generate up to 10 more. You would specify $NPR = 30 in  your
   CHIL  program. When your subroutine is entered, IBUF(I),I=1,20 would be set
   to   the  parameter  value  or  to  X'FFFF'  for  any  missing  parameters.
   IBUF(I),I=21,30 would be set to X'FFFF'.
 
   Parameters which you generate should be in the range 0  -  16383  or  less.
   Any  missing  parameters  should be left set to X'FFFF'. In the interest of
   speed, USERSUBS should avoid calling other routines  on  an  Event-by-Event
   basis.  Subroutine and function calls take at least 10 microseconds and the
   linkage time increases as the number of subprogram arguments increase.
 
                        USER COMMAND PROCESSOR - USERCMP
 
   The function of the user  command  processor  is  to  process  user-defined
   commands  for set-up purposes etc. When the main program receives a command
   UCOM, it strips off the UCOM as well as blanks between UCOM  and  the  next
   non-blank  character  and  calls  USERCMP  with  the remaining buffer as an
   argument. A skeleton user command processor is shown on the next page.
    
   02-Feb-02 ........... U350  CHIL (UNIX version) - WTM ............ PAGE  22
 
   U350.190  USER-SUPPLIED SUBROUTINES - COMMAND PROCESSOR
 
   The skeleton user command processor given below  illustrates  some  of  the
   basic  functions  normally  required.  Note:  the  use  of  routines  GREAD
   (reformats input line), FINAME  (picks  up  file  name),  FILMAN  (creates,
   opens  &  closes  files),  MILV  (pause proof decoder of integer & floating
   number fields), and UMESSO (message sender).
 
         SUBROUTINE USERCMP(IWD)
         INTEGER*4 IWD(20),LWD(2,40),ITYP(40),NAMFIL(6),MESBUF(13,3)
         COMMON/MYCOM/ IV(50),NI
         EQUIVALENCE (KMD,LWD(1,1))
         DATA MESBUF/
        1'UNRECOGNIZED UCOM COMMAND - IGNORED                 ',
        2'SYNTAX ERROR IN UCOM COMMAND - IGNORED              ',
        3'ERROR DECODING LIST OF INTEGER NUMBERS              '/
   C
         CALL GREAD(IWD,LWD,ITYP,NF,1,80,NTER)    ;RE-FORMAT INPUT LINE
   C
         IF(KMD.EQ.'FILE') GO TO 100              ;TEST COMMAND TYPE
         IF(KMD.EQ.'ICON') GO TO 200              ;  "     "     "
         GO TO 510                                ;ERROR IF NOT FOUND
   C
   C     PICK UP FILE-NAME AND OPEN A FILE ***********************************
   C
     100 CALL FINAME(IWD,5,80,NAMFIL,IERR)        ;GET FILE NAME
         IF(IERR.NE.0) RETURN                     ;TST FOR ERROR
         LU=1                                     ;SPECIFY LOGICAL UNIT
         CALL FILMAN(2,NAMFIL,LU,0,0,0,0,0,ISTAT) ;OPEN THE FILE
         CALL OPERR(ISTAT)                        ;REPORT ANY ERROR
         IF(ISTAT.NE.0) RETURN                    ;TEST FOR ERROR
   C
   C     DO WHATEVER  -  AND RETURN ******************************************
   C
         RETURN
   C
   C     DECODE A LIST OF INTEGER VALUES *************************************
   C
     200 IF(NTER.NE.0) GO TO 520                  ;TEST FOR GREAD ERROR
         NI=0                                     ;INIT # OF VALUES
         DO 210 J=2,NF                            ;LOOP ON NF-1 FIELDS
         CALL MILV(LWD(1,J),IV(J-1),XX,KIND,IERR) ;DECODE INTO IV(J-1)
         IF(IERR.NE.0.OR.KIND.NE.1) GO TO 530     ;TEST FOR ERROR
     210 CONTINUE
         NI=NF-1                                  ;SET # OF VALUES
         RETURN
   C
   C     SEND ERROR MESSAGES *************************************************
   C
     510 CALL UMESSO(1,MESBUF(1,1))
         RETURN
     520 CALL UMESSO(1,MESBUF(1,2))
         RETURN
     530 CALL UMESSO(1,MESBUF(1,3))
         RETURN
         END
    
   02-Feb-02 ........... U350  CHIL (UNIX version) - WTM ............ PAGE  23
 
 
 
   U350.190  USER-SUPPLIED SUBROUTINES - MESSAGES FROM
 
   Information may be sent to the terminal and/or the  log-file  by  means  of
   routine  UMESSO  which  is  included  in the main package. The procedure is
   this: Set the text that you wish to be transmitted into an INTEGER*4  array
   of  dimension 13 and call UMESSO to send it. The following code illustrates
   the different options:
 
         INTEGER*4 MESBUF(13)
   C
         DATA MESBUF/'THIS IS A MESSAGE   ',8*4H    /
   C
         CALL UMESSO(1,MESBUF)   ;Send to terminal only
         CALL UMESSO(2,MESBUF)   ;Send to log-file only
         CALL UMESSO(3,MESBUF)   ;Send to terminal and log-file
 
   U350.200  HOW TO CREATE CUSTOMIZED CHIL-BASED TASKS
 
   (1)....Use the editor to create  the  desired  USERSUBS,  USERCMP  and  any
          other  required  routines  and save all such routines on one or more
          files (sub1.f, sub2.f ... for example).
 
   (2)....Copy   /usr/hhirf/scan.make   (or  /home/upak/scan.make)  into  your
          directory and use it as a template to construct your own  customized
          make.file.  These  files  are  listed  below.  The  portions  of the
          make-file that you may need to replace are shown bold faced.
 
   For DECstation Users
   DIRA= /usr/hhirf/
   DIRB= /usr/users/milner/Dscan/
   OBJS= $(DIRA)scan.o $(DIRB)dummysubs.o
   LIBS= $(DIRA)scanlib.a $(DIRA)orphlib.a
   scan: $(OBJS) $(LIBS)
   	f77 -O2 $(OBJS) $(LIBS) -o scan
 
   For SPARCstation Users
   DIRA= /home/upak/
   DIRB= /home/milner/Dscan/
   OBJS= $(DIRA)scan.o $(DIRB)dummysubs.o
   LIBS= $(DIRA)scanlib.a $(DIRA)orphlib.a
   scan: $(OBJS) $(LIBS)
   	f77 -O2 $(OBJS) $(LIBS) -o scan
   ---------------------------------------------------------------------------
 
                         LOGICAL UNITS AND COMMON BLOCKS
 
   CHIL-based SCAN programs use - LOGICAL UNITS 4,5,6,7,10,14
   CHIL-based PRESCAN progs use - LOGICAL UNITS 5,6,7,8,10,14
   User supplied routines should NOT attempt to use these LOGICAL UNITS.
   All CHIL-based tasks use COMMON BLOCK labels /AAA/ through /ZZZ/
   User supplied routines should not use these COMMON BLOCK labels .
 
   See LEMO document for how to create customized prescan programs.
    
   02-Feb-02 ........... U350  CHIL (UNIX version) - WTM ............ PAGE  24
 
   U350.210  DIRECTORY FILE - STRUCTURE
 
   STRUCTURE OF .drr-FILE - FIRST RECORD (128 BYTES) *************************
 
   JDIRF(1-3)   - 'HHIRFDIR0001'
   JDIRF(4)     - # of histograms on .his-file
   JDIRF(5)     - # of half-words on .his-file
   JDIRF(7-12)  - YR,MO,DA,HR,MN,SC (date, time of CHIL run)
   JDIRF(13-32) - TEXT (entered in CHIL via $TEX command)
 
   STRUCTURE OF .drr-FILE - DIRECTORY ENTRY (128 BYTES) **********************
 
   IDIRH(1)     - Histogram dimensionality (max = 4)
   IDIRH(2)     - Number of half-words per channel (1 or 2)
   IDIRH(3-6)   - Histogram Parm#'s (up to 4 parameters)
   IDIRH(7-10)  - Length of raw    parameters (pwr of 2)
   IDIRH(11-14) - Length of scaled parameters (pwr of 2)
   IDIRH(15-18) - MIN channel# list
   IDIRH(19-22) - MAX channel# list
   IDIRF(12)    - Disk offset in half-words (1st word# minus 1)
   IDIRF(13-15) - X-Parm label
   IDIRF(16-18) - Y-Parm label
   XDIRF(19-22) - Calibration constants (up to 4 FP numbers)
   IDIRF(23-32) - Sub-title (40 bytes) (entered via $TIT cmd)
 
   STRUCTURE OF .drr-FILE - ID-LIST (32 ID'S/RECORD) *************************
 
   IDLST(1)     - ID number of 1st histogram defined
   IDLST(2)     - ID number of 2nd histogram defined
                -
   DEFINITIONS OF COMMON/DIR/ ************************************************
 
    COMMON/DIR/KLOC(6),JHSP(4),LENG(4),ND,NHW,LENH,LENT,IOF,LDF,
   &NHIS,LEND(4),LENS(4),MINC(4),MAXC(4),CONS(4),ITEX(20),
   &ITIT(10),LABX(3),LABY(3),MSER(10),KFILT
 
   KLOC(I),I=1,6  = YR,MO,DAY,HR,MIN,SEC
   JHSP(I),I=1,4  = Histogram parameters
   LENG(I),I=1,4  = HIST "Lengths" - (MIXC(I)-MINC(I)+1)
   LEND(I),I=1,4  = Raw   -Data "Lengths" in channels
   LENS(I),I=1,4  = Scaled-Data "Lengths" in channels
   MINC(I),I=1,4  = MIN channel# list
   MAXC(I),I=1,4  = MAX channel# list
   CONS(I),I=1,4  = CAL constants
   ITEX(I),I=1,20 = TEXT  from $TEX CHIL-entry
   ITIT(I),I=1,10 = TITLE from $TIT CHIL-entry
   LABX(I),I=1,3  = X-Parm label if any
   LABY(I),I=1,3  = Y-Parm label if any
   ND   = Dimensionality of histogram (# parms and # lengths)
   NHW  = # half-words/channel
   LENH = # of half-words in this histogram
   LENT = Not defined
   IOF  = Disk "offset" of 1st WD of this histogram (1st WD # -1)
          (in half-words)
   LDF  = Length of disk file "user.his" in half-words
   NHIS = Total # of histograms on file "user.his"
    
   02-Feb-02 ........... U350  CHIL (UNIX version) - WTM ............ PAGE  25
 
 
   U350.220  EXAMPLES
 
   EXAMPLE-1 - SIMPLE PROGRAM DEFINING SYMBOLS & USING LOOPS & GATES
 
   $TEX XFER EXP 150Nd+154Sm with 2-PPAC 3-GE 4-NAI (P2G3N4)
   $LSTL = 8192                  ;Tape record length (bytes)
   $NPR  = 21                    ;Number of parameters/event (max)
   $LPR 1 TO 5,2  = 8192         ;Length of parameters 1,3,5 - GE Energy
   $LPR 2 TO 6,2  = 2048         ;Length of parameters 2,4,6 - GE Time
   $LPR 7 TO 21,1 = 2048         ;Length of all other parameters
   $DIP GE(3),GT(3)              ;Define names - GE  Energy & GE  Time
   $DIP NAE(4),NAT(4)            ;Define names - NAI Energy & NAI Time
   $DIP LX(1),LY(1)              ;Define names - PPAC Left-X & Left-Y
   $DIP RX(1),RY(1)              ;Define names - PPAC Right-X & Right-Y
   $DIP DT(1)                    ;Define name  - Delta Time
   $DIP H(1)                     ;Define name  - NAI Total Energy
   $DIP K(1)                     ;Define name  - NAI Multiplicity
   $ASS  GE(1 TO 3) = 1,2        ;Assign Parameter# to GE  Energy
   $ASS  GT(1 TO 3) = 2,2        ;Assign Parameter# to GE  Time
   $ASS NAE(1 TO 4) = 7,2        ;Assign Parameter# to NAI Energy
   $ASS NAT(1 TO 4) = 8,2        ;Assign Parameter# to NAI Time
   $ASS H(1)  = 15               ;Assign Parameter# to NAI Total Energy
   $ASS K(1)  = 16               ;Assign Parameter# to NAI Multiplicity
   $ASS LX(1) = 17               ;Assign Parameter# to PPAC Left-X
   $ASS LY(1) = 18               ;Assign Parameter# to PPAC Left-Y
   $ASS RX(1) = 19               ;Assign Parameter# to PPAC Right-X
   $ASS RY(1) = 20               ;Assign Parameter# to PPAC Right-Y
   $ASS DT(1) = 21               ;Assign Parameter# to Delta Time
   $H32                          ;Specify 32-bits/channel histogramming
 
   $HID 1
   $TIT PPAC's
         H(RX(1),LX(1)) L(256,256)             ;PPAC L-X vs R-X    ID = 1
         H(RY(1),LY(1)) L(256,256)             ;PPAC L-Y vs R-Y    ID = 2
         H(DT(1),LX(1)) L(256,256)             ;PPAC L-X vs D-T    ID = 3
   $TIT K vs H
         H(K(1),H(1))   L(64,128)              ;K vs H             ID = 4
   $TIT GE RAW
   $HID 11
         DO GEE I=1,3                          ;GE Energy   ID = 11 to 13
         H(GE(I)) L(4096)  G(GE(I) 0,8191)
   GEE   CONTINUE
   $HID 21
         DO GET I=1,3                          ;GE Time     ID = 21 to 23
         H(GT(I)) L(256)   G(GT(I) 0,2047)
   GET   CONTINUE
   $TIT NAI RAW
   $HID 31
         DO NAIE I=1,4                         ;NAI Energy  ID = 31 to 34
         H(NAE(I)) L(1024) G(NAE(I) 0,2047)
   NAIE  CONTINUE
   $HID 41
         DO NAIT I=1,4                         ;NAI Time    ID = 41 to 44
         H(NAT(I)) L(256)  G(NAT(I) 0,2047)
   NAIT  CONTINUE
    
   02-Feb-02 ........... U350  CHIL (UNIX version) - WTM ............ PAGE  26
 
 
 
   EXAMPLE-2 - PRESCAN EXAMPLE SELECTING EVENTS SATISFYING 1 CONDITION
 
   $NPR 18                    ;SPECIFY NUMBER OF PARAMETERS
   $LPR 1 TO  2 = 64          ;SPECIFY LENGTH OF PARAMETERS 1 & 2
   $LPR 3 TO 18 = 2048        ;SPECIFY LENGTH OF PARAMETERS 3 TO 18
   $LSTL        = 8192        ;SPECIFY TAPE RECORD LENGTH IN BYTES
   $BAN (256) 1               ;REQUEST FREE-FORM-GATE FROM ban-file
                              ;(X-LENGTH = 256, ID-NUMBER = 1)
   $BAF DCE2.ban              ;GIVE NAME OF BAN-FILE
 
         IFU(B(18,16 1)) 100  ;TEST X,Y-PARAMETERS (18,16) AGAINST GATE
                              ;GO TO 100 IF GATE NOT SATISFIED
 
         CALL REPACK1 1,18    ;OTHERWISE, SAVE EVENT IN OUTPUT STREAM
 
     100 CONTINUE
    
   02-Feb-02 ........... U350  CHIL (UNIX version) - WTM ............ PAGE  27
 
   EXAMPLE-4  - PROGRAM USING IF-STATEMENTS, COMPUTED GOTO'S & LOOPS
 
   $LSTL=8192                               ;TAPE RECORD LENGTH (BYTES)
   $NPR =240                                ;SPECIFY # OF PARAMETERS
   $LPR   1 TO 240,1=2048                   ;ASSIGN PARAMETER LENGTHS
   $LPR 219         =64                     ;   "      "        "
   $LPR 220 TO 222,1=1024                   ;   "      "        "
   $LPR 223 TO 228,1=256                    ;   "      "        "
   $LPR 229 TO 239,2=8192                   ;   "      "        "
   $DIP NAI(72),GE(6)                       ;DEFINE PARAMETER NAMES
   $DIP TOTH(1),TOTK(1)                     ;   "      "        "
   $DIP LAMD(3),PHI(1)                      ;   "      "        "
   $ASS NAI(1 TO 72)=1,3                    ;ASSIGN VALUES TO NAMES
   $ASS GE(1 TO 6)  =229,2                  ;   "      "    "   "
   $ASS TOTH(1)     =218                    ;   "      "    "   "
   $ASS TOTK(1)     =219                    ;   "      "    "   "
   $ASS LAMD(1 TO 3)=220,1                  ;   "      "    "   "
   $ASS PHI(1)      =224                    ;   "      "    "   "
   $GLST  64,1 1,15 16,18 19,21 22,24 25,60 ;SPECIFY SIMPLE GATE-LIST
   $H16                                     ;SPECIFY 16-BITS/CHANNEL
         IFS(G(TOTH(1),320,2000))10         ;TST TOTH VS SIMPLE GATE
         IFU(G(TOTK(1),1,8))     1000       ;TST TOTK VS SIMPLE GATE
      10 DO 15 I=1,6                        ;LOOP ON 6 GELI'S
         IFS(G(GE(I),1431,1440)) 20         ;TST VS GATE (1431,1440)
      15 CONTINUE
         GO TO 1000                         ;RETURN IF NONE HIT
 
                                            ;TST TOTK VS GATE-LIST IN
      20 IFC(GS(TOTK(1),1,1,5))100,200,300,400,500
                                            ;COMPUTED GOTO
         GO TO 1000                         ;RETURN IF NONE HIT
 
     100 H(NAI(2))  L(1024)                 ;FOR TOTK = 1 TO 15
         DO 110 I=3,72                      ;HISTOGRAM NAI(2)-NAI(72)
         OH(NAI(I)) L(1024)                 ;ALL IN SAME SPACE
     110 CONTINUE
         GO TO 1000
     200 H(NAI(2))  L(1024)                 ;FOR TOTK = 16 TO 18
         DO 210 I=3,72
         OH(NAI(I)) L(1024)
     210 CONTINUE
         GO TO 1000
     300 H(NAI(2))  L(1024)                 ;FOR TOTK = 19 TO 21
         DO 310 I=3,72
         OH(NAI(I)) L(1024)
     310 CONTINUE
         GO TO 1000
     400 H(NAI(2))  L(1024)                 ;FOR TOTK = 22 TO 24
         DO 410 I=3,72
         OH(NAI(I)) L(1024)
     410 CONTINUE
         GO TO 1000
     500 H(NAI(2))  L(1024)                 ;FOR TOTK = 25 TO 60
         DO 510 I=3,72
         OH(NAI(I)) L(1024)
     510 CONTINUE
    1000 CONTINUE
    
   02-Feb-02 ........... U350  CHIL (UNIX version) - WTM ............ PAGE  28
 
 
   EXAMPLE-5 - PROG USING FREE-FORM-GATES, USERSUB1, RANGES, LOOPS, ETC
 
   $TEX PRE-SCANNED GMR DATA WITH NN SUM (3 LEVEL)
 
   $LSTL = 8192                            ;TAPE RECORD LENGTH (BYTES)
 
   $NPR  = 248                             ;NUMBER OF PARAMETERS
 
   $LPR   1 TO 248,1 = 2048                ;ASSIGN PARAMETER LENGTHS
   $LPR   2 TO 215,3 = 1024                ;   "      "        "
   $LPR 218 TO 234,1 = 8192                ;   "      "        "
   $LPR 220 TO 235,3 = 2048                ;   "      "        "
   $LPR 217          = 4096                ;   "      "        "
   $LPR 239          = 1024                ;   "      "        "
   $LPR 236          = 4096                ;   "      "        "
 
   $DIP NAE(72),NAT(72),NAS(72)            ;DEFINE PARAMETER NAMES
   $DIP PE(6),PDE(6),PT(6)                 ;   "      "        "
   $DIP SUMH(1),FOLD(1),EMAX(1),PART(1)    ;   "      "        "
   $DIP VECP(1),COST(1),COSR(1),PHIR(1)    ;   "      "        "
   $DIP COSE(1),FFG(1)                     ;   "      "        "
 
   $ASS NAE(1 TO 72) = 1,3                 ;ASSIGN VALUES TO NAMES
   $ASS NAT(1 TO 72) = 2,3                 ;   "      "    "   "
   $ASS NAS(1 TO 72) = 3,3                 ;   "      "    "   "
   $ASS PE( 1 TO  6) = 218,3               ;   "      "    "   "
   $ASS PDE(1 TO  6) = 219,3               ;   "      "    "   "
   $ASS PT( 1 TO  6) = 220,3               ;   "      "    "   "
   $ASS SUMH(1) =236                       ;   "      "    "   "
   $ASS FOLD(1) =237                       ;   "      "    "   "
   $ASS EMAX(1) =238                       ;   "      "    "   "
   $ASS PART(1) =239                       ;   "      "    "   "
   $ASS VECP(1) =240                       ;   "      "    "   "
   $ASS COST(1) =241                       ;   "      "    "   "
   $ASS COSR(1) =242                       ;   "      "    "   "
   $ASS PHIR(1) =243                       ;   "      "    "   "
   $ASS COSE(1) =244                       ;   "      "    "   "
   $ASS FFG(1)  =248                       ;   "      "    "   "
 
   $BAN (1024) 1                           ;REQUEST FREE-FORM GATE
   $BAF GPOST.ban/85                       ;GIVE NAME OF BAN-FILE
 
   $H32                                    ;SPECIFY 32-BITS/CHANNEL
 
         H(SUMH(1)) L(512)                             ;SUMH SINGLES
         H(PART(1)) L(512)                             ;PART SINGLES
 
         H(PART(1),SUMH(1)) L(256,256) R(70,255 0,255) ;SUMH VS PART
 
         IFU(B(PART(1),SUMH(1),1)) ZIP                 ;TST BANANA-GATE
 
         CALL USERSUB1                                 ;CALL USERSUB1
                                                       ;TO DO THE MAGIC
 
                            (CONTINUED ON NEXT PAGE)
    
   02-Feb-02 ........... U350  CHIL (UNIX version) - WTM ............ PAGE  29
 
 
   EXAMPLE-5  (CONTINUED)
 
         H(NAS(2)) L(512)                              ;OVERLAY
         DO 50 J=3,72                                  ;NAS(2 THRU 72)
         OH(NAS(J)) L(512)                             ;BANANA-GATED
      50 CONTINUE
 
         H(NAE(2))  L(512)                             ;OVERLAY
         DO 60 J=3,72                                  ;NAE(2 THRU 72)
         OH(NAE(J)) L(512)                             ;BANANA-GATED
      60 CONTINUE
 
         H(PART(1),SUMH(1)) L(256,256) R(70,255 0,255) ;SUMH VS PART
                                                       ;BANANA-GATED
 
         H(NAS(2) ,SUMH(1)) L(256,256) R(0,255 0,127)  ;OVERLAY (2-72)
         DO 100 J=3,72                                 ;SUMH VS NAS
         OH(NAS(J),SUMH(1)) L(256,256) R(0,255 0,127)  ;BANANA-GATED
     100 CONTINUE
 
         H(NAE(2) ,SUMH(1)) L(256,256) R(0,255 0,127)  ;OVERLAY (2-72)
         DO 200 J=3,72                                 ;SUMH VS NAE
         OH(NAE(J),SUMH(1)) L(256,256) R(0,255 0,127)  ;BANANA-GATED
     200 CONTINUE
 
         IFU(G(VECP(1),950,1023)) ZIP                  ;1ST GATE ON VECP
                                                       ;IF SATISFIED,
         H(NAS(2) ,SUMH(1)) L(256,256) R(0,255 0,127)  ;OVERLAY (2-72)
         DO 300 J=3,72                                 ;SUMH VS NAS
         OH(NAS(J),SUMH(1)) L(256,256) R(0,255 0,127)  ;FOR 1ST VECP-GATE
     300 CONTINUE                                      ;PLUS BANANA-GATE
 
         IFU(G(VECP(1),980,1020)) ZIP                  ;2ND GATE ON VECP
                                                       ;IF SATISFIED,
         H(NAS(2) ,SUMH(1)) L(256,256) R(0,255 0,127)  ;OVERLAY (2-72)
         DO 400 J=3,72                                 ;SUMH VS NAS
         OH(NAS(J),SUMH(1)) L(256,256) R(0,255 0,127)  ;FOR 2ND VECP-GATE
     400 CONTINUE                                      ;PLUS BANANA-GATE
 
         H(SUMH(1),COSR(1)) L(256,256) R(0,127 0,255)  ;COSR VS SUMH
         H(SUMH(1),COST(1)) L(256,256) R(0,127 0,255)  ;COST VS SUMH
         H(SUMH(1),COSE(1)) L(256,256) R(0,127 0,255)  ;COSE VS SUMH
         H(SUMH(1),PHIR(1)) L(256,256) R(0,127 0,255)  ;PHIR VS SUMH
                                                       ;FOR 2ND VECP-GATE
                                                       ;PLUS BANANA-GATE
   ZIP   CONTINUE
    
   02-Feb-02 ........... U350  CHIL (UNIX version) - WTM ............ PAGE  30
 
 
 
   U350.230  COMMENTS AND WARNINGS
 
                                 Warning on NPAR
 
   Don't set NPAR (the maximum number of parameters  per  event)  to  be  less
   than  it  actually is in an attempt to save on unpack time or whatever. Any
   event found to contain more than NPAR parameters is trashed!
 
                       Comment & Warning on Bit-numbering
 
   CHIL defines bit-numbers such that the lo-order bit  is  number-1  and  the
   hi-order  bit  (of  a 16-bit word) is number-16. See SEC# U350.170 for some
   examples if you are still uncertain.
 
                             Warning on Expressions
 
   CHIL evaluates EXPRESSIONS left-to-right  -  NOT  like  FORTRAN.  See  SEC#
   U350.170 for some examples.
 
            Warning on Banana-Gate Specifications - B(PX,PY IDA,IDB)
 
   At a given X-coordinate - Banana IDA+1 must lie above Banana IDA
 
                           - Banana IDA+2 must lie above Banana IDA+1
 
                           - Banana IDA+N must lie above Banana IDA+N-1
 
   This  rule  is  made in the interest of speed. If you can't live by it, you
   will have to do your Banana gating one at a time.
