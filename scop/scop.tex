   08-Feb-06 ...... U235  SCOP - Scaler Display for Operators ....... PAGE   1
 
   Sec  Page Contents
 
   010    1  Introduction
 
   020    2  How to Get Started
 
   030    3  List of Commands
 
   040    4  Creation of a Scaler Initialization File - snit-file
 
   050    4  Comments
 
 
   U235.010  Introduction - features
 
   o......Scop  provides  a CAMAC scaler interface between experiments and the
          accelerator operators.
 
   o......Scaler "rate data" (in counts/sec -  normalized  to  the  computer's
          internal  clock)  is provided to the operators via a "connection" to
          the EPICS accelerator control system.
 
   o......When first started scop  will ask you to  specify  a  beamline.  The
          possible beamline names are listed.
 
   o......The  scalers  can  then  be displayed by the operators in windows on
          the same screens they use for tuning.
 
   o......As experimenters, you need only tell  the  operators  that  you  are
          sending  scaler  rate  data  to  them, specify the beamline name and
          which scaler/s they should monitor. scop  permits  up  to  8  scaler
          rates to be displayed.
 
   o......scop  also  displays  (at the experimentors terminal) the count rate
          (in counts/sec) of the scalers being sent to the  operators  in  the
          form  of  software  generated  meters.  Normalization  is  via   the
          computer's internal clock.
 
   o......Scalers  to  be  displayed are specified via the usual "snit file" -
          the same syntax as for program scad. (see SEC# 040 for an example)
 
                    The Figure Below Illustrates the Display
 
   3 scalers  -     MASTER         PSAC-LEFT      PSAC-RIGHT     are displayed
   counts/sec =         32,             1500,           7900     approximately
 
 
 
 
 
 
         (See ORPHAS Handbook Tab-8 for a picture of an example display)
 
 
 
 
 
 
 
 
   Note: The inside poniter indicates the power-of-10 scale factor and is  the
   same as that given in the banner.
    
   08-Feb-06 ...... U235  SCOP - Scaler Display for Operators ....... PAGE   2
 
 
   U235.020  How to Get Started
 
   o......Make a snit-file for only the scalers to be displayed (max# = 8)
 
   o......If you are logging (from lochost) onto another platform (remhost)
          via telnet (instead of ssh)  you will need enable the display on
          your platform (lochost). To do this:
 
          Before you login, Type: xhost +remhost
 
          After  you login, Type: setenv DISPLAY lochost:0.0
 
   o......Type: setenv VME vmex - where vmex denotes the name of the vme
                                  processor that you are using - vme1, vme2..
 
   o......Type: scop            ;To start and attempt to process operator.sca
 
   o......Type: scop  filename  ;To start and process filename
 
   o......Type: snit  filename  ;To process another file after starting scop
 
   o......Type: tst             ;To test the display with generated data
 
          or
 
   o......Type: run             ;To do CAMAC reads & display
 
   o......Type: Ctrl/C          ;To interrupt read/display
 
   ---------------------------------------------------------------------------
 
   o......When  first  started  scop   will ask you to specify a beamline. The
          possible beamline names from which you can choose are listed.  These
          are currently rms, drs, enge, bl21 & bl23.
 
   o......Also  when  scop  is  first  started,  it attempts to open a default
          snit-file operator.sca. If this file  is  found,  it  is  processed,
          displays  are  generated  and  you  are  ready  to  run (or tst). If
          operator.sca is not found, you must specify another file  using  the
          snit filename command.
 
   ---------------------------------------------------------------------------
    
   08-Feb-06 ...... U235  SCOP - Scaler Display for Operators ....... PAGE   3
 
 
 
   U235.030  List of Commands
 
   LON              Turns ON  output to scop.log  (default)
   LOF              Turns OFF output to scop.log
 
   CMD   filename   Processes commands from filename
 
   SNIT  filename   Opens & processes standard snit-file
 
   RAV   N          Sets to display an average of N readings
                    allowed value of N is 0 to 100
                    N = 0 or blank sets N = 5 (the default)
 
   DPS   N          Specifies display rate to be N/sec
                    allowed value of N is 0 to 20
                    N = 0 or blank sets 10 displays/sec (the default)
 
   REVV             Sets to reverse color mapping on next FIG
 
   FIG              Does a FIG for the # of scalers defined (1 to 8)
 
   STAT             Displays/logs current status
 
   DLIN             Set  experimenter's display to linear (default)
   DLOG             Set  experimenter's display to log
 
   DOF              Turn experimenter's display OFF
                    (DLIN, DLOG & DOF do not affect operator's display)
   ---------------------------------------------------------------------------
   glim NAME lo hi  Set limits (lo hi) for NAMEd graphic scaler
 
   glim NAME on     Enable  limits for NAMEd graphic scaler display
   glim NAME off    Disable limits for NAMEd graphic scaler display
   glim NAME null   Deletes limits for NAMEd graphic scaler display
 
   glim on          Enable  limits for ALL   graphic scaler displays
   glim off         Disable limits for ALL   graphic scaler displays
   glim null        Deletes limits for ALL   graphic scaler displays
 
   glim sho         Display all graphic scaler limits
   ---------------------------------------------------------------------------
   GLIM defined limits generate visual alarms on experimentors display:
 
   IF(rate.GT.HI) - Displays flashing yellow disks at upper left & right
   IF(rate.LT.LO) - Displays flashing white  disks at lower left & right
 
   GLIM defined limits have NO EFFECT on Operator's display
   ---------------------------------------------------------------------------
 
   TST              Tests operation - no scalers required
 
   RUN              Start read/display
 
   CTRL/C           Interrupts read & display operation
 
   END              Ends scop
    
   08-Feb-06 ...... U235  SCOP - Scaler Display for Operators ....... PAGE   4
 
 
 
   U235.040  Creation of a Scaler Initialization File - snit-file
 
   Use  the  editor  to  prepare a scaler initialization file (snit file - the
   filename extension is usually .sca). The general form of the snit  file  is
   shown below:
 
   ---------------------------------------------------------------------------
   LABEL C N A
     .
   LABEL C N A
     .
   LABEL C N A ECL
     .
   LABEL C N A ECL
     .
   $END                          (flags the end of scalers to be read)
   ---------------------------------------------------------------------------
   Note: No computed scalers are supported by scop
   ---------------------------------------------------------------------------
 
   LABEL..denotes  a  unique  label  (11 characters max) which must contain no
          imbedded blanks or the characters + - / *
 
   C......denotes the scaler module crate number.
 
   N......denotes the scaler module slot number.
 
   A......denotes the scaler sub-address (numbers start at 0).
 
   ECL....denotes an ECL-scaler (requires special read/clear functions).
 
   ---------------------------------------------------------------------------
 
   Example snit file
 
   Integrator 9 17  2
   Ge.Trigs   9 17  4
   See.Trigs  9 17  5
   $END
 
 
   U235.050  Comments
 
   o......This program was written  in  attempt  to  provide  the  accelerator
          operators with a "cable free" tuning aid.
 
   o......Remember  that  when  you interrupt scop via a ctrl/c, the operators
          will no longer be recieving rate data.
 
   o......This is the reason that I have not included the "send  data  to  the
          operator"  feature  in  scad.  There may be reasom to interrupt scad
          more often to do such things  as  seting  limits,  changing  display
          types, telling alarms to "shut up" (if play ever works again), etc.
 
   o......So  the  idea  for  scop is that you set up the few scalers that the
          operators need to monitor and start it & leave it alone!'
